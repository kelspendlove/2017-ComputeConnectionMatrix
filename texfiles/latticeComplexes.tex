%!TEX root = ../main.tex



\section{$J(L)$-Filtered Cell Complexes}

In applications, data often come in the form of a cell complex $(X,\kappa,\preceq)$ filtered by a partial order $(P,\leq)$.  This is codified in terms of a map $f:(X,\kappa, \preceq)\to (P,\leq)$ which restricts to a poset morphism $f:(X,\preceq)\to (P,\leq)$.  In practice, we work with respect to some lattice $L\in {\bf FDLat}$ and its poset of join-irreducibles.

\begin{defn}
{\em
Let $(X,\kappa,\preceq)$ be a cell complex.  Let $L\in FDLat$.  A {\em $J(L)$-filtered poset} is a map $f:(X,\kappa,\preceq)\to (J(L),\leq)$ such that $f:(X,\preceq)\to (J(L),\leq)$ is order preserving.
}
\end{defn}

In the context of the computational Conley theory, one starts with a multivalued map $\cF: X^n\rightrightarrows X^n$ on the top-cells of a complex.    The top-cells are thought of as vertices in a directed graph whose edges are given by $\xi_1\to \xi_2$ if $\xi_2\in \cF(\xi_1)$.  The strongly connected components of this directed graph have a partial order $J(L)$ which may be computed efficiently via Tarjan's algorithm.   This partial-ordering on top cells gives an implicit ordering on all lower-dimensional cells (i.e. faces) of $X$.  The implicit ordering yields $J(L)$-filtered complex $(X,\kappa,\preceq)\to (J(L),\leq)$.




If $f:(X,\kappa,\preceq)\to (J(L),\leq)$ is $J(L)$-filtered, then $f:(X,\preceq)\to (J(L),\leq)$ is a poset morphism.  In this case, Birkhoff's theorem provides a lattice homomorphism $O(f):L \to O(X,\preceq)$.  $O(X,\preceq)$ is lattice of downsets of the face poset and for a cell complex $(X,\kappa,\preceq)$ the downsets of $(X,\leq)$ correspond to the subcomplexes of $(X,\kappa,\leq)$. Moreover, they provide a basis for the subcomplexes of $C(X)$.   Thus a $J(L)$-filtered complex gives rise to a parameterization of subcomplexes in $(X,\kappa,\preceq)$.



% In computational dynamics, $f$ is a combinatorial model for dynamics.  The fibers of $f$ parameterize the recurrent sets, and the order structure on $(P,\leq)$ organizes the gradient-like behavior.  

%We give a few illustrations of how a cell fibration may arise in practice:
%
%\begin{description}
%
%\item[Applied Topology]  Within applied topology, often data comes as a height function $X\to \R$, and one examines the change in the topology of the sublevel sets $f^{-1}(-\infty,t]$.  For instance, when $X$ is a collection of pixels, a new function is then defined on a cubical complex $\cX$ corresponding to the image such that the sublevel sets are subcomplexes.  This produces a cell fibration $(\cX,\preceq)\to (\R,\leq)$.  
%
%\item[Morse Theory] For a Morse function $f:M\to \R$ on smooth manifold $M$ one often examines the flow defined generated by $\dot x = -\nabla f(x)$.  The fixed points of the flow are indexed by a poset~\cite{smale} and their unstable manifolds carve out a CW decomposition of the manifold.  The map sending each cell in the CW-complex to its index within the poset is a cell fibration.
%
%\item[Dynamics] In computational dynamics, especially the database approaches, one often has a transitive relation defined on a cubical complex.  The transitive relation partitions the complex into recurrent and gradient-like behavior which takes the form of a chain fibration.  See the braids paper for a concrete example.
%
%\item[Combinatorics] Mrozek's multivector field
%
%\end{description}

\section{Lattice-Filtered Chain Complexes}\label{sec:cf}

In this section we axiomize what we observed from Birkhoff's theorem.  We introduce {\em lattice-filtered chain complexes}, which will be our primary data structure.   Using this, we will introduce our analogue of the connection matrix.  For this section, let $k$ be a field and $L\in \bFDLat$.  All chain complexes will be from $\bCh(k)$.

\begin{defn}
{\em
An {\em $L$-filtered chain complex} is a lattice homomorphism $f:L\to Sub(C,d)$.
}
\end{defn}

We form a category of $L$-filtered chain complexes, which we denote $\bCh(L)$.  To discuss the morphisms of this category, we introduce the notion of a filtered map.  Let $h:L\to Sub(C)$ and $g:L\to Sub(D)$ be filtered chain complexes. A map $\phi:C\to D$ is {\em filtered} if $\phi(h(q))\subseteq g(q)$ for each $q\in L$.  For instance, it is straightforward that for $L\to Sub(C,d)$ the differential $d:C\to C$ is filtered.  The morphisms of $\bCh(L)$ are precisely the filtered chain maps.

\begin{rem}
Let $h:L\to Sub(C,d)$ be in $Ch(k,L)$.  Define $C_q := h(q)$. It is straightforward that the collection $\{C_q\}_{q\in L}$ is filtered chain complex in the sense of~\cite[Section 7]{salamon}.
\end{rem}


A $J(L)$-filtered cell complex gives rise to an $L$-filtered chain complex in the following fashion.   Using Birkhoff's theorem $f:(X,\kappa,\preceq)\to (J(L),\leq)$ becomes $O(f):L\to Sub(X,\preceq)$.  $Sub(X,\preceq)$ is the lattice of downsets in the face poset.  Each element of $Sub(X,\preceq)$ is a basis for the subcomplex of $C(X)$ spanned by the downset.   Thus the map defined by $L\ni q\mapsto span(O(f)(q))\in Sub(C(X))$ is an $L$-filtered chain complex.  We call this the $L$-filtered chain complex associated to $f$.




\section{The Homotopy Category of Filtered Complexes}\label{sec:homotopy}

Given our category of $Ch(k,L)$ we can now describe a homotopy category of these objects.

\begin{defn}
{\em
Let $f:L\to Sub(C,d_C)$ and $g:L\to Sub(D,d_D)$ be $L$-filtered chain complexes.  Let $\phi,\psi:f\to g$.  We say $\phi$ and $\psi$ are {\em chain homotopic} if there is a map $h:C\to D$ such that
\begin{enumerate}
\item $f-g = hd_C+d_Dh$
\item $h$ is filtered
\end{enumerate}  
}
\end{defn}

The map $h$ is called a {\em filtered chain homotopy}.  We write $\psi\sim \phi$ if $\psi$ and $\phi$ are filtered chain homotopic.  It is straightforward that this is an equivalence relation.  

\begin{defn}
{\em
Let $f:L\to Sub(C)$ and $f':L\to Sub(C)$ be $L$-filtered complexes.  Let $\phi:f\to f'$ and $\psi:f'\to f$ be filtered chain maps.  A {\em filtered chain homotopy equivalence} consists of a quadruple $(\psi,\phi,h,g)$ such that 
\begin{enumerate}
\item $\psi\phi-id_C= hd_C + d_Ch$
\item $\phi\psi-id_D = gd_D+d_Dg$
\end{enumerate}
}
\end{defn}


We can now describe the homotopy category $K(k,L)$.  The objects are $L$-filtered chain complexes.  The morphisms are given by the homotopy equivalence classes, i.e. $Hom_K(f,g) = Hom_{Ch}(f,g)/\sim$ where $\sim$ is the homotopy equivalence relation.   Isomorphisms in $K(k,L)$ correspond to filtered chain equivalences.




\section{The Conley Complex}

Speaking formally,~\cite{fran} defines a connection matrix as a boundary operator on Conley indices.~\cite{salamon} define a connection matrix as a chain complex with an appropriate boundary operator.    We introduce the idea of a Conley complex as a way of clarifying these ideas.  The Conley complex has the property that its boundary operator is a connection matrix in the sense of~\cite{fran}.   

\begin{defn}
{\em A Conley complex is an $L$-filtered complex $f:L\to Sub(A,d_A)$ such that $d_A(f(q))\subseteq d_A(f(Pred(q))$ for any $q\in J(L)$.
}
\end{defn}

The condition of the Conley complex implies that the induced differential of the subquotient $h(q)/h(Pred(q))$ is zero.  That is, $H(h(q)/h(Pred(q))) = h(q)/h(Pred(q))$.   Thus each subquotient corresponding to a join-irreducible is a classically a Conley index.  


%\begin{rem}
%This is what~\cite{salamon} dub a `connection matrix'.  We will be more in line with Franzosa and refer to the boundary operator $d_A$ of a Conley complex as the connection matrix.
%\end{rem}

Let $f:L\to Sub(C)$ be a filtered chain complex.  We say that $f':L\to Sub(A,d_A)$ is a Conley complex for $f$ if $f'$ and $f$ are isomorphic in $K(k,L)$.  The boundary operator $d_A$ is called a {\em connection matrix} for $f$. 



\section{Relationship to Chain Complex Braids}

\subsection{$J(L)$-Splitting}

In applications the chain complexes we work with are associated to cell complexes.  This implies that the chain complex $C(X)$ has a canonical basis. In this case we can outline a relationship between the category $\bCh(L)$ and $\bCCB(J(L))$.


In~\cite[Section 7]{salamon} the splitting of a filtered chain complex is introduced.   For $h:L\to Sub(C,d)$ the chain complex $C$ may be written as $C=\bigoplus_{q\in J(L)} h(q)/h(Pred(q))$.  One can realize the quotient spaces $h(q)/h(Pred(q))$ as subspaces of $C$.  Choosing bases for these subspaces gives a filtered basis, i.e. a basis $B=\{b_\alpha\}$ of $C$ with a map $f:B\to J(L)$ such that $\{b\in B: f(b)\leq p\}$ is a basis for $h(p)$.

In our case, if $L\to Sub(C(X),d)$ is derived from $f:(X,\kappa,\leq)\to (P,\leq)$ then $X$ describes a basis for $C(X)$ and the map $f:X\to J(L)$ is a filtered basis.   Thus we can rewrite $C(X)$ as $C(X)=\bigoplus_{p\in J(L)} h(q)/h(Pred(q))$ and $d$ is an upper triangular boundary map with respect to this splitting.  We call this the $J(L)$-splitting.


In particular, for Conley complexes the $J(L)$-splitting has a particular form.  Since $\partial(h(q)\subseteq h(Pred(q))$ we have for $h(q)/h(Pred(q))$ the induced differentials are zero, and thus $h(q)/h(Pred(q)) = H_\bullet(H(q)/h(Pred(q))$ where the homology is thought of as an object of $Ch_0$.  Thus for a Conley complex we may write the splitting as $C(X) = \bigoplus_{p\in J(L)} H(h(p)/h(Pred(p))$.


In summary, this splitting is canonical since $X$ is a canonical basis for $C(X)$.  From the $J(L)$-splitting we can construct a chain complex braid.





%In this case $C(X)$ has a canonical basis generated by the cells $X$.  An $L$-filtered complex $L\to Sub(C(X),d)$ has a splitting $C(X)=\bigoplus_{q\in J(L)} C_q/C_{Pred(q)}$.  The boundary map $d$ is upper triangular with respect to this splitting.


\subsection{Functor to $\bCCB(J(L))$}

We work with the full subcategory $\bCh_X(L)$ of $\bCh(L)$ whose objects are $L$-filtered chain complexes associated to some filtered cell complex.  There is a functor $F:\bCh_X(k,L)\to \bCCB(J(L))$.

Consider $h:L\to Sub(C(X),d)$.  There is a $J(L)$-splitting $C(X)=\bigoplus_{q\in J(L)} h(q)/h(Pred(q))$. Then $d$ is upper triangular with respect to the splitting.  We use the following observation from Franzosa:


\begin{prop}[\cite{fran}, Proposition 3.4]
Given an upper triangular boundary map $$\Delta:\bigoplus_{q\in J(L)} C_q\to \bigoplus_{q\in J(L)} C_q$$ the collection, denoted $\cC\Delta(J(L))$, consisting of the chain complexes $C(I)$ with boundary map $\partial(I)$ for each $I\in I(J(L))$ and the obvious chain maps $i(I,IJ)$ and $p(IJ,J)$ for each $(I,J)\in I_2(J(L))$ is a chain complex braid over $J(L)$.
\end{prop}


Thus there is an assignment $F:\bCh_X(k,L)\to \bCCB(J(L))$.   Moreover, due to our observation in the previous section, we have a Conley complex $L\to Sub(C,d)$ has a splitting $\bigoplus_{p\in J(L)} H(h(p)/h(Pred(p))$ and $d$ is upper triangular with respect to this splitting.  This is precisely what Franzosa identifies as a connection matrix.  Therefore Conley complexes get sent under this assignment to chain complex braids induced from a connection matrix.



Morphisms of $L$-filtered complexes can be written as matrices with respect to the filtered bases.   Since these morphisms are filtered, they are upper triangular with respect to $J(L)$.  A morphism of lattice-filtered chain complexes $\phi:f\to g$ induces an upper-triangular map on the associated $J(L)$-decompositions $\phi:\bigoplus_{q\in J(L)}C_q\to \bigoplus_{q\in J(L)} C_q'$ such that $\phi\partial_f = \partial_g\phi$.  The following observation shows that this induces a morphism between the associated chain complex braids $\cC(f)\to \cC(g)$.

\begin{prop}[\cite{atm}, Proposition 3.2]
Let $\Delta,\Delta'$ be an upper triangular boundary map for $\bigoplus_{q\in J(L)} C_q$ and $\bigoplus_{q\in J(L) C_q'}$, respecitvely.  If $T:\bigoplus_{q\in J(L)} C_q\to \bigoplus_{q\in J(L)} C_q'$ is upper triangular with $T\Delta = \Delta'T$ then $\cT:=\{T(I)\}_{I\in I(P)}$ is a chain complex braid morphism from $\cC(\Delta)\to \cC(\Delta')$.
\end{prop}

In conclusion, there is a functor from $F:Ch_X(k,L)\to CCB(J(L))$.  Moreover, this functor preserves the homotopy equivalence relationship.

\begin{prop}
Let $h:L\to Sub(C)$ and $h':L\to Sub(D)$ be filtered homotopy equivalent.  Then $F(h)$ and $F(h')$ are homotopy equivalent chain complex braids.
\end{prop}

\begin{cor}
Let $h:L\to Sub(C)$ and $h':L\to Sub(D)$ be filtered homotopy equivalent. Then $\cH(F(h))$ and $\cH(F(h'))$, the graded module braids induced by the chain complex braids $F(h)$ and $F(h')$, are isomorphic. 
\end{cor}


Finally, we formalize the previous discussion with a theorem relating our language to Franzosa's.  The proof is a straightforward from our previous observations.

\begin{thm}\label{thm:cfcm}
Let $h\in Ch_X(k,L)$.  Let $g:L\to Sub(A,d_A)$ be a Conley complex for $h$.  Then $d_A$ is a connection matrix, in the sense of Franzosa~\cite[Definition 3.6]{fran}, for $\cH(F(h))$, the graded module braid induced by the chain complex braid $F(h)$.
\end{thm}


 %We'll want to show that a Conley complex gets sent to a connection matrix.  We'll also want to show that a connection matrix for a particular filtered complex gets sent to a connection matrix for its chain complex braid.

%Let $f:L\to Sub(C,d_C)$ be an $L$-filtered complex.  Consider the splitting of the $L$-filtered complex $C=\bigoplus_{q\in J(L)} C_q$.  With respect to this splitting the boundary operator $d_C$ is upper triangular with respect to $J(L)$. The following observation of Franzosa implies that for $f:L\to Sub(C)$ coherent there is an induced chain complex braid $\cC(f)$.











%\subsection{$J(L)$-Splitting}
%
%
%In this section we introduce the $J(L)$-splitting.  For applications, our category is the $Ch(X,\kappa,\preceq)$.  This is a distinguished basis for the chain complex.
%
%
%\begin{enumerate}
%\item ~\cite[Section 7]{salamon} introduced the splitting of a filtered chain complex.
%
%\item There always exists a filtered basis, and we may write $d$ as an upper triangular boundary map with respect to this basis.
%
%\item In the case where we use $Ch(X,\kappa,\preceq)$ we have a canonical basis.  This implies there is a canonical representation of $d$.   This canonical representation means we can express a canonical chain complex braid.
%
%\end{enumerate}
%
%%In~\cite[Section 7]{salamon} it is pointed out that for any convex set $I$ of $J(L)$ there is a subquotient of $C$ which is well-defined up to a canonical isomorphism.   This bears some resemblance to the Conley form and will be explored in a future paper.  For the purposes of this paper, we only need to notice that for any $q\in J(L)$ there is a well-defined predecessor $Pred(q)$.  To $q$ we associate the subquotient $f(q)/f(Pred(q))$.  The following theorem is only provided in some discussion in~\cite{salamon}.
%
%
%For our purposes, this will generalize the splitting lemma for a chain complexes.   However, taking a cue from~\cite{salamon} this may be viewed in another light.  Namely, as defining the associated graded object of the lattice-filtered chain complex.  Due to the duality in Birkhoff's theorem, the associated graded of $L\to Sub(C)$ is (a chain complex) graded over $J(L)$.\footnote{Associated graded objects are useful for defining a filtered derived category.}
%
%
%
%\begin{thm}\label{thm:splitting}
%Let $h:L\to Sub(C)$ be an $L$-filtered chain complex.  Then there is a splitting $C=\bigoplus_{p\in J(L)} C_q/C_{Pred(q)}$ such that for any $\alpha\in L$ we have $C_\alpha = \bigoplus_{q\in J(L), q\leq \alpha} C_q/C_{Pred(q)}$.
%\end{thm}
%%\begin{proof}
%%
%% We will first construct subspaces associated to each $q\in J(L)$ then choose a basis for these subspaces.  Let $q\in J(L)$.  Consider the set $\{p_1,\ldots,p_n\}$ of maximal $p_i\in J(L)$ with $p_i < q$.  Then $\bigvee_i p_i = Pred(q)$.   Since $f$ is coherent, $D_{Pred(q)} = D_{p_1}+\ldots + D_{p_n}$.  Choose a subspace $V_q$ such that $D_q = V_q \oplus D_{Pred(q)}$.  Notice that for any minimal $q\in J(L)$ we have $Pred(q)=0_L$.  Thus $D_{Pred(q)}=0$ implying $V_q = D_q$.
%%
%%We claim that $C = \bigoplus_{q\in J(L)} V_q$.  We first show that $V_q\cap V_p=0$ for $q\neq p\in J(L)$.  Let $x\in V_q\cap V_p$.  Then $x\in D_q\cap D_p = D_{q\wedge p}$.  However, $D_{q\wedge p}\subseteq D_{Pred(q)}$ and $D_{q\wedge p}\subseteq D_{Pred(p)}$.  Thus $x=0$ by choice of $V_q$ and $V_p$.  
%%
%%Now we wish to show that $\bigoplus_{q\in J(L)} V_q$ span $C$.  We will prove this by strong induction.  We will induct over a linear extension of $L$.  The base case is to consider the minimal element, $0_L\in L$.  By (4) of~\ref{def:cf} if $f(x)=0_L$ then $x=0$, which is in the span.   Now fix $p\in L$.  The strong inductive hypothesis is to assume that for any $q< p$ any $x$ with $f(x)=q$ is in span $\bigoplus_{q\in J(L)} V_q$.  Let $p\in L$.  By Lemma~\ref{lem:join} we may write $p$ as an irredundant join $p=\bigvee_i q_i$ with $q_i \in J(L)$. Notice if $p\in J(L)$ then the decomposition is trivially written as $p=p$.  Coherence implies that $D_p = D_{q_1}+D_{q_2}+\ldots+D_{q_n}$.  Thus $x= \sum_i \lambda_i x_{q_i}$.  There are two cases.  First, if $p\not\in J(L)$, then $q_i< p$ and each $x_{q_i}$ belongs to the span.  For the second case, $p\in J(L)$ and we may write $D_p = V_p \bigoplus D_{Pred(p)}$.  Thus $x = v_p + x_{Pred(p)}$.  Since $Pred(p)<p$ the inductive hypothesis implies that $x_{Pred(p)}$ is in the span. 
%%
%%
%%Choose a basis $B_q$ for each $V_q$.  We have shown that $\bigsqcup_{q\in J(L)} B_q$ is a basis for $C$ and $\bigsqcup_{q\leq p, q\in J(L)} B_q$ is a basis for $D_p$.   We now show $f(b)=q$ for $b\in B_q$. Suppose that $f(b)\neq q$.  As $b\in B_q\subset D_q$ then $f(b)< q$.  Therefore $f(b)\leq Pred(q)$, implying $b\in D_{Pred(q)}$ and forcing $b=0$ by our choice of $V_q$.  This contradicts our choice of $b$, therefore $f(b)=q$. 
%%
%%\end{proof}
%
%
%The $C_q/C_{Pred(q)}$ may be realized as subspaces of $C$.  Thus we may split $C$ as $C=\bigoplus_{q\in J(L)}  V_q$ with $V_q\subset C$ and $V_q\cong C_q/C_{Pred(q)}$.  As a corollary, we can choose a basis for each of the $V_q$ to get a filtered basis.
%
%\begin{cor} \label{cor:basis}
%Let $h:L\to Sub(C)$ be an $L$-filtered chain complex.  There is a basis $B=\{b_\alpha\}$ of $C$ with a map $f:B\to J(L)$ such that $\{b\in B: f(b)\leq p\}$ is a basis for $h(p)$.
%\end{cor}
%
%
%Let $L\to Sub(C,d)$ be a $L$-filtered chain complex.  There is a splitting Theorem~\ref{thm:splitting} given by Corollary~\ref{cor:basis} gives a filtered basis $B$.  With respect to this basis, the boundary map $d$ induces an upper triangular map endomorphism of  $C_q/C_{Pred(q)}$. 
%


%\subsection{Relationship}
%
%There is a functor from $Ch(k,L)$ to chain complex braids.    We'll want to show that a Conley complex gets sent to a connection matrix.  We'll also want to show that a connection matrix for a particular filtered complex gets sent to a connection matrix for its chain complex braid.
%
%Let $f:L\to Sub(C,d_C)$ be an $L$-filtered complex.  Consider the splitting of the $L$-filtered complex $C=\bigoplus_{q\in J(L)} C_q$.  With respect to this splitting the boundary operator $d_C$ is upper triangular with respect to $J(L)$. The following observation of Franzosa implies that for $f:L\to Sub(C)$ coherent there is an induced chain complex braid $\cC(f)$.
%
%\begin{prop}[\cite{fran}, Proposition 3.4]
%Given an upper triangular boundary map $$\Delta:\bigoplus_{q\in J(L)} C_q\to \bigoplus_{q\in J(L)} C_q$$ the collection, denoted $\cC\Delta(J(L))$, consisting of the chain complexes $C(I)$ with boundary map $\partial(I)$ for each $I\in I(J(L))$ and the obvious chain maps $i(I,IJ)$ and $p(IJ,J)$ for each $(I,J)\in I_2(J(L))$ is a chain complex braid over $J(L)$.
%\end{prop}
%
%
%Moreover, morphisms of $L$-filtered complexes induce upper triangular maps between the associated graded.   A morphism of lattice-filtered chain complexes $\phi:f\to g$ induces an upper-triangular map on the associated $J(L)$-decompositions $\phi:\bigoplus_{q\in J(L)}C_q\to \bigoplus_{q\in J(L)} C_q'$ such that $\phi\partial_f = \partial_g\phi$.  The following observation shows that this induces a morphism between the associated chain complex braids $\cC(f)\to \cC(g)$.
%
%\begin{prop}[\cite{atm}, Proposition 3.2]
%Let $\Delta,\Delta'$ be an upper triangular boundary map for $\bigoplus_{q\in J(L)} C_q$ and $\bigoplus_{q\in J(L) C_q'}$, respecitvely.  If $T:\bigoplus_{q\in J(L)} C_q\to \bigoplus_{q\in J(L)} C_q'$ is upper triangular with $T\Delta = \Delta'T$ then $\cT:=\{T(I)\}_{I\in I(P)}$ is a chain complex braid morphism from $\cC(\Delta)\to \cC(\Delta')$.
%\end{prop}
%
%In conclusion, there is a functor from $Ch(k,L)\to CCB(J(L))$.
%
%
%\begin{enumerate}
%\item Connection matrices are sent to connection matrices
%\item Homotopy equivalent $L$-filtered complexes are sent to homotopy equivalent chain complex braids
%\end{enumerate}
%
%
%\begin{prop}
%Let $h:L\to Sub(C)$ and $h':L\to Sub(D)$ be filtered homotopy equivalent.  Then $F(h)$ and $F(h')$ are homotopy equivalent chain complex braids.
%\end{prop}
%
%\begin{cor}
%Let $h:L\to Sub(A)$ be a Conley complex for $g:L\to Sub(C)$.  Then $F(h)$
%\end{cor}
%
%\begin{rem}
%Perhaps it is more correct to say that both chain complex braids and $L$-filtered chain complexes have the some associated graded.  
%\end{rem}
%





 
