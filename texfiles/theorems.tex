%!TEX root = ../main.tex

\section{Main Theorem}\label{sec:thm}



The next theorem is proved algorithmically with algebraic Morse theory.  This is the analogue to Franzosa's existence of connection matrices and~\cite{salamon} proof of existence.


\begin{thm}\label{thm:exist}
Let $h\in Ch_X(k,L)$.  Then there exists a Conley complex $g:L\to Sub(A,d_A)$ which is homotopy equivalent to $h$.
\end{thm}
%\begin{proof}
%$f_i$ is coherent.  Thus there exists a basis by Theorem~\ref{thm:cfdecomp}, (\ref{thm:cfdecomp:basis}).  Use~\cite[Algorithm 3.6]{focm} to obtain an acyclic partial matching.  This induces a new fibration $f_{i+1}:M\to L$.  The number of basis elements must decrease monotonically, thus this process stabilizes at some point.
%
%\end{proof}







%Another way of thinking of a connection fibration is as some sort of initial object in the homotopy equivalence class.


