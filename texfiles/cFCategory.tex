%!TEX root = ../main.tex



\section{A Category of Chain Fibrations}


For the purposes of this paper, we will fix the target lattice $L$.  We will construct a category whose objects are chain fibrations over $L$.  We call this category $\bChF(k, L)$.  As $k$ is also fixed, we will suppress the dependence on $k$ and $L$ and denote this $\bChF$.

\begin{defn}
{\em
Let $f:C\to L$ and $f':C'\to L$ be chain fibrations.   A map $\phi:C\to C'$ is {\em order-preserving} between $f$ and $f'$ if we have that $$f'\circ \phi \leq f$$
}
\end{defn}



Notice that for $f:(C,\partial)\to L$ the boundary operator $\partial:C\to C$ is order-preserving.  For $f,f'\in \bChF$ we set $Hom(f,f')$ to be the set of order-preserving chain maps $C\to C'$.  We will call such maps {\em fibration morphisms}.

\begin{prop}\label{prop:compCF}
Let $\phi\in Hom(C,C')$ with $\phi$ injective.  Let $f\in \bChF$ with $f:C'\to L$.  Then the composition $f\circ\phi:C\to L$ is a chain fibration.
\end{prop}
\begin{proof}
Let $q\in L$.  Since $f$ is a chain fibration, $O_f(q)$ is a subcomplex in $\bChF(C',L)$.  Since $\phi$ is a chain map and $O_f(q)$ a subcomplex it is straightforward that $\phi^{-1}(O_f(q))$ is a subcomplex.   Let $h=f\circ \phi$.  Then there is a map $O_h:L\to S(C)$.  We wish to show that the $O_h$ is a lattice homomorphism.  Let $p,q\in L$.  Then $$h(p \vee q) = (f\circ \phi)^{-1}(B(p \vee q)) = \phi^{-1}(f^{-1}(B(p \vee q))) = \phi^{-1}(O_f(p\vee q)) = \phi^{-1}(O_f(p)+O_f(q))$$  Then $\phi^{-1}(O_f(p\vee q)) = \phi^{-1}(O_f(p)+O_f(q)) = \phi^{-1}(O_f(p)) + \phi^{-1}(O_f(q))$ by the properties of preimages.

$h(p \wedge q) = (f\circ \phi)^{-1}(B(p \wedge q)) = \phi^{-1}(f^{-1}(B(p \wedge q)))) = \phi^{-1}(O_f(p \wedge q)) = \phi^{-1}(O_f(p)\cap O_f(q)) = \phi^{-1}(O_f(p))\cap \phi^{-1}(O_f(q))$. The last equality follows from the properties of preimages.

Finally we must check that $0_L\mapsto 0_{S(C)}$ and $1_L\mapsto 1_{S(C)}$ under $h$.  $O_f(0_L) = 0_{S(C')}$ since $O_f$ is a lattice morphism.  Then $\phi^{-1}(0_{S(C')}=0_{S(C)}$ since $\phi$ is injective.  Furthermore $O_f(1_L)=C'$ as $O_f$ is a lattice morphism and  $\phi^{-1}(C') = C$.  Thus $O_h(0_L) = 0_{S(C)}$ and $O_h(1_L) = C$.
\end{proof}

%\begin{cor}\label{prop:comp}
%Let $f:C'\to L$ be a coherent chain fibration.  Let $\phi\in Hom(C,C')$ be injective.  Then $f\circ \phi$ is coherent.  
%\end{cor}
%\begin{proof}
%To show that $f\circ \phi$ is coherent, we must show that the map $L\ni p\mapsto (f\circ \phi)^{-1}(p)\in Sub(C')$ is a lattice homomorphism.  Fix $p,q\in L$.    Let $D_p = f^{-1}p$ and $D_q = f^{-1}q$.  Then $D_p \cap D_q = \cap D_{p\wedge q}$ and $D_p+D_q = \cap D_{p\vee q}$.  
%
%We have that $D_p = (f\circ \phi)^{-1}(p) = \phi^{-1}(f^{-1}D_p)$.
%
%$$(f\circ \phi)^{-1}p = $$
%
%
%\end{proof}


%A $P$-filtered chain complex $\Phi:L\to C(X)$ induces a function $f:C\to L$ as follows: define $f(x) = \min {q\in L: x\in \Phi(q)\}$.  



%\begin{rem}
%For $\phi\in Hom(C,C')$ and $f\in \bChF(C',L)$ we have a chain fibration $f\circ\phi:C\to L$.  Thus $\phi$ induces a map between $$\phi^*:\bChF(C';L)\to \bChF(C;L)$$ by $$\phi^*(f) \mapsto f'\circ \phi$$ 
%\end{rem}

%
%\begin{rem}
%The coherent chain fibrations form a subcategory of $\bChF(L)$.
%
%\end{rem}



Let $f:C\to L,g:C'\to L$ be chain fibrations.  A fibration morphism $\phi:f\to g$ induces an upper-triangular map on the associated $J(L)$-decompositions $\phi:\bigoplus_{q\in J(L)}C_q\to \bigoplus_{q\in J(L)} C_q'$ such that $\phi\partial_f = \partial_g\phi$.  The following observation shows that this induces a morphism between the associated chain complex braids $\cC(f)\to \cC(g)$.

\begin{prop}[\cite{atm}, Proposition 3.2]
Let $\Delta,\Delta'$ be an upper triangular boundary map for $\bigoplus_{q\in J(L)} C_q$ and $\bigoplus_{q\in J(L) C_q'}$, respecitvely.  If $T:\bigoplus_{q\in J(L)} C_q\to \bigoplus_{q\in J(L)} C_q'$ is upper triangular with $T\Delta = \Delta'T$ then $\cT:=\{T(I)\}_{I\in I(P)}$ is a chain complex braid morphism from $\cC(\Delta)\to \cC(\Delta')$.
\end{prop}

Therefore there is a functor $F:\bChF(L)\to {\bf CCB}(J(L))$.


\subsection{Relationship to $J(L)$-filtered complexes}

A chain fibration $f:C\to L$ induces a $J(L)$-filtered complex $O_f:L\to S(C)$ by definition. We saw that a morphism of $J(L)$-filtered complexes is a map $\phi:A\to B$ that preserves the filtration, i.e. $\phi(A_\alpha)\subset B_\alpha$.  We reinterpreted that as $\phi$ induces a map $\Phi:S(A)\to S(B)$ such that $\Phi(\rho(q))\leq \sigma(q)$.



\begin{prop}
Let $f:C\to L$ and $g:C'\to L$ be chain fibrations.  Let $\phi:f\to g$ be a fibration morphism.  Then $\phi$ induces a $J(L)$-filtered morphism.
\end{prop}
\begin{proof}
We have that $g\circ\phi \leq f$.  Fix $p\in L$.  Let $A_q := O_f(q)$.  Then $g(\phi(A_q))\subset \{p\leq q\}$.  Thus $g(\phi(A_q))\subset O_g(q)$.  Therefore $\phi$ is a $J(L)$-filtered morphism.  
\end{proof}


We now show that a $J(L)$ filtered complex $\rho:L\to S(C)$ also defines a function $f:C\to L$ given by 
\begin{align}\label{eqn:cf:birkhoff}
x\mapsto \min\{q\in L:x\in \rho(q)\}
\end{align}
Observe the set is nonempty as $x\in C=1_{S(C)}=\rho(1_L)$.  Furthermore, it has a minimum as $\rho$ is a lattice homomorphism, i.e. if $x\in \rho(q),\rho(p)$ then $x\in \rho(q)\cap \rho(p)=\rho(p\wedge q)$.  

\begin{prop}\label{prop:cf:birkhoff}
Let $\rho:L\to S(C)$ be a $J(L)$-filtered chain complex.  For $f$ given by Eqn.~\ref{eqn:cf:birkhoff} we have $O_f = \rho$. 
\end{prop}
\begin{proof}
 Let $q\in L$.  We must show that $O_f(q) = \rho (q)$.  Notice $$ O_f(q) = f^{-1}\{p\leq q\} = \{x\in C:f(x)\leq q\} = \{x\in C: \min\{r\in L: x\in \rho(r)\}\leq q\} $$

If $x\in \rho(q)$ then $q\in \{r\in L:x\in \rho(r)\}$ so $ \min \{r\in L:x\in \rho(r)\}\leq q$.  Thus $x\in O_f(q)$.  If $x\in O_f(q)$, then $x\in \rho(r)$ for some $r\leq q$.  Then $x\in p(r)+ p(q) = p(r\vee q ) = p(q)$.
\end{proof}

Proposition~\ref{prop:cf:birkhoff} shows that this construction is an assignment from $J(L)$-filtered complexes to chain fibrations over $L$.     We show that a morphism of $J(L)$-filtered complexes determines a morphism of chain fibrations.  

\begin{prop}
Let $\rho:L\to S(A),\sigma:L\to S(B)$ be $J(L)$-filtered chain complexes and $f_\rho:A\to L, g_\sigma:B\to L$ be the associated chain fibrations.  Let $\phi:A\to B$ be a $J(L)$-filtered morphism of $\rho$ and$ \sigma$.  Then $\phi$ is a fibration morphism.
\end{prop}
\begin{proof}
Let $x\in C$.  We must show that $g_\sigma(\phi(x))\leq f_\rho(x)$.  Let $q:= f(x)\in L$.  By Proposition~\ref{prop:cf:birkhoff} we have that $O_f(q) = \rho(q)$ and $O_g(q) = \sigma(q)$.   Since $\phi$ is a $J(L)$-filtered morphism $\phi(\rho(q))\subseteq \sigma(q) = O_g(q)=f^{-1}\{p\leq q\}$. Therefore $g(\phi(x))\leq q = f(x)$.
\end{proof}

%A more hands on proof can also be done.
%\begin{prop}
%Let $\phi:A\to B$ be a $J(L)$-filtered morphism of $\rho:L\to S(A)$ and $\sigma:L\to S(B)$.  We show that $\phi$ is a fibration morphism of $f_\rho\to g_\sigma$.
%\end{prop}
%\begin{proof}
%Let $x\in A$.  We'll show that $g_\sigma(\phi(x))\leq f_\rho(x)$.   Define $R:= \{r\in L:x\in \rho(r)\}$ and $T:=\{t\in L:\phi(x)\in \sigma(t)\}$.  Notice that $f_\rho(x):= \min R$ and $g_\sigma(\phi(x)) = \min T$.  If $R\subset T$ then $\min T\leq \min R$.  Let $r\in R$.  Then $x\in p(r)$.  Since $\phi$ is a $J(L)$ morphism we have $\phi(p(r))\subset \sigma(r)$.  Thus $\phi(x)\in \phi(p(r))\subset \sigma(r)$.  This implies that $r\in T$. 
%\end{proof}




 
%This is an analogue of the Birkhoff construction of Theorem~\ref{thm:birkhoff} for $J(L)$-filtered chain complexes and chain fibrations over $L$.  

This correspondence is an equivalence of categories.  We record this as a corollary.

\begin{cor}
The categories of $J(L)$-filtered chain complexes and chain fibrations over $L$ are equivalent.
\end{cor}

















 
