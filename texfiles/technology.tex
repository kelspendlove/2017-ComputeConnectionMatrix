%!TEX root = ./main.tex



\section{Technology}

In this section we will introduce some technology for manipulating the combinatorial models.  In many ways the ideas we introduce are in the vein of combinatorialization of dynamical systems.  Most of these objects have continuous analogues.


\begin{defn}[Lattice Predecessor]
{\em
For a lattice $L$ define $Pred:L\backslash \{\emptyset\} \to L$ via $Pred(p) = \bigwedge \{q: \text{$p$ covers $q$}\}$, i.e. as the met of the immediate predecessors of $p$.  In the case of a lattice of down sets, this corresponds to removing all of the maximal elements at once.  Note for join irreducible elements that $Pred$ yields the unique predecessor.
}
\end{defn}



\begin{defn}[Chain Fibration]
{\em
A {\em chain fibration}, or just {\em fibration}, is a function $f:\cC\to L$ between a chain complex $\cC$ and a finite distributive lattice $L$ such that
\begin{enumerate}
\item $f(a+ b) \leq f(a)\vee f(b)$

\item $f\circ \partial(a) \leq f(a)$

\item $f(\alpha a) = f(a)$ for $\alpha\in \F$ and $\alpha \neq 0$ 

\item $f(0) = 0_L$

\item (Update) $D_p = span~f^{-1}\{q\in J(L):q\leq p\}$ (this condition guarantees that the pullback of the join-irreduicibles acts similar to a `basis' for any down subcomplex

\end{enumerate}
}
\end{defn}

The join-irreducibles of a lattice are very important objects.  In fact they form a poset which acts similar to a basis for a vector space.

\begin{defn}
{\em
Given a fibration $f:\cC\to L$, we define the following.  For each $p\in L$ define the {\em down subcomplex} $\cD_p := f^{-1}(\{q\in L: q \leq p\})$.  Define the {\em center subcomplex} $\cC_p:= (\cD_p,\cD_{Pred(p)})$
}
\end{defn}
\begin{defn}
Intervals of lattice - $I(J(L))$
\end{defn}

\begin{defn}
For $I\in I(J(L))$ by $q\leq I$ we mean that $q\leq p$ for some $p\in I$.
\end{defn}
%
%\begin{defn}
%For $I\in I(J(L))$ define $P(I):= \bigwedge_{p\in I} Pred(p)$
%\end{defn}
%
%\begin{defn}[Lattice Successor]
%For $I\in I(J(L))$ define $S(I):= \bigvee_{p\in I} p\in I$.  Notice for $I=\{p\}$ that $S(I) = p$.
%\end{defn}

%\begin{defn}
%{\em
%Given a fibration $f:\cC\to L$, we define the following.  For $I\in I(J(L))$ define the {\em down subcomplex} $\cD_I:= f^{-1}(\{q\in L: q \leq S(I)\})$.  Define the {\em center subcomplex} $\cC_I:= (\cD_I,\cD_{P(I)})$.
%}
%\end{defn}

We must justify the name subcomplex.  We do this by proving down subcomplexes and center subcomplexes are subspaces and quotient spaces, respectively.

\begin{prop}
Let $f:\cC\to L$ be a chain fibration.  Let $p\in L$.  Then
\begin{enumerate}
\item $D_p$ is a subspace.  
\item $D_{Pred(p)}$ is a subspace.  
\end{enumerate}
\end{prop}
\begin{proof}
For $(1)$, we have three things to prove:
\begin{enumerate}
\item $0\in D_p$ as $0_L \leq p$ and $f(0)=0_L$.
\item Let $a,b\in D_p$. We want to show that $a+b\in D_p$.  By the definition of chain fibration $f(a+b)\leq f(a)\vee f(b)\leq p \vee p = p$.
\item Let $a\in D_p$.  Let $\alpha\neq 0\in \F$.  We wish to show $\alpha a\in D_p$. $f(\alpha a) = f(a) \in D_p$.
\end{enumerate}

For $(2)$ the argument is quite similar.
\end{proof}

Thus $D_p$ and $D_{Pred(p)}$ are subspaces and $D_{Pred(p)}\subseteq D_p$, so there is a natural quotient space $D_p/D_{Pred(p)}$.  This is the center subcomplex $\cC_p$.  Now to see that these subspaces are subcomplexes, let $a\in D_p$.  Then $f(a)\leq p$.  By the property of being a chain fibration $$f(\partial a) \leq f(a) \leq p$$  Thus $\partial(D_p)\subseteq D_p$. Similarly $\partial(D_{Pred(p)})\subseteq D_{Pred(p)}$. Thus $\partial$ induces a well-defined boundary map on the quotient $\cC_p := D_p/D_{Pred(p)}$.



Sometimes it is easier to characterize center subcomplexes as actual subspaces instead of quotient vector spaces.  This leads to the following observation.
\begin{prop}
The inclusion $i:D_{Pred(p)}\hookrightarrow D_p$ which leads to the center subcomplex $D_p/D_{Pred(p)}$ can also be computed by $D_p/D_{Pred(p)}\cong D_p \cap Keri^*$ where $i^*$ is the adjoint of $i$.
\end{prop}


%Let $(I,J)\in I_2(J(L))$.  We make some claims.

%\begin{prop}
%$S(I)\leq S(IJ)$.  $P(I)=P(IJ)$.  $S(IJ)=S(J)$.  $P(IJ)\leq P(J)$.
%\end{prop}
%\begin{proof}
%$S(I)=\bigvee_{p\in I} p \leq \bigvee_{p\in IJ} p = S(IJ)$.  Notice that for $p\in J$, $Pred(p)$ is unique and $Pred(p)\geq I$ since $IJ$ is adjacent.  Thus $P(I)=P(IJ)$
%
%Notice that for $p\in J$ $P(IJ)=\bigwedge_{p\in IJ} Pred(p)\leq \bigwedge_{p\in J} Pred(p)$.
%
%
%\end{proof}

%Thus $D_I\subseteq D_{IJ}$.  $D_{Pred(I)}=D_{Pred(IJ)}$. $D_{IJ}=D_J$.  $D_{Pred(IJ)}\subseteq D_{Pred(J)}$.


\subsection{Some Possible Theorems}

In a very real way the condition $(5)$ implies that the join-irreducibles are acting as basis elements.  However, the pullback of a lattice element $f^{-1}(q)$ is not in general a subspace.  We should write some possible theorems that make this relationship explicit.




\begin{prop}
The downsubcomplex at $p$ can be written as the direct sum of center subcomplexes $$D_p = \bigoplus_{q\in J(L), q\leq p} C_q$$
\end{prop}



\subsection{Morphisms}
\begin{defn}[Fibration Morphism]
{\em
A {\em fibration morphism} from $f:\cC\to L$ to $f':\cC'\to L'$ consists of a chain map $\phi:\cC\to \cC'$ and a lattice morphism $\phi':L\to L'$ such that $$f'\circ \phi \leq \phi'\circ f$$

 If $L=L'$ and $\phi' = id$, then we say $(\phi,\phi')$, or $\phi$, is a {\em strict fibration morphism}.
}
\end{defn}

The intuition here is that the chain map $\phi$ can kill off various parts of a chain by sending it to zero.  Thus the chain may lose its place in the lattice to a lower part by a higher-dimension of the chain is killed.

We now describe the category of chain fibrations.  The objects are chain fibrations $f:\cC\to L$.  The morphisms are fibration morphisms.  Composition is given by the following rule: $(\phi',l')\circ (\phi,l) = (\phi'\circ \phi,l'\circ l)$.

We now describe special objects in the chain fibration category $\cC\cF(\F)$.  This is the notion of a {\em connection fibration}.  This is our analogue to the connection matrix of Franzosa.

\begin{prop}
$(\phi',')\circ (\phi,l)$ is a chain fibration.
\end{prop}


\begin{defn}[Connection Fibration]
{\em
A {\em connection fibration} is a fibration $f:\cC\to L$ such that every center subcomplex has a trivial boundary map.  That is, for every $p\in L$, the complex $(\cD_p,\cD_{Pred(p)})$ has a zero boundary map.
}
\end{defn}

\begin{prop}
It is equivalent to require that the boundary map restricted to join irreducibles is zero.
\end{prop}

%\begin{defn}
%Let $f:\cC\to L$ and $f':\cC'\to L'$ be fibrations.  A {\em degree +1 map of fibrations} is a degree +1 map $\gamma:\cC\to \cC'$ and a lattice morphism $l:L\to L'$ such that for every $p\in L$ $\gamma(\cD_p)\subset \cD'_{l(p)}$.
%\end{defn}


%\begin{defn}[Fibration Homotopy]
%{\em
%Let $f:\cC\to L$ and $f':\cC'\to L'$ be fibrations.  Let $(\phi,l),(\phi',l')\in Hom(f,f')$.  A {\em fibration homotopy} from $(\phi,l)$ to $(\phi',l')$ is a degree +1 map of fibrations $(\gamma,\delta)$  such that:
%
%}
%
%In the case where $L=L'$ and $l=id$, we say that $(\gamma,l)$, or just $\gamma$, is a {\em strict fibration homotopy}.
%\end{defn}


\begin{defn}[Fibration Homotopy]
{\em
Let $f:\cC\to L,f':\cC'\to L'$ be fibrations.  Let $(\phi:\cC\to \cC',\phi':L\to L')$ and $(\psi:C\to C, \psi:L\to L')$ be fibration morphisms.  A {\em fibration homotopy} from $\phi$ to $\psi$ is a degree +1 map $\gamma$ such that:
\begin{enumerate}
\item $\phi - \psi = \partial\circ \gamma + \gamma\circ \partial$
\item $f'\gamma x \leq \phi' fx \vee \psi ' fx$

%\item $f'\phi x \wedge f'\psi x \leq f'\gamma x \leq f'\phi x \vee f'\psi x$
\end{enumerate}


%\item $id_{C'} - \phi\circ \psi = \partial'\circ \gamma'+\gamma'\circ \partial'$

%Suppose that there exist fibration homotopies $(\gamma:C\to C, l:L\to L')$ and $(\gamma':C'\to C,l':L'\to L)$ such that 
%\begin{enumerate}
%\item $id_C - \psi\circ \phi = \partial\circ \gamma + \gamma\circ \partial$
%\item $id_{C'} - \phi\circ \psi = \partial'\circ \gamma'+\gamma'\circ \partial'$
%\end{enumerate}
%Then we say that $((\phi,\phi'),(\psi,\psi'))$ is a fibration equivalence between $f$ and $f'$, and we say that $f$ and $f'$ are {\em equivalent}.  In the case where $L=L'$ and $\phi'=\psi'=l=l'=id$ (i.e. striction fibration morphism and strict fibration homotopies) we say $(\phi,\psi)$ is a {\em strict fibration equivalence} and that $f$ and $f'$ are {\em strictly equivalent}.
In the case that $\phi'=\psi'=id$ then $\gamma$ {\em strict fibration homotopy} from $\phi$ to $\psi$.  In this case, (2) reduces to $f'\gamma x \leq f(x)$.
}
\end{defn}

\begin{rem}
Condition (1) makes the constraint $f(\partial\gamma+\gamma\partial)\leq f$ however as far as I can tell (2) is still needed.
\end{rem}

If there exists a fibration homotopy from $\phi$ to $\psi$ we say they are {\em homotopic} and denote this by $\phi\sim \psi$.

\begin{prop}
 $\phi \sim \psi$ is an equivalence relation on $Hom(f,f')$.
\end{prop}
\begin{proof}
We have three things to prove.

\begin{enumerate}
\item Reflexive: Let $(\phi,\phi') \in Hom(f,f')$. We must show that $\phi\sim \phi$.  Define $\gamma=0$.

\item Symmetric:  Let $\phi, \psi \in Hom(f,f')$.  Suppose that $\phi\sim \psi$.  We must show $\psi\sim \phi$.  Choose $-\gamma$.

\item Transitive: Suppose $\phi\sim \psi$ and $\psi \sim \theta$.  We must show $\phi\sim \theta$.  We have that there exists $\gamma:\phi \two \psi$ and $\gamma':\psi \two \theta$.  Choose $h:= \gamma+\gamma'$.  Then 

$$\phi - \theta = (\phi-\psi)+(\psi-\theta) = (\partial\gamma + \gamma\partial) + (\partial\gamma'+\gamma'\partial) = \partial(\gamma+\gamma')+(\gamma+\gamma')\partial = \partial h + h \partial$$

For the second condition, we have that $$f'hx = f'(\gamma x+\gamma' x) = f'\gamma x \vee f'\gamma' x \leq f'x \vee f' x = f'x$$

\end{enumerate}
\end{proof}


We've shown that $\sim$ is an equivalence relation on the Hom set $Hom(f,f')$.    We now consider the category $F_0$ obtained by the quotient $Hom_{F_0}(f,f') := Hom_F(f,f')/\sim$.

Our construction of $F(\F)$ and $F_0(\F)$ is analogous to $Kom(\F)$ and $K(\F)$.

%
%\begin{prop}
%Fibration equivalence and strict fibration equivalence are both equivalence relations.
%\end{prop}
%
%\begin{proof}
%We begin by proving strict fibration equivalence relation.  
%\end{proof}

\begin{rem}
{\em
{\em Transition matrices} as $Aut(f:\cC\to L)$ for a connection fibration in the appropriate $K(A)$ category.
}
\end{rem}



\begin{thm}[Connection Matrix Theorem]
Let $f:C\to L$ be a fibration which is strictly equivalent to a connection fibration $g:C'\to L$.  Then for every $p\in L$, $C'_p\cong H_\bullet(C_p)$ and the boundary map $\Delta$ of $C'$ is a {\em connection matrix} in the sense of Franzosa: (1) there is a chain complex braid $\cB'$ we may build out of $H_\bullet(C_p)$ and $\Delta$, (2) there is a chain complex braid $\cB$ associated to $f$, and (3) the fibration equivalence $(\phi,\psi)$ yields a chain complex braid morphism $\Psi:\cB'\to \cB$.  In particular, for each interval $I\subset J(L)$ let $a = \bigvee \{p\in J(L):p < I\}\in L$ and $b=a\vee I\in L$.  Consider the complexes $C_I:=(\cD_b, \cD_a)$ and $C_I':= (\cD_b',\cD_a')$.  See that $\psi(\cD'_b)\subset \cD_b$ and $\psi(\cD_a')\subset \cD_a$.  It follows that $\psi$ induces a chain map from $C_I'\to C_I$.  Denote this induced chain map as $\Psi(I)$.  Franzosa's $\Psi$ is the mapping from intervals $I$ to the chain maps $\Psi(I)$.  We show that the chain complex braid morphism $\Psi$ induces an isomorphism of graded module braids.
\end{thm}


\begin{rem}
Is there a (faithful) functor from category of chain fibrations to category of chain complex braids? 
\end{rem}

\begin{rem}
What's the different between quasi-isomorphic objects in the fibration category and connection matrices?  I.e. if a connection fibration is quasi isomorphic to a chain fibration does it push forward under the function to a connection matrix for the image of the chain fibration?
\end{rem}

\begin{rem}
When is quasi-isomorphic the same as chain homotopic? homotopy category of free abelian groups equivalent to the derived category of free abelian groups.
\end{rem}

\begin{rem}
My guess is that if one has a chain complex braid where one can split the chain groups for each pair $(I,J)$ into $I$ and $J$, then one can make a chain fibration out of this by direct summing the chain groups of singletons.  %A sufficient condition is probably that this works for singletons and singleton pairs.
\end{rem}


%
%\subsection{Fibrations}
%
%In the previous section we showed that intervals in the face poset of a complex form subcomplexes.  In order to parameterize subcomplexes of $\cX$ which are of interest, we can begin instead by parameterizing intervals.  This is done with a notion called a {\em fibration}.  
%
%\begin{defn}
%Let $(\cX,\preceq)$ be the face poset of a complex $(\cX,\kappa)$.  A {\bf fibration} is an order preserving map $(\cX,\preceq)\to (P,\leq)$ for some poset $(P,\leq)$.
%\end{defn}
%
%A fibration $\varphi$ provides a method of organizing subcomplexes of $(\cX,\kappa)$ by presenting them as fibers of $\varphi$.  In the literature such maps are sometimes referred to as `poset fibrations'~\cite{koz}.    In analogy to bundle theory, the poset $(P,\leq)$ may be thought of as a base space.  We call it a {\em target poset}.  The target poset parameterizes a wealth of data for the complex $\cX$.  Most basically, the fibers of $\varphi$ will form the building blocks of the theory.   More generally, the lower sets of $\P$ parameterize lower sets (attractors) of $\cX$.  Most generally, intervals of $\P$ parameterize of subcomplexes of $\cX$.   This fact is captured in the next proposition.
%
%
%\begin{prop}
%Let $\varphi:(\cX,\preceq)\to (P,\leq)$ be a fibration.  If $I\subseteq P$ is an interval then $\varphi^{-1}(I)$ is an interval in $(\cX,\preceq)$.   
%\end{prop}
%
%We set $X_I:= \varphi^{-1}(I)$.   By Proposition~\ref{prop:subcomplex} $X_I$ is a subcomplex and therefore $\partial_I:=\partial_{X_I}$ is a boundary operator on $C_\bullet(X_I)$.  This implies that the homology $H_\bullet(X_I):=H_\bullet((C_\bullet(X_I),\partial_I)$ is well-defined. %$H_\bullet\varphi^{-1}(I)$ is called the {\em Conley index} of $I$ and is denoted $CH_\bullet(I)$.
%
%
%
%\begin{prop}\label{prop:del:split}
%$$\partial_{IJ}=$$
%\end{prop}
%
%For adjacent intervals $(I,J)$ the canonical basis of $C_\bullet(X_{IJ})$ splits as $C_\bullet(X_I)\oplus C_\bullet(X_J)$.  The inclusion map $\iota_{X_I,X_{IJ}}$ and projection map $\pi_{X_{IJ},X_J}$ are chain maps by Propositions~\ref{prop:imap} and~\ref{prop:pmap}. Thus we have a short exact sequence of chain complexes: $$0\to (C_\bullet(X_I),\partial_I)\xrightarrow{i_{X_I,X_{IJ}}} (C_\bullet(X_{IJ}),\partial_{IJ})\xrightarrow{\pi_{X_{IJ},X_J}} (C_\bullet(X_J),\partial_J)\to 0$$
%
%
%
%This gives rise to a long exact sequence which relates the homology: 
%\begin{align}
%\label{seq:conleyExact}
%\ldots \to H_\bullet (X_I)\to H_\bullet( X_{IJ}) \to H_\bullet(X_J)\to H_{\bullet-1}(X_I)\to \ldots
%\end{align}
%
%
%These exact sequences can be thought of as providing a means by which to move between fibers of $\varphi$.  The homology of these fibers will be central to our theory.  We will be interested the direct sum $$H^\oplus_\bullet(P):=\bigoplus_{p\in P}H_\bullet(X_p) $$
%
%
%Let $I\subseteq P$.  We set $H^\oplus_\bullet(I)=\bigoplus_{p\in I} H_\bullet(X_p)$.   An endomorphism $\Delta(I):H^\oplus_\bullet(I)\to H^\oplus_\bullet(I)$ can be thought of as a matrix of linear maps:  $$[\Delta(p,q):H_\bullet(p)\to H_\bullet(q)\mid p,q\in I]$$
%
%\begin{defn}
%A matrix $\Delta(I)$ is {\bf diagonal with respect to $I$} if $\Delta(p,q)=0$ for $p\neq q$.  
%\end{defn}
%
%
%\begin{defn}
%A matrix $\Delta(I)$ is {\bf upper triangular with respect to $I$} if $\Delta(p,q)=0$ for $p\nless q$.  
%\end{defn}
%
%\begin{defn}
%A matrix $\Delta(I)$ is {\bf a boundary map} if each $\Delta(p,q)$ is a degree -1 map and $\Delta(I)\circ \Delta(I) = 0$.
%\end{defn}
%
%
%\begin{prop}
%If $\Delta(P)$ is an upper triangular boundary map then $\Delta(I)$ upper triangular boundary map for any interval $I$.
%\end{prop}
%
%\begin{prop}
%The composition of upper triangular boundary maps is upper triangular.
%\end{prop}
%
%
%Equipping $H^\oplus_\bullet(I)$ with boundary map $\Delta(I)$ transforms it into a chain complex $(H^\oplus_\bullet(I),\Delta(I))$.  Therefore it has well-defined homology.   Upper triangularity of $\Delta(I)$ means it respects the poset structure.  If $(I,J)$ is an adjacent pair of intervals, then with the usual inclusion and projection maps there is a short exact sequence:
%
%$$0\to (H^\oplus_\bullet(I),\Delta(I)) \to (H^\oplus_\bullet(IJ),\Delta(IJ))\to (H^\oplus_\bullet(J),\Delta(J))\to 0$$
%
%
%This gives a long exact sequence:
%\begin{align}
%\label{seq:connectionExact}
%\ldots \to H_\bullet(H^\oplus_\bullet(I))\to H_\bullet(H^\oplus_\bullet(IJ))\to H_\bullet(H^\oplus_\bullet(J))\to H_{\bullet-1}(H^\oplus_\bullet(I))\to \ldots
%\end{align}
%
%
%
%
%\begin{defn}
%An upper triangular boundary map $\Delta(P)$ is a {\bf boundary connection} for a fibration $\varphi:(\cX,\preceq)\to (P,\leq)$ if for all intervals $I\subseteq P$ there are isomorphisms $\phi(I): H_\bullet(H^\oplus_\bullet(X_I),\Delta(I))\to H_\bullet(X_I)$ such that for all adjacent pairs $(I,J)$ of intervals the following diagram commutes
%%\begin{enumerate}
%%\item If $I=\{p\}$ then $\phi(p)$ is the identity
%%\item 
%\begin{align*}
%\xymatrixrowsep{0.3in}
%\xymatrixcolsep{0.3in}
%\xymatrix{
%\cdots \ar@{->}[r] & H_\bullet(H^\oplus_\bullet(I)) \ar@{->}[r] \ar@{->}[d]_{\simeq}^{\phi_\bullet(I)} & H_\bullet(H^\oplus_\bullet(IJ)) \ar@{->}[r] \ar@{->}[d]_{\simeq}^{\phi_\bullet(IJ)} & H_\bullet(H^\oplus_\bullet(J)) \ar@{->}[r] \ar@{->}[d]_{\simeq}^{\phi_\bullet(J)} & H_{\bullet-1}(H^\oplus_\bullet(I)) \ar@{->}[r] \ar@{->}[d]_{\simeq}^{\phi_{\bullet-1}(I)} & \cdots \\
%\cdots \ar@{->}[r] & H_\bullet(I) \ar@{->}[r] & H_\bullet(IJ) \ar@{->}[r] & H_\bullet(J) \ar@{->}[r] & H_{\bullet-1}(I) \ar@{->}[r] & \cdots
%}
%\end{align*} 
%where the top row is~(\ref{seq:connectionExact}) and bottom row is~(\ref{seq:conleyExact}). 
%
%%\end{enumerate}
%
%\end{defn}
%%
%%
%%\begin{prop}
%%If $\Delta(\P)$ is a connection matrix then the homomorphisms $\phi(I)$ are isomorphisms for all $I$.
%%\end{prop}
%
%
%
%%\myline
%
%
%
%\begin{rem}
%Our definition of {\bf boundary connection} models that of the connection matrix from~\cite{mcr}.  See Remark 3.6 of~\cite{bar} for an idea about the equivalence between this definition and Franzosa's original definition using braids.
%\end{rem}
%
%


 
