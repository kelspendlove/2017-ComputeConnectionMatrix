%!TEX root = ../main.tex




\section{$J(L)$-Graded Cell Complexes}

In this section, we make it explicit how to get from applications to the category of lattice-filtered complexes via Birkhoff's theorem.  In applications, data often come in the form of a cell complex $(X,\kappa,\preceq)$ graded by a partial order $(P,\leq)$.  This is codified in terms of a map $f:(X,\kappa, \preceq)\to (P,\leq)$ which restricts to a poset morphism $f:(X,\preceq)\to (P,\leq)$.  In practice we think of $P$ as being derived from some lattice $L=O(P)$, i.e. as the poset of join-irreducibles of $L\in {\bf FDLat}$.  However, via Birkhoff's theorem one could also regard the lattice as being obtained from the poset.

\begin{defn}
{\em
Let $X$ be a cell complex.  Let $L\in FDLat$.  A {\em $J(L)$-graded cell complex} is a map $f:X\to (J(L),\leq)$ such that $f:(X,\preceq)\to (J(L),\leq)$ is order preserving.
}
\end{defn}

%In the context of the computational Conley theory, one starts with a multivalued map $\cF: X^n\rightrightarrows X^n$ on the top-cells of a complex.    The top-cells are thought of as vertices in a directed graph whose edges are given by $\xi_1\to \xi_2$ if $\xi_2\in \cF(\xi_1)$.  The strongly connected components of this directed graph have a partial order $J(L)$ which may be computed efficiently via Tarjan's algorithm.   This partial-ordering on top cells gives an implicit ordering on all lower-dimensional cells (i.e. faces) of $X$.  The implicit ordering yields $J(L)$-filtered complex $(X,\kappa,\preceq)\to (J(L),\leq)$.

In the context of the computational Conley theory, such structures typically appear as the result of transversality models.  One starts with a multivalued map $\cF: X^n\rightrightarrows X^n$ on the top-cells of a complex.    The top-cells are thought of as vertices in a directed graph whose edges are given by $\xi_1\to \xi_2$ if $\xi_2\in \cF(\xi_1)$.  The strongly connected components of this directed graph have a partial order $J(L)$ which may be computed efficiently via Tarjan's algorithm.   This partial-ordering on top cells gives an implicit ordering on all lower-dimensional cells (i.e. faces) of $X$.  The implicit ordering yields $J(L)$-filtered complex $(X,\kappa,\preceq)\to (J(L),\leq)$.




Let $Cell(J(L))$ be the collection of $J(L)$-graded cell complexes.  In the propositions below we'll show that for $f\in Cell(J(L))$ there is an associated lattice-filtered chain complex in $Ch(L)$ and an associated chain complex braid in $CCB(J(L))$.


\begin{prop}
There is an assignment $\cL:Cell(J(L))\to Ch(L)$.
\end{prop}
\begin{proof}
A $J(L)$-graded cell complex gives rise to an $L$-filtered chain complex in the following fashion.   Using Birkhoff's theorem $f:(X,\kappa,\preceq)\to (J(L),\leq)$ becomes $O(f):L\to Sub_{Cl}(X,\preceq)$.  $Sub_{Cl}(X,\preceq)$ is the lattice of downsets in the face poset, i.e. closed subcomplexes.  Each element of $Sub_{Cl}(X,\preceq)$ is a basis for the associated subcomplex of $C(X)$ spanned by the downset.   Thus the map defined by $L\ni q\mapsto span(O(f)(q))\in Sub(C(X))$ is an $L$-filtered chain complex. 
\end{proof}


\begin{prop}
There is an assignment $\cB:Cell(J(L))\to CCB(J(L))$.
\end{prop}
\begin{proof}
The collection of intervals $I(J(L))$ have pre-images $f^{-1}(I)$.  The preimage of this map is a convex set in $(X,\preceq)$ and therefore a clopen subcomplex.   It is straightforward that the collection $\{C(I)\}_{I\in I(J(L))}$ with $C(I) := C(f^{-1}(I))\}$ satisfies the axioms of a chain complex braid. 
\end{proof}

\subsection{Splittings with Cell Complexes}


In applications the chain complexes typically arise as associated to cell complexes.  This implies that the chain complex $C(X)$ has a canonical basis.  Likewise for a $J(L)$-graded cell complex, there is a canonical filtered basis.

\begin{prop}
Let $f:X\to \sJ(L)$ be a graded cell complex.  Then 
\begin{enumerate}
\item The associated chain complex $(C(X),d)$ has a splitting $$C(X) = \bigoplus_{q\in J(L)} C(X_q)$$
\item The boundary operator is upper triangular with respect to this splitting.
\end{enumerate}
\end{prop}
\begin{proof}
For a $J(L)$-graded cell complex $f:X\to \sP$ we have $X=\bigsqcup_{q\in \sP} X_q$ where $X_q=f^{-1}(q)$.  
\end{proof}

Consider the lattice-filtered complex $\cL(f):L\to Sub(C(X),d)$ associated to $f:X\to J(\sL)$.  The splitting of the previous proposition in a $J(L)$-splitting for $\cL(f)$.  Morever it is canonical and $X=\bigsqcup_{q\in J(\sL)} X_q$ is a filtered basis for $C(X)$, in the sense that the union $\bigsqcup_{q\in \gamma} X_q$ is a basis for $\cL(\gamma)$.



\begin{prop}[\cite{fran}, Proposition 3.4]\label{prop:UT}
Given an upper triangular boundary map $$\Delta:\bigoplus_{q\in J(L)} C_q\to \bigoplus_{q\in J(L)} C_q$$ the collection, denoted $\cC\Delta(J(L))$, consisting of the chain complexes $C(I)$ with boundary map $\partial(I)$ for each $I\in I(J(L))$ and the obvious chain maps $i(I,IJ)$ and $p(IJ,J)$ for each $(I,J)\in I_2(J(L))$ is a chain complex braid over $J(L)$.
\end{prop}

If $h:L\to Sub(C)$ is a lattice-filtered chain complex with a $J(L)$-splitting $C^\oplus(J(L))$, then we call the chain complex braid of Proposition~\ref{prop:UT} {\em chain complex braid subordinate to $C^\oplus(J(L))$}.

\begin{lem}
Let $f$ be a $J(L)$-graded cell complex.  Let $\cL(f)$ be the associated filtered chain complex.  Then since $\cL(f)$ has a $J(L)$-splitting.  Then the chain complex braid subordinate to $C^\oplus(J(L))$ is precisely $\cB(f)$.
\end{lem}


\subsection{Conley Complexes and Franzosa's Connection Matrices}


\begin{prop}[\cite{atm}, Proposition 3.2]\label{prop:UTMap}
Let $\Delta,\Delta'$ be an upper triangular boundary map for $\bigoplus_{q\in J(L)} C_q$ and $\bigoplus_{q\in J(L) C_q'}$, respecitvely.  If $T:\bigoplus_{q\in J(L)} C_q\to \bigoplus_{q\in J(L)} C_q'$ is upper triangular with $T\Delta = \Delta'T$ then $\cT:=\{T(I)\}_{I\in I(P)}$ is a chain complex braid morphism from $\cC(\Delta)\to \cC(\Delta')$.
\end{prop}

%In conclusion, there is a functor from $F:Ch_X(k,L)\to CCB(J(L))$.  Moreover, this functor preserves the homotopy equivalence relationship.

%\begin{prop}
%Let $h:L\to Sub(C)$ and $h':L\to Sub(D)$ be filtered homotopy equivalent.  Then $F(h)$ and $F(h')$ are homotopy equivalent chain complex braids.
%\end{prop}
%
%\begin{cor}
%Let $h:L\to Sub(C)$ and $h':L\to Sub(D)$ be filtered homotopy equivalent. Then $\cH(F(h))$ and $\cH(F(h'))$, the graded module braids induced by the chain complex braids $F(h)$ and $F(h')$, are isomorphic. 
%\end{cor}


\begin{thm}
Let $f$ be a $J(L)$-graded cell complex.  Suppose that $h:L\to Sub(A,d_A)$ is a Conley complex for the associated lattice-filtered chain complex in $\cL(f)$ in $Ch(L)$ and that $h$ has a $J(L)$-splitting.  Then $d_A$ is a connection matrix, in the sense of Franzosa~\cite[Definition 3.6]{fran}, for $\cH(\cB(f))$, the graded module braid induced by the chain complex braid $\cB(f)$.
\end{thm}
\begin{proof}
If $A$ has a $J(L)$-splitting then it splits as $$A=\bigoplus_{q\in J(L)} h(q_-)=\bigoplus_{q\in J(L)} H(h(q_-))$$

 With respect to this splitting $d_A$ is an upper triangular boundary map on Conley indices.
 
 Take $(A,d_A)$ and generate a chain complex braid $\cC(A)$ via Proposition~\ref{prop:UT}.  We want to show that $\cH(\cC)$ is isomorphic to $\cH(\cB(f))$.
 
 We'll have this if there is a quasi-isomorphism of $\cC\to \cB(f)$.  We claim that this comes from the filtered homotopy equivalences $(\phi,\psi)$ and that $\cB(f)$ is precisely the chain complex associated to the $J(L)$-splitting of $\cL(f)$.  This is because with respect to the splittings these can be written as upper triangular boundary maps of graded chain complexes.  Then via Proposition~\ref{prop:UTMap} these induce morphisms of chain complex braids.  Moreover the filtered homotopy equivalence equation holds.  Therefore this induces an isomorphism on the associated braids.
 

\end{proof}

% We call this the $L$-filtered chain complex associated to $f$.  This describes a map from the set of $J(L)$-graded cell complexes to $Cell(J(L))$ to lattice-filtered complexes.  In practice we work in the full subcategory of $Ch(L)$ generated by the image of this map, which we call $Ch_X(L)$.
%
%  We call the full subcategory generated by the image of this map $CCB_X(J(L))$.



%If $f:(X,\kappa,\preceq)\to (J(L),\leq)$ is $J(L)$-filtered, then $f:(X,\preceq)\to (J(L),\leq)$ is a poset morphism.  In this case, Birkhoff's theorem provides a lattice homomorphism $O(f):L \to O(X,\preceq)$.  $O(X,\preceq)$ is lattice of downsets of the face poset and for a cell complex $(X,\kappa,\preceq)$ the downsets of $(X,\leq)$ correspond to the subcomplexes of $(X,\kappa,\leq)$. Moreover, they provide a basis for the subcomplexes of $C(X)$.   Thus a $J(L)$-filtered complex gives rise to a parameterization of subcomplexes in $(X,\kappa,\preceq)$.





%
%In this case we can outline a relationship between the category $\bCh_X(L)$ and $\bCCB(J(L))$.
%
%
%
%
%In~\cite[Section 7]{salamon} the splitting of a filtered chain complex is introduced.   For $h:L\to Sub(C,d)$ the chain complex $C$ may be written as $C=\bigoplus_{q\in J(L)} h(q)/h(Pred(q))$.  One can realize the quotient spaces $h(q)/h(Pred(q))$ as subspaces of $C$.  Choosing bases for these subspaces gives a filtered basis, i.e. a basis $B=\{b_\alpha\}$ of $C$ with a map $f:B\to J(L)$ such that $\{b\in B: f(b)\leq p\}$ is a basis for $h(p)$.

 


%In summary, this splitting is canonical since $X$ is a canonical basis for $C(X)$ .  From the $J(L)$-splitting we can construct a chain complex braid.


%
%\section{Chain Complex Braids for Graded Cell Complexes}
%
%There is a natural chain complex braid associated with a graded cell complex.  The collection of intervals $I(J(L))$ have pre-images $f^{-1}(I)$.  The preimage of this map is a convex set in $(X,\preceq)$ and thus a subcomplex.   The collection $\{C(f^{-1}(I))\}_{I\in I(J(L))}$ can be seen to satisfy the axioms of a chain complex braid.  Thus there is a map $Cell(J(\sL))\to CCB(J(L))$.  We call the full subcategory generated by the image of this map $CCB_X(J(L))$.
%
%
%
%\subsection{Relationship}
%
%The $J(\sL)$-graded cell complexes generate the subcategory $Ch_X(L)$.  They also generate the subcategory $CCB_X(J(L)$.  We wish to show that these two subcategories are equivalent.
%
%
%
%
%The graded cell complexes generate a subcategory of $Ch(L)$ that we label $Ch_X(L)$.  On this subcategory we can define a functor $F:Ch_X(L)\to CCB(J(L))$ which has some nice properties.
%
%
%
%
%Let $h\in Ch_X(L)$ with $h:L\to Sub(C(X),d)$.  There is a $J(L)$-splitting $C(X)=\bigoplus h(q)/h(Pred(q))$. Then $d$ is upper triangular with respect to the splitting.  We use the following observation from Franzosa:
%
%\begin{prop}[\cite{fran}, Proposition 3.4]
%Given an upper triangular boundary map $$\Delta:\bigoplus_{q\in J(L)} C_q\to \bigoplus_{q\in J(L)} C_q$$ the collection, denoted $\cC\Delta(J(L))$, consisting of the chain complexes $C(I)$ with boundary map $\partial(I)$ for each $I\in I(J(L))$ and the obvious chain maps $i(I,IJ)$ and $p(IJ,J)$ for each $(I,J)\in I_2(J(L))$ is a chain complex braid over $J(L)$.
%\end{prop}
%
%Moreover, the assignment $f\to Ch_X(L)\to CCB(J(L))$ commutes with the assignment $f\to CCB(J(L))$.
%
%
%
%Thus there is an assignment $F:\bCh_X(k,L)\to \bCCB(J(L))$.   Moreover, due to our observation in the previous section, if $h\in Ch_X(L)$ is a Conley complex then the splitting is $$\bigoplus_{p\in J(L)} H_\bullet(h(p)/h(Pred(p))$$ and $d$ is upper triangular with respect to this splitting.  Therefore $d$ is an upper triangular boundary map on Conley indices; precisely what Franzosa identifies as a connection matrix.  Thus Conley complexes get sent under this assignment to chain complex braids induced from a `connection matrix'.
%
%
%
%
%Finally, we formalize the previous discussion with a theorem relating our language to Franzosa's.  The proof is a straightforward from our previous observations.
%
%\begin{thm}\label{thm:cfcm}
%Let $h\in Ch_X(k,L)$.  Let $g:L\to Sub(A,d_A)$ be a Conley complex for $h$.  Then $d_A$ is a connection matrix, in the sense of Franzosa~\cite[Definition 3.6]{fran}, for $\cH(F(h))$, the graded module braid induced by the chain complex braid $F(h)$.
%\end{thm}
%


%\subsection{$J(L)$-Splitting}
%
%In applications the chain complexes we work with are associated to cell complexes.  This implies that the chain complex $C(X)$ has a canonical basis. In this case we can outline a relationship between the category $\bCh(L)$ and $\bCCB(J(L))$.
%
%
%In~\cite[Section 7]{salamon} the splitting of a filtered chain complex is introduced.   For $h:L\to Sub(C,d)$ the chain complex $C$ may be written as $C=\bigoplus_{q\in J(L)} h(q)/h(Pred(q))$.  One can realize the quotient spaces $h(q)/h(Pred(q))$ as subspaces of $C$.  Choosing bases for these subspaces gives a filtered basis, i.e. a basis $B=\{b_\alpha\}$ of $C$ with a map $f:B\to J(L)$ such that $\{b\in B: f(b)\leq p\}$ is a basis for $h(p)$.
%
%In our case, if $L\to Sub(C(X),d)$ is derived from $f:(X,\kappa,\leq)\to (P,\leq)$ then $X$ describes a basis for $C(X)$ and the map $f:X\to J(L)$ is a filtered basis.   Thus we can rewrite $C(X)$ as $C(X)=\bigoplus_{p\in J(L)} h(q)/h(Pred(q))$ and $d$ is an upper triangular boundary map with respect to this splitting.  We call this the $J(L)$-splitting.
%
%
%In particular, for Conley complexes the $J(L)$-splitting has a particular form.  Since $\partial(h(q)\subseteq h(Pred(q))$ we have for $h(q)/h(Pred(q))$ the induced differentials are zero, and thus $h(q)/h(Pred(q)) = H_\bullet(H(q)/h(Pred(q))$ where the homology is thought of as an object of $Ch_0$.  Thus for a Conley complex we may write the splitting as $C(X) = \bigoplus_{p\in J(L)} H(h(p)/h(Pred(p))$.
%
%
%In summary, this splitting is canonical since $X$ is a canonical basis for $C(X)$.  From the $J(L)$-splitting we can construct a chain complex braid.
%




%In this case $C(X)$ has a canonical basis generated by the cells $X$.  An $L$-filtered complex $L\to Sub(C(X),d)$ has a splitting $C(X)=\bigoplus_{q\in J(L)} C_q/C_{Pred(q)}$.  The boundary map $d$ is upper triangular with respect to this splitting.

%
%\subsection{Functor to $\bCCB(J(L))$}
%
%We work with the full subcategory $\bCh_X(L)$ of $\bCh(L)$ whose objects are $L$-filtered chain complexes associated to some filtered cell complex.  There is a functor $F:\bCh_X(k,L)\to \bCCB(J(L))$.
%
%Consider $h:L\to Sub(C(X),d)$.  There is a $J(L)$-splitting $C(X)=\bigoplus_{q\in J(L)} h(q)/h(Pred(q))$. Then $d$ is upper triangular with respect to the splitting.  We use the following observation from Franzosa:
%
%
%\begin{prop}[\cite{fran}, Proposition 3.4]
%Given an upper triangular boundary map $$\Delta:\bigoplus_{q\in J(L)} C_q\to \bigoplus_{q\in J(L)} C_q$$ the collection, denoted $\cC\Delta(J(L))$, consisting of the chain complexes $C(I)$ with boundary map $\partial(I)$ for each $I\in I(J(L))$ and the obvious chain maps $i(I,IJ)$ and $p(IJ,J)$ for each $(I,J)\in I_2(J(L))$ is a chain complex braid over $J(L)$.
%\end{prop}
%
%
%Thus there is an assignment $F:\bCh_X(k,L)\to \bCCB(J(L))$.   Moreover, due to our observation in the previous section, we have a Conley complex $L\to Sub(C,d)$ has a splitting $\bigoplus_{p\in J(L)} H(h(p)/h(Pred(p))$ and $d$ is upper triangular with respect to this splitting.  This is precisely what Franzosa identifies as a connection matrix.  Therefore Conley complexes get sent under this assignment to chain complex braids induced from a connection matrix.
%
%
%
%Morphisms of $L$-filtered complexes can be written as matrices with respect to the filtered bases.   Since these morphisms are filtered, they are upper triangular with respect to $J(L)$.  A morphism of lattice-filtered chain complexes $\phi:f\to g$ induces an upper-triangular map on the associated $J(L)$-decompositions $\phi:\bigoplus_{q\in J(L)}C_q\to \bigoplus_{q\in J(L)} C_q'$ such that $\phi\partial_f = \partial_g\phi$.  The following observation shows that this induces a morphism between the associated chain complex braids $\cC(f)\to \cC(g)$.
%
%\begin{prop}[\cite{atm}, Proposition 3.2]
%Let $\Delta,\Delta'$ be an upper triangular boundary map for $\bigoplus_{q\in J(L)} C_q$ and $\bigoplus_{q\in J(L) C_q'}$, respecitvely.  If $T:\bigoplus_{q\in J(L)} C_q\to \bigoplus_{q\in J(L)} C_q'$ is upper triangular with $T\Delta = \Delta'T$ then $\cT:=\{T(I)\}_{I\in I(P)}$ is a chain complex braid morphism from $\cC(\Delta)\to \cC(\Delta')$.
%\end{prop}
%
%In conclusion, there is a functor from $F:Ch_X(k,L)\to CCB(J(L))$.  Moreover, this functor preserves the homotopy equivalence relationship.
%
%\begin{prop}
%Let $h:L\to Sub(C)$ and $h':L\to Sub(D)$ be filtered homotopy equivalent.  Then $F(h)$ and $F(h')$ are homotopy equivalent chain complex braids.
%\end{prop}
%
%\begin{cor}
%Let $h:L\to Sub(C)$ and $h':L\to Sub(D)$ be filtered homotopy equivalent. Then $\cH(F(h))$ and $\cH(F(h'))$, the graded module braids induced by the chain complex braids $F(h)$ and $F(h')$, are isomorphic. 
%\end{cor}
%
%
%Finally, we formalize the previous discussion with a theorem relating our language to Franzosa's.  The proof is a straightforward from our previous observations.
%
%\begin{thm}\label{thm:cfcm}
%Let $h\in Ch_X(k,L)$.  Let $g:L\to Sub(A,d_A)$ be a Conley complex for $h$.  Then $d_A$ is a connection matrix, in the sense of Franzosa~\cite[Definition 3.6]{fran}, for $\cH(F(h))$, the graded module braid induced by the chain complex braid $F(h)$.
%\end{thm}




% In computational dynamics, $f$ is a combinatorial model for dynamics.  The fibers of $f$ parameterize the recurrent sets, and the order structure on $(P,\leq)$ organizes the gradient-like behavior.  

%We give a few illustrations of how a cell fibration may arise in practice:
%
%\begin{description}
%
%\item[Applied Topology]  Within applied topology, often data comes as a height function $X\to \R$, and one examines the change in the topology of the sublevel sets $f^{-1}(-\infty,t]$.  For instance, when $X$ is a collection of pixels, a new function is then defined on a cubical complex $\cX$ corresponding to the image such that the sublevel sets are subcomplexes.  This produces a cell fibration $(\cX,\preceq)\to (\R,\leq)$.  
%
%\item[Morse Theory] For a Morse function $f:M\to \R$ on smooth manifold $M$ one often examines the flow defined generated by $\dot x = -\nabla f(x)$.  The fixed points of the flow are indexed by a poset~\cite{smale} and their unstable manifolds carve out a CW decomposition of the manifold.  The map sending each cell in the CW-complex to its index within the poset is a cell fibration.
%
%\item[Dynamics] In computational dynamics, especially the database approaches, one often has a transitive relation defined on a cubical complex.  The transitive relation partitions the complex into recurrent and gradient-like behavior which takes the form of a chain fibration.  See the braids paper for a concrete example.
%
%\item[Combinatorics] Mrozek's multivector field
%
%\end{description}



 
