%!TEX root = ./main.tex


\section{Preliminaries}\label{sec:prelims}


In this section we review the necessary mathematical prerequisites.  We begin with the computational Conley paradigm.


\subsection{Computational Dynamics}

In the last few decades an algorithmic approach to Conley's approach to dynamical systems~\cite{conley} has been established~\cite{kmv, cmdb, cmdbchaos}.  This combinatorial-topological framework is central to our motivation for computing the connection matrix.  It has been made clear that one of the most prominent objects of the theory is the lattice of attractors and lattice of attracting blocks~\cite{kmv,lsa,lsa2,salamon}.

Let $f:X\times Z\to X$ be a dynamical system with $X,Z$ compact metric spaces.  We recall the standard pipeline:

\begin{itemize}
\item Select {\em grids} $\cX,\cZ$ on $X$ and $Z$.  In practice $\cX,\cZ$ are cubical complexes.
\item Compute a lattice of subcomplexes which serve as attracting blocks of $f$
\item For each join irreducible, compute a Conley index -  an algebraic topological invariant which provides a coarse measurement of the unstable dynamics associated with $M_\zeta$. This may be done efficiently by interpreting $M_\zeta$ as a subcomplex of $\cX$ and using computational homology 
\item Invoke theorems~\cite{cmdb,cmdbchaos} to lift the computational results to rigorous results for the continuous system $f$ 

%\item Construct an {\em outer approximation} $\cF:\cX\times \cZ\to \cX$ of $f$, i.e. a relation between $\cX\times \cZ$ and $\cX$ such that for all $\xi\in \cX$ and $\zeta\in \cZ$ $$f_{|\zeta|}(|\xi|)\subseteq int |(\cF(\xi,\zeta)|$$
%\item For $\zeta\in \cZ$, $\cF_\zeta := \cF(\cdot, \zeta)$ be decomposed into recurrent and gradient-like parts, in analogy to Conley's decomposition theorem~\cite{conley,kmv}.  This is done by interpreting $\cF_\zeta$ as a directed graph with vertex set $\cX$ and applying Tarjan-like algorithms~\cite{cmdbchaos}
%\item For each recurrent set $M_\zeta$ of $\cF_\zeta$ compute a Conley index - an algebraic topological invariant which provides a coarse measurement of the unstable dynamics associated with $M_\zeta$. This may be done efficiently by interpreting $M_\zeta$ as a subcomplex of $\cX$ and using computational homology algorithms~\cite{cmdbchaos}.
%\item Invoke theorems~\cite{cmdb,cmdbchaos} to lift the computational results to rigorous results for the continuous system $f$
\end{itemize}

As we will show, the lattice of subcomplexes can (via Birkhoff's theorem) be codified in morphism $(X,\preceq)\to (P,\leq)$ between posets, where $(X,\preceq)$ is the face poset of $X$ and $(P,\leq)$ is the set of join-irreducibles.


The connection matrix is a boundary operator on the Conley indices and promotes the Conley theory into a homology theory.  In this setting, an algorithm for the connection matrix promotes the computational Conley theory to a computational homology theory.




\subsection{Algebraic Topology}\label{sec:prelims:AT}

We review some algebraic topology.  This exposition follows~\cite{weibel}.  Let $\F$ be a field.  A {\em chain complex} $C_\bullet$ of $\F$-vector spaces is a family $\{C_n\}_{n\in \N}$ of vector spaces over field $\F$ together with linear maps $\partial=\partial_n:C_n\to C_{n-1}$.  When the context is clear we will abbreviate $C_\bullet$ by $C$.  A morphism $f:A\to B$ is a {\em chain map}, that is a family of linear maps $f_n:A_n\to B_n$ such that $f_{n-1}\partial^A = \partial^B f_n$. Chain complexes and chain maps make up a category denoted $Ch(\F)$.  

A chain complex $B$ is called a {\em subcomplex} of $C$ if each $B_n$ is a subspace of $C_n$ and $\partial(B_n)\subset B_{n-1}$, i.e. that the inclusion map $i:B\to C$ is a chain map.  In this case we may assemble the quotients $C_n/B_n$ into a chain complex denoted $C/B$ called the {\em quotient complex}.   The $n$th homology of $C$ is the quotient $H_n(C):= \ker \partial_n/im \partial_{n+1}$.  The graded vector space $H_\bullet(C) := \{H_n(C)\}_{n\in \N}$ is the {\em homology} of $C_\bullet$.  Chain maps induce linear maps on homology.  A chain map $A\to B$ is a {\em quasi-isomorphism} if the maps $H_n(A)\to H_n(B)$ are all isomorphisms.

Two chain maps $f,g:A\to B$ are {\em chain homotopic} if there exists degree +1 maps $h_n:A_n\to B_{n+1}$ such that $$f-g = h\partial_A+\partial_Bh$$  We say that $f:A\to B$ is a {\em chain homotopy equivalence} if there is a chain map $g:B\to A$ such that $fg$ and $gf$ are chain homotopic to the respective identity maps of $A$ and $B$.  Chain homotopy equivalence is an equivalence relation on $Hom(A,B)$.  The set of such equivalence classes $Hom_K(A,B)$ is an abelian group.  The category $K$ consisting of chain complexes with hom sets given by $Hom_K(A,B)$ is called the homotopy category.  The isomorphisms in this category are precisely the equivalence classes of the chain homotopy equivalences.


The rest of this exposition follows~\cite{focm,mn}.  This particular definition of complex dates back to Lefschetz.

\begin{defn}
{\em
Consider a finite graded set $\cX = \bigsqcup_{q\in \Z} \cX_q$ along with a function $\kappa:\cX\times\cX\to \F$ and denote $\xi\in \cX_q$ by $\dim \xi = q$.  Then $(\cX,\kappa)$ is a {\em complex} if the following hold:
\begin{enumerate}
\item \label{cond:1} For each $\xi$ and $\xi'$ in $\cX$:
$$\kappa(\xi,\xi')\neq 0\quad\text{implies}\quad \dim \xi = \dim \xi'+1$$
\item\label{cond:2} For each $\xi$ and $\xi''$ in $\cX$,
$$\sum_{\xi'\in \cX} \kappa(\xi,\xi')\cdot \kappa(\xi',\xi'')=0$$
\end{enumerate}
}
\end{defn}

An element $\xi\in \cX$ is called a {\em cell} and $\dim \xi$ is the {\em dimension} of $\xi$.  The function $\kappa$ is the {\em incidence function} of the complex $(\cX,\kappa)$.   The {\em face partial order} $\preceq$ is induced on the elements of $\cX$ by the transitive closure of the generating relation $\prec$ given as follows: For $\xi, \xi'\in \cX$ $$\xi' \prec \xi \quad \text{if} \quad \kappa(\xi,\xi')\neq 0$$
Let $(\cX,\kappa)$ be a complex.  The {\em associated chain complex} consists of free vector spaces $C_q(\cX)$ where the basis elements are the cells $\xi \in \cX_q$ and the boundary operator is generated by the maps $$\partial_q \xi := \sum_{\xi' \in \cX} \kappa(\xi, \xi')\xi'$$ It is straightforward to verify that the associated chain complex of a complex is indeed a chain complex.











\subsection{Order Theory}\label{sec:prelims:order}

Order theory is the study of posets and lattices.  Order theory has a strong relationship to both algebraic topology and dynamical systems, e.g.~\cite{salamon,lsa,lsa2}.  An intuition for posets, lattices and their correspondence via Birkhoff's theorem will be very helpful for understanding the paper.


\subsubsection{Posets}

A morphism of posets is a map $h:(P,\leq_P)\to (Q,\leq_Q)$ such that if $p\leq_P q$ then $h(p)\leq_Q h(q)$. Posets and their morphisms form the category {\bf Poset}.

The face poset $(X,\preceq)$ provides a powerful method of thinking about a complex $(X,\kappa)$.  One reason is that subcomplexes of $X$ have a nice characterization in terms of {\em convex sets} of $(X,\preceq)$.   Let $P$ be a poset.  An {\em upper set} of $P$ is a subset $U\subset P$ such that if $p\in U$ and $p\leq q$ then $q\in U$.  For $p\in P$ the {\em upset} at $p$ is $\uparrow(p):=\{q\in P:p \leq q\}$.  A {\em lower set} of $P$ is a set $D\subset P$ such that if $q\in D$ and $p\leq q$ then $p\in D$.  The {\em downset} at $q$ is $\downarrow(q):=\{p\in P: p \leq q\}$.  A subset $I\subset P$ is an {\em convex set} if $p,q\in I, r\in P$ and $ p < r < q$ implies that $r\in I$.  Any convex set in $P$ can be obtained by an intersection of a lower and upper set.  
 
\begin{defn}
{\em
A collection $(I_1,\ldots, I_N)$ of convex sets of $(P,\leq)$ is called {\em adjacent} if
\begin{enumerate}
\item $I_1,\ldots,I_n$ are mutually disjoint
\item $\bigcup_{i=1}^n I_i$ is a convex set in $P$
\item $p\in I_i, q\in I_j, i < j$ imply $q \nless p$
\end{enumerate}
}
\end{defn}

We will be primarily interested in adjacent pair of convex sets $(I,J)$.  We write the union (a convex set) $I\cup J$ as $IJ$.  We will denote the set of convex sets as $I(P)$ and the set of adjacent tuples and triples of convex sets as $I_2(P)$ and $I_3(P)$.  This notation agrees with~\cite{fran}.
%The convex sets are particularly important for complexes:

\begin{prop}\label{prop:subcomplex}
Let $(\cX,\kappa)$ be a complex.  Let $(\cX,\preceq)$ be its face poset.  If $\cX'$ be an convex set in $(\cX,\preceq)$ then $(\cX',\partial_{\cX'})$ is a subcomplex.
\end{prop}
\begin{proof}
We must show that $\partial_{\cX'}\circ \partial_{\cX'}=0$.  If $\xi,\xi''\in \cX'$ and $\xi'' \prec \xi$ then all $\xi'\in \cX$ such that $\xi'' \prec \xi' \prec \xi$ must be in $\cX'$ since $\cX'$ is an interval.  Thus~(\ref{cond:1}) and~(\ref{cond:2}) imply that $\sum_{\xi'\in \cX'} \kappa(\xi, \xi')\cdot \kappa(\xi',\xi'')=0$ which implies $\partial_{\cX'}^2=0$.
\end{proof}

\begin{cor}\label{cor:clsubcomplex}

If $\cX'$ is a lower set then this is a closed subcomplex.

\end{cor}

Posets have a topology known as the Alexandrov topology.  It is easy to see that a map $h:(P,\leq_P)\to (Q,\leq_Q)$ is a morphism of posets if and only if $h$ is continuous with respect to the Alexandrov topologies.


\subsubsection{Lattices}

\begin{defn}
{\em
A {\em lattice} is a set $L$ with the binary operations $\vee,\wedge:L\times L\to L$ satisfying the following axioms:

\begin{enumerate}
\item (idempotent) $a\wedge a = a \vee a = a$ for all $a\in L$
\item (commutative) $a\wedge b = b\wedge a$ and $a\vee b = b \vee a$ for all $a,b\in L$
\item (associative) $a\wedge b(b\wedge c) = (a\wedge b)\wedge c$ and $a\vee(b\vee c) = (a\vee b)\vee c$ for all $a,b,c\in L$
\item (absorption $a\wedge (a\vee b) = a\vee (a\wedge b)=a$ for all $a,b\in L$

A lattice $L$ is {\em distributive} if it satisfies the additional axiom:

\item (distributive) $a\wedge (b\vee c) = (a\wedge b)\vee (a\wedge c)$ and $a\vee (b\wedge c) = (a\vee b) \wedge (a\vee c)$ for all $a,b,c\in L$

A lattice $L$ is {\em bounded} if there exist elements $0_L$ and $1_L$ such that

\item $0_L\wedge a = 0_L, 0_L\vee a = a, 1_L\wedge a = a, 1_L\vee a = 1_L$ for all $a\in L$
\end{enumerate}
}
\end{defn}

A lattice morphism $h:L\to M$ is a map such that if $a,b\in L$ then $f(a\wedge b) = f(a)\wedge f(b)$ and $f(a\vee b) = f(a)\vee f(b)$.  If $L$ and $M$ are bounded lattices then we also require that $f(0_L)=0_M$ and $f(1_L)=1_M$.    Bounded, distributive lattices and their morphisms form the category {\bf BDLat}.

We say that $q$ {\em covers} $p$ if $p\leq q$ and there does not exist an $r$ with $p\leq r \leq q$.  If $q$ covers $p$ then we say $p$ is a {\em predecessor} of $q$.  An element $a\in L$ is {\em join-irreducible} if it has a unique predecessor.   A subset $K\subset L$ is called a sublattice of $L$ if $a,b\in K$ implies that $a\vee b\in K$ and $a\wedge b\in K$.  A lattice $L$ has an associated poset structure given by $a\leq b$ if $a=a\wedge b$ or if $b=a\vee b$.

\begin{defn}
{\em
For a lattice $L$ define $Pred:L\backslash \{0_L\} \to L$ via $Pred(p) = \bigwedge \{q: \text{$p$ covers $q$}\}$, i.e. the meet of the predecessors.
}
\end{defn}

 Notice for join irreducible elements that $Pred$ yields the unique predecessor.

\begin{lem}[\cite{roman}, Theorem 4.29]\label{lem:join}
Let $L$ be a bounded distributive lattice.  Any $p\in L$ can written as the irredundant join of join-irreducibles.
\end{lem}


\subsubsection{Birkhoff's Correspondence}

For a lattice $L$ we denote its join-irreducibles as $J(L)$.  $J(L)$ has a poset structure.  $J$ is a functor $J:{\bf BDLat}\to {\bf Poset}$.  For a poset $(P,\leq)$ we denote set of downsets by $O(P)$. $O(P)$ has the structure of a distributive lattice.  $O$ is a functor $O:{\bf Poset}\to {\bf BDLat}$.  This is formalized via Birkhoff's theorem.  See~\cite{lsa,lsa2,salamon} for a discussion in the context of dynamics.

\begin{thm}[\cite{lsa}]\label{thm:birkhoff}
$J$ and $O$ are contravariant functors and provide an equivalence of categories {\bf Poset} and {\bf BDLat}, i.e. $$L\cong O(J(L))\quad\quad P\cong J(O(P))$$

\end{thm}

It is often the case that lattices are related to algebraic structures via a study of their substructures.  For instance, consider a vector space $V$.  It is straightforward to verify that the collection of subspaces of $V$ forms a bounded lattice under the operations $\cap$ and $+$ (span). However this lattice of subspace is not distributive.  Similarly, for a chain complex $C$, it is again straightforward that the collection of subcomplexes of $C$ form a bounded lattice under the operations $\cap$ and $+$.  We denote this lattice $S(C)$.  Again, note that $S(C)$ is not distributive.









