%!TEX root = ./main.tex

\section{The Homotopy Category of Chain Fibrations}\label{sec:homotopy}


  In this section we will build a homotopy category for $\bChF(L)$.  This construction will be analogous to the relationship between $K(\F)$ and $Ch(\F)$, which was discussed in Section~\ref{sec:prelims:AT}.

\begin{defn}
{\em
Let $f,g\in \bChF(L)$.  Let $\phi,\psi:f\to g$ be fibration morphisms.  A {\em fibration homotopy} from $\gamma:\phi\two \psi$ is a map that is (1) a chain homotopy between $\phi$ and $\psi$ and (2) order-preserving $f\to g$, i.e.
\begin{enumerate}
\item $\phi - \psi = \partial_{C'}\circ \gamma + \gamma\circ \partial_C$
\item $g\gamma x \leq f(x)$
\end{enumerate}
}
\end{defn}


%\begin{rem}
%Condition (1) makes the constraint $f(\partial\gamma+\gamma\partial)\leq f$ however as far as I can tell (2) is still needed.
%\end{rem}

If there exists a fibration homotopy from $\phi$ to $\psi$ we say they are {\em homotopic} and denote this by $\phi\sim \psi$.  

\begin{prop}
 Let $f,g\in \bChF(L)$.  $\phi \sim \psi$ is an equivalence relation on $Hom(f,g)$.
\end{prop}
%\begin{proof}
%We have three things to prove.
%
%\begin{enumerate}
%\item Reflexive: Let $\phi \in Hom(f,g)$. We must show that $\phi\sim \phi$.  Define $\gamma=0$.
%
%\item Symmetric:  Let $\phi, \psi \in Hom(f,g)$.  Suppose that $\phi\sim \psi$ by $\gamma$. We must show $\psi\sim \phi$.  Choose $-\gamma$.
%
%\item Transitive: Suppose $\phi\sim \psi$ and $\psi \sim \theta$.  We must show $\phi\sim \theta$.  We have that there exists $\gamma:\phi \two \psi$ and $\gamma':\psi \two \theta$.  Choose $h:= \gamma+\gamma'$.  Then 
%
%$$\phi - \theta = (\phi-\psi)+(\psi-\theta) = (\partial\gamma + \gamma\partial) + (\partial\gamma'+\gamma'\partial) = \partial(\gamma+\gamma')+(\gamma+\gamma')\partial = \partial h + h \partial$$
%
%For the second condition, we have that $$f'hx = f'(\gamma x+\gamma' x) \leq f'\gamma x \vee f'\gamma' x \leq f'x \vee f' x = f'x$$
%
%\end{enumerate}
%\end{proof}


The define the category $\bD(L)$ whose objects are coherent chain fibrations over $L$ and whose hom-sets areobtained by the quotient $Hom_{\bD(L)}(f,g) := Hom_{\bChF(L)}(f,g)/\sim$.  This is analogous to the derived category.

\begin{defn}
{\em
Let $f,g\in \bChF(L)$.  We say $f,g$ are {\em homotopy equivalent} if there exists morphisms $\phi:f\to g$ and $\psi:g\to f$ such that $\phi\circ \psi\sim id_g$ and $\psi\circ \phi \sim id_f$. Such maps $\phi,\psi$ are representatives of isomorphisms in $\bD(L)$.
}
\end{defn}

The next theorem is proved algorithmically with algebraic Morse theory:


\begin{thm}\label{thm:exist}
Let $f\in \bChF(L)$.  Then there is a connection fibration $g$ which is homotopy equivalent to $f$.
\end{thm}
\begin{proof}
$f_i$ is coherent.  Thus there exists a basis by Theorem~\ref{thm:cfdecomp}, (\ref{thm:cfdecomp:basis}).  Use~\cite[Algorithm 3.6]{focm} to obtain an acyclic partial matching.  This induces a new fibration $f_{i+1}:M\to L$.  The number of basis elements must decrease monotonically, thus this process stabilizes at some point.

\end{proof}


\begin{rem}
Thus any coherent chain fibration has a a connection fibration within its homotopy equivalence class.  This is the analogue of Franzosa's existence of a connection matrix~\cite{fran}.
\end{rem}


The upshot of this is that an equivalence of chain fibrations induces an equivalence of their associated chain complex braids.  Therefore we have the following:

\begin{prop}
Let $f,g$ be coherent and $f\sim g$. Then $H\cC(f)$ and $H\cC(g)$, the graded module braids induced by the chain complex braids $\cC(f)$ and $\cC(g)$, are isomorphic.
\end{prop}

Finally, we may relate our construction to Franzosa's with the following theorem:

\begin{thm}\label{thm:cfcm}
Let $f\in \bChF(L)$.  Let $g:(C,\Delta)\to L$ be a connection fibration for $f$.  Then $\Delta$ is a connection matrix, in the sense of Franzosa~\cite[Definition 3.6]{fran}, for $H\cC(f)$, the graded module braid induced by the chain complex braid $\cC(f)$.
\end{thm}


Another way of thinking of a connection fibration is as some sort of initial object in the homotopy equivalence class.


