%!TEX root = ./main.tex
\section{Chain Fibrations}\label{sec:cf}

In this section we will introduce our fundamental tools: cell fibrations and chain fibrations.


\subsection{The Model of Cell Fibrations}

In applications data often comes in the form of a cell complex $(X,\kappa)$ filtered by a partial order $(P,\leq)$.  This is codified in terms of an order-preserving map $f:(X,\preceq)\to (P,\leq)$, where $(X,\preceq)$ is the face poset of the cell complex $(X,\kappa)$. As we are explicitly thinking of $(X,\preceq)$ as being a face poset of some cell complex, we call such order-preserving maps {\em cell fibrations}.


% In computational dynamics, $f$ is a combinatorial model for dynamics.  The fibers of $f$ parameterize the recurrent sets, and the order structure on $(P,\leq)$ organizes the gradient-like behavior.  

%We give a few illustrations of how a cell fibration may arise in practice:
%
%\begin{description}
%
%\item[Applied Topology]  Within applied topology, often data comes as a height function $X\to \R$, and one examines the change in the topology of the sublevel sets $f^{-1}(-\infty,t]$.  For instance, when $X$ is a collection of pixels, a new function is then defined on a cubical complex $\cX$ corresponding to the image such that the sublevel sets are subcomplexes.  This produces a cell fibration $(\cX,\preceq)\to (\R,\leq)$.  
%
%\item[Morse Theory] For a Morse function $f:M\to \R$ on smooth manifold $M$ one often examines the flow defined generated by $\dot x = -\nabla f(x)$.  The fixed points of the flow are indexed by a poset~\cite{smale} and their unstable manifolds carve out a CW decomposition of the manifold.  The map sending each cell in the CW-complex to its index within the poset is a cell fibration.
%
%\item[Dynamics] In computational dynamics, especially the database approaches, one often has a transitive relation defined on a cubical complex.  The transitive relation partitions the complex into recurrent and gradient-like behavior which takes the form of a chain fibration.  See the braids paper for a concrete example.
%
%\item[Combinatorics] Mrozek's multivector field
%
%\end{description}


\subsection{Chain Fibrations}

We lift the concept of a cell fibration to that of a {\em chain fibration} by promoting cells to chains.  Lifting to chains will facilitate the use of algebraic Morse theory.  This will also help us relate more directly to chain complex braids.



\begin{defn}\label{def:cf}
{\em
Let $(C,\partial)$ be a chain complex and $L$ be a bounded, finite distributive lattice.  A {\em chain fibration} is a function $f:C\to L$ such that
\begin{enumerate}
\item $f\circ \partial(a) \leq f(a)$

\item $f(a+ b) \leq f(a)\vee f(b)$

\item $f(\lambda a) = f(a)$ for $\lambda\in \F$ and $\lambda \neq 0$ 

\item $f^{-1}(0_L) = \{0 \in C_n: n\in \N\}$

\end{enumerate}
}
\end{defn}


These conditions ensure that $f$ organizes subcomplexes of $C$.  For $p\in L$ the set $\{q\in L:q\leq p\}$ is a sublattice of $L$.  We will be interested in the preimage of this sublattice $B(p,f):=f^{-1}\{q\in L: q\leq p\}$. When $f$ is fixed or clear from context, we will denote $B(p)=B(p,f)$.



\begin{prop}\label{prop:cfDown}
Let $f:C\to L$ be a chain fibration and let $p\in L$.  Then $B(p,f):=f^{-1}\{q\in L: q\leq p\}$ is a subcomplex.  
\end{prop}
\begin{proof}
$B(p)$ inherits a grading from $C$. Let $B_n\subset C_n$  be the $n$th component of $B(p)$.  Let $a\in B_n$.  Then property (1) gives $f\partial (a)\leq f(a)\leq p$, implying $\partial a\in B_n$.  Therefore $\partial(B_n)\subset B_n$.  To see $B_n$ is a subspace, observe that:
\begin{enumerate}
\item $0\in B_n$ as $0_L \leq p$ and $f(0_n)=0_L$ by property (4)
\item For any $a,b\in B_n$ we have $f(a+b)\leq f(a)\vee f(b)\leq p \vee p = p$ from property (2).  Therefore $a+b\in B_n$.
\item For any $a\in B_n$ and $\lambda\in \F$ with $\lambda\neq 0$ we have $f(\lambda a ) = f(a) \leq p$ from (3). Therefore $\lambda a \in B_n$.
\end{enumerate}
\end{proof}

Chain fibrations are a model of cell fibrations.  We've seen that a map $h:(X,\preceq)\to (P,\leq)$ is a poset morphism if and only the preimages of lower sets are lower sets (continuous in the Alexandrov topology).  A similar result holds for chain fibrations.  If $f:C\to L$ is any function such that the preimages of the sublattices $\{q\in L:q\leq p\}$ are subcomplexes for each $p\in L$ then $f$ is nearly a chain fibration:

\begin{prop}\label{prop:CFDefEquiv}
Let $f:C\to L$ be a function such that $B(p)\subseteq C$ is a subcomplex of $C$ for each $p\in L$.  Then conditions $(1)-(3)$ hold.  Furthermore a weaker condition of (4), namely $f(0_i) = 0_L$ for each $i\in \N$ holds.
\end{prop}
\begin{proof}
Let $f:C\to L$ satisfy the hypothesis. We will argue by contradiction.

\begin{enumerate}
\item  Suppose there exists $a\in C$ such that $f\partial(a) \not\leq f(a)$.   Then $\partial f(a)\not\in B(f(a))$, implying $B(f(a))$ is not a subcomplex.  This contradicts our hypothesis. Therefore $f\partial (a) \leq f(a)$.

\item Suppose there exists $a,b\in C$ such that $f(a+b)>f(a)\vee f(b)$.  Define $p:=f(a)\vee f(b)$.  Consider $B(p,f)$. Then $a,b\in B(p)$ but $a+b\not\in B(p)$.  Therefore $B(p)$ is not a subcomplex, which contradicts our hypothesis.  Thus $a+b \in C$.

\item Suppose there exists $a\in C$ and $\lambda\in \F, \lambda\neq 0$ such that $f(\lambda a)\neq f(a)$.  If $f(\lambda a)$ and $f(a)$ are not comparable, then $\lambda a \not\in B(f(a))$, thus $B$ is not a subcomplex.  If $f(\lambda a)$ and $f(a)$ are comparable, then either  $f( a ) < f(\lambda a)$ or $f(\lambda a)< f(a)$.  If $f(a)<f(\lambda a)$ then $\lambda a\not\in B(f(a))$.  Then $B(f(a))$ is not a subcomplex.  If $f(\lambda a)< f(a)$ then $a\not\in B(f(\lambda a))$.  Thus $B(f(\lambda a))$ is not a subcomplex.  All of these contradict our hypothesis.  Therefore $f(\lambda a ) = f(a)$ for all $a$ and $\lambda \neq 0$.
\end{enumerate}

 Finally, suppose there exists $n\in N$ such that $f(0_n)=p\neq 0_L$.  Then $0_n\not\in B_n(Pred(p))$.  Therefore $B(Pred(p))$ is not a subcomplex.  This is yet another contradiction.

\end{proof}

\begin{cor}\label{cor:cfEquiv}
Let $f:C\to L$ be a function such that $B(p)\subseteq C$ is a subcomplex for each $p\in L$ and (4) of Definition~\ref{def:cf} holds.  Then $f$ is a chain fibration.
\end{cor}


Given a cell fibration $g:(X,\preceq)\to (P,\leq)$ one may define a map $f:C(X)\to O(P,\leq)$ where $O(P)$ is the lattice of lower sets of $P$ as follows: for $\sum \lambda_i a_i$ with $\lambda_i\neq 0$
\begin{align} \label{eqn:cell2chain}
f(\sum_i \lambda_i a_i):= \bigcup_i \downarrow (g(a_i))\quad\quad\quad\quad\quad\quad
f(0_d):= 0_L
\end{align}

\begin{prop}
Let $g:(X,\preceq)\to (P,\leq)$ be a cell fibration.  Let $f:C(X)\to O(P,\leq)$ be defined as in Eqn.~(\ref{eqn:cell2chain}).  Then $f$ is a chain fibration.
\end{prop}
\begin{proof}


It is easy to see from the definition that $f(\sum \lambda_i a_i)$ is indeed a lower set of $P$ since it is expressed as a union of down sets.  Thus $f$ is indeed a map $C(X)\to O(P)$.  There are four things to show.

\begin{enumerate}
\item Let $x\in C(X)$. If $x=0$, then $\partial x =0$ and $f\partial x = fx$.  If $x\neq 0$, then $x = \sum_i \lambda_i a_i$ for $\lambda_i\neq 0$ and $a_i\in (X,\preceq)$.  Thus $$f\partial x = f\partial (\sum_i \lambda_i a_i) =  f (\sum_i \lambda_i \partial a_i) $$  

We can write $\partial a_i = \sum_j \beta_{i,j} y_{i,j}$ for $y_{i,j}\in (X,\preceq)$ and $y_{i,j}\preceq a_i$.  Then $$f\partial x = f \sum_i \lambda_i (\sum_j \beta_{i,j} y_{i,j}) \subseteq \bigcup_{i,j} \downarrow ( g(y_{i,j})) $$  

The final relation is not a strict equality as there may be some cancellation of $y_{i,j}$.  As $y_{i,j}\preceq x_i$ we have $g(y_{i,j})\leq g(a_i)$, implying $\downarrow (g(y_{i,j}))\subseteq \downarrow ( g(a_i))$.  Therefore $$f(\partial x)\subseteq  \bigcup_{i,j} \downarrow ( g(y_{i,j})) \subseteq \bigcup_i \downarrow( g(a_i)) = f(x)$$


\item Let $a,b\in C_n(X)$.  Since $X^n$ forms a basis we have $a = \sum_i \lambda_i x_i$ and $b=\sum_i \beta_i x_i$ for $x_i\in (X,\preceq)$.  As it may be that some $\beta_i=0$ or $\lambda_i=0$,  $$f(a),f(b)\subseteq \bigcup_i \downarrow g(x_i) $$  Now $$f(a+b) = f(\sum_i (\lambda_i+\beta_i)x_i) \subseteq \bigcup_i \downarrow(g(x_i))$$  Again, the final relation is not strict equality, as it may be the case that $\lambda_i+\beta_i=0$ for some $i$.  

\item $f(\lambda a) = f(a)$ for $\lambda\neq 0$. Let $a\in C$.  Then we may write $a = \sum_i \lambda_i x_i$ with $\lambda_i\neq 0$.  Thus $\lambda\lambda_i\neq 0$.  Then $$f(\lambda a) = f(\lambda \sum_i \lambda_i x_i) = f(\sum_i \lambda \lambda_i x_i) = \bigcup_i \downarrow g(x_i) = f(a)$$

\item  $f^{-1}(0_L) = \{0 \in C_n:n\in \N\}$.  We have $\{0\in C_n:n\in \N\}\subset f^{-1}(0_L)$ as by the construction we set $f(0)=0_L$ for $0\in C_n$.  Now let $a\in f^{-1}(0_L)$.  If $a\neq 0$, then we may write $a = \sum_i \lambda_i x_i$ for $\lambda_i\neq 0$ for $x_i\in (X,\preceq)$.  Thus $$f(a) = \bigcup_i \downarrow (g(x_i)) \neq \emptyset$$  This forces $a=0$.

\end{enumerate}  

%We will use Corollary~\ref{cor:cfEquiv} to show that $f$ is a chain fibration.  As $g:(X,\preceq) \to (P,\leq)$ is a cell fibration, one may apply the contravariant functor provided by Birkhoff's theorem (Theorem~\ref{thm:birkhoff}) to obtain a lattice homomorphism $O(g):O(P)\to O(X,\preceq)$. Notice that $O(X,\preceq)$ are lower sets of the complex $(X,\preceq)$.  Therefore they are closed subcomplexes by Corollary~\ref{cor:clsubcomplex}.  For each $p\in L$, $O(g)(p)$ gives a distinguished basis for $f^{-1}(q\in L:q\leq p$.  Therefore these are subcomplexes.  
%
%
%
%
%
%
%
%Then $D_L:=O(g)(L)$ is a closed subcomplex in $(X,\preceq)$ with a distinguished basis.  We claim that $O(g)(L)$ is a distinguished basis for $D_L$.  Let $x\in D_L$.  We can write $x = \sum \sigma_i a_i$ with $a_i\in X$.  We want to show that $a_i\in O(g)(L)$.  Since $x\in D_L$ we have $f(\sum \sigma_i a_i) = f(x) \leq L$.  By definition $f(\sum\sigma_i a_i) = \bigcup_i \downarrow(f_c(a_i))$.  Thus $\downarrow(f_c(a_i))\leq L$ in the partial order (inclusion) on $O(P)$.  Therefore $f_c(a_i)\in L$.  So $a_i\in O(f_c)(L)$.
 
% From the fact that $O(f_c)$ is a lattice homomorphism it follows that the mapping $L\mapsto D_L$ is also a lattice homomorphism.



\end{proof}
 
 For a cell fibration $f:(\cX,\preceq)\to (P,\leq)$ we call $f:C(X)\to O(P,\leq)$ defined as in Eqn.~(\ref{eqn:cell2chain}) the {\em associated chain fibration}.





%We give a few illustrations of how a chain fibration may arise in practice.  These two examples are very similar.
%
%\begin{description}
%
%\item[Applied Topology] We may define a `Rips complex' using the idea of chain fibration, as in Usher, Zhang, Example 2.4.   Let $X$ be a point cloud embedded in $\R^n$.  Let $C_\bullet(X)$ be the chain complex associated to the complete simplicial complex $K$ on $X$.  Then each simplex $\sigma$ of $K$ corresponds to a chain in $C_\bullet(X)$ and has an associated diameter $\alpha\in \R_+$.  Define $f(\sum_i \lambda_i \sigma_i):= \max_i \{\alpha_i\in \R_+:\lambda_i\neq 0\}$.  We could also map our point cloud generated simplicial complex to a more complex lattice, which has not previously been considered.
%
%\item[Morse Theory]  Let $X$ be a closed manifold and $f$ be a Morse function on $X$.  Let $\{p_i\}$ be the collection of critical points.  Then the collection $\{f (p_i)\}\cup \{0\}$ form a sublattice. Then there is a corresponding Morse complex $M_\bullet(X;f)$.  Let $C=M$.  For any $x\in C$, with $x=\sum_i \lambda_i p_i$ where each $p_i$ is a critical point of index $i$, we define $f(\sum_i \lambda_i p_i) = \max_i \{f(p_i):\lambda_i \neq 0\}$.


%\end{description}


We now describe {\em connection fibrations}, which are particularly simple chain fibrations. This is our analogue of the connection matrix of Franzosa.

\begin{defn}\label{def:connection}
{\em
A {\em connection fibration} is a chain fibration $f:C\to L$ if $f(\partial x) < f(x)$ for all $x\in C$.
}
\end{defn}

%\begin{defn}\label{def:connection}
%{\em
%A {\em connection fibration} is a chain fibration $f:C\to L$ such that every center subcomplex has a trivial boundary map.  For each $p$ we have that $(C_p,\partial_p)$ the center subcomplex at $p$ has $\partial_p \equiv 0$.
%}
%\end{defn}

%
%
%
%\begin{rem}
%Here's an alternate definition of connection fibration that may be easier: $f$ is a {\em connection fibration} if  $f:C\to L$ such that for all $x\in C$ we have $f\partial x < f x$.
%\end{rem}





 
