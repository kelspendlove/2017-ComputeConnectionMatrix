%!TEX root = ../main.tex

\section{Persistent Homology}\label{sec:PH}

Let $(T,\leq)=\{0,1,\ldots, n\}\subseteq \N$ be the total order with order $\leq$ inherited from $\N$.  Let $\sT=\sO(T)$.  Consider a $\sT$-filtered complex $h\colon \sT\to Sub(C,d)$.  


Let $(C_a,d_a) = h(a),d|_{h(a)}$.  Then $Z_a= \ker d_a$ is a subcomplex of $C$.  Another way to describe this is as $Z_a = Z\cap h(a)$ where $Z=\ker d$.   Let $B_a = B\cap h(a)$ where $B=im(d)$.  This is $B_a = im(d_a)$.  

{\bf Some intro to persistence, typically think of as functor and calculate ranks}


Since $\sT$ is a total order, we have chains 
\[
B_0\subseteq B_1\subseteq \ldots \subseteq  B_n\quad \quad\text{and} \quad\quad Z_0\subseteq Z_1 \subseteq \ldots \subseteq Z_n
\]
subject to the condition $B_i\subseteq Z_i$.  These generate a distributive lattice of cyclic subcomplexes of the form 
\[
Z_i\wedge B_j \quad\text{ and }\quad Z_i \vee B_j\quad\quad \text{ where $i<j$}
\]






Thus for $i<j$ we may define the $(i,j)$ persistent homology subcomplex as the subquotient

\[
PH_{i,j}= \frac{Z_i}{Z_i\wedge B_j} \cong \frac{Z_i\vee B_j}{B_j}
\]

Since these are cyclic quotient complexes, we may define the $p$th persistent homology group of $(i,j)$ as the $p$th chain group of the subquotient complex.

One is often interested in the classes that live from $i$ to and die at $j$.  This can be calculated as from the sequence:
\[
\xymatrixrowsep{0.03in}
\xymatrixcolsep{0.45in}
\xymatrix{
Z_i/B_i \ar[r]^{i^{i,j-1}} & Z_{j-1}/B_{j-1} \ar[r]^{i^{j-1,j}} & Z_j/B_j
}
\]

The image of the first map intersect with the kernel of the second is computed as:
\[
\frac{Z_i\wedge B_j}{Z_i\wedge B_{j-1}} \cong \frac{(Z_i\wedge B_j)\vee B_{j-1}}{B_{j-1}}
\]



The {\em (i,j) quotient complex $Q_{i,j}$} is defined as 
\[
\frac{\frac{Z_i\wedge B_j}{Z_i\wedge B_{j-1}} }{\frac{Z_{i-1}\wedge B_j}{Z_{i-1}\wedge B_{j-1}}} \quad\quad\text{ or }\quad\quad \frac{\frac{(Z_i\wedge B_j)\vee B_{j-1}}{B_{j-1}} } {\frac{(Z_{i-1}\wedge B_j)\vee B_{j-1}}{B_{j-1}}}
\]

By the second isomorphism theorem applied to the right hand quotient, we have 
\[
Q_{i,j} \cong \frac{(Z_i\wedge B_j)\vee B_{j-1} } {(Z_{i-1}\wedge B_j)\vee B_{j-1}}
\]

%\[
%Q_{i,j} = \frac{PH_{i,j} \wedge Z_{j-1}\wedge B_j}{PH_{i-1,j-1}\wedge Z_{j-1}\wedge B_j} = \frac{\frac{Z_i}{Z_i\wedge B_{j-1}} \wedge Z_{j-1}\wedge B_j}{\frac{Z_{i-1}}{Z_{i-1}\wedge B_{j-1}}\wedge Z_{j-1}\wedge B_j}
%\]
%\[
%\frac{\frac{Z_i}{Z_i \wedge B_{j-1}} \wedge Z_{j-1}\wedge B_j}{\frac{Z_{i-1}{Z_{i-1}\wedge B_{j-1}} \wedge Z_{j-1}\wedge B_j}
%\]

For $Q_{i,j}$ the integer $\mu^{i,j}_p$ defined as the rank of the $p$-th chain group is the number of independent $p$-dimensional classes that are born at $C_i$ and die entering $C_j$.  We can calculate $\mu_p^{i,j}$ as 
\[
\mu_p^{i,j} = 
\]


\begin{prop}
$\sL$-filtered chain homotopy equivalence induces isomorphisms of all persistent homology groups.
\end{prop}

\subsection{Computing Persistence From a Tower of Contractions}


There are algorithms for computing persistent homology based on discrete Morse theory, these can be found in~\cite{pawel,real}.  At each level of the filtration, one gets a splitting homotopy.  One matches 

{\bf Algorithm}
\begin{enumerate}
\item Given a filtration $f:\cX\to \sT$ where $\sT = \{0,\ldots, n\}$
\item {\bf for i=0\ldots n do}
\item Find a matching $\cA_i,w_i:\cQ_i\to \cK_i$ using Algorithm that matches king to highest queen in boundary
\item Apply~\cite[Algorithm 3.12]{focm} to produce a graded splitting homotopy $\Gamma:C^\oplus(\sP)\to C^\oplus(\sP)$ and graded chain contraction
\[
\xymatrixrowsep{0.03in}
\xymatrixcolsep{0.3in}
\xymatrix{
M_i(\sP)  \ar[r]<3pt>^{\Phi} & \ar[l]<3pt>^{\Psi} M_{i-1}^\oplus(\sP) \ar@(ul,ur)^{\Gamma}
}
\]
%\item Apply~\cite[Algorithm 3.6]{focm} to the fibers $\{X_q\colon X_q = f^{-1}(q)\}$ to produce a graded acyclic partial matching $(\cA,w:\cQ\to \cK)$

\end{enumerate}


{\bf review algorithm}.  

With our previous results, this implies we can cook up a contraction from the tower.  The left hand side the homology, i.e. generators with infinite generators.


\[
\xymatrixrowsep{0.03in}
\xymatrixcolsep{0.3in}
\xymatrix{
P  \ar[r]<3pt>^{i} & \ar[l]<3pt>^{\pi} C \ar@(u,r)^{h}
}
\]
 



For lattices that are not totally ordered we may do the same for any $a< b$ via 
\[
\frac{Z_a}{Z_a\wedge B_b} \cong \frac{Z_a\vee B_b}{B_b}
\]

It is straightforward that for any filtered homotopy equivalence we have isomorphisms on the persistent homology groups.

\begin{prop}
If $f\cong g$ in ${\bf KLFC}$ then $f$ and $g$ have the same persistent homology.
\end{prop}


If $\sL$ is totally ordered we may apply the persistence equivalence theorem of~\cite{} to get that their diagrams are the same.

\begin{cor}
Same persistence diagram.
\end{cor}

In fact, there should be a category that holds all persistent homology groups and morphisms between them, see Zeeman's work.  Filtered homotopy equivalence should imply isomorphism of categories.

%
%
%Consider $h\colon \sL\to Sub(C,d)$.  
%
%
%Let $\sT=\{0,\ldots, |\sP|\}\subseteq \N$ with order $\leq$ inherited from $(\N,\leq)$.  A {\em linear extension of $\sP$} is a bijective poset morphism $\sP\to \sT$.  By Birkhoff's theorem the composition $X\xrightarrow{f}\sP\to \sT$ induces a lattice morphism $\sO(f):\sO(\sT)\to Sub_{Cl}(X,\leq)$.  The map $\sO(f)$ is called a {\em filtration} as $\sO(\sT)$ is of the form
%\[
%\ldots \subseteq [0,n] \subseteq [0,n+1]\subseteq \ldots
%\]
%
%and $im\sO(f)$ 
%\[
%\ldots \subseteq X^n \subseteq X^{n+1}\subseteq \ldots
%\]
%
%The inclusion $\iota:X$