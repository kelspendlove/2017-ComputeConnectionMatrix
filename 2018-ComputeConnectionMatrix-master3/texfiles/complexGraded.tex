%!TEX root = ../main.tex


\section{$\sP$-Graded Complexes}\label{sec:grad}

In Section~\ref{sec:grad} and \ref{sec:lfc} we'll introduce objects which are a marriage between homological algebra and order theory.  These a straightforward reformulation of the ideas within~\cite[Section 7]{robbin:salamon2}.  In particular we use categorical language, and explicitly use an appropriate homotopy category for connection matrix theory over fields.    The relationship between posets and lattices is encapsulated by Birkhoff's theorem, which is recalled in Section~\ref{sec:birkhoff}.  This relationship is reflected in the homological algebra of this section.  Namely, in the difference between the concepts of a {\em lattice-filtered} object and a {\em poset-graded} chain complex.  

We begin with the concepts needed for applications.  In applications, data often come in the form of a cell complex $X$ graded by a partial order $(\sP,\leq)$.  This is codified in terms of a map $f:X\to (\sP,\leq)$ which restricts to a poset morphism $f:(X,\preceq)\to (\sP,\leq)$.   One can see how these structures arise in~\cite{braids}.  

\begin{defn}
{\em
Let $X$ be a cell complex.  Let $\sL\in FDLat$.  A {\em $J(\sL)$-graded cell complex} is a map $f:X\to (J(L),\leq)$ such that $f:(X,\preceq)\to (J(L),\leq)$ is order preserving.  We call the map $f$ the {\em valuation}.
}
\end{defn}


Let $Cell(J(\sL))$ be the collection of $J(\sL)$-graded cell complexes.  

\begin{ex}
In applications one often begins with a cell complex $X$ and function $\tilde f:X^n\to \R$.  For instance, in imaging data one may have a two dimensional cubical complex with greyscale values on the 2-cells. Let $\sT=f(X^n)$ with the inherited order from $\R$.  $(\sT,\leq)$ is a finite poset since $X^n$ is finite.  We may extend $\tilde f$ to a graded cell complex $f:X\to \sT$ via
\[
X\ni \sigma \rightsquigarrow \min\{f(\eta): \eta\in star(\sigma)\cap X^n\} \in \sT
\]
The fact that $f$ is a poset morphism follows from the observation that if $\sigma \leq \tau$ then $star(\tau)\subseteq star(\sigma)$.   As $f:X\to \sT$ is a $\sT$-graded cell complex we may consider the Birkhoff transform $\sO(f)\colon \sO(\sT)\to Sub_{Cl}(X,\leq)$.  Since $\sT$ is totally ordered, the image $\sO(f)(\sT)$ is a filtration of $X$.
\end{ex}

%For a $J(L)$-graded cell complex $f:X\to \sP$ we have $X=\bigsqcup_{q\in \sP} X_q$ where $X_q=f^{-1}(q)$.  
%
%\begin{prop}
%There is an assignment $Cell(\sJ(\sL))\to GCC(\sJ(\sL))$.  A $\sJ(\sL)$-graded cell complex $f:X\to \sJ(\sL)$ determines an associated $\sJ(\sL)$-graded chain complex $C^\oplus(\sJ(\sL)),\partial)$ $$C(X) = \bigoplus_{q\in J(L)} C(X_q)$$
%
%\end{prop}


\subsection{$\sP$-Graded Chain Complexes}


A $\sP$-graded chain complex is a chain complex $(C,\Delta)$ with decomposition $$C=\bigoplus_{q\in \sP} C_q$$ and boundary operator determined by its components $\Delta_{qp}:C_q\to C_p$ subject to the condition 
\begin{align}\label{eqn:ut}
\Delta_{qp}\neq 0\implies p\leq q
\end{align}  The collection $\{\Delta_{qp}\}$ can be thought of as a matrix of linear maps. If a linear map satisfies condition~(\ref{eqn:ut}) we call it {\em $\sP$-graded}.  In~\cite{fran} this condition is referred to as {\em upper triangularity with respect to $\sP$}.   If $\Delta_{pp} = 0$ for all $p$ then the boundary map $\Delta$ is called a {\em connection matrix}.  In this case $\Delta$ can be interpreted as a map
\begin{align}\label{eqn:cm}
\Delta \colon \bigoplus_{q\in \sP} H_\bullet(C_q,\Delta_{qq})\to \bigoplus_{q\in \sP} H_\bullet(C_q,\Delta_{qq})
\end{align}

The identification of $\Delta$ with the matrix structure is the genesis of the phrase {\em connection matrix}.  We denote $\sP$-graded chain complexes by $(C^\oplus(\sP),\Delta)$ and often abbreviate this with $C^\oplus(P)=(C^\oplus(P),\Delta(P))$.  If $\Delta$ is a connection matrix for $(C^\oplus(\sP),\Delta)$ then we say $C^\oplus(\sP)$ is a {\em Conley complex}.  Equation~\ref{eqn:cm} implies that $\Delta$ is boundary map on Conley indices.  Therefore the notion of Conley complex is analogous to that of a Morse complex.


\begin{ex}
For a $J(L)$-graded cell complex $f:X\to \sP$ we have $X=\bigsqcup_{q\in \sP} X_q$ where $X_q=f^{-1}(q)$.   This implies a $\sJ(\sL)$-graded cell complex $f:X\to \sJ(\sL)$ determines an associated $\sJ(\sL)$-graded chain complex $C^\oplus(\sJ(\sL)),\partial)$ 
\[
C(X) = \bigoplus_{q\in J(L)} C(X_q)
\]
Therefore there is an assignment $\cC:Cell(\sJ(\sL))\to {\bf GCC}(\sJ(\sL))$. 
\end{ex}

\begin{ex}
Let $M$ be a closed manifold and $\varphi:M\times \R\to M$ a Morse-Smale gradient flow.  The set $\sP$ of fixed points are partially ordered by the flow and there is a poset morphism $\mu\colon \sP\to \N$ which assigns each $p$ its Morse index, i.e. the dimensionality of its unstable manifold.   The associated Morse-Witten complex may be written $$C_M = \bigoplus_{p\in \sP} C_p$$ where $C_p$ is the chain complex which is all $0$ except a copy of $k$ in the $\mu(p)$ position: $$\ldots \to 0 \to k_{\mu(p)}\to 0\to \ldots$$

The boundary map $\Delta$ is defined using trajectories and is there $\sP$-graded. In particular, when $k=\Z_2$ the entry $\Delta_{qp}$ counts the number of flow lines from $q$ to $p$ modulo two.  The homology $H_\bullet(C_M)$ is isomorphic to the singular homology of $M$.
\end{ex}


A morphism of $\sP$-graded chain complexes is a $\sP$-graded chain map.   We call the category of $\sP$-graded chain complexes $\bf{GCC(\sP)}$.   Two $\sP$-graded chain maps $\Phi,\Psi:(C^\oplus(\sP),\Delta)\to (D^\oplus(\sP),\Delta')$ are {\em chain homotopic} if there  is a $\sP$-graded degree+1 map $\Gamma:C^\oplus(\sP)\to D^\oplus(\sP)$ such that 
\[
\Phi-\Psi = \Gamma\Delta + \Delta'\Gamma
\]
 The map $\Gamma$ is called a {\em $\sP$-graded chain homotopy}.  We write $\Psi\sim \Phi$ if $\Psi$ and $\Phi$ are graded chain homotopic.  It is straightforward that this is an equivalence relation.
 
 \begin{defn}
 {\em
Let $(C^\oplus(\sP),\Delta)$ and $(D^\oplus(\sP),\Delta')$ be $\sP$-graded complexes.  Let $\Phi:C^\oplus(\sP)\to D^\oplus(\sP)$ and $\Psi:D^\oplus(\sP)\to C^\oplus(\sP)$.  A {\em graded chain homotopy equivalence} consists of a quadruple $(\Phi,\Psi,\Gamma,\Sigma)$ such that
\begin{enumerate}
\item $\Psi\Phi - id_C = \Gamma\Delta + \Delta \Gamma$
\item $\Phi\Psi - id_D = \Sigma\Delta' + \Delta'\Sigma$
\end{enumerate}
}
 \end{defn}
 
 
 The associated associated homotopy category ${\bf KGCC(\sP)}$ has $\sP$-graded chain complexes as objects.  The morphisms are given by the homotopy equivalence classes, i.e. $$Hom_{\bf KGCC}(C^\oplus(\sP), D^\oplus(\sP)) = Hom_{\bf GCC}(C^\oplus(\sP),D^\oplus(\sP))\slash\sim$$ where $\sim$ is the graded homotopy equivalence relation.  Isomorphisms in ${\bf KGCC(\sP)}$ correspond to graded chain equivalences.  For a subset $I\subseteq \sP$ we set 
\[
C^\oplus(I) = \bigoplus_{p\in I} C_p\quad\quad \quad \Delta(I) = \pi_{P,I} \circ \Delta \circ \iota_{I,P}
\]
 where $\iota_{I,P}:C^\oplus(I)\to C^\oplus(P)$ and $\pi_{P,I}:C^\oplus(P)\to C^\oplus(I)$ are the natural inclusion and projection.  In general $C^\oplus(I)$ is neither a subcomplex nor a chain complex.  When $I$ is convex in $\sP$ it is straightforward that $\Delta(I)\circ\Delta(I) = 0$ and $(C^\oplus(I),\Delta(I)$ is a chain complex.  In fact, it may be regarded as $(\sI,\leq)$-graded chain complex where $(\sI,\leq)$ is the restriction of $(\sP,\leq)$ to $\sI$.   
 
  For each lower set $a\in \sO(\sP)$ we have $\Delta(C^\oplus(a))\subseteq C^\oplus(a)$ is $\Delta$ is $\sP$-graded.  Thus $\Delta(a)=\Delta|_{C^\oplus(a)} $. This implies that $(C^\oplus(a),\Delta(a))$ is a subcomplex of $(C^\oplus(\sP),\Delta)$.    
  
  
%  \myline
%  
%  
%  Every $\sP$-graded chain complex $(C^\oplus(\sP),\Delta)$ determines an $\sL$-filtered chain complex $h:\sO(\sP)\to Sub(C,\Delta)$ via $$\sO(\sP)\ni a \rightsquigarrow (C^\oplus(a),\Delta(a))\in Sub(C,\Delta)$$
%
%
%A morphism of $\sP$-graded chain complexes induces a morphism between the associated lattice-filtered complexes.  This is captured as a corollary of the next result, whose proof is straightforward.
%
%
%\begin{prop}
%There is a functor $\mathfrak{L}\colon {\bf GCC}(\sJ(\sL))\to \bCF(\sL)$.
%\end{prop}
%
%\begin{prop}
%If $(C^\oplus(\sJ(\sL)),\Delta)$ is a Conley complex then $\mathfrak{L}(C^\oplus(\sJ(\sL)),\Delta)$ is a Conley-filtering.
%\end{prop}
%
%
%
%
%One can reverse the process and construct a $\sJ(\sL)$-graded chain complex from an $\sL$-filtered chain complex.  However such constructions are typically not unique nor functorial.
%
%
%\begin{defn}
%{\em
%Consider $f:\sL\to Sub(C,d)$.  A {\em graded basis for $h$} is a collection $\cB\subseteq C$ with a map $\nu:\cB\to \sJ(\sL)$ such that for each $q\in \sO(\sJ(\sL))$ the set $\nu^{-1}(q)$ is a basis for $f(q)$.
% The function $\nu$ is called the {\em valuation}. As $\sO(\sJ(\sL)) \cong \sL$ we'll often refer to $\nu^{-1}(q)$ for $q\in \sL$.  
%}
%\end{defn}
%
%The next proposition shows that for $\sL$-filtered chain complexes over fields, one can always find a graded basis.  Note that distributivity of $\sL$ is a necessary ingredient in this result.
%
%\begin{prop}\label{prop:bases}
%For any $\sL$-filtered complex $f:\sL\to Sub(C,d)$ there is a graded basis $(\cB,\nu)$.
%\end{prop}
%\begin{proof}
%We first construct subspaces associated to each $q\in J(L)$, then we may select a basis for these subspaces.  Let $q\in J(L)$.  We have $Pred(q)=\bigvee_i p_i$.  Thus $f(Pred(q)) = f(p_1)+\ldots + f(p_n)$.    Choose a subspace $V_q$ such that $f(q) = V_q \oplus f(Pred(q))$. Notice that for any $q$ that covers $0_L$ we have $Pred(q)=0_L$, so $f(Pred(q)) = 0$ and $V_q = f(q)$.
%
%We'll show that $V_q\cap V_p=0$ for $q\neq p$.  Let $x\in V_q\cap V_p$.  Then $x\in f(q)\cap f(p) = f(q\wedge p)$.  However, $f(q\wedge p)\subseteq f(Pred(q))$ and $f(p\wedge p)\subseteq f(Pred(p))$.  Thus $x=0$ by choice of $V_q$ and $V_p$.  
%
%Choose a basis $\cB_q$ for each $V_q$.  Since $V_q\cap V_p=0$ for $p\neq q$ these bases are pairwise disjoint. Let $\cB=\bigsqcup_q \cB_q$ and define $\nu:\cB\to J(L)$ by $\nu(x) = q$ if $x\in \cB_q$.
%
%
%  It remains to show that for any $a\in \sL$ the set $\nu^{-1}(a) = \bigsqcup_{q\in a} \cB_q $ is a basis for $f(q)$.  We'll do this using an inductive argument.  Let $q\in \sL$ and assume that for each $p<q$ we have $\nu^{-1}(p)$ is a basis for $f(p)$.  Let $x\in f(q)$.  Since $f(q) = V_q\oplus f(Pred(q))$ we have $x = x_q+x_{Pred(q)}$.  $V_q$ is spanned by $\cB_q$.  We have $Pred(q) = \bigvee_i p_i$ with $p_i\in \sJ(\sL)$.  Thus $f(Pred(q)) = f(p_1) + \ldots + f(p_n)$.  
%  
%  
%%   It suffices to show that it spans.  Let $x\in f(q)$.  
%%  
%%  
%%  Now we'll argue that $\bigoplus_{q\in J(L)} V_q$ span $C$.
%%
%%Now we wish to show that $\bigoplus_{q\in J(L)} V_q$ span $C$.  We will prove this by strong induction.  We will induct over a linear extension of $L$.  The base case is to consider the minimal element, $0_L\in L$.  By (4) of~\ref{def:cf} if $f(x)=0_L$ then $x=0$, which is in the span.   Now fix $p\in L$.  The strong inductive hypothesis is to assume that for any $q< p$ any $x$ with $f(x)=q$ is in span $\bigoplus_{q\in J(L)} V_q$.  Let $p\in L$.  By Lemma~\ref{lem:join} we may write $p$ as an irredundant join $p=\bigvee_i q_i$ with $q_i \in J(L)$. Notice if $p\in J(L)$ then the decomposition is trivially written as $p=p$.  Coherence implies that $D_p = D_{q_1}+D_{q_2}+\ldots+D_{q_n}$.  Thus $x= \sum_i \lambda_i x_{q_i}$.  There are two cases.  First, if $p\not\in J(L)$, then $q_i< p$ and each $x_{q_i}$ belongs to the span.  For the second case, $p\in J(L)$ and we may write $D_p = V_p \bigoplus D_{Pred(p)}$.  Thus $x = v_p + x_{Pred(p)}$.  Since $Pred(p)<p$ the inductive hypothesis implies that $x_{Pred(p)}$ is in the span. 
%\end{proof}
%
%
%\begin{cor}
%\label{prop:Lsplitting}
%Any $\sL$-filtered complex $f\colon \sL\to Sub(C,\partial)$ determines a $\sJ(\sL)$-graded chain complex $(C^\oplus(\sJ(\sL)),\Delta)$  such that $$f(a)= (C^\oplus(a),\Delta(a))$$
%
%and in particular for $a\leq b$ we have $f(b)/f(a) \cong (C^\oplus(b-a),\Delta(b-a))$.
%
%%\begin{enumerate}
%%\item for any $a\in L$ $$\bigoplus_{q\leq a} C_q \cong f(a)$$
%%\end{enumerate}
%\end{cor}
%
%We call the associated $\sP$-graded complex a $J(L)$-splitting.  In general splitting is not unique, and depend upon the choice of graded basis.  
%
%We highlight the following simple objects of $\bCF$. In the filtered case, this is the analogue of the connection matrix.
%
%\begin{defn}
%{\em 
%An $\sL$-filtering $f\colon \sL \to Sub(C,\partial)$ is a \em{Conley filtering} if
%\[
%\partial(f(a)) \subseteq f(Pred(a))
%\]
%for all $a\in J(\sL)$.
%}
%\end{defn}
%
% These are central objects to the Conley theory, as with regard to the $J(L)$-splitting, the boundary map can be interpreted as a map on the direct sum of homology.
%
%\begin{cor}
%Let $f\colon \sL \to Sub(C,\partial)$  be a Conley filtering. Then $f$ determines a $\sJ(\sL)$-graded chain complex $(C^\oplus(\sJ(\sL)),\Delta)$ such that 
%\[
%H_\bullet(f(a)/f(Pred(a))) \cong C^\oplus(a-Pred(a))
%\]
%\end{cor}
%
%In this case we may interpret $\Delta$ as an upper triangular boundary map on homology
%\[
%\Delta \colon \bigoplus_{a\in \sJ(\sL)} H(f(a)/f(Pred(a)) \to \bigoplus_{a\in \sJ(\sL)} H(f(a)/f(Pred(a))
%\]
%
%
%\begin{thm}
%Let $f:X\to \sJ(\sL)$ be a graded cell complex and $\cL(f)$ be the associated $\sL$-filtered chain complex..  Let $(C^\oplus(\sJ(\sL)),\Delta)$ be the associated $\sJ(\sL)$-graded chain complex.  If $(A^\oplus(\sJ(\sL)),\Delta_A)$ is a Conley complex which is homotopy-equivalent to $(C^\oplus(\sJ(\sL)),\Delta)$ then $\mathfrak{L}(A^\oplus(\sJ(\sL)),\Delta_A)$ is a Conley filtering which is filtered homotopy equivalent to $\cL(f)$.
%\end{thm}
%
%
%
%








