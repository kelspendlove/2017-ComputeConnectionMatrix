%!TEX root = ../main.tex


\section{Historical Remarks and Context}

Historically, the connection matrix theory was first developed by R. Franzosa in a sequence of papers based on his dissertation, directed by C. Conley~\cite{fran,fran2,fran3}.  Franzosa's treatment uses a {\em chain complex braid} indexed over a poset $\sJ(\sL)$ of join-irreducibles.  The chain complex braid can be understood as a data structure that stores the singular chain data associated to a lattice of attracting blocks. These objects constitute the category $\bCCB(\sJ(\sL))$ and are reviewed in Section~\ref{sec:CMT}.   Graded module braids are data structures for storing the homological information contained in a chain complex braid.  Graded module braids form a category $\bGMB(\sJ(\sL))$ and there is a functor $\mathfrak{H}\colon \bCCB(\sJ(\sL))\to \bGMB(\sJ(\sL))$ which is analogous to a homology functor. The connection matrix theory for maps developed by D. Richeson also uses the structures of chain complex braids and graded module braids~\cite{richeson}.   The contribution of J. Robbin and D. Salamon to the connection matrix theory both addresses maps and merges the theory with order theoretic principles~\cite{robbin:salamon2}.  They introduced the idea of a chain complex being either graded by a poset $\sJ(\sL)$ or filtered by a lattice $\sL$.   These respectively constitute the categories $\bGCC(\sJ(\sL))$ and $\bLFC(\sL)$ and are described in Sections~\ref{sec:grad} and \ref{sec:lfc}.

In this paper, we address how these two approaches fit together.  We emphasize the fact that in applications data come in the form $\sJ(\sL)$-graded cell complex which determines three distinct objects: a $\sJ(\sL)$-graded chain complex, a $\sL$-filtered chain complex and a chain complex braid over $\sJ(\sL)$.   In terms of computations we may think of the following diagram.  %From the data we obtain a $\sJ(\sL)$-graded cell complex which has an associated $\sJ(\sL)$-graded chain complex.   

\begin{align}\label{dia:concept}
\xymatrixrowsep{0.35in}
\xymatrixcolsep{0.45in}
\xymatrix{
& & Cell(\sJ(\sL)) \ar@{-->}[d]^{\cC} \ar@{-->}[ddl]_{\cL} \ar@{-->}[ddr]^{\cB} & &  \\
& & \bGCC(\sJ(\sL)) \ar[dl]_{\mathfrak{L}} \ar[dr]^{\mathfrak{B}} & &  \\
&\bLFC(\sL)  & &  \bCCB(\sJ(\sL))&
}
\end{align}

The dashed arrows are assignments while the solid arrows are functors.  These are described in Sections~\ref{sec:grad}--\ref{sec:CMT}.   In this paper we show that connection matrices may be computed for both $\bLFC(\sL)$ and $\bCCB(\sJ(\sL))$  by utilizing graded discrete-algebraic Morse theory within the category $\bGCC(\sJ(\sL))$.
%
%
%\begin{prop}
%We have the following isomorphism of subcategories:
%\[
%\cL(Cell(\sJ(\sL)))\cong \cC(Cell(\sJ(\sL))) \cong \cB(Cell(\sJ(\sL)))
%\]
%\end{prop}
%
%For computations, we develop the notion of a contraction for graded and filtered categories.  Not for braids since Franzosa does not use homotopy equivalence.


\section{Preliminaries}\label{sec:prelims}

In this section we recall  the necessary mathematical prerequisites.   For a more complete discussion the reader is referred to \cite{davey:priestley, roman} for order theory, \cite{lefschetz, gelfand, weibel} for algebraic topology, and \cite{focm,mn,sko} for discrete Morse theory.


\subsection{Notation}

Boldface font is used to denote categories and Franktur font to denote functors.  Sans-serif font is used for order-theoretic structures, such as posets and lattices.  Capital greek letters are used for graded morphisms (graded chain complexes, chain complex braids) and lower case objects filtered by a lattice (lattice-filtered complexes).  Calligraphic font is used for notation related to cell complexes.


\subsection{Homology}\label{sec:prelims:AT}

Let $k$ be a field.  
A {\em chain complex} $C_\bullet$ of $k$-vector spaces is a family $\{C_n\}_{n\in \Z}$ of vector spaces over $k$ together with linear maps $\partial=\partial_n \colon C_n\to C_{n-1}$, called \emph{boundary maps}, such that $\partial\circ\partial =0$.  
When the context is clear we will abbreviate $C_\bullet$ by $C$. 
A morphism $\phi\colon A\to B$ of chain complexes $A$ and $B$ is a {\em chain map}, that is a family of linear maps $\phi_n \colon A_n\to B_n$ such that $\phi_{n-1}\partial^A = \partial^B \phi_n$. 
Chain complexes and chain maps constitute a category denoted $\bCh(k)$.  A chain complex $(C,d)$ is {\em cyclic} if $d_n=0$ for all $n$.  The cyclic complexes form a subcategory $\bCh_0(k)$.  A chain complex is called {\em acyclic} if it is exact.

A chain complex $B$ is called a {\em subcomplex} of $C$ if each $B_n$ is a subspace of $C_n$ and $\partial^B= \partial^C |_B$ is a boundary map for $B$.    
If $B$ is a subcomplex of $C$, then the quotients $C_n/B_n$ assemble into a chain complex denoted $C/B$ called the {\em quotient complex}.   If $\phi\colon A\to B$ is a chain map then $\ker(f)$ and $im(f)$ are subcomplexes of $A$ and $B$ respectively.

The $n$-th \emph{homology} of $C$ is the quotient $H_n(C):= \ker \partial_n/im \partial_{n+1}$.  The graded vector space $H_\bullet(C) := \{H_n(C)\}_{n\in \Z}$ is the {\em homology} of $C_\bullet$.  Chain maps induce linear maps on homology.  By equipping homology $H_\bullet(C)$ with a set of zero differentials we'll often may regard it as cyclic complex, i.e. an object of $\bCh_0(k)$.   A chain complex $C$ is acyclic if and only if $H_\bullet(C)=0$.   Two chain maps $\phi,\psi \colon A\to B$ are {\em chain homotopic} if there exists degree +1 maps $h_n\colon A_n\to B_{n+1}$ such that 
\[
\phi-\psi = h\partial^A+\partial^Bh.
\]  
Chain homotopic maps induce the same map on homology.    We say that $\phi \colon A\to B$ is a {\em chain homotopy equivalence} if there is a chain map $\psi \colon B\to A$ such that $f\circ g$ and $g\circ f$ are chain homotopic to the respective identity maps of $A$ and $B$.  A {\em chain contraction}, sometimes called {\em contracting homotopy}, is a homotopy $h:C\to C$ such that $id_C = dh+hd$.  A contraction is a homotopy between $id_C$ and the zero morphism, which immediately implies that $C$ is cyclic.

Chain homotopy equivalence is an equivalence relation on $Hom(A,B)$ in ${\bf Ch(k)}$.  
Denoting the set of  such equivalence classes by $Hom_K(A,B)$, we note that $Hom_K(A,B)$ is an abelian group under composition.  
The category $\bK(k)$ whose objects are chain complexes with hom-sets given by $Hom_K(A,B)$ is called the \emph{homotopy category}.  
The isomorphisms in this category are precisely the equivalence classes of the chain homotopy equivalences.  Every chain complex $C$ over a field is chain homotopy equivalent to its homology $H_\bullet (C)$~\cite{weibel}, implying that in the category $\bK(k)$ a chain complex $C$ is isomorphic to its homology.


\subsection{Posets}
A \emph{partial order} $\leq$ is a reflexive, antisymmetric, transitive binary relation.
A set $\sP$ with a partial order is called a \emph{partially ordered set} (poset).  We let $<$ be the relation on $\sP$ such that $x<y$ if and only if $x\leq y$ and $x\neq y$.
A morphism of posets is a map $\nu \colon(\sP,\leq)\to (\sQ,\leq)$ such that if $p\leq q$ then $\nu(p)\leq \nu(q)$.  
We restrict our attention to finite posets, which form the category $\bFPoset$.  

We say that $q$ {\em covers} $p$ if $p\leq q$ and there does not exist an $r$ with $p< r < q$.  
If $q$ covers $p$ then $p$ is a {\em predecessor} of $q$.   An {\em upper set} of $\sP$ is a subset $\sU\subseteq \sP$ such that if $p\in \sU$ and $p\leq q$ then $q\in \sU$.  For $p\in \sP$ the {\em upper set at $p$} is $\uparrow(p):=\{q\in \sP:p \leq q\}$.  The collection of upper sets is denoted by $\sU(\sP)$.
A {\em lower set} of $\sP$ is a set $\sD\subset \sP$ such that if $q\in \sD$ and $p\leq q$ then $p\in \sD$.  The {\em lower set at $q$} is $\downarrow(q):=\{p\in \sP: p \leq q\}$.  We denote the collection of lower sets by $\sO(\sP)$.  Any lower set can be obtained by a union of lower sets of the form $\downarrow(q)$.  In fact, $\sO(\sP)$ are the closed sets of the Alexandroff topology of the poset $\sP$.  Under a poset morphism, the preimage of a lower set is a lower set.  Simiarly, the preimage of an upper set is an upper set.

A subset $\sI\subset \sP$ is an {\em convex set} if $p,q\in \sI, r\in \sP$ and $ p \leq r \leq q$ implies that $r\in \sI$.  Any convex set in $\sP$ can be obtained by an intersection of a lower and upper set.  The collection of convex sets is denoted $I(\sP)$.  For a poset morphism, the preimage of a convex set is a convex set.
 


\subsection{Lattices}

\begin{defn}
{\em
A {\em lattice} is a set $\sL$ with the binary operations $\vee,\wedge \colon \sL\times \sL\to \sL$ satisfying the following axioms:

\begin{enumerate}
\item (idempotent) $a\wedge a = a \vee a = a$ for all $a\in \sL$
\item (commutative) $a\wedge b = b\wedge a$ and $a\vee b = b \vee a$ for all $a,b\in \sL$
\item (associative) $a\wedge (b\wedge c) = (a\wedge b)\wedge c$ and $a\vee(b\vee c) = (a\vee b)\vee c$ for all $a,b,c\in \sL$
\item (absorption $a\wedge (a\vee b) = a\vee (a\wedge b)=a$ for all $a,b\in \sL$

A lattice $\sL$ is {\em distributive} if it satisfies the additional axiom:

\item (distributive) $a\wedge (b\vee c) = (a\wedge b)\vee (a\wedge c)$ and $a\vee (b\wedge c) = (a\vee b) \wedge (a\vee c)$ for all $a,b,c\in \sL$

A lattice $\sL$ is {\em bounded} if there exist elements $0_\sL$ and $1_\sL$ such that

\item $0_\sL\wedge a = 0_\sL, 0_\sL\vee a = a, 1_\sL\wedge a = a, 1_\sL\vee a = 1_\sL$ for all $a\in \sL$
\end{enumerate}
}
\end{defn}

A lattice morphism $f \colon \sL\to \sM$ is a map such that if $a,b\in \sL$ then $f(a\wedge b) = f(a)\wedge f(b)$ and $f(a\vee b) = f(a)\vee f(b)$.  
If $\sL$ and $\sM$ are bounded lattices then we also require that $f(0_\sL)=0_\sM$ and $f(1_\sL)=1_\sM$.    
It is straightforward that every finite lattice is bounded. 
Finite distributive lattices and their morphisms form the category $\bFDLat$.

A lattice $\sL$ has an associated poset structure given by $a\leq b$ if $a=a\wedge b$ or, equivalently, if $b=a\vee b$.
An element $a\in \sL$ is {\em join-irreducible} if it has a unique predecessor.   
The set of join-irreducible elements of $\sL$ is denoted by $\sJ(\sL)$.
The function $Pred \colon \sJ(\sL)\to \sL$ is defined by taking each join-irreducible element to its predecessor.  
A subset $\sK\subset \sL$ is  a sublattice of $\sL$ if $a,b\in \sK$ implies that $a\vee b\in \sK$ and $a\wedge b\in \sK$.  

\begin{ex}
{\em
Let $(C,\partial)$ be a chain complex.  
The associated \emph{subcomplex lattice}, denoted by $\Sub(C,\partial)$, consists of all  subcomplexes of $(C,\partial)$ with the operations $\wedge := \cap$ and $\vee := +$ (span).
$\Sub(C,\partial)$ is a bounded lattice, but  is not distributive in general.
}
\end{ex}

\subsection{Birkhoff Correspondence and Theorems}\label{sec:birkhoff}
As indicated above, given a finite distributive lattice $\sL$,  $\sJ(\sL)$ has a poset structure.
In the opposite direction, given a finite poset $(\sP,\leq)$ the collection of downsets $\sO(\sP)$ is a bounded distributive lattice under $\wedge = \cap$ and $\vee = \cup$.  The following theorem often goes under the moniker `Birkhoff's Representation Theorem'.

\begin{thm}[\cite{davey:priestley}]\label{thm:birkhoff}
$\sJ$ and $\sO$ are contravariant functors from ${\bf FDLat}$ to ${\bf FPoset}$ and ${\bf FPoset}$ to ${\bf FDLat}$, respectively.  We represent this via the following diagram:
\[
\xymatrixrowsep{0.15in}
\xymatrixcolsep{0.15in}
\xymatrix{
\sK \ar[dd]_{h} && & \sJ(\sK) \\
& \ar@{=>}[r]^{\sJ}  &&\\
\sL &&& \sJ(\sL) \ar[uu]_{\sJ(h)}
}\quad\quad
\xymatrix{
\sP \ar[dd]_{\nu} &&& \sO(\sP) \\
&  \ar@{=>}[r]^{\sO} &\\
\sQ &&& \sO(\sQ) \ar[uu]_{\sO(\nu)}
}
\]
The formulas for the morphisms are given by
\begin{align*}
J(h)(a) = \min h^{-1}(\uparrow a),\quad a\in \sJ(\sL)\\
\sO(\nu)(a) = \nu^{-1}(a), \quad a\in \sO(\sQ)
\end{align*}
Furthermore,
\[
\sL\cong \sO(\sJ(\sL))\quad\text{and}\quad \sP\cong \sJ(\sO(\sP)).
\]
\end{thm}

These pair of functors are called the {\em Birkhoff transforms}.   An element $a\in \sO(\sP)$ is a lower set of $\sP$.  Therefore it has a complementary (upper)-set $a^c$ such that $a\cup a^c = \sP$.    For $a\leq b\in \sO(\sP)$ we define $$a-b:=a\cap b^c$$   This is a convex set of $\sP$.  For a general lattice $\sL$ with $a,b\in \sL$ we may define $a-b$ by identifying $\sL$ with $\sO(\sJ(\sL))$ via the Birkhoff transform.  We'll use this operation often as a method of constructing convex sets in $\sP$.  This idea is introduced and axiomized as the {\em Conley form} in~\cite{kmv3}.  

We'll use another theorem, due to Birkhoff, in the Section~\ref{sec:PH} on persistent homology.

\begin{thm}\label{thm:birkhoff:chains}
The free modular lattice $\sM$ generated by two finite chains $\cT,\cT'$ is finite and distributive.  A typical element $X\in \sM$ may be written uniquely in the normal form
\[
X = (T_1\cap T_1') \cup (T_2\cap T_2') \cup\ldots \cup (T_s\cap T_s')
\]
where $T_1\subset T_2\subset \ldots \subset T_s$ in $\sT$ and $T_1'\supset T_2'\supset \ldots \supset T_s'$ in $\sT'$.
\end{thm}


 %The {\em Conley form} is defined as the map $\sO(\sP)\times \sO(\sP)\to \sI(\sP)$ via 
%\[
%(a,b)\rightsquigarrow a-b:=a\cap b^c
%\]

%\subsection{Conley Form}
%
%Let $\sP$ be a poset.  The {\em Conley form} is defined as the map $\sO(\sP)\times \sO(\sP)\to \sI(\sP)$ via 
%\[
%(a,b)\rightsquigarrow a-b:=a\cap b^c
%\]
%
%The Conley form satisfies the following axioms:
%
%\begin{enumerate}
%\item $(a\wedge b)-a = b-a$ and $a - (a\wedge b) = a-b$
%\item $(a\wedge c)-(b\wedge d) = (a-b)\wedge (c-d)$
%\item $1-0 = 1$ and $0-1 = 0$
%\item $a-b = 0$ implies $b\geq a$
%\end{enumerate}
%
%For any $\sL$ in ${\bf FDLat}$ we may define a Conley form on $\sL$ by identifiying $\sL$ with $\sO(\sJ(\sL))$ via the Birkhoff transform.


\subsection{Cell Complexes}\label{sec:prelims:cell}

Since our ultimate focus is on data analysis, we are interested in combinatorial topology.  We make use of the following complex, whose definition is inspired by~\cite{lefschetz}.  Recall that $k$ is a field.

\begin{defn}
\label{defn:cellComplex}
{\em
A {\em cell complex} $(\cX,\preceq,\kappa,\dim)$ is an object $(\cX,\preceq)$ of $\bFPoset$ together with two associated functions $\dim\colon \cX\to \N$ and $\kappa\colon \cX\times \cX\to k$ subject to the following conditions:
\begin{enumerate}
\item $\dim\colon(\cX,\preceq)\to (\N,\leq)$ is a poset morphism;
\item  For each $\xi$ and $\xi'$ in $\cX$:
\[
\kappa(\xi,\xi')\neq 0\quad\text{implies } \xi'\preceq \xi \quad\text{and}\quad \dim(\xi) = \dim(\xi')+1;
\]
\item\label{cond:three} For each $\xi$ and $\xi''$ in $\cX$,
\[
\sum_{\xi'\in X} \kappa(\xi,\xi')\cdot \kappa(\xi',\xi'')=0.
\]
\end{enumerate}
}
\end{defn}

For simplicity we typically write $\cX=(\cX,\preceq,\kappa,\dim)$.  
The partial order $\preceq$ is the {\em face partial order}.
$\cX$ as a graded set with respect to $\dim$, i.e.\ $\cX = \bigsqcup_{n\in \N} \cX^n$ with $\cX^n = \dim^{-1}(n)$.  
An element $\xi\in \cX$ is called a {\em cell} and $\dim \xi$ is the {\em dimension} of $\xi$. 
The function $\kappa$ is the {\em incidence function} of the complex.    The data of $(\kappa,\dim)$ is logically equivalent to a chain complex $C(\cX) = \{C_n(\cX)\}_{n\in\N}$ where $C_n(\cX)$ is the vector space over $k$ with basis elements given by the cells $\xi\in \cX^n$ and the boundary operator is $\partial_n\colon C_n(\cX) \to C_{n-1}(\cX)$ is defined by
\[
\partial_n( \xi) := \sum_{\xi' \in \cX} \kappa(\xi, \xi')\xi'.
\]
Condition~(\ref{cond:three}) of Definition~\ref{defn:cellComplex} is equivalent to $\partial_{n-1}\partial_n = 0$.  For hand computations one often typically finds themselves working the chain complex $C(\cX)$ instead of $\kappa$.  On the other hand, $\kappa$ is used for storage and manipulation via computer.  An important point is that in Definition~\ref{defn:cellComplex} the poset $\leq$ cannot a priori be derived from $\kappa$ or $\partial$.    In many definitions of cell complexes it is derived from the incidence function via the implication
\begin{align}\label{eqn:poset}
\kappa (\xi,\xi')\neq 0 \implies \xi'\leq \xi
\end{align}

A morphism of cell complexes $\cX$ and  $\cX'$ is a chain map with respect to the boundary operators $\partial^\cX$ and $\partial^{\cX'}$ that sends cells to cells.  Cell complexes and their morphisms form a category $Cell$. In general, we find morphisms of cell complexes to restrictive.  Instead, we embed the category of cell complexes in chain complexes via $Cell\hookrightarrow Ch(k)$ and work with morphisms in $Ch(k)$.


Typically the cells we consider have some realization in a topological space.  For instance, see~\cite{braids}.   This becomes a point of consideration for subcomplexes of a cell complex, which are more delicate than for chain complexes.  In general, a {\em subcomplex} of a cell complex $X$ is a subset of $Y\subseteq X$ such that $(Y,\leq)$ is a convex set.  This implies that the incidence function $(Y,\kappa|_Y)$ is again a cell complex. Since cell complexes have an undergirding poset, this warrants a particular type of subcomplex.  We say that $Y$ is a {\em closed subcomplex} of $X$ if it is a lower set in $(X,\leq)$, i.e. basis element of $O(X,\leq)$.   Likewise a {\em closed subcomplex} of the associated chain complex $C(X)$ is a subcomplex with basis an element of $O(\sP)$.  We say that $Y$ is an {\em open subcomplex} of $X$ if it is an upper set in $(X,\leq)$.  Open subcomplexes correspond to quotient complexes of $C(X)$ via $C(Y)\cong C(X)/C(X-Y)$.

\begin{ex}
{\em
Let $(X,\leq,\kappa)$ be a cell complex.  The associated {\em lattice of closed subcomplexes}, denoted by $\Sub_{Cl}(C(X),\partial)$ consists of all closed subcomplexes of $(C(X),\partial)$ with operations $\wedge:= \cap$ and $\vee := +$ (span).  It is straightforward that $O(\sP)$ and $\Sub_{Cl}(C(X))$ are isomorphic.  Therefore this is a distributive lattice.  There is a lattice monomorphism $\Sub_{Cl}(C(X))\to \Sub(C(X),\partial)$.
}
\end{ex}

We define the star and closures: 
\[
star(\xi):= \{\xi': \xi \leq \xi'\}\quad\quad \text{ and } \quad\quad cl(\xi) := \{\xi':\xi'\leq \xi\}
\]

The star defines an open subcomplex while the closure defines a closed subcomplex.  In order-theoretic terms these are the upper and lower set of $(X,\leq)$ at $\xi$.


%This is codified in terms of a map $f:(X,\kappa, \preceq)\to (P,\leq)$ which restricts to a poset morphism $f:(X,\preceq)\to (P,\leq)$.  In practice we think of $P$ as being derived from some lattice $L=O(P)$, i.e. as the poset of join-irreducibles of $L\in {\bf FDLat}$.  

%However, via Birkhoff's theorem one could also regard the lattice as being obtained from the poset.  One can see how these structures arise in~\cite{braids}.


\subsection{Discrete Morse Theory}
We review the use of discrete Morse theory to compute homology of complexes. Our exposition will be brief and will follow~\cite{focm}.  See also~\cite{sko, real}.

\begin{defn}
{\em
An {\em acyclic partial matching} of cell complex $\cX$ consists of a partition of $\cX$ into three sets $\cA$, $\cK$, and $\cQ$ along with a bijection $w:\cQ\to \cK$ such that the following hold:
 \begin{enumerate}
 \item {\em Incidence:} $\kappa(w(Q),Q)\neq 0$
 \item {\em Acyclicity:} the transitive closure of the binary relation $$Q' \ll Q \text{ if and only if } \kappa (w(Q),Q')\neq 0$$
 generates a partial order $\leq$ on $\cQ$.
 \end{enumerate}
 }
 \end{defn} 
 
 Acyclic partial matchings are sometimes called discrete vector fields.  We may lift the partial matching to a degree+1 map $V:C_\bullet (\cX)\to C_{\bullet+1}(\cX)$ by defining it using the distinguished basis:
 \[
 V(x) = 
\begin{cases}
\kappa (\xi, \xi') w(x) & x\in \cQ \\
0 & \text{otherwise}
\end{cases}
 \]
 
 We use an acyclic partial matching $(\cA,w\colon\cQ\to \cK)$ of $\cX$ to construct a new chain complex. This is done through the observation that acyclic partial matchings produce degree+1 maps $C_\bullet(\cX)\to C_{\bullet+1}(\cX)$ called {\em splitting homotopies}.  Splitting homotopies will play a central role in the paper, and we'll review these in depth in Section~\cite{}.  Further references to the use of splitting homotopies within discrete Morse theory can be found in~\cite{sko}.  The following proposition is from~\cite{focm}, however we make a sign change to agree with the exposition in Section~\ref{sec:reductions}.
  
 \begin{prop}[\cite{focm},Proposition 3.9]\label{prop:matchinghomotopy}
An acyclic partial matching $(\cA,w)$ induces a unique linear map $\gamma:C_\bullet(X)\to C_{\bullet+1}(X)$ so that $id_\cX-\partial \gamma$ is canonical with $im\gamma = C(K)$ and $ker\gamma = C(A)\oplus C(K)$.  It is given by the formula
\[
\gamma = \sum_{i\geq 0} V(1-dV)^i
\]
\end{prop}
 
  Let $\iota_\cA:C_\bullet(A)\to C_\bullet(X)$ and $\pi_\cA\colon C_\bullet(X)\to C_\bullet(\cA)$ be the canonical inclusion and projection.  Define $\psi:C_\bullet(X)\to C_\bullet(A), \phi:C_\bullet(A)\to C_\bullet(\cX)$ and $\partial^\cA:C_\bullet(A)\to C_{\bullet-1}(\cA)$ by $$\psi:=\pi_\cA\circ (id_\cX-\partial \gamma) \quad\quad  \phi:= (id_\cX-\gamma \partial)\circ \iota_\cA \quad\quad \partial^\cA:= \psi\circ \partial\circ \phi $$
 
  
 \begin{thm}[\cite{focm}, Theorem 3.10]\label{thm:focm:red}
 $(C_\bullet(\cA),\partial^\cA)$ is a chain complex and $\psi,\phi$ are chain equivalences.  In particular,
 \[
 \psi\circ \phi = id_\cA\quad\quad \phi\circ\psi - id_\cX = \partial\gamma+\gamma\partial
 \]
 \end{thm}
 
 As a corollary $H_\bullet(C_\bullet(\cA))\cong H_\bullet(C_\bullet(\cX))$.  We can now make some remarks on computation.  Acyclic partial matchings are relatively easy to produce, see~[Algorithm 3.6 (Coreduction-based Matching)]\cite{focm}.   Moreover, given an acyclic partial matching there is an efficient algorithm to produce the associated splitting homotopy~\cite[Algorithm 3.12 (Gamma Algorithm)]{focm}.  
 
 
 

