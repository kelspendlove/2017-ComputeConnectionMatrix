\documentclass[11pt, reqno]{amsart}
 \usepackage[margin=1in]{geometry}
\usepackage{amsmath, amssymb, amsthm}
\usepackage[osf]{mathpazo}
\usepackage[euler-digits]{eulervm}
\usepackage{mathrsfs}
% Xy-Pic
\usepackage[all,2cell]{xy}
\UseAllTwocells
% Commutative diagram
\usepackage{pb-diagram}
\usepackage{pb-xy}

\setcounter{tocdepth}{2}
\renewcommand{\mathbf}{\mathbold}
\renewcommand{\P}{\mathbb{P}}
\newcommand{\Q}{\mathbb{Q}}
\newcommand{\sABlock}{{\mathsf{ ABlock}}}

\newcommand{\incar}{\ar@{^{(}->}}
\newcommand{\proar}{\ar@{->>}}
\newcommand{\two}{\Rightarrow}

 \newcommand{\under}{\mathbin{\mkern-3mu/\mkern-6mu/}\mkern-3mu}
	\newcommand{\fiber}{\under}

\usepackage{tikz}
\usetikzlibrary{matrix,arrows}
\usepackage{blkarray}

\usepackage[pdftex]{hyperref}
 \hypersetup{
    colorlinks,%
    citecolor=red,%
    filecolor=black,%
    linkcolor=blue,%
    urlcolor=blue     % can put red here to visualize the links
 }
 \urlstyle{same}

\newcommand{\myline}{\setlength{\parindent}{0in}\rule{\textwidth}{1pt}\setlength{\parindent}{0.1in}}


\setlength{\parindent}{0.1in}

\usepackage[osf]{mathpazo}
\usepackage[euler-digits]{eulervm}
%\usepackage{fourier}
%\usepackage{librebaskerville}
%\usepackage[T1]{fontenc}

\newcommand{\Qint}{\mathbf{Q}_{\text{Int}}}
\newcommand{\Con}{\mathrm{CH}}
\DeclareMathOperator{\Int}{\mathrm{Int}}

\newcommand{\Icat}{\mathrm{\mbox{{\bf Int}}}}

%\newcommand{\bI}{\mathbf{I}}
\newcommand{\bJ}{\mathbf{J}}
\newcommand{\Ch}{\mathrm{\textbf{Ch}}}


\newcommand{\inc}{\hookrightarrow}
\newcommand{\proj}{\twoheadrightarrow}

\newcommand{\SES}{\mathcal{S}}
\newcommand{\LES}{\mathcal{L}}
\newcommand{\CS}{\Con^{\oplus}} %conley-(direct)sum??
\newcommand{\HS}{H}

%\setcounter{tocdepth}{3}

% THEOREMS
\newtheorem{thm}{Theorem}[section]
\newtheorem{lem}[thm]{Lemma}
\newtheorem{defn}[thm]{Definition}
\newtheorem{cor}[thm]{Corollary}
\newtheorem{prop}[thm]{Proposition}
\newtheorem{ex}[thm]{Example}
\newtheorem{rem}[thm]{Remark}
\newtheorem{exer}[thm]{Exercise}
\newtheorem{alg}[thm]{Algorithm}
\newtheorem{com}[thm]{Comment}
\newtheorem{conj}[thm]{Conjecture}
%\newtheorem*{defn*}{Definition}

% COUNTERS
%\renewcommand{\thesection}{\thechapter.\arabic{section}}
%\renewcommand{\theequation}{\thesection.\arabic{theorem}}
%\renewcommand{\theequation}{\thesection.\arabic{equation}}
%\renewcommand{\thefigure}{\thechapter.\arabic{figure}}
%\newcommand\step{\stepcounter{theorem}}



% SET-RELATED MACROS
%\newcommand{\set}[1]{\left| {#1}\right|}
\newcommand{\setof}[1]{\left\{ {#1}\right\}}
\newcommand{\mvmap}{\raisebox{-0.2ex}{$\,\overrightarrow{\to}\,$}}
\newcommand{\setdef}[2]{\left\{{#1}\,\left|\,{#2}\right.\right\}}
\newcommand{\bdf}{\hbox{\rm bd$_f$}}
\newcommand{\supp}[1]{\lfloor {#1}\rfloor}

\newcommand{\cb}{\mbox{\sf{cb}}}
\newcommand{\bd}{\mbox{\sf{bd}}}
\newcommand{\birth}{\text{\sf{b}}}


%BOLD MACROS
\newcommand{\bCCB}{{\bf CCB}}
\newcommand{\bCh}{{\bf Ch}}
\newcommand{\bFDLat}{{\bf FDLat}}

% BOLD LETTERS
\newcommand{\bg}{{\bf g}}
\newcommand{\bh}{{\bf h}}
\newcommand{\bk}{{\bf k}}
\newcommand{\bq}{{\bf q}}
\newcommand{\bu}{{\bf u}}
\newcommand{\bv}{{\bf v}}
\newcommand{\bw}{{\bf w}}
\newcommand{\bx}{{\bf x}}
\newcommand{\by}{{\bf y}}
\newcommand{\bz}{{\bf z}}


\newcommand{\bA}{{\bf A}}
\newcommand{\bB}{{\bf B}}
\newcommand{\bC}{{\bf C}}
\newcommand{\bD}{{\bf D}}
\newcommand{\bG}{{\bf G}}
\newcommand{\bK}{{\bf K}}
\newcommand{\bM}{{\bf M}}
\newcommand{\bR}{{\bf R}}
\newcommand{\bI}{{\bf I}}


% BLACKBOARD BOLD LETTERS
\newcommand{\C}{{\mathbb{C}}}
\newcommand{\D}{{\mathbb{D}}}
\newcommand{\F}{{\mathbb{F}}}
\newcommand{\N}{{\mathbb{N}}}
\newcommand{\R}{{\mathbb{R}}}
\newcommand{\T}{{\mathbb{T}}}
\newcommand{\Z}{{\mathbb{Z}}}


% HOMOTOPY SYMBOL -- CHANGE THIS
\newcommand{\sh}{{\mathsf h}}

% FRAK LETTERS
%\newcommand{\cF}{{\mathfrak{f}}} % FOR MULTIVALUED MAPS

% CALIGRAPHIC LETTERS
\newcommand{\cA}{{\mathcal A}}
\newcommand{\cB}{{\mathcal B}}
\newcommand{\cC}{{\mathcal C}}
\newcommand{\cD}{\mathcal{D}}
\newcommand{\cE}{{\mathcal E}}
\newcommand{\cF}{{\mathcal F}}
\newcommand{\cG}{{\mathcal G}}
\newcommand{\cH}{{\mathcal H}}
\newcommand{\cI}{{\mathcal I}}
\newcommand{\cJ}{{\mathcal J}}
\newcommand{\cK}{{\mathcal K}}
\newcommand{\cL}{{\mathcal L}}
\newcommand{\cM}{{\mathcal M}}
\newcommand{\cN}{{\mathcal N}}
\newcommand{\cO}{{\mathcal O}}
\newcommand{\cP}{{\mathcal P}}
\newcommand{\cQ}{{\mathcal Q}}
\newcommand{\cR}{{\mathcal R}}
\newcommand{\cS}{{\mathcal S}}
\newcommand{\cT}{{\mathcal T}}
\newcommand{\cU}{{\mathcal U}}
\newcommand{\cV}{{\mathcal V}}
\newcommand{\cW}{{\mathcal W}}
\newcommand{\cX}{{\mathcal X}}
\newcommand{\cY}{{\mathcal Y}}
\newcommand{\cZ}{{\mathcal Z}}

\newcommand{\cAKQ}{{\mathcal{AKQ}}}
\newcommand{\wed}{\mathop{w}\nolimits}
\newcommand{\ci}{{\mathsf{i}}}
\newcommand{\cj}{{\mathsf{j}}}

% San Serif LETTERS
\newcommand{\sA}{{\mathsf A}}
\newcommand{\sB}{{\mathsf B}}
\newcommand{\sC}{{\mathsf C}}
\newcommand{\sD}{{\mathsf D}}
\newcommand{\sE}{{\mathsf E}}
\newcommand{\sH}{{\mathsf H}}
\newcommand{\sI}{{\mathsf I}}
\newcommand{\sJ}{{\mathsf J}}
\newcommand{\sK}{{\mathsf K}}
\newcommand{\sL}{{\mathsf L}}
\newcommand{\sM}{{\mathsf M}}
\newcommand{\sN}{{\mathsf N}}
\newcommand{\sP}{{\mathsf P}}
\newcommand{\sQ}{{\mathsf Q}}
\newcommand{\sR}{{\mathsf R}}
\newcommand{\sT}{{\mathsf T}}
\newcommand{\sU}{{\mathsf U}}
\newcommand{\sV}{{\mathsf{V}}}
\newcommand{\sW}{{\mathsf W}}
\newcommand{\sZ}{{\mathsf Z}}


\newcommand{\im}{{\mathsf i}}
\newcommand{\jm}{{\mathsf j}}
\newcommand{\km}{{\mathsf k}}
\newcommand{\lm}{{\mathsf l}}


\newcommand{\fF}{{\mathsf F}}
\newcommand{\fG}{{\mathsf G}}
\newcommand{\fH}{{\mathsf H}}
\newcommand{\fM}{{\mathsf M}}
\newcommand{\fS}{{\mathsf S}}

\newcommand{\bs}{{\mathbf s}}
\newcommand{\be}{{\mathbf e}}

\newcommand{\ind}{{\text{ind}}}
\newcommand{\mul}{{\text{mul}}}


\newcommand{\Que}{{\sf{Que}}}


% MISC CHARACTERS
%\newcommand{\Chi}{\raise .75ex\hbox{$\chi$}}

% MISC COMMANDS
%\newcommand{\proof}{\noindent{\em Proof:$\quad$}}
%\newcommand{\eproof}{\hfill{\vrule height5pt width5pt depth0pt}\medskip}


%  Maps and Arrows

\def\mapright#1{\stackrel{#1}{\longrightarrow}}
\def\smapright#1{\stackrel{#1}{\rightarrow}}

\def\mapleft#1{\stackrel{#1}{\longleftarrow}}
\def\mapdown#1{\Big\downarrow\rlap{$\vcenter{\hbox{$\scriptstyle#1$}}$}}
\def\mapup#1{\Big\uparrow\rlap{$\vcenter{\hbox{$\scriptstyle#1$}}$}}
\def\mapne#1{\nearrow\rlap{$\vcenter{\hbox{$\scriptstyle#1$}}$}}
\def\mapse#1{\searrow\rlap{$\vcenter{\hbox{$\scriptstyle#1$}}$}}
\def\mapsw#1{\swarrow\rlap{$\vcenter{\hbox{$\scriptstyle#1$}}$}}
\def\mapnw#1{\nwarrow\rlap{$\vcenter{\hbox{$\scriptstyle#1$}}$}}

\def\relup#1{\stackrel{#1}{\to}}


\def\setof#1{\left\{{#1}\right\}}
% For <blah>:
\def\ang#1{\langle {#1} \rangle}
% For <blah1, blah2>
\def\ip#1{\left\langle {#1} \right\rangle}
% For math words without mbox
\def\w#1{\mbox{#1}}
% For homology classes?
\def\eqcls#1{\left[#1\right]}

\newcommand{\inv}{^{-1}}
\newcommand{\id}{\text{id}}

% INDEXING MACRO

\newcommand{\symindex}[1]{{\index{#1}}}

% MATH 
%\newcommand{\isdef}{\stackrel{\rm def}{=}}
%\newcommand{\opensubset}{\stackrel{\rm open}{\subset}}
%\newcommand{\conv}{{\hbox{conv}\,}}
%\newcommand{\dist}{{\rm dist}\,}
%\newcommand{\rank}{\hbox{\rm rank}\,}
%\newcommand{\cl}{{\rm cl}\,}%{\text{cl}}
%\newcommand{\interior}{\hbox{\rm int}\,}
%\newcommand{\diam}{{\rm diam}\,}


% geometric
%\newcommand{\cl}{\mathop{\mathrm{cl}}\nolimits}
%\newcommand{\bd}{\mathop{\mathrm{bd}}\nolimits}
%\newcommand{\st}{\mathop{\mathrm{st}}\nolimits}

%algebraic
%\newcommand{\bdy}{\mathop{\mathrm{bdy}}\nolimits}
%\newcommand{\cbdy}{\mathop{\mathrm{cbdy}}\nolimits}
%\newcommand{\cbd}{\mathop{\mathrm{cbd}}\nolimits}  % to be removed


%\newcommand{\Int}{\mathop{\mathrm{int}}\nolimits} 
%\newcommand{\Inv}{\mathop{\mathrm{Inv}}\nolimits}
%\newcommand{\id}{\mathop{\mathrm{id}}\nolimits}
%
%
%\newcommand{\dist}{{\rm dist}}
%\newcommand{\rank}{\hbox{\rm rank}}
%\newcommand{\image}{\mathop{\rm image}}

\newcommand{\interior}{\hbox{\rm int}}
\newcommand{\diam}{{\rm diam}}

% MATH OPERATORS
\DeclareMathOperator{\vol}{vol}
\DeclareMathOperator{\img}{img}
\newcommand{\HG}{{H}}
\DeclareMathOperator{\dist}{d}
\DeclareMathOperator{\Oh}{O}

%\DeclareMathOperator{\diam}{diam}


% HATS

\newcommand{\hxi}{\widehat{\xi}}
\newcommand{\heta}{\widehat{\eta}}
\newcommand{\hbeta}{\widehat{\beta}}
\newcommand{\hzeta}{\widehat{\zeta}}

\newcommand{\higma}{\widehat{\sigma}}
\newcommand{\hphi}{\widehat{\phi}}
\newcommand{\hpsi}{\widehat{\psi}}
\newcommand{\hamma}{\widehat{\gamma}}
\newcommand{\happa}{\widetilde{\kappa}}
\newcommand{\bambda}{{\bf \lambda}}

% MISC
\newcommand{\divs}{{\bf ~|~}}
\newcommand{\pprec}{{\curlyeqprec}}
\newcommand{\mord}{{\lhd~}}

% MISC
%\newcommand{\bh}{{\bf h}}
%\newcommand{\divs}{{\bf ~|~}}
\newcommand{\tab}{\hspace{15pt}}
\newcommand{\Tub}{\text{Tub}}


% Make commands for the fancy quotes


\newenvironment{shadequote}%
{\begin{snugshade}\begin{quote}}
{\hfill\end{quote}\end{snugshade}}

\definecolor{shadecolor}{rgb}{0.8,0.8,0.8}

%\newcommand{\tab}{\hspace*{0.5em}}

\newcommand{\sweetline}{ 
\begin{center}
\nointerlineskip\vspace{-0.1in}
        $\diamond$\hfill\rule{0.90\linewidth}{1.0pt}\hfill$\diamond$
\par\nointerlineskip\vspace{0.1in}
\end{center}}


\title{A Computational Framework for Connection Matrices}
\author{Shaun Harker, Konstantin Mischaikow, Kelly Spendlove}
\date{April 25, 2018}

\begin{document}

\maketitle

\begin{abstract}
The connection matrix is a powerful algebraic topological tool from Conley index theory that captures relationships between isolated invariant sets.  Conley index theory is a topological generalization of Morse theory, in which the connection matrix subsumes the role of the Morse boundary operator.  Over the last few decades, the ideas of Conley has been cast into a purely computational framework~\cite{kmv,lsa,cmdb}.  In this paper we introduce a framework for computing the connection matrix.  This contribution transforms the computational Conley theory into a computational homological theory for dynamical systems.  Within these margins we have two goals:
\begin{enumerate}
\item We introduce a homotopy category which models the connection matrix theory.  We identify objects of this category which correspond to connection matrices and may be computed within the computational Conley theory paradigm.
\item We describe an algorithm for this computation based on discrete/algebraic Morse theory; we advertise publicly available packages in python and C++ which use the algorithm these computations.
\end{enumerate}

\end{abstract}

\myline



%
%!TEX root = ./main.tex




\section{Introduction}\label{sec:intro}


The last few decades have seen the development of computational methods which allow one to obtain rigorous results and computer-assisted proofs in dynamical systems theory~\cite{bl,mm,w}.   One efficacious approach for obtaining rigor with computers has been to refocus the subject of mathematical investigation from invariant sets to objects which are robust under perturbations, such as isolating neighborhoods.  Such objects are the subject of study in the Conley theory~\cite{conley}.  This theory has turned out to be intrinsically combinatorial: most of the continuous structures have their combinatorial analogues.  The combinatorial analogues may be processed and analyzed with graph theory and computational algebraic topology and one can lift the results back to the continuous system.  One of the fundamental results of the theory is that Conley's notions of chain recurrence and Lyapunov functions (primary movers in the proof of his decomposition theorem~\cite{conley}) can be put on algorithmic footing~\cite{bk,kmv}.  Both the global dynamics, in terms of Morse decompositions, and the local homological invariants, in the form of the Conley index, have proven to be efficiently computable~\cite{cmdb,cmdbchaos,kmm}.  The most ambitious of these projects are aimed at capturing rigorous global dynamics across wide ranges of parameters~\cite{cmdb,cmdbchaos,dsgrn}. 

The highest strata of the Conley theory is occupied by the connection matrix.  R. Franzosa first formalized and introduced the connection matrix in a series of papers based on his dissertation under Conley~\cite{fran,fran2,fran3}. Speaking basically, the connection matrix is an algebraic topological tool which codifies the homological Conley theoretic data and their relationships.  It may be thought of as the Conley theoretic generalization of the Morse boundary operator. Therefore, the connection matrix contains information about the structure of both local and global dynamics.  % The connection matrix has been used to prove the existence of invariant sets such as connecting orbits~\cite{dhmo}.  

In this paper we introduce a computational framework for computing the connection matrix within the computational Conley theory.  We introduce the notion of {\em chain fibration} which models the idea of a connection matrix (Section~\ref{sec:cf}).  Moreover, the construction is intimately connected to the ideas of Robbin and Salamon's approach toward dynamics~\cite{salamon}.  In Section~\ref{sec:homotopy} we build a homotopy category for chain fibrations and relate this to Franzosa's original construction.  This relationship implies that Theorem~\ref{thm:exist} can be viewed as an algorithmic existence result which is the analogy Franzosa's existence theorem for connection matrices~\cite[Theorem 3.8]{fran}.   This puts the connection matrix on algorithmic footing, akin to the algorithmic treatment of Conley's decomposition theorem within~\cite{kmv}.

We use discrete Morse theory for computations.  As this is a computational technique, we introduce it with the algorithm in Section~\ref{sec:computation}.  Discrete Morse theory has gained much prominence within applied topology.  As this technique is related to our computations, the connection matrix should prove useful within communities outside of dynamics, such as applied topology and topological data analysis.  We will make comments within the text to indicate such applications.

 That being said, our motivation is dynamics.  The steps taken in this paper are the first along a path toward developing a much larger theory of connection matrices, transition matrices and their computation.  The primary function of the connection matrix is to transform the Conley theoretic data into a homology theory and ultimately our efforts are toward a computational homological theory of dynamical systems.  This paper falls within the push being made toward developing such a computational theory of dynamical systems~\cite{cmdbProject,dsgrnProject}.  

\subsection{Function of a Connection Matrix}

We believe the connection matrix has application both within and beyond computational dynamics.  Here are some different interpretations of its function:

\begin{enumerate}
\item as a data reduction technique {\em at the level of chain complexes}, capable of recovering all homological, persistent homological information (see~\cite{braids} for an application to a Morse theory on braids)
\item a simple/minimal representative of the derived/homotopy category (as implied by Theorems~\ref{thm:exist},~\ref{thm:cfcm}), this is related to (1)
\item as an algebraic model of the dynamics, carrying homological connecting orbit data and able to construct semi-conjugacies of the global attractor~\cite{scalar,dhmo,models}
\item the engine which transforms the computational Conley theory into a truly computational homological theory for dynamical systems

\end{enumerate}

%In the next two subsections we give examples of how one can think of the connection matrix by relationship with Morse theory and persistent homology.

%
%\subsection{Connection Matrix \& Morse Theory}
%
%The connection matrix is a boundary operator on the Conley indices of the isolated invariant sets.  It is the generalization of the Morse boundary operator.  It is the engine that transforms the Conley theory into a homological theory akin to the Morse theory and in the axiomatic sense of Eilenberg-Maclane~\cite{mc,rv,rvII,sal}.
%
%We given a simple example of the connection matrix theory in the setting of a filtration. Consider the filtration $$X_0 \subseteq X_1$$
%
%There is a short exact sequence of chain groups: $$0\to C_\bullet(X_0)\to C_\bullet(X_1)\to C_\bullet(X_1,X_0)\to 0$$
%This gives rise to a long exact sequence relating homology groups: $$\ldots \to H_\bullet(X_0)\to H_\bullet(X_1)\to H_\bullet(X_1,X_0)\xrightarrow{\partial} H_{\bullet-1}(X_0)\to \ldots$$
%
%One can show that $\Delta$ is a boundary operator on $H_\bullet(X_0)\oplus H_\bullet(X_1,X_0)$ where
%\[
%\Delta := \begin{pmatrix}
%0 & \partial\\
%0 & 0
%\end{pmatrix}
%\]
%
%In this case the map $\Delta$ is thought of as a {\em connection matrix}.  It recovers the original homology of $X_1$: $$H_\bullet\big(H_\bullet(X_0)\oplus H_\bullet(X_1,X_0),\Delta\big)\cong H_\bullet(X_1)$$
%
%%The sets $M_i = X_{i+1}\backslash X_i$ are akin to Smale's basic sets, or Conley's Morse sets.  The relative homology $H_\bullet(M_i) \cong H_\bullet(X_{i+1},X_i)$ contains information relating to the stability of basic set.  The connection matrix is a boundary operator $$\Delta:\bigoplus_{1\leq i < n} H_\bullet(X_{i+1},X_i)\to \bigoplus_{1\leq i < n} H_\bullet(X_{i+1},X_i)$$ In its most basic application, the connection matrix can reconstruct the homology of the lower sets, i.e. 
%
%A basic observation is that if $H_\bullet(X_1)\cong H_\bullet(X_0)\oplus H_\bullet(X_1,X_0)$ this forces $\partial\equiv 0$.  The contrapositive being that if $\partial\neq 0$ then $H_\bullet(X_1)$ does not split as such.  Thus there is some `connecting orbit' behavior linking $X_1$ and $X_0$.
%
%
%The connection matrix being the key ingredient in forming a homological theory is relevant in applications.  In has powerful implications for the `Conley database' projects of~\cite{cmdb,bush,cmdbchaos, bm,dsgrn} and it turns these into computational homological theories of dynamical systems.
%
%
%
%
%\subsection{Connection Matrix \& Persistent Homology}
%
%The connection matrix can be used to compute the persistent homology of a filtration.  Consider again a filtration of chain complexes $$X_0\subset X_1 \subset \ldots \subset X_n$$ indexed over a total order $\{0,1,\ldots, n\}$.
%
%One can pass to homology $$H X_0\to H X_1 \to \ldots \to H X_n$$
%
%Such a sequence of vector spaces generalizes to a {\em persistence module}, a functor $\Z\to Vec$.  Persistence modules are understood in terms of their {\em persistent homology}: a a unique decomposition in terms of a direct sum of indecomposable persistence modules guaranteed by a structure theorem.
%
%If one has the connection matrix $\Delta$ for the filtration $X$ one can recover the persistence in the following fashion.  Denote the downset of $k$ by $O(k)$, i.e. $O(k)=\{0,1,\ldots,k\}$.  Let $\Delta_{O(k)}$ be the restriction to $\bigoplus_{j\in O(k)} H(X_j/X_{j-1})$.  Then we can form a filtration of chain complexes $$H(X_0)\hookrightarrow (\bigoplus_{j\in O(1)} H (X_j/X_{j-1}),\Delta_{O(1)}) \hookrightarrow \ldots \hookrightarrow (\bigoplus_{j\in O(n)} (H(X_j/X_{j-1}), \Delta_{O(n)})$$
%
%Denote $H (\Delta_{O(k)}) = H(\bigoplus_{j\in O(k)} H(X_j/X_{j-1)},\Delta_{O(k)})$.  Standard connection matrix theory gives us an isomorphism between the sequences:
%
%\begin{align*}
%\xymatrixrowsep{0.3in}
%\xymatrixcolsep{0.3in}
%\xymatrix{
% H (\Delta_0)  \ar@{->}[r] \ar@{->}[d]_{\simeq}^{\phi_0} & H(\Delta_{O(1)}) \ar@{->}[r] \ar@{->}[d]_{\simeq}^{\phi_{O(1)}} & \ldots  \ar@{->}[r]  & H(\Delta_{O(n)})  \ar@{->}[d]_{\simeq}^{\phi_{O(n)}}  \\
%H(X_0) \ar@{->}[r] & H(X_1) \ar@{->}[r] & \ldots \ar@{->}[r] & H(X_n) 
%}
%\end{align*} 
%
%This extends to an isomorphism of the persistence modules and the Persistence Equivalence Theorem~\cite{} gives us that the persistence diagrams are the same.
%
%




%%
%!TEX root = ./main.tex


\section{Preliminaries}\label{sec:prelims}


In this section we review the necessary mathematical prerequisites.  We begin with the computational Conley paradigm.


\subsection{Computational Dynamics}

In the last few decades an algorithmic approach to Conley's approach to dynamical systems~\cite{conley} has been established~\cite{kmv, cmdb, cmdbchaos}.  This combinatorial-topological framework is central to our motivation for computing the connection matrix.  It has been made clear that one of the most prominent objects of the theory is the lattice of attractors and lattice of attracting blocks~\cite{kmv,lsa,lsa2,salamon}.

Let $f:X\times Z\to X$ be a dynamical system with $X,Z$ compact metric spaces.  We recall the standard pipeline:

\begin{itemize}
\item Select {\em grids} $\cX,\cZ$ on $X$ and $Z$.  In practice $\cX,\cZ$ are cubical complexes.
\item Compute a lattice of subcomplexes which serve as attracting blocks of $f$
\item For each join irreducible, compute a Conley index -  an algebraic topological invariant which provides a coarse measurement of the unstable dynamics associated with $M_\zeta$. This may be done efficiently by interpreting $M_\zeta$ as a subcomplex of $\cX$ and using computational homology 
\item Invoke theorems~\cite{cmdb,cmdbchaos} to lift the computational results to rigorous results for the continuous system $f$ 

%\item Construct an {\em outer approximation} $\cF:\cX\times \cZ\to \cX$ of $f$, i.e. a relation between $\cX\times \cZ$ and $\cX$ such that for all $\xi\in \cX$ and $\zeta\in \cZ$ $$f_{|\zeta|}(|\xi|)\subseteq int |(\cF(\xi,\zeta)|$$
%\item For $\zeta\in \cZ$, $\cF_\zeta := \cF(\cdot, \zeta)$ be decomposed into recurrent and gradient-like parts, in analogy to Conley's decomposition theorem~\cite{conley,kmv}.  This is done by interpreting $\cF_\zeta$ as a directed graph with vertex set $\cX$ and applying Tarjan-like algorithms~\cite{cmdbchaos}
%\item For each recurrent set $M_\zeta$ of $\cF_\zeta$ compute a Conley index - an algebraic topological invariant which provides a coarse measurement of the unstable dynamics associated with $M_\zeta$. This may be done efficiently by interpreting $M_\zeta$ as a subcomplex of $\cX$ and using computational homology algorithms~\cite{cmdbchaos}.
%\item Invoke theorems~\cite{cmdb,cmdbchaos} to lift the computational results to rigorous results for the continuous system $f$
\end{itemize}

As we will show, the lattice of subcomplexes can (via Birkhoff's theorem) be codified in morphism $(X,\preceq)\to (P,\leq)$ between posets, where $(X,\preceq)$ is the face poset of $X$ and $(P,\leq)$ is the set of join-irreducibles.


The connection matrix is a boundary operator on the Conley indices and promotes the Conley theory into a homology theory.  In this setting, an algorithm for the connection matrix promotes the computational Conley theory to a computational homology theory.




\subsection{Algebraic Topology}\label{sec:prelims:AT}

We review some algebraic topology.  This exposition follows~\cite{weibel}.  Let $\F$ be a field.  A {\em chain complex} $C_\bullet$ of $\F$-vector spaces is a family $\{C_n\}_{n\in \N}$ of vector spaces over field $\F$ together with linear maps $\partial=\partial_n:C_n\to C_{n-1}$.  When the context is clear we will abbreviate $C_\bullet$ by $C$.  A morphism $f:A\to B$ is a {\em chain map}, that is a family of linear maps $f_n:A_n\to B_n$ such that $f_{n-1}\partial^A = \partial^B f_n$. Chain complexes and chain maps make up a category denoted $Ch(\F)$.  

A chain complex $B$ is called a {\em subcomplex} of $C$ if each $B_n$ is a subspace of $C_n$ and $\partial(B_n)\subset B_{n-1}$, i.e. that the inclusion map $i:B\to C$ is a chain map.  In this case we may assemble the quotients $C_n/B_n$ into a chain complex denoted $C/B$ called the {\em quotient complex}.   The $n$th homology of $C$ is the quotient $H_n(C):= \ker \partial_n/im \partial_{n+1}$.  The graded vector space $H_\bullet(C) := \{H_n(C)\}_{n\in \N}$ is the {\em homology} of $C_\bullet$.  Chain maps induce linear maps on homology.  A chain map $A\to B$ is a {\em quasi-isomorphism} if the maps $H_n(A)\to H_n(B)$ are all isomorphisms.

Two chain maps $f,g:A\to B$ are {\em chain homotopic} if there exists degree +1 maps $h_n:A_n\to B_{n+1}$ such that $$f-g = h\partial_A+\partial_Bh$$  We say that $f:A\to B$ is a {\em chain homotopy equivalence} if there is a chain map $g:B\to A$ such that $fg$ and $gf$ are chain homotopic to the respective identity maps of $A$ and $B$.  Chain homotopy equivalence is an equivalence relation on $Hom(A,B)$.  The set of such equivalence classes $Hom_K(A,B)$ is an abelian group.  The category $K$ consisting of chain complexes with hom sets given by $Hom_K(A,B)$ is called the homotopy category.  The isomorphisms in this category are precisely the equivalence classes of the chain homotopy equivalences.


The rest of this exposition follows~\cite{focm,mn}.  This particular definition of complex dates back to Lefschetz.

\begin{defn}
{\em
Consider a finite graded set $\cX = \bigsqcup_{q\in \Z} \cX_q$ along with a function $\kappa:\cX\times\cX\to \F$ and denote $\xi\in \cX_q$ by $\dim \xi = q$.  Then $(\cX,\kappa)$ is a {\em complex} if the following hold:
\begin{enumerate}
\item \label{cond:1} For each $\xi$ and $\xi'$ in $\cX$:
$$\kappa(\xi,\xi')\neq 0\quad\text{implies}\quad \dim \xi = \dim \xi'+1$$
\item\label{cond:2} For each $\xi$ and $\xi''$ in $\cX$,
$$\sum_{\xi'\in \cX} \kappa(\xi,\xi')\cdot \kappa(\xi',\xi'')=0$$
\end{enumerate}
}
\end{defn}

An element $\xi\in \cX$ is called a {\em cell} and $\dim \xi$ is the {\em dimension} of $\xi$.  The function $\kappa$ is the {\em incidence function} of the complex $(\cX,\kappa)$.   The {\em face partial order} $\preceq$ is induced on the elements of $\cX$ by the transitive closure of the generating relation $\prec$ given as follows: For $\xi, \xi'\in \cX$ $$\xi' \prec \xi \quad \text{if} \quad \kappa(\xi,\xi')\neq 0$$
Let $(\cX,\kappa)$ be a complex.  The {\em associated chain complex} consists of free vector spaces $C_q(\cX)$ where the basis elements are the cells $\xi \in \cX_q$ and the boundary operator is generated by the maps $$\partial_q \xi := \sum_{\xi' \in \cX} \kappa(\xi, \xi')\xi'$$ It is straightforward to verify that the associated chain complex of a complex is indeed a chain complex.











\subsection{Order Theory}\label{sec:prelims:order}

Order theory is the study of posets and lattices.  Order theory has a strong relationship to both algebraic topology and dynamical systems, e.g.~\cite{salamon,lsa,lsa2}.  An intuition for posets, lattices and their correspondence via Birkhoff's theorem will be very helpful for understanding the paper.


\subsubsection{Posets}

A morphism of posets is a map $h:(P,\leq_P)\to (Q,\leq_Q)$ such that if $p\leq_P q$ then $h(p)\leq_Q h(q)$. Posets and their morphisms form the category {\bf Poset}.

The face poset $(X,\preceq)$ provides a powerful method of thinking about a complex $(X,\kappa)$.  One reason is that subcomplexes of $X$ have a nice characterization in terms of {\em convex sets} of $(X,\preceq)$.   Let $P$ be a poset.  An {\em upper set} of $P$ is a subset $U\subset P$ such that if $p\in U$ and $p\leq q$ then $q\in U$.  For $p\in P$ the {\em upset} at $p$ is $\uparrow(p):=\{q\in P:p \leq q\}$.  A {\em lower set} of $P$ is a set $D\subset P$ such that if $q\in D$ and $p\leq q$ then $p\in D$.  The {\em downset} at $q$ is $\downarrow(q):=\{p\in P: p \leq q\}$.  A subset $I\subset P$ is an {\em convex set} if $p,q\in I, r\in P$ and $ p < r < q$ implies that $r\in I$.  Any convex set in $P$ can be obtained by an intersection of a lower and upper set.  
 
\begin{defn}
{\em
A collection $(I_1,\ldots, I_N)$ of convex sets of $(P,\leq)$ is called {\em adjacent} if
\begin{enumerate}
\item $I_1,\ldots,I_n$ are mutually disjoint
\item $\bigcup_{i=1}^n I_i$ is a convex set in $P$
\item $p\in I_i, q\in I_j, i < j$ imply $q \nless p$
\end{enumerate}
}
\end{defn}

We will be primarily interested in adjacent pair of convex sets $(I,J)$.  We write the union (a convex set) $I\cup J$ as $IJ$.  We will denote the set of convex sets as $I(P)$ and the set of adjacent tuples and triples of convex sets as $I_2(P)$ and $I_3(P)$.  This notation agrees with~\cite{fran}.
%The convex sets are particularly important for complexes:

\begin{prop}\label{prop:subcomplex}
Let $(\cX,\kappa)$ be a complex.  Let $(\cX,\preceq)$ be its face poset.  If $\cX'$ be an convex set in $(\cX,\preceq)$ then $(\cX',\partial_{\cX'})$ is a subcomplex.
\end{prop}
\begin{proof}
We must show that $\partial_{\cX'}\circ \partial_{\cX'}=0$.  If $\xi,\xi''\in \cX'$ and $\xi'' \prec \xi$ then all $\xi'\in \cX$ such that $\xi'' \prec \xi' \prec \xi$ must be in $\cX'$ since $\cX'$ is an interval.  Thus~(\ref{cond:1}) and~(\ref{cond:2}) imply that $\sum_{\xi'\in \cX'} \kappa(\xi, \xi')\cdot \kappa(\xi',\xi'')=0$ which implies $\partial_{\cX'}^2=0$.
\end{proof}

\begin{cor}\label{cor:clsubcomplex}

If $\cX'$ is a lower set then this is a closed subcomplex.

\end{cor}

Posets have a topology known as the Alexandrov topology.  It is easy to see that a map $h:(P,\leq_P)\to (Q,\leq_Q)$ is a morphism of posets if and only if $h$ is continuous with respect to the Alexandrov topologies.


\subsubsection{Lattices}

\begin{defn}
{\em
A {\em lattice} is a set $L$ with the binary operations $\vee,\wedge:L\times L\to L$ satisfying the following axioms:

\begin{enumerate}
\item (idempotent) $a\wedge a = a \vee a = a$ for all $a\in L$
\item (commutative) $a\wedge b = b\wedge a$ and $a\vee b = b \vee a$ for all $a,b\in L$
\item (associative) $a\wedge b(b\wedge c) = (a\wedge b)\wedge c$ and $a\vee(b\vee c) = (a\vee b)\vee c$ for all $a,b,c\in L$
\item (absorption $a\wedge (a\vee b) = a\vee (a\wedge b)=a$ for all $a,b\in L$

A lattice $L$ is {\em distributive} if it satisfies the additional axiom:

\item (distributive) $a\wedge (b\vee c) = (a\wedge b)\vee (a\wedge c)$ and $a\vee (b\wedge c) = (a\vee b) \wedge (a\vee c)$ for all $a,b,c\in L$

A lattice $L$ is {\em bounded} if there exist elements $0_L$ and $1_L$ such that

\item $0_L\wedge a = 0_L, 0_L\vee a = a, 1_L\wedge a = a, 1_L\vee a = 1_L$ for all $a\in L$
\end{enumerate}
}
\end{defn}

A lattice morphism $h:L\to M$ is a map such that if $a,b\in L$ then $f(a\wedge b) = f(a)\wedge f(b)$ and $f(a\vee b) = f(a)\vee f(b)$.  If $L$ and $M$ are bounded lattices then we also require that $f(0_L)=0_M$ and $f(1_L)=1_M$.    Bounded, distributive lattices and their morphisms form the category {\bf BDLat}.

We say that $q$ {\em covers} $p$ if $p\leq q$ and there does not exist an $r$ with $p\leq r \leq q$.  If $q$ covers $p$ then we say $p$ is a {\em predecessor} of $q$.  An element $a\in L$ is {\em join-irreducible} if it has a unique predecessor.   A subset $K\subset L$ is called a sublattice of $L$ if $a,b\in K$ implies that $a\vee b\in K$ and $a\wedge b\in K$.  A lattice $L$ has an associated poset structure given by $a\leq b$ if $a=a\wedge b$ or if $b=a\vee b$.

\begin{defn}
{\em
For a lattice $L$ define $Pred:L\backslash \{0_L\} \to L$ via $Pred(p) = \bigwedge \{q: \text{$p$ covers $q$}\}$, i.e. the meet of the predecessors.
}
\end{defn}

 Notice for join irreducible elements that $Pred$ yields the unique predecessor.

\begin{lem}[\cite{roman}, Theorem 4.29]\label{lem:join}
Let $L$ be a bounded distributive lattice.  Any $p\in L$ can written as the irredundant join of join-irreducibles.
\end{lem}


\subsubsection{Birkhoff's Correspondence}

For a lattice $L$ we denote its join-irreducibles as $J(L)$.  $J(L)$ has a poset structure.  $J$ is a functor $J:{\bf BDLat}\to {\bf Poset}$.  For a poset $(P,\leq)$ we denote set of downsets by $O(P)$. $O(P)$ has the structure of a distributive lattice.  $O$ is a functor $O:{\bf Poset}\to {\bf BDLat}$.  This is formalized via Birkhoff's theorem.  See~\cite{lsa,lsa2,salamon} for a discussion in the context of dynamics.

\begin{thm}[\cite{lsa}]\label{thm:birkhoff}
$J$ and $O$ are contravariant functors and provide an equivalence of categories {\bf Poset} and {\bf BDLat}, i.e. $$L\cong O(J(L))\quad\quad P\cong J(O(P))$$

\end{thm}

It is often the case that lattices are related to algebraic structures via a study of their substructures.  For instance, consider a vector space $V$.  It is straightforward to verify that the collection of subspaces of $V$ forms a bounded lattice under the operations $\cap$ and $+$ (span). However this lattice of subspace is not distributive.  Similarly, for a chain complex $C$, it is again straightforward that the collection of subcomplexes of $C$ form a bounded lattice under the operations $\cap$ and $+$.  We denote this lattice $S(C)$.  Again, note that $S(C)$ is not distributive.










%%
%!TEX root = ../main.tex


\section{$\sP$-Graded Complexes}\label{sec:grad}

In Section~\ref{sec:grad} and \ref{sec:lfc} we'll introduce objects which are a marriage between homological algebra and order theory.  These a straightforward reformulation of the ideas within~\cite[Section 7]{robbin:salamon2}.  In particular we use categorical language, and explicitly use an appropriate homotopy category for connection matrix theory over fields.    The relationship between posets and lattices is encapsulated by Birkhoff's theorem, which is recalled in Section~\ref{sec:birkhoff}.  This relationship is reflected in the homological algebra of this section.  Namely, in the difference between the concepts of a {\em lattice-filtered} object and a {\em poset-graded} chain complex.  

We begin with the concepts needed for applications.  In applications, data often come in the form of a cell complex $X$ graded by a partial order $(\sP,\leq)$.  This is codified in terms of a map $f:X\to (\sP,\leq)$ which restricts to a poset morphism $f:(X,\preceq)\to (\sP,\leq)$.   One can see how these structures arise in~\cite{braids}.  

\begin{defn}
{\em
Let $X$ be a cell complex.  Let $\sL\in FDLat$.  A {\em $J(\sL)$-graded cell complex} is a map $f:X\to (J(L),\leq)$ such that $f:(X,\preceq)\to (J(L),\leq)$ is order preserving.  We call the map $f$ the {\em valuation}.
}
\end{defn}


Let $Cell(J(\sL))$ be the collection of $J(\sL)$-graded cell complexes.  

\begin{ex}
In applications one often begins with a cell complex $X$ and function $\tilde f:X^n\to \R$.  For instance, in imaging data one may have a two dimensional cubical complex with greyscale values on the 2-cells. Let $\sT=f(X^n)$ with the inherited order from $\R$.  $(\sT,\leq)$ is a finite poset since $X^n$ is finite.  We may extend $\tilde f$ to a graded cell complex $f:X\to \sT$ via
\[
X\ni \sigma \rightsquigarrow \min\{f(\eta): \eta\in star(\sigma)\cap X^n\} \in \sT
\]
The fact that $f$ is a poset morphism follows from the observation that if $\sigma \leq \tau$ then $star(\tau)\subseteq star(\sigma)$.   As $f:X\to \sT$ is a $\sT$-graded cell complex we may consider the Birkhoff transform $\sO(f)\colon \sO(\sT)\to Sub_{Cl}(X,\leq)$.  Since $\sT$ is totally ordered, the image $\sO(f)(\sT)$ is a filtration of $X$.
\end{ex}

%For a $J(L)$-graded cell complex $f:X\to \sP$ we have $X=\bigsqcup_{q\in \sP} X_q$ where $X_q=f^{-1}(q)$.  
%
%\begin{prop}
%There is an assignment $Cell(\sJ(\sL))\to GCC(\sJ(\sL))$.  A $\sJ(\sL)$-graded cell complex $f:X\to \sJ(\sL)$ determines an associated $\sJ(\sL)$-graded chain complex $C^\oplus(\sJ(\sL)),\partial)$ $$C(X) = \bigoplus_{q\in J(L)} C(X_q)$$
%
%\end{prop}


\subsection{$\sP$-Graded Chain Complexes}


A $\sP$-graded chain complex is a chain complex $(C,\Delta)$ with decomposition $$C=\bigoplus_{q\in \sP} C_q$$ and boundary operator determined by its components $\Delta_{qp}:C_q\to C_p$ subject to the condition 
\begin{align}\label{eqn:ut}
\Delta_{qp}\neq 0\implies p\leq q
\end{align}  The collection $\{\Delta_{qp}\}$ can be thought of as a matrix of linear maps. If a linear map satisfies condition~(\ref{eqn:ut}) we call it {\em $\sP$-graded}.  In~\cite{fran} this condition is referred to as {\em upper triangularity with respect to $\sP$}.   If $\Delta_{pp} = 0$ for all $p$ then the boundary map $\Delta$ is called a {\em connection matrix}.  In this case $\Delta$ can be interpreted as a map
\begin{align}\label{eqn:cm}
\Delta \colon \bigoplus_{q\in \sP} H_\bullet(C_q,\Delta_{qq})\to \bigoplus_{q\in \sP} H_\bullet(C_q,\Delta_{qq})
\end{align}

The identification of $\Delta$ with the matrix structure is the genesis of the phrase {\em connection matrix}.  We denote $\sP$-graded chain complexes by $(C^\oplus(\sP),\Delta)$ and often abbreviate this with $C^\oplus(P)=(C^\oplus(P),\Delta(P))$.  If $\Delta$ is a connection matrix for $(C^\oplus(\sP),\Delta)$ then we say $C^\oplus(\sP)$ is a {\em Conley complex}.  Equation~\ref{eqn:cm} implies that $\Delta$ is boundary map on Conley indices.  Therefore the notion of Conley complex is analogous to that of a Morse complex.


\begin{ex}
For a $J(L)$-graded cell complex $f:X\to \sP$ we have $X=\bigsqcup_{q\in \sP} X_q$ where $X_q=f^{-1}(q)$.   This implies a $\sJ(\sL)$-graded cell complex $f:X\to \sJ(\sL)$ determines an associated $\sJ(\sL)$-graded chain complex $C^\oplus(\sJ(\sL)),\partial)$ 
\[
C(X) = \bigoplus_{q\in J(L)} C(X_q)
\]
Therefore there is an assignment $\cC:Cell(\sJ(\sL))\to {\bf GCC}(\sJ(\sL))$. 
\end{ex}

\begin{ex}
Let $M$ be a closed manifold and $\varphi:M\times \R\to M$ a Morse-Smale gradient flow.  The set $\sP$ of fixed points are partially ordered by the flow and there is a poset morphism $\mu\colon \sP\to \N$ which assigns each $p$ its Morse index, i.e. the dimensionality of its unstable manifold.   The associated Morse-Witten complex may be written $$C_M = \bigoplus_{p\in \sP} C_p$$ where $C_p$ is the chain complex which is all $0$ except a copy of $k$ in the $\mu(p)$ position: $$\ldots \to 0 \to k_{\mu(p)}\to 0\to \ldots$$

The boundary map $\Delta$ is defined using trajectories and is there $\sP$-graded. In particular, when $k=\Z_2$ the entry $\Delta_{qp}$ counts the number of flow lines from $q$ to $p$ modulo two.  The homology $H_\bullet(C_M)$ is isomorphic to the singular homology of $M$.
\end{ex}


A morphism of $\sP$-graded chain complexes is a $\sP$-graded chain map.   We call the category of $\sP$-graded chain complexes $\bf{GCC(\sP)}$.   Two $\sP$-graded chain maps $\Phi,\Psi:(C^\oplus(\sP),\Delta)\to (D^\oplus(\sP),\Delta')$ are {\em chain homotopic} if there  is a $\sP$-graded degree+1 map $\Gamma:C^\oplus(\sP)\to D^\oplus(\sP)$ such that 
\[
\Phi-\Psi = \Gamma\Delta + \Delta'\Gamma
\]
 The map $\Gamma$ is called a {\em $\sP$-graded chain homotopy}.  We write $\Psi\sim \Phi$ if $\Psi$ and $\Phi$ are graded chain homotopic.  It is straightforward that this is an equivalence relation.
 
 \begin{defn}
 {\em
Let $(C^\oplus(\sP),\Delta)$ and $(D^\oplus(\sP),\Delta')$ be $\sP$-graded complexes.  Let $\Phi:C^\oplus(\sP)\to D^\oplus(\sP)$ and $\Psi:D^\oplus(\sP)\to C^\oplus(\sP)$.  A {\em graded chain homotopy equivalence} consists of a quadruple $(\Phi,\Psi,\Gamma,\Sigma)$ such that
\begin{enumerate}
\item $\Psi\Phi - id_C = \Gamma\Delta + \Delta \Gamma$
\item $\Phi\Psi - id_D = \Sigma\Delta' + \Delta'\Sigma$
\end{enumerate}
}
 \end{defn}
 
 
 The associated associated homotopy category ${\bf KGCC(\sP)}$ has $\sP$-graded chain complexes as objects.  The morphisms are given by the homotopy equivalence classes, i.e. $$Hom_{\bf KGCC}(C^\oplus(\sP), D^\oplus(\sP)) = Hom_{\bf GCC}(C^\oplus(\sP),D^\oplus(\sP))\slash\sim$$ where $\sim$ is the graded homotopy equivalence relation.  Isomorphisms in ${\bf KGCC(\sP)}$ correspond to graded chain equivalences.  For a subset $I\subseteq \sP$ we set 
\[
C^\oplus(I) = \bigoplus_{p\in I} C_p\quad\quad \quad \Delta(I) = \pi_{P,I} \circ \Delta \circ \iota_{I,P}
\]
 where $\iota_{I,P}:C^\oplus(I)\to C^\oplus(P)$ and $\pi_{P,I}:C^\oplus(P)\to C^\oplus(I)$ are the natural inclusion and projection.  In general $C^\oplus(I)$ is neither a subcomplex nor a chain complex.  When $I$ is convex in $\sP$ it is straightforward that $\Delta(I)\circ\Delta(I) = 0$ and $(C^\oplus(I),\Delta(I)$ is a chain complex.  In fact, it may be regarded as $(\sI,\leq)$-graded chain complex where $(\sI,\leq)$ is the restriction of $(\sP,\leq)$ to $\sI$.   
 
  For each lower set $a\in \sO(\sP)$ we have $\Delta(C^\oplus(a))\subseteq C^\oplus(a)$ is $\Delta$ is $\sP$-graded.  Thus $\Delta(a)=\Delta|_{C^\oplus(a)} $. This implies that $(C^\oplus(a),\Delta(a))$ is a subcomplex of $(C^\oplus(\sP),\Delta)$.    
  
  
%  \myline
%  
%  
%  Every $\sP$-graded chain complex $(C^\oplus(\sP),\Delta)$ determines an $\sL$-filtered chain complex $h:\sO(\sP)\to Sub(C,\Delta)$ via $$\sO(\sP)\ni a \rightsquigarrow (C^\oplus(a),\Delta(a))\in Sub(C,\Delta)$$
%
%
%A morphism of $\sP$-graded chain complexes induces a morphism between the associated lattice-filtered complexes.  This is captured as a corollary of the next result, whose proof is straightforward.
%
%
%\begin{prop}
%There is a functor $\mathfrak{L}\colon {\bf GCC}(\sJ(\sL))\to \bCF(\sL)$.
%\end{prop}
%
%\begin{prop}
%If $(C^\oplus(\sJ(\sL)),\Delta)$ is a Conley complex then $\mathfrak{L}(C^\oplus(\sJ(\sL)),\Delta)$ is a Conley-filtering.
%\end{prop}
%
%
%
%
%One can reverse the process and construct a $\sJ(\sL)$-graded chain complex from an $\sL$-filtered chain complex.  However such constructions are typically not unique nor functorial.
%
%
%\begin{defn}
%{\em
%Consider $f:\sL\to Sub(C,d)$.  A {\em graded basis for $h$} is a collection $\cB\subseteq C$ with a map $\nu:\cB\to \sJ(\sL)$ such that for each $q\in \sO(\sJ(\sL))$ the set $\nu^{-1}(q)$ is a basis for $f(q)$.
% The function $\nu$ is called the {\em valuation}. As $\sO(\sJ(\sL)) \cong \sL$ we'll often refer to $\nu^{-1}(q)$ for $q\in \sL$.  
%}
%\end{defn}
%
%The next proposition shows that for $\sL$-filtered chain complexes over fields, one can always find a graded basis.  Note that distributivity of $\sL$ is a necessary ingredient in this result.
%
%\begin{prop}\label{prop:bases}
%For any $\sL$-filtered complex $f:\sL\to Sub(C,d)$ there is a graded basis $(\cB,\nu)$.
%\end{prop}
%\begin{proof}
%We first construct subspaces associated to each $q\in J(L)$, then we may select a basis for these subspaces.  Let $q\in J(L)$.  We have $Pred(q)=\bigvee_i p_i$.  Thus $f(Pred(q)) = f(p_1)+\ldots + f(p_n)$.    Choose a subspace $V_q$ such that $f(q) = V_q \oplus f(Pred(q))$. Notice that for any $q$ that covers $0_L$ we have $Pred(q)=0_L$, so $f(Pred(q)) = 0$ and $V_q = f(q)$.
%
%We'll show that $V_q\cap V_p=0$ for $q\neq p$.  Let $x\in V_q\cap V_p$.  Then $x\in f(q)\cap f(p) = f(q\wedge p)$.  However, $f(q\wedge p)\subseteq f(Pred(q))$ and $f(p\wedge p)\subseteq f(Pred(p))$.  Thus $x=0$ by choice of $V_q$ and $V_p$.  
%
%Choose a basis $\cB_q$ for each $V_q$.  Since $V_q\cap V_p=0$ for $p\neq q$ these bases are pairwise disjoint. Let $\cB=\bigsqcup_q \cB_q$ and define $\nu:\cB\to J(L)$ by $\nu(x) = q$ if $x\in \cB_q$.
%
%
%  It remains to show that for any $a\in \sL$ the set $\nu^{-1}(a) = \bigsqcup_{q\in a} \cB_q $ is a basis for $f(q)$.  We'll do this using an inductive argument.  Let $q\in \sL$ and assume that for each $p<q$ we have $\nu^{-1}(p)$ is a basis for $f(p)$.  Let $x\in f(q)$.  Since $f(q) = V_q\oplus f(Pred(q))$ we have $x = x_q+x_{Pred(q)}$.  $V_q$ is spanned by $\cB_q$.  We have $Pred(q) = \bigvee_i p_i$ with $p_i\in \sJ(\sL)$.  Thus $f(Pred(q)) = f(p_1) + \ldots + f(p_n)$.  
%  
%  
%%   It suffices to show that it spans.  Let $x\in f(q)$.  
%%  
%%  
%%  Now we'll argue that $\bigoplus_{q\in J(L)} V_q$ span $C$.
%%
%%Now we wish to show that $\bigoplus_{q\in J(L)} V_q$ span $C$.  We will prove this by strong induction.  We will induct over a linear extension of $L$.  The base case is to consider the minimal element, $0_L\in L$.  By (4) of~\ref{def:cf} if $f(x)=0_L$ then $x=0$, which is in the span.   Now fix $p\in L$.  The strong inductive hypothesis is to assume that for any $q< p$ any $x$ with $f(x)=q$ is in span $\bigoplus_{q\in J(L)} V_q$.  Let $p\in L$.  By Lemma~\ref{lem:join} we may write $p$ as an irredundant join $p=\bigvee_i q_i$ with $q_i \in J(L)$. Notice if $p\in J(L)$ then the decomposition is trivially written as $p=p$.  Coherence implies that $D_p = D_{q_1}+D_{q_2}+\ldots+D_{q_n}$.  Thus $x= \sum_i \lambda_i x_{q_i}$.  There are two cases.  First, if $p\not\in J(L)$, then $q_i< p$ and each $x_{q_i}$ belongs to the span.  For the second case, $p\in J(L)$ and we may write $D_p = V_p \bigoplus D_{Pred(p)}$.  Thus $x = v_p + x_{Pred(p)}$.  Since $Pred(p)<p$ the inductive hypothesis implies that $x_{Pred(p)}$ is in the span. 
%\end{proof}
%
%
%\begin{cor}
%\label{prop:Lsplitting}
%Any $\sL$-filtered complex $f\colon \sL\to Sub(C,\partial)$ determines a $\sJ(\sL)$-graded chain complex $(C^\oplus(\sJ(\sL)),\Delta)$  such that $$f(a)= (C^\oplus(a),\Delta(a))$$
%
%and in particular for $a\leq b$ we have $f(b)/f(a) \cong (C^\oplus(b-a),\Delta(b-a))$.
%
%%\begin{enumerate}
%%\item for any $a\in L$ $$\bigoplus_{q\leq a} C_q \cong f(a)$$
%%\end{enumerate}
%\end{cor}
%
%We call the associated $\sP$-graded complex a $J(L)$-splitting.  In general splitting is not unique, and depend upon the choice of graded basis.  
%
%We highlight the following simple objects of $\bCF$. In the filtered case, this is the analogue of the connection matrix.
%
%\begin{defn}
%{\em 
%An $\sL$-filtering $f\colon \sL \to Sub(C,\partial)$ is a \em{Conley filtering} if
%\[
%\partial(f(a)) \subseteq f(Pred(a))
%\]
%for all $a\in J(\sL)$.
%}
%\end{defn}
%
% These are central objects to the Conley theory, as with regard to the $J(L)$-splitting, the boundary map can be interpreted as a map on the direct sum of homology.
%
%\begin{cor}
%Let $f\colon \sL \to Sub(C,\partial)$  be a Conley filtering. Then $f$ determines a $\sJ(\sL)$-graded chain complex $(C^\oplus(\sJ(\sL)),\Delta)$ such that 
%\[
%H_\bullet(f(a)/f(Pred(a))) \cong C^\oplus(a-Pred(a))
%\]
%\end{cor}
%
%In this case we may interpret $\Delta$ as an upper triangular boundary map on homology
%\[
%\Delta \colon \bigoplus_{a\in \sJ(\sL)} H(f(a)/f(Pred(a)) \to \bigoplus_{a\in \sJ(\sL)} H(f(a)/f(Pred(a))
%\]
%
%
%\begin{thm}
%Let $f:X\to \sJ(\sL)$ be a graded cell complex and $\cL(f)$ be the associated $\sL$-filtered chain complex..  Let $(C^\oplus(\sJ(\sL)),\Delta)$ be the associated $\sJ(\sL)$-graded chain complex.  If $(A^\oplus(\sJ(\sL)),\Delta_A)$ is a Conley complex which is homotopy-equivalent to $(C^\oplus(\sJ(\sL)),\Delta)$ then $\mathfrak{L}(A^\oplus(\sJ(\sL)),\Delta_A)$ is a Conley filtering which is filtered homotopy equivalent to $\cL(f)$.
%\end{thm}
%
%
%
%









%%
%!TEX root = ../main.tex
\section{Lattice-Filtered  Complexes}\label{sec:lfc}


We again begin with data analysis perspective, and define the object for chain complexes.  Recall from Section~\ref{} that the notion of subcomplex for a cell complex is more general than for a chain complex.  We work with closed subcomplexes, the lattice of which is denoted $Sub_{Cl}$.

\begin{defn}
{\em
Let $\sL\in \bFDLat$.
An \emph{$\sL$-filtering of a complex $\cX$} is a lattice homomorphism $f\colon \sL \to Sub_{Cl}(\cX,\partial)$.
The function $f$ is called an \emph{$\sL$-filtering}.
}
\end{defn}

\begin{ex}
Let $M$ be a closed manifold and $\varphi:\R\times M\to M$ a Morse-Smale gradient flow and $\sP$ be the set of fixed points and $\mu\colon \sP\to \N$ the assignment of Morse indices.  Each unstable manifold $W_u(p)$ is a $\mu(p)$-cell and the manifold $M$ admits a cellular decomposition $(\cM,\kappa,\leq,\mu)$.  Furthermore, the closures $cl(W_u(p))$ is a subcomplex of the cell complex.  The map 
\[
\sO(\sP) \ni a \rightsquigarrow \bigcup_{p\in a} cl(W_u(p))
\]

is a $\sO(\sP)$-filtered cell complex.
\end{ex}

\subsection{Lattice-Filtered Chain Complexes}


\begin{defn}
{\em
Let $\sL\in \bFDLat$.
An \emph{$\sL$-filtering of a chain complex $(C,\partial)$ over a field $k$} is a lattice homomorphism $f\colon \sL \to Sub(C,\partial)$.
The function $f$ is called an \emph{$\sL$-filtering}.
}
\end{defn}


Notice that since $f$ is a lattice homomorphism we have $0_L\rightsquigarrow 0$ and $1_L \rightsquigarrow (C,d)$.  Let $f\colon \sL \to Sub(C)$ and $g\colon \sL \to Sub(D)$ be $\sL$-filterings. 
A linear map $\phi\colon C\to D$ is {\em $\sL$-filtered} if
\[
\phi(f(a)) \subset g(a),\quad\text{for all $a\in\sL$}.
\]




\begin{ex}
{\em
For any $\sL$-filtering $f\colon \sL \to Sub(C,\partial)$,
the boundary map $\partial\colon C\to C$ is $\sL$-filtered.
}
\end{ex}

\begin{ex}

%A $J(L)$-graded cell complex gives rise to an $L$-filtered chain complex in the following fashion.   

Let $f\colon X\to \sJ(\sL)$ be a graded cell complex.  By Birkhoff's theorem there is a lattice homomorphism $O(f):\sL\to Sub_{Cl}(X,\preceq)$.  Each element of $Sub_{Cl}(X,\leq)$ is a basis for the associated closed subcomplex of $C(X)$ spanned by the downset.   Thus the map defined by $$L\ni q\rightsquigarrow span(O(f)(q))\in Sub(C(X))$$ is an $L$-filtered chain complex.  Therefore there is an assignment $\cL:Cell(\sJ(\sL))\to \bCF(\sL)$.


%There is an assignment $\cL:Cell(J(L))\to \bCF(\sL)$ and a graded-cell complex $f:X\to \sJ(\sL)$ is a graded basis for $\cL(f)$.
 
%$$Cell(\sJ(\sL))\to {\bf GCC(\sJ(\sL))} \to \bCF(\sL)$$
\end{ex}






The category $\bCF(\sL,k)$ has objects consisting of $\sL$-filterings of chain complexes over $k$.  The morphisms are the $\sL$-filtered chain maps.  For $f,g$ in $\bCF(\sL,k)$ we write $\phi\colon f\to g$ to indicate that $\phi$ is in $Hom_{\bCF}(f,g)$.   Let $f \colon \sL\to Sub(C,\partial^C)$ and $g \colon \sL\to Sub(D,\partial^D)$ be $\sL$-filterings over a field $k$.  
Two $\phi,\psi \colon f\to g$  $\sL$-filtering chain homomorphisms are {\em chain homotopic} if $h\colon f\to g$ is an $\sL$-filtering morphism satisfying
\[
\phi -\psi = h\partial^C+\partial^Dh.
\]
The $\sL$-filtering morphism $h$ is called a {\em filtered chain homotopy}.  
We write $\psi\sim \phi$ if $\psi$ and $\phi$ are filtered chain homotopic.  
It is straightforward that this is an equivalence relation.    Let $f \colon \sL\to Sub(C,\partial^C)$ and $f' \colon \sL\to Sub(D,\partial^C)$ be $\sL$-filtered complexes.  
Let $\phi \colon f\to f'$ and $\psi \colon f'\to f$ be filtered chain maps.  
A {\em filtered chain homotopy equivalence} consists of a quadruple $(\psi,\phi,h,g)$ such that 
\begin{enumerate}
\item $\psi\phi-id_C= h \partial^C + \partial^C h$
\item $\phi\psi-id_D = g \partial^D+\partial^D g$
\end{enumerate}



The homotopy category $\bK\bCF(\sL,k)$ has $\sL$-filtered chain complexes as objects.
The morphisms are given by the homotopy equivalence classes, i.e.\ $Hom_{\bK\bCF}(f,g) = Hom_{\bCF}(f,g)/\sim$ where $\sim$ is the homotopy equivalence relation.   
Isomorphisms in $\bK\bCF(\sL,k)$ correspond to filtered chain equivalences.





%
%\subsection{$\sP$-Graded Chain Complexes}\label{sec:lfc:grad}
%
%
%
%A $\sP$-graded chain complex is a chain complex $(C,\Delta)$ with decomposition $$C=\bigoplus_{q\in \sP} C_q$$ and boundary operator determined by its components $\Delta_{qp}:C_q\to C_p$ subject to the condition 
%\begin{align}\label{eqn:ut}
%\Delta_{qp}\neq 0\implies p\leq q
%\end{align}  The collection $\{\Delta_{qp}\}$ can be thought of as a matrix of linear maps. If a linear map satisfies condition~(\ref{eqn:ut}) we call it {\em $\sP$-graded}.  In~\cite{fran} this condition is referred to as {\em upper triangularity with respect to $\sP$}.   If $\Delta_{pp} = 0$ for all $p$ then the boundary map $\Delta$ is called a {\em connection matrix}.  In this case $\Delta$ can be interpreted as a map
%\begin{align}\label{eqn:cm}
%\Delta \colon \bigoplus_{q\in \sP} H_\bullet(C_q,\Delta_{qq})\to \bigoplus_{q\in \sP} H_\bullet(C_q,\Delta_{qq})
%\end{align}
%
%The identification of $\Delta$ with the matrix structure is the genesis of the phrase {\em connection matrix}.  We denote $\sP$-graded chain complexes by $(C^\oplus(\sP),\Delta)$ and often abbreviate this with $C^\oplus(P)=(C^\oplus(P),\Delta(P))$.  If $\Delta$ is a connection matrix for $(C^\oplus(\sP),\Delta)$ then we say $C^\oplus(\sP)$ is a {\em Conley complex}.  Equation~\ref{eqn:cm} implies that $\Delta$ is boundary map on Conley indices.  Therefore the notion of Conley complex is analogous to that of a Morse complex.
%
%A morphism of $\sP$-graded chain complexes is a $\sP$-graded chain map.   We call the category of $\sP$-graded chain complexes $\bf{GCC(\sP)}$.   Two $\sP$-graded chain maps $\Phi,\Psi:(C^\oplus(\sP),\Delta)\to (D^\oplus(\sP),\Delta')$ are {\em chain homotopic} if there  is a $\sP$-graded degree+1 map $\Gamma:C^\oplus(\sP)\to D^\oplus(\sP)$ such that 
%\[
%\Phi-\Psi = \Gamma\Delta - \Delta\Gamma
%\]
% The map $\Gamma$ is called a {\em graded chain homotopy}.  The associated associated homotopy category ${\bf KGCC(\sP)}$ has $\sP$-graded chain complexes as objects and morphisms given by homotopy equivalences classes.     For a subset $I\subseteq \sP$ we set 
%\[
%C^\oplus(I) = \bigoplus_{p\in I} C_p\quad\quad \quad \Delta(I) = \pi_{P,I} \circ \Delta \circ \iota_{I,P}
%\]
% where $\iota_{I,P}:C^\oplus(I)\to C^\oplus(P)$ and $\pi_{P,I}:C^\oplus(P)\to C^\oplus(I)$ are the natural inclusion and projection.  In general $C^\oplus(I)$ is neither a subcomplex nor a chain complex.  When $I$ is convex in $\sP$ it is straightforward that $\Delta(I)\circ\Delta(I) = 0$ and $(C^\oplus(I),\Delta(I)$ is a chain complex.  In fact, it may be regarded as $(\sI,\leq)$-graded chain complex where $(\sI,\leq)$ is the restriction of $(\sP,\leq)$ to $\sI$.   
% 
%  For each lower set $a\in \sO(\sP)$ we have $\Delta(C^\oplus(a))\subseteq C^\oplus(a)$ is $\Delta$ is $\sP$-graded.  Thus $\Delta(a)=\Delta|_{C^\oplus(a)} $. This implies that $(C^\oplus(a),\Delta(a))$ is a subcomplex of $(C^\oplus(\sP),\Delta)$.    
  
  
 We can now explore the relationship between graded and filtered complexes.  Every $\sP$-graded chain complex $(C^\oplus(\sP),\Delta)$ determines an $\sL$-filtered chain complex $h:\sO(\sP)\to Sub(C,\Delta)$ via $$\sO(\sP)\ni a \rightsquigarrow (C^\oplus(a),\Delta(a))\in Sub(C,\Delta)$$


This defines an assignment $\mathfrak{L}\colon {\bf GCC}(\sJ(\sL))\to \bCF(\sL)$.  A morphism of $\sP$-graded chain complexes induces a morphism between the associated lattice-filtered complexes.  This is captured as a corollary of the next result.


\begin{prop}\label{prop:filt:functor}
The assignment $\mathfrak{L}\colon {\bf GCC}(\sJ(\sL))\to \bCF(\sL)$ is an additive functor.  Moreover $\mathfrak{L}$ descends to a functor on homotopy categories $K(\mathfrak{L}):{\bf KGCC}\to {\bf KLFC}$.
\end{prop}
\begin{proof}
It is straightforward that $\mathfrak{L}$ is an additive functor.  Let $(C^\oplus(\sP),\Delta)$ and  $(D^\oplus(\sP),\Delta')$  be $\sJ(\sL)$-graded complexes.  Denote by $f\colon \sL\to Sub(C,\Delta)$ and $g\colon \sL\to Sub(D,\Delta')$ the associated filtered complexes $\mathfrak{L}(C^\oplus(\sP))$ and $\mathfrak{L}(D^\oplus(\sP))$.  Any $\sJ(\sL)$-graded homotopy $\Gamma:C^\oplus(\sP)\to D^\oplus(\sP)$ is an $\sL$-filtered homotopy $f\to g$ since
\[
\Gamma(f(a))= \Gamma(C^\oplus(a)) \subseteq D^\oplus(a) = g(a)
\]
Therefore any graded homotopy equivalence is a filtered homotopy equivalence and  shows that the functor $\mathfrak{L}$ descends to the homotopy categories.
\end{proof}


\begin{prop}\label{prop:filt:functor2}
The functor $\mathfrak{L}\colon {\bf GCC}(\sJ(\sL))\to \bCF(\sL)$ fits into the following commutative diagram with the assignments $\cC$ and $\cL$ (denoted by dashes arrows)
\[
\xymatrixrowsep{0.35in}
\xymatrixcolsep{0.35in}
\xymatrix{
Cell(\sJ(\sL)) \ar@{-->}[dr]_{\cL} \ar@{-->}[r]^{\cC} & {\bf GCC(\sJ(\sL))} \ar[d]^{\mathfrak{L}} \\
 & {\bf LFC(\sL)}
}
\]
\end{prop}

We highlight the following simple objects of $\bCF$. In the filtered case, this is the analogue of the Conley complex and connection matrix.  This is formalized in the next by Proposition~\ref{prop:filt:cm}.



\begin{defn}
{\em 
An $\sL$-filtering $f\colon \sL \to Sub(C,\partial)$ is a \em{Conley filtering} if
\[
\partial(f(a)) \subseteq f(Pred(a))
\]
for all $a\in J(\sL)$.
}
\end{defn}



\begin{prop}\label{prop:filt:cm}
If $(C^\oplus(\sJ(\sL)),\Delta)$ is a Conley complex then $\mathfrak{L}(C^\oplus(\sJ(\sL)),\Delta)$ is a Conley-filtering.
\end{prop}



We've discussed how the functor $\mathfrak{L}$ constructs filtered chain complexes from graded ones.  One can reverse the process and construct a $\sJ(\sL)$-graded chain complex from an $\sL$-filtered chain complex.  However such constructions are typically not unique nor functorial.


\begin{defn}
{\em
Consider $f:\sL\to Sub(C,d)$.  A {\em graded basis for $h$} is a collection $\cB\subseteq C$ with a map $\nu:\cB\to \sJ(\sL)$ such that for each $q\in \sO(\sJ(\sL))$ the set $\nu^{-1}(q)$ is a basis for $f(q)$.
 The function $\nu$ is called the {\em valuation}. As $\sO(\sJ(\sL)) \cong \sL$ we'll often refer to $\nu^{-1}(q)$ for $q\in \sL$.  
}
\end{defn}

Graded bases are closely related to the idea of graded cell complexes, as introduced in Section~\ref{sec:grad}.  In particular, a graded bases determines a graded cell complex if one defines the underlying poset using Eqn.~\ref{eqn:poset}.  On the other hand, if one begins with a graded-cell complex $f:X\to \sJ(\sL)$ then $f$ is a graded basis for $\cL(f)$, the associated $\sL$-filtered chain complex.   The next proposition shows that for $\sL$-filtered chain complexes over fields, one can always find a graded basis.  Note that distributivity of $\sL$ is a necessary ingredient in this result.

\begin{prop}\label{prop:bases}
For any $\sL$-filtered complex $f:\sL\to Sub(C,d)$ there is a graded basis $(\cB,\nu)$.
\end{prop}
\begin{proof}
We first construct subspaces associated to each $q\in J(L)$, then we may select a basis for these subspaces.  Let $q\in J(L)$.  We have $Pred(q)=\bigvee_i p_i$.  Thus $f(Pred(q)) = f(p_1)+\ldots + f(p_n)$.    Choose a subspace $V_q$ such that $f(q) = V_q \oplus f(Pred(q))$. Notice that for any $q$ that covers $0_L$ we have $Pred(q)=0_L$, so $f(Pred(q)) = 0$ and $V_q = f(q)$.

We'll show that $V_q\cap V_p=0$ for $q\neq p$.  Let $x\in V_q\cap V_p$.  Then $x\in f(q)\cap f(p) = f(q\wedge p)$.  However, $f(q\wedge p)\subseteq f(Pred(q))$ and $f(p\wedge p)\subseteq f(Pred(p))$.  Thus $x=0$ by choice of $V_q$ and $V_p$.  

Choose a basis $\cB_q$ for each $V_q$.  Since $V_q\cap V_p=0$ for $p\neq q$ these bases are pairwise disjoint. Let $\cB=\bigsqcup_q \cB_q$ and define $\nu:\cB\to J(L)$ by $\nu(x) = q$ if $x\in \cB_q$.


  It remains to show that for any $a\in \sL$ the set $\nu^{-1}(a) = \bigsqcup_{q\in a} \cB_q $ is a basis for $f(q)$.  We'll do this using an inductive argument.  Let $q\in \sL$ and assume that for each $p<q$ we have $\nu^{-1}(p)$ is a basis for $f(p)$.  Let $x\in f(q)$.  Since $f(q) = V_q\oplus f(Pred(q))$ we have $x = x_q+x_{Pred(q)}$.  $V_q$ is spanned by $\cB_q$.  We have $Pred(q) = \bigvee_i p_i$ with $p_i\in \sJ(\sL)$.  Thus $f(Pred(q)) = f(p_1) + \ldots + f(p_n)$.  
  
  
%   It suffices to show that it spans.  Let $x\in f(q)$.  
%  
%  
%  Now we'll argue that $\bigoplus_{q\in J(L)} V_q$ span $C$.
%
%Now we wish to show that $\bigoplus_{q\in J(L)} V_q$ span $C$.  We will prove this by strong induction.  We will induct over a linear extension of $L$.  The base case is to consider the minimal element, $0_L\in L$.  By (4) of~\ref{def:cf} if $f(x)=0_L$ then $x=0$, which is in the span.   Now fix $p\in L$.  The strong inductive hypothesis is to assume that for any $q< p$ any $x$ with $f(x)=q$ is in span $\bigoplus_{q\in J(L)} V_q$.  Let $p\in L$.  By Lemma~\ref{lem:join} we may write $p$ as an irredundant join $p=\bigvee_i q_i$ with $q_i \in J(L)$. Notice if $p\in J(L)$ then the decomposition is trivially written as $p=p$.  Coherence implies that $D_p = D_{q_1}+D_{q_2}+\ldots+D_{q_n}$.  Thus $x= \sum_i \lambda_i x_{q_i}$.  There are two cases.  First, if $p\not\in J(L)$, then $q_i< p$ and each $x_{q_i}$ belongs to the span.  For the second case, $p\in J(L)$ and we may write $D_p = V_p \bigoplus D_{Pred(p)}$.  Thus $x = v_p + x_{Pred(p)}$.  Since $Pred(p)<p$ the inductive hypothesis implies that $x_{Pred(p)}$ is in the span. 
\end{proof}


\begin{cor}
\label{prop:Lsplitting}
Any $\sL$-filtered complex $f\colon \sL\to Sub(C,\partial)$ determines a $\sJ(\sL)$-graded chain complex $(C^\oplus(\sJ(\sL)),\Delta)$  such that $$f(a)= (C^\oplus(a),\Delta(a))$$

and in particular for $a\leq b$ we have $f(b)/f(a) \cong (C^\oplus(b-a),\Delta(b-a))$.

%\begin{enumerate}
%\item for any $a\in L$ $$\bigoplus_{q\leq a} C_q \cong f(a)$$
%\end{enumerate}
\end{cor}

We call the associated $\sP$-graded complex a $J(L)$-splitting.  In general splitting is not unique, and depend upon the choice of graded basis.   The Conley filtered chain complexes are central in the theory, as with regard to the $J(L)$-splitting, the boundary map is a connection matrix.

\begin{cor}
Let $f\colon \sL \to Sub(C,\partial)$  be a Conley filtering. Then $f$ determines a $\sJ(\sL)$-graded Conley complex $(C^\oplus(\sJ(\sL)),\Delta)$ such that
\[
H_\bullet(f(a)/f(Pred(a))) \cong C^\oplus(a-Pred(a))
\]
\end{cor}

In this case we may interpret $\Delta$ as an $\sP$-graded boundary map on homology
\[
\Delta \colon \bigoplus_{a\in \sJ(\sL)} H(f(a)/f(Pred(a)) \to \bigoplus_{a\in \sJ(\sL)} H(f(a)/f(Pred(a))
\]


Finally, in analogy to Proposition~\ref{prop:grad:cmiso}, we have that chain homotopy equivalent Conley filterings are chain isomorphic.

 \begin{prop}\label{prop:filt:cmiso}
 Let $f:\sL\to Sub(C,d)$ and $g:\sL\to Sub(D,d')$ be $\sL$-Conley filtered chain complexes.  If $f$ and $g$ are filtered chain equivalent, then $f$ and $g$ are filtered chain isomorphic.
 
 \end{prop}
 \begin{proof}
From Proposition~\ref{prop:bases} there are graded bases $(\cB,\nu)$ and $(\cB',\nu')$ for $f$ and $g$ respectively.  Let $C^\oplus(\sJ(\sL))$ and $D^\oplus(\sJ(\sL))$ be the associated $\sJ(\sL))$-graded Conley complexes.  The $\sL$-filtered maps $\phi:C\to D$ and $\psi:D\to C$ can be written as matrices $\Phi$ and $\Psi$ with respect to the bases $\cB$ and $\cB'$.  In this way they determine $\sJ(\sL)$-graded maps on $C^\oplus(\sJ(\sL))$ and $D^\oplus(\sJ(\sL))$.  From Proposition~\ref{prop:grad:cmiso} the maps $\Phi$ and $\Psi$ are chain isomorphisms.  Thus $\phi$ and $\psi$ are chain isomorphisms.
  \end{proof}

With the theory we've built up thus far, the following proof is straightforward.

\begin{thm}
Let $f:X\to \sJ(\sL)$ be a graded cell complex and $\cL(f)$ be the associated $\sL$-filtered chain complex..  Let $(C^\oplus(\sJ(\sL)),\Delta)$ be the associated $\sJ(\sL)$-graded chain complex.  If $(A^\oplus(\sJ(\sL)),\Delta_A)$ is a Conley complex which is homotopy-equivalent to $(C^\oplus(\sJ(\sL)),\Delta)$ then $\mathfrak{L}(A^\oplus(\sJ(\sL)),\Delta_A)$ is a Conley filtering which is filtered homotopy equivalent to $\cL(f)$.
\end{thm}

\begin{proof}
Since $(A^\oplus(\sJ(\sL)),\Delta_A)$ is a Conley complex $\mathfrak{L}(A^\oplus(\sJ(\sL)),\Delta_A)$ is $\sL$-Conley filtering follows from Proposition~\ref{prop:filt:cm}.  Since $(A^\oplus(\sJ(\sL)),\Delta_A)$ and $(C^\oplus(\sJ(\sL)),\Delta)$ are graded homotopy equivalent, $\mathfrak{L}(A^\oplus(\sJ(\sL)),\Delta_A)$ and $\mathfrak{L}(C^\oplus(\sJ(\sL)),\Delta)$ are isomorphic in ${\bf KLFC}$ by Proposition~\ref{prop:filt:functor}.  By Proposition~\ref{prop:filt:functor} 
\[
\mathfrak{L}(C^\oplus(\sJ(\sL)),\Delta) = \mathfrak{L}\circ \cC(f) = \cL(f)
\]

Therefore $\mathfrak{L}(C^\oplus(\sJ(\sL)),\Delta)$ is isomorphic to $\cL(f)$ in ${\bf KLFC}$.

\end{proof}










%%
%!TEX root = ../main.tex

\section{Connection Matrix Theory}\label{sec:CMT}

In this section we will review the connection matrix theory.  Unfortunately, the theory comes with a fairly large overhead of mathematical machinery.  This section is included for completeness, and the Conley theory cognoscenti.  For non-experts, it may be skipped upon first reading.

The connection matrix theory was first developed by R. Franzosa in a sequence of papers based on his dissertation, which was directed by C. Conley~\cite{fran2,fran,fran3}.  These ideas were reinterpreted by J. Robbin and D. Salamon in their paper~\cite{salamon}.  We will discuss both approaches.  The connection matrix is a generalization of the Morse boundary operator for the Conley theory.   It is a boundary operator defined on Conley indices.  Its basic utility is to prove existence of connecting orbits~\cite{mpmw}.  At a higher level, it serves as an algebraic representation of global dynamics and may used to construct (semi)-conjugacies of the global attractor~\cite{dhmo,mcmodels,scalar}. Its preeminent function is to complete the Conley theory to a homological theory~\cite{mc}.  

We first review the connection matrix theory as presented by Franzosa in~\cite{fran}.

\subsection{The Category Of Chain Complex Braids}
It was Conley's observation~\cite{conley} that focusing on the attractors of a dynamical system provides a generalization of the Spectral Decomposition of Smale~\cite[Theorem 6.2]{smale}.  There is a lattice structure to the attractors of a dynamical system~\cite{salamon,lsa,lsa2}.  Thus a natural object of study in Conley theory is some finite sublattice of attractors, and an associated sublattice of attracting blocks.  A sublattice of attracting blocks is what Franzosa calls an index filtration.

In his work, Franzosa introduces the notion of a {\em chain complex braid} as a data structure to hold the chains that arise from the topological data within the index lattice.  The chain complex braid is organized by the poset of join-irreducibles.  Implicit in Franzosa's work is a description of a category for chain complex braids over a fixed poset.  We now describe this category and denote it by ${\bf CCB}(P,\leq)$.

\begin{defn}
{\em
A sequence of chain complexes and chain maps $$C_1\xrightarrow{i} C_2 \xrightarrow{p} C_3$$
is called {\em weakly exact} if $i$ is injective, $p\circ i = 0$ and $p:C_2/im(i)\to C_3$ induces an isomorphism on homology.
}
\end{defn}
\begin{rem}
Every short exact sequence is weakly exact.  Franzosa needs weakly for complications which arise when working with singular homology and a lattice of attracting blocks.
\end{rem}

\begin{defn}
{\em
A {\em chain complex braid} over $(P,\leq)$ is a collection of chain complexes and chain maps such that:
\begin{enumerate}
\item for each $I\in I(P)$ there is a chain complex $(C(I),\partial(I))$
\item for each $(I,J)\in I_2(P)$ there are chain maps $$i(I,IJ):C(I)\to C(IJ)\quad\quad p(IJ,J):C(IJ)\to C(J)$$ which satisfy:
\begin{enumerate}
\item $C(I)\xrightarrow{i(I,IJ)} C(IJ)\xrightarrow{p(IJ,J)} C(J)$ is weakly exact,
\item if $I$ and $J$ are noncomparable then $p(JI,I)i(I,IJ)=id|_{C(I)}$
\item if $(I,J,K)\in I_3(P)$ then the following braid diagram commutes:
\[
\xymatrixrowsep{0.03in}
\xymatrixcolsep{0.3in}
\xymatrix{
& & C(J) \incar[dr] & &  \\
& C(IJ) \proar[ur] \incar[dr] & & C(JK) \proar[dr] &  \\
C(I) \incar[ur] \ar@{_{(}->}[rr]& & C(IJK) \proar[rr] \proar[ur]& & C(K) 
}
\]
\end{enumerate}

\end{enumerate}
}
\end{defn}

Chain complex braids are the objects of our category ${\bf CCB}(P,\leq)$.  A morphism $\Psi:\cC\to \cC'$ between chain complex braids $\cC$ and $\cC'$ is a collection of chain maps $\Psi(I):C(I)\to C'(I)$ for each $I\in I(<)$ such that for $(I,J)\in I_2(<)$ the following diagram commutes:
\[
\xymatrixcolsep{0.4in}
\xymatrixrowsep{0.4in}
\xymatrix{
C(I) \incar[r] \ar@{->}[d]_{\Psi(I)} & C(IJ) \ar@{->}[d]_{\Psi(IJ)} \proar[r] & C(J) \ar@{->}[d]^{\Psi(J)}  \\
C'(I) \incar[r] & C'(IJ) \proar[r] & C'(J)
}
\] 

Different index lattices for the same dynamical system may yield different chain complex braids.  However the homology of these chain groups are an invariant.  This is the motivation for the idea of a graded module braid, which formalizes the notion of `homology' for a chain complex braid.


\begin{defn}
{\em
A {\em graded module braid} over $(P,\leq)$ is a collection of graded modules and maps between graded modules satisfying:
\begin{enumerate}
\item for each $I\in I(P)$ there is a graded module $G(I)$
\item for each $(I,J)\in I_2(P)$ there are maps:
\begin{align*}
i(I,IJ):G(I)\to G(IJ) \text{ of degree 0,}\\
p(IJ,J):G(IJ)\to G(J) \text{ of degree 0,}\\
\partial(J,I):G(J)\to G(I) \text{ of degree -1}
\end{align*}
which satisfy:
\begin{enumerate}
\item $\ldots \xrightarrow{i} G(I)\to G(IJ)\xrightarrow{p} G(J) \xrightarrow{\partial} \ldots$ is exact,
\item if $I$ and $J$ are noncomaprable then $p(JI,I)i(I,IJ)=id|_{G(I)}$
\item if $(I,J,K)\in I_3(P)$ then the following braid diagram commutes:
\[
\xymatrixrowsep{0.15in}
\xymatrixcolsep{0.3in}
\xymatrix{
\vdots \ar@{->}[d] && \cdots \ar@{->}[drr] \ar@{->}[dll] && \vdots \ar@{->}[d]\\
G(I)\ar@/_2pc/[dd]_{i} \ar@{->}[drr]_{i} &&  && G(K)\ar@{->}[dll]_{\partial} \ar@/^2pc/[dd]_{\partial}  \\
& &G(IJ) \ar@{->}[dll]_{i} \ar@{->}[drr]_{p} &&\\
G(IJK) \ar@/_2pc/[dd]_{p}  \ar@{->}[drr]_{p} &&  && G(J)\ar@{->}[dll]_{i}  \ar@/^2pc/[dd]_{\partial}  \\
& &G(JK) \ar@{->}[dll]_{p} \ar@{->}[drr]_{\partial} &&\\
G(K)\ar@/_2pc/[dd]_{\partial}    \ar@{->}[drr]_{\partial} &&  && G(I)\ar@{->}[dll]_{i}   \ar@/^2pc/[dd]_{i} \\
& &G(IJ) \ar@{->}[dll]_{p} \ar@{->}[drr]_{i} &&\\
G(J)\ar@{->}[d] \ar@{->}[drr] &&  && G(IJK)\ar@{->}[dll]  \ar@{->}[d] \\
\vdots && \cdots && \vdots}
\] 
%C(I) \incar[r] \ar@{->}[d]_{\Psi(I)} & C(IJ) \ar@{->}[d]_{\Psi(IJ)} \proar[r] & C(J) \ar@{->}[d]^{\Psi(J)}  \\
%C'(I) \incar[r] & C'(IJ) \proar[r] & C'(J)
\end{enumerate}

\end{enumerate}
}
\end{defn}

A morphism $\theta:\cG\to \cG'$ of graded module braids is a collection of linear maps $\theta(I):G(I)\to G'(I)$, $I\in I(P)$ such that for each $(I,J)\in I_2(P)$ the following diagram commutes:
\[
\xymatrixrowsep{0.4in}
\xymatrixcolsep{0.45in}
\xymatrix{
\ldots \ar@{->}[r] & G(I) \ar@{->}[d]^{\theta(I)} \ar@{->}[r]^{i} & G(IJ) \ar@{->}[d]_{\theta(IJ)} \ar@{->}[r]^{p} & G(J) \ar@{->}[d]_{\theta(I)} \ar@{->}[r]^{\partial} & G(I) \ar@{->}[d]_{\theta(I)} \ar@{->}[r] & \ldots\\
\ldots \ar@{->}[r] & G'(I) \ar@{->}[r]^{i} & G'(IJ) \ar@{->}[r]^{p} & G'(J) \ar@{->}[r]^{\partial} & G'(I) \ar@{->}[r] &\ldots
}
\]
%As observed by Robbin and Salamon~\cite{sal}, the correct way to think about ${\bf CCB}(P,\leq)$ is as a homotopy category.

We label the category of graded module braids over $(P,\leq)$ by ${\bf GB}(P,\leq)$.

Franzosa describes a functor from $\cH:{\bf CCB}(P,\leq)\to {\bf GB}(P,\leq)$ which is the analogy of homology.  This is basically the content of~\cite[Proposition 2.7]{fran}.

\subsection{Connection Matrices}

The connection matrix is a boundary operator on Conley indices.  The Conley indices are graded vector spaces.  Let $(P,\leq)$ be a poset.  Let $\{V_p\}_{p\in P}$ be a collection of graded vector spaces indexed by $P$.  Define $$V^\oplus(P) := \bigoplus_{p\in P} V_p$$  For collections $\{V_p\}_{p\in P}$ and $\{W_p\}_{p\in P}$ a morphism $\phi(P):V^\oplus(P)\to W^\oplus(P)$ may be thought of as a matrix of linear maps: $$[\phi(p,q):V_p\to W_q\mid p,q\in P]$$


As $V^\oplus(P),W^\oplus(P)$ are indexed by a poset, we say that a morphism $\phi(P)$ is {\em diagonal with respect to $P$} if $\phi(p,q)=0$ for $p\neq q$.  It is {\em upper triangular with respect to $P$} if $\phi(p,q)=0$ for $p\nless q$.  We will also be concerned with endomorphisms.  An endomorphism $\Delta(P):V^\oplus(P)\to V^\oplus(P)$ is {\em a boundary map} if each $\Delta(p,q)$ is a degree -1 map and $\Delta(P)\circ \Delta(P) = 0$.  Identifying an endomorphism $\Delta(P)$ with its matrix structure is origin of the term {\em connection matrix}.  

One often examines subspaces of $V^\oplus(P)$.  For $I\subseteq P$ we define $V^\oplus(I)= \bigoplus_{p\in I} V_p$.  We let $\phi(I)$ be the restriction of $\phi(P)$ to $V^\oplus(I)\to V^\oplus(I)$.  It is easy to see that $\phi(I) = \pi_{P,I} \circ \phi(P) \circ \iota_{I,P}$ where $\iota_{I,P}:V^\oplus(I)\to V^\oplus(P)$ and $\pi_{P,I}:V^\oplus(P)\to V^\oplus(I)$ are the natural inclusion and projection.  The following sequence of propositions shows that upper triangular boundary maps can be used to build chain complex braids.




\begin{prop}[Proposition 3.2,~\cite{fran}]
If $\Delta(P):V^\oplus(P)\to V^\oplus(P)$ is an upper triangular boundary map then $\Delta(I)$ upper triangular boundary map for any convex set $I$.   
\end{prop}

\begin{prop}[Proposition 3.3,~\cite{fran}]\label{prop:fran:3.3}
If $\Delta(P):V^\oplus(P)\to V^\oplus(P)$ is an upper triangular boundary map, then for any adjacent pair $(I,J)\in I_2(P)$ the following sequence is a short exact sequence of chain complexes: $$0\to (V^\oplus(I),\Delta(I)\to (V^\oplus(IJ),\Delta(IJ))\to (V^\oplus(J),V(J))\to 0$$ where the maps are the natural inclusion and projections.
\end{prop}

\begin{prop}[\cite{fran}, Proposition 3.4]
Given an upper triangular boundary map $\Delta(P):C^\oplus(P)\to C^\oplus(P)$ the collection, denoted $C\Delta(P)$, consisting of the chain complexes $C(I)$ with boundary map $\partial(I)$ for each $I\in I(P)$ and the obvious chain maps $i(I,IJ)$ and $p(IJ,J)$ for each $(I,J)\in I_2(P)$ is a chain complex braid over $P$.
\end{prop}






\begin{defn}
{\em
Let $\cG$ be a graded module braid.  Let $\{C(p)\}_{p\in P}$ be a collection of graded vector spaces.  Let $\Delta:C^\oplus(P)\to C^\oplus(P)$ be an upper triangular boundary map.  Let $\cC$ be the associated chain complex braid.  If $\cH(\cC)\cong \cG$, then $\Delta$ is a {\em C-connection matrix} for $\cC$.  If in addition $\Delta(p,p)=0$ for all $p\in P$ then $\Delta$ is a {\em connection matrix} for $\cC$.
}
\end{defn}

It is clear that if $\Delta(p,p)=0$ for all $p\in P$ then for each $p$ $HC(p)\cong C(p)$ and $\Delta$ may be written as a boundary map $\Delta:\bigoplus_{p\in P} HC(p)\to \bigoplus_{p\in P} HC(p)$.

One way to motivate the connection matrix is to observe that chain complex braids generated by upper triangular boundary maps are particularly simple.  For instance given $I\in I(P)$ $C(I) = \bigoplus_{p\in I} C(p)$.  Below is one of Franzosa's theorems on existence of connection matrices:

\begin{thm}[\cite{fran}, Theorem 4.8]
Let $\cC$ be a chain complex braid over $(P,\leq)$.  Let $B=\{B(p)\}_{p\in P}$ be a collection of free chain complexes such that $HB(p) \cong HC(p)$, the homology of the chain complex $C(p)$ in $\cC$.  Then there exists an upper triangular boundary map, $$\Delta:\bigoplus_{p\in P} B(p)\to \bigoplus_{p\in P}B(p)$$ and a chain map $\Psi:\cB\to \cC$, from the chain complex braid induced by $\Delta$ such that $\cH\Psi:\cH\cB\to \cH\cC$ is a graded module braid isomorphism.
\end{thm}

Here's a simple application of Franzosa's theorem.  Let $\cC$ be a chain complex braid.  Choose $B = \{C(p)\}_{p\in P}$.  The theorem says that there exists an upper triangular boundary map $\Delta:C^\oplus(P)\to C^\oplus(P)$ and a chain map $\Psi:\cB\to \cC$ such that $\cH\Psi$ is an isomorphism.  Therefore for any chain complex braid there is a simple representative (one generated by an upper triangular boundary map) in its {\em derived equivalence class}.  If $HC(p)$ is free, then we may choose $B=\{HC(p)\}$ where $HC(p)$ is considered as chain complex with zero differentials.  In this case $\Delta$ is a connection matrix.


\subsection{Connection Matrix Redux}



J. Robbin and D. Salamon have a related development of the connection matrix~\cite{salamon}.   Instead of a chain complex braid, their idea is to assign a subcomplex to each lower set of the poset.  As the lower sets form a lattice, the assignment is required to be a lattice homomorphism.  A $P$-filtered chain complex is a chain complex $(C,\partial)$ with a collection of subcomplexes $\{C_\alpha\}_{\alpha\in O(P)}$ such that $$C_{\alpha\cap \beta} = C_\alpha\cap C_\beta, \quad C_{\alpha\cup \beta} = C_\alpha + C_\beta,\quad C_\emptyset = \{0\},\quad C_P = C$$ and that $$\partial(C_\alpha)\subset C_\alpha$$

A mapping $\phi:A\to B$ of $P$-filtered chain complexes is said to {\em preserve the filtration} if $\phi(A_\alpha)\subset B_\alpha$.  A morphism of $P$-chain map $\phi:A\to B$ of $P$-filtered chain complexes is a chain map $A\to B$ which preserves the filtration.

It is implicit in~\cite{salamon} that the correct model for connection matrix theory is homotopy theory.    Two $P$-chain maps $\phi,\psi:A\to B$ are called $P$-chain homotopic if there is a map $\gamma:A\to B$ which preserves the filtration and satisfies $$\phi-\psi = \partial_B \circ \gamma + \gamma\circ \partial_A$$

Two $P$-filtered chain complexes $A$ and $B$ are called $P$-chain equivalent if there are morphisms $\phi:A\to B$ and $\psi:B\to A$ such that both $\phi\circ \psi:B\to A$ and $\psi\circ \phi:A\to B$ are $P$-chain homotopic to the identity.  

A $P$-connection matrix is a $P$-filtered chain complex $(C,\Delta)$ with the property that $$\Delta(C_\beta)\subset C_{\beta \backslash p}$$ whenever $p$ is maximal in $\beta$.  

Using the idea of homotopy equivalence you can build a homotopy category as in Section~\ref{sec:prelims:AT}.  This makes homology implicit, and one then does not need to define the equivalent of a `graded module braid'.  


%%
%!TEX root = ../main.tex


\section{Reductions and Discrete Morse Theory}\label{sec:reductions}

In this section we'll build the theoretical tools for computing Conley-filtered complexes.  In Section~\ref{sec:reductions:alg} we detail a computational version of the theory presented here.  We'll first review the tools for the chain complexes and the category $\bCh(k)$ before proceeding to graded and filtered versions within the categories $\bGCC(\sP)$ and $\bLFC(\sL)$, respectively.  Much of the material may feel redundant, as we will port results from chain complexes to graded and filtered versions.


  In computational homological algebra, one often finds a simpler representative with which to compute homology.  A model for this is the notion of {\em reduction}, which is a particular type of chain homotopy equivalence.  The notion also goes under the moniker {\em strong deformation retract} or sometimes chain contraction~\cite{sko2}.\footnote{We've previously introduced the term {\em chain contraction} was  in Section~\ref{sec:prelims:AT} which agrees with~\cite{weibel}.  This idea should not be confused with reduction.}  It can be found in Eilenberg and MacLane, homological perturbation theory~\cite{barnes:lambe} and forms the basis for effective homology theory and algebraic Morse theory~\cite{sko,sko2}.  Roughly, a reduction is a method of data reduction for a chain complex without losing any information with respect to homology.

\begin{defn}
{\em A {\em reduction} is a diagram of chain complexes
\[
\xymatrixrowsep{0.03in}
\xymatrixcolsep{0.3in}
\xymatrix{
C  \ar@(u,l)_{\gamma}  \ar[r]<3pt>^{\psi} & \ar[l]<3pt>^{\phi} M
}
\]
where $\phi,\psi$ are chain maps and $\gamma$ is a degree+1 map satisfying the identities:
\begin{enumerate}
\item $\phi\psi = id_M$
\item $\phi\psi = id_C-(\gamma d+d\gamma)$
\item $\gamma^2 = \gamma\phi = \psi\gamma = 0$
\end{enumerate}
}
\end{defn}

 From the definition it is clear that $\phi$ is a monomorphism and $\psi$ is an epimorphism.  In applications, one calls $M$ the {\em reduced complex}.  When reductions arise from discrete-algebraic Morse theory it is sometimes called the {\em Morse complex}.  The point is that the reduced complex often has much smaller cardinality than $C$, leading to efficient computation of homology.  Reductions may be obtained from special degree +1 maps called splitting homotopies.

\begin{defn}
{\em
Let $(C,d)$ be a chain complex.  A {\em splitting homotopy} is a degree +1 map $h:C\to C$ such that $h^2=0$ and $h dh = h$.
}
\end{defn}

The conditions $d^2=h^2=0$ and $hdh = h$ ensure that $hd+dh$ is a projection.  Therefore $\pi=id_C-(hd+dh)$ is a projection onto the complementary subspace.  Since $\pi$ is a projection, there is a splitting of $C$ into subcomplexes:
\[
C=\ker\pi\oplus im(\pi)
\] The image $(M,d^M)=(im(\pi),d|_{im(\pi)})$ is a subcomplex of $C$.  We have the following reduction:
\begin{align}\label{reduction:homotopy}
\xymatrixrowsep{0.03in}
\xymatrixcolsep{0.3in}
\xymatrix{
C  \ar@(u,l)_{\gamma}  \ar[r]<3pt>^{ \pi } & \ar[l]<3pt>^{i } M
}
\end{align}

We can calculate the differential $d^M$ via $$d\pi = d(id-(hd+dh)) = d-dhd + ddh = d-dhd$$

Finally, it is straightforward that the remaining identities $\gamma i=\pi\gamma = 0$ are easily verified. Furthermore, $\ker\pi$ is a subcomplex of $C$ and $h|_{\ker\pi}$ is a chain contraction, since $id_{\ker\pi} = dh+hd$.  This implies that $\ker\pi$ is an acyclic, i.e. $H_\bullet(\ker\pi) = 0$.



\begin{prop}\label{prop:cont:homiso}
Reductions and splitting homotopies are in bijective correspondence, up to isomorphism.
\end{prop}
\begin{proof}
Consider two reductions:
\[
\xymatrixrowsep{0.03in}
\xymatrixcolsep{0.3in}
\xymatrix{
M  \ar[r]<3pt>^{i} & \ar[l]<3pt>^{\pi} C \ar@(ul,ur)^{h} \ar[r]<3pt>^{\pi'}  & M'  \ar[l]<3pt>^{i'}
}
\] 

It is straightforward from the side conditions that the compositions $\pi\circ i'$ and $\pi'\circ i$ are inverses.  Therefore $M$ and $M'$ are chain isomorphic.
\end{proof}

\begin{ex}
{\em
Let $\cX$ be a cell complex and $(\cA,w:\cQ\to \cK)$ an acyclic partial matching.  By Proposition~\ref{prop:matchinghomotopy} there exists a unique splitting homotopy $\gamma$.  From Theorem~\ref{thm:focm:red} defining the maps
$$\psi:=\pi_\cA\circ (id_\cX-\partial \gamma) \quad\quad  \phi:= (id_\cX-\gamma \partial)\circ \iota_\cA \quad\quad \partial^\cA:= \psi\circ \partial\circ \phi $$
leads to a reduction:
\begin{align}\label{reduction:dmt}
\xymatrixrowsep{0.03in}
\xymatrixcolsep{0.3in}
\xymatrix{
(C_\bullet(\cX),\partial^\cX) \ar@(ur,ul)_{\gamma} \ar[r]<3pt>^{\psi} & \ar[l]<3pt>^{\phi} (C_\bullet(\cA),\partial^\cA) 
}
\end{align}
Notice that this is a different reduction than the one defined in Diagram~\ref{reduction:homotopy}.  However, we have $(C_\bullet(\cA),\partial^\cA) \cong (M,d^M)$ from Proposition~\ref{prop:cont:homiso}.  In contrast to Diagram~\ref{reduction:homotopy} using the reduction of Diagram~\ref{reduction:dmt} has the property that the Morse complex is comprised of critical cells of the matching.   
%This is pointed out in Forman's paper~\cite{cell} in the difference between Sections 7 and 8.
}
\end{ex}


  We say a reduction is {\em strict} if the $M$ is cyclic.  We say a splitting homotopy $h$ is {\em perfect} if $d=dhd$. 

\begin{prop}
Strict reductions and perfect splitting homotopies are in bijective correspondence.
\end{prop}
\begin{proof}
If the reduction is strict then $di\pi = id^M\pi = 0$.  We have $i\pi = id_C-dh-hd$ and apply $d$ to each side to get $$0=d(i\pi) = d(id_C-dh-hd) = d-dhd$$

Conversely, if $\gamma$ is perfect then with $M=im(\pi)$ the differential $d^M$ is calculated as $$d^M=d-dhd=0$$ 

Therefore $M$ is cyclic and the reduction is strict.
\end{proof}

A perfect splitting homotopy implies $im(\pi)\cong H_\bullet (C)$.  This allows the homology to be read from the reduction without computation.  In addition, we have $d^Ci = id^M = 0$.  Therefore $i:M\to \ker d_C$ and $i(M)$ gives representatives for the homology in the original complex $C$. In the case of fields, perfect splittings always exist, see.  This implies a chain complex $C$ and its homology $H(C)$ always fit into a reduction. Moreover any reduction of homology, or more generally a cyclic complex, is minimal in the sense that the two complexes are isomorphic.

\begin{prop}\label{prop:cont:cyclic}
Let $(C,d)$ be a cyclic chain complex. For the reduction
\[
\xymatrixrowsep{0.03in}
\xymatrixcolsep{0.3in}
\xymatrix{
C  \ar@(u,l)_{\gamma}  \ar[r]<3pt>^{\psi} & \ar[l]<3pt>^{\phi} M
}
\]
We have $M\cong C$.
\end{prop}
\begin{proof}
We have $\psi \circ \phi  = id_M$.  If $C$ is cyclic then $d=0$ and $\phi\circ\psi = id_C-(\gamma d+d\gamma) = id_C$.
\end{proof}

In this sense, the homology $H(C)$ is the algebraic core of a chain complex and the minimal model for $C$ with respect to reductions.  This result will have analogues in the graded and filtered cases.  Finally, we show that reductions compose.  This observation can be found in~\cite{} and we provide a proof for completeness.

\begin{prop}
Consider the sequence of reductions:
\[
\xymatrixrowsep{0.03in}
\xymatrixcolsep{0.35in}
\xymatrix{
C\ar@(u,l)_{\gamma} \ar[r]<3pt>^{\psi } & \ar[l]<3pt>^{\phi} M \ar@(ur,ul)_{\gamma'}  \ar[r]<3pt>^{\psi '} & \ar[l]<3pt>^{\phi'} M' 
}
\]
Then there is a reduction 
\[
\xymatrixrowsep{0.03in}
\xymatrixcolsep{0.3in}
\xymatrix{
C  \ar@(u,l)_{\gamma''}  \ar[r]<3pt>^{\psi'' } & \ar[l]<3pt>^{\phi'' } M'
}
\]
with the maps given by the formulas
\[
\phi'' = \phi \circ \phi ' \quad\quad \psi'' = \psi'\circ  \psi \quad\quad \gamma'' = \gamma + \phi \circ \gamma '\circ \psi
\]

\end{prop}
\begin{proof}
Elementary computations show that 
\[
\pi''\circ i'' = id_{M''}\quad\text{and}\quad i'' \circ \pi'' = id_C-(d\Gamma + \Gamma d)
\]

The side conditions follow from the side conditions for $\gamma$ and $\gamma'$
\begin{enumerate}
\item  $\Gamma^2 = (\gamma+i\gamma'\pi)(\gamma+i\gamma'\pi) = \gamma^2 +  (\gamma i)\gamma'\pi + i\gamma'(\pi\gamma)  + i\gamma' (\pi i)\gamma'\pi = i (\gamma'\gamma' )\pi = 0$
\item $ \Gamma \circ i'' = (\gamma + i\gamma'\pi)(i\circ i') = (\gamma i)i'  + \pi' (\pi i)\gamma '\pi =  (\pi'\gamma) '\pi = 0$
\item $\pi''\circ \Gamma = (\pi'\pi)(\gamma+i\gamma'\pi) = \pi' (\pi\gamma) + \pi'( \pi i) \gamma'\pi = (\pi'\gamma' )\pi = 0$
\end{enumerate}

\end{proof}

An inductive argument gives the following result:


\begin{prop}\label{prop:cont:tower}
For a tower of reductions:
\[
\xymatrixrowsep{0.03in}
\xymatrixcolsep{0.35in}
\xymatrix{
C   \ar@(ur,ul)_{\gamma_0} \ar[r]<3pt>^{\psi_0} & \ar[l]<3pt>^{\phi_0 } M_0 \ar@(ur,ul)_{\gamma_1} \ar[r]<3pt>^{\psi_1} & \ar[l]<3pt>^{\phi_1 }  \ldots \ar[r]<3pt>^{\psi_{n-1}}  & \ar[l]<3pt>^{\phi_{n-1} }M_{n-1} \ar@(ur,ul)_{\gamma_n}  \ar[r]<3pt>^{\psi_n} & \ar[l]<3pt>^{\phi_n} M_n
}
\]

\begin{enumerate}
\item there is a reduction 
\begin{align}\label{dia:cont:tower}
\xymatrixrowsep{0.03in}
\xymatrixcolsep{0.45in}
\xymatrix{
 C \ar@(ur,ul)_{\gamma^{n+1} } \ar[r]<3pt>^{\psi_{n,0}} & \ar[l]<3pt>^{\phi_{n,0}} M_n
}
\end{align}
with chain maps given by the compositions
\[
 \psi_{m,0} = \prod_{i=0}^n \psi_i \quad\quad \phi_{m,0} = \prod_{i=0}^m \phi_i 
\]
and the degree+1 map given by
\[
\gamma_{n+1}= \gamma + \sum_{i=0}^n \phi_{i,0}\circ  \gamma_i\circ \psi_{i,0} \quad\quad 
\]
\item $\gamma$ is a splitting homotopy and $\gamma$ is perfect if any $\gamma_i$ is perfect.

\end{enumerate}
\end{prop}
\begin{proof}
(1) follows from induction.  That $\gamma$ is a splitting homotopy follows since Diagram~\ref{dia:cont:tower} is a reduction.  If $\gamma_i$ is perfect, then $M_i$ is cyclic.  Thus $M_j$ is cyclic for any $j\geq i$ by Proposition~\ref{prop:cont:cyclic}. In particular $M_n$ is cyclic and Diagram~\ref{dia:cont:tower} is strict.  Therefore $\gamma$ is a perfect splitting homotopy.
\end{proof}




\subsection{Graded Reductions}

Filtered and graded versions of the theory are obtained by regarding the diagram in the appropriate category.  A {\em $\sP$-graded reduction} is a diagram in the category $\bGCC(\sP)$
 \[
\xymatrixrowsep{0.03in}
\xymatrixcolsep{0.3in}
\xymatrix{
C^\oplus(\sP)\ar@(ur,ul)_{\Gamma}  \ar[r]<3pt>^{\Psi} & \ar[l]<3pt>^{\Phi} M^\oplus(\sP) 
}
\]

 An $\sP$-graded reduction is {\em strict} if $M^\oplus(\sP)$ is a Conley complex.  A $\sP$-graded splitting homotopy is a degree+1 map $\Gamma:C^\oplus(\sP)\to C^\oplus(\sP)$ that is both a splitting homotopy and $\sP$-graded.    For $C^\oplus(\sP)$ a graded splitting homotopy $\Gamma\colon C^\oplus(\sP)\to C^\oplus(\sP)$ is {\em perfect} if for each $p$ the splitting homotopy $\Gamma_{pp}:C_p\to C_p$ is perfect, i.e. we have that $$\Delta_{pp} = \Delta_{pp}\Gamma_{pp}\Delta_{pp}$$  
  
 Again, one may define $\Pi=id_C-(\Gamma\Delta+\Delta\Gamma)$ and $M=im(\Pi)$.  Then $M$ is a subcomplex of $C$, $\Pi \circ i = id_M$ and $i\circ \Pi = id_C-(\Gamma\Delta+\Delta\Gamma)$.  
  
\begin{prop}\label{prop:grad:contract}
Let $C^\oplus(\sP)$ be a $\sP$-graded complex.  Let $\Gamma$ be a graded splitting homotopy and $\Pi = id_C-(\Gamma\Delta+\Delta\Gamma)$.  Define $M = im(\Pi)$.  Then $M$ is $\sP$-graded.
\end{prop}
\begin{proof}
Let $M_p = C_p\cap M$.  Then $$M=\bigoplus_{p\in \sP} M_p$$

Moreover $\Delta|_M$ is $\sP$-graded since $\Delta$ is $\sP$-graded.
\end{proof}

\begin{prop}
Graded splitting homotopies and $\sP$-graded reductions are in bijective correspondence.  Furthermore, perfect graded splitting homotopy and strict $\sP$-graded reductions are in bijective correspondence.
\end{prop}
\begin{proof}
The first result follows from Proposition~\ref{prop:grad:contract}.  A strict reduction implies the equation $$i\Delta^M\Pi = \Delta (i\circ \Pi) = \Delta(id_C-\Gamma\Delta-\Delta\Gamma) = \Delta-\Delta\Gamma\Delta$$  Since these maps are $\sP$-graded we have $$0=i_{pp}\Delta^M_{pp}\Pi_{pp}= (i\Delta^M\Pi)_{pp}  =(\Delta-\Delta\Gamma\Delta)_{pp}= \Delta_{pp}-\Delta_{pp}\Gamma_{pp}\Delta_{pp}$$

Conversely, let $\gamma$ be a perfect graded splitting homotopy.  The differential on $M=im(\Pi)$ is calculated as $\Delta^M = \Delta-\Delta\Gamma\Delta$.  Since the maps $\Delta$ and $\Gamma$ are $\sP$-graded (upper triangular), we have $$\Delta^M_{pp} = (\Delta-\Delta\Gamma\Delta)_{pp} = \Delta_{pp}-\Delta_{pp}\Gamma_{pp}\Delta_{pp} = 0$$
\end{proof}

Observe that in a strict reduction $\Phi_{pp}\colon M_p\to \ker \Delta_{pp}$ since $\Delta_{pp} \Phi_{pp} = \Phi_{pp}\Delta^M = 0$.  Therefore the images $\Phi_{pp}(M_p)$ are representatives of the homology $H_\bullet(C_p,\Delta_{pp})$.    We may also show that Conley complexes are minimal with respect to reductions.  Again, this is an elementary observation, but it mirrors the above Proposition~\ref{prop:cont:cyclic}.

\begin{prop}
Let $C^\oplus(C)$ be a Conley complex.  Any reduction
 \[
\xymatrixrowsep{0.03in}
\xymatrixcolsep{0.3in}
\xymatrix{
C^\oplus(\sP)\ar@(ur,ul)_{\Gamma}  \ar[r]<3pt>^{\Psi} & \ar[l]<3pt>^{\Phi} M^\oplus(\sP) 
}
\]
is strict. Moreover $M^\oplus(\sP)\cong C^\oplus(\sP)$.
\end{prop}
\begin{proof}
The differential on $M$ is calculated by $\Delta^M = \Delta-\Delta\Gamma\Delta$.  We have 
\[
\Delta^M_{pp} = (\Delta-\Delta\Gamma\Delta)_{pp} = \Delta_{pp}-\Delta_{pp}\Gamma_{pp}\Delta_{pp} = 0
\]
There $M^\oplus$ is a Conley complex.  Since $i$ and $\Pi$ is a chain homotopy equivalence, invoking Proposition~\ref{prop:grad:cmiso} shows that $M^\oplus(\sP)$ and $C^\oplus(\sP)$ are graded chain isomorphic. 
\end{proof}

For a tower of graded reductions, we have the following.  This is analogous to Proposition~\ref{prop:cont:tower}.

\begin{prop}\label{prop:cont:gtower}
For a tower of $\sP$-graded reductions:
\[
\xymatrixrowsep{0.03in}
\xymatrixcolsep{0.35in}
\xymatrix{
C^\oplus(\sP)    \ar@(ur,ul)_{\Gamma_0} \ar[r]<3pt>^{\Psi_0} & \ar[l]<3pt>^{\Phi_0 } M_0^\oplus(\sP)  \ar@(ur,ul)_{\Gamma_1} \ar[r]<3pt>^{\Psi_1} & \ar[l]<3pt>^{\Phi_1 }  \ldots \ar[r]<3pt>^{\Psi_{n-1}}  & \ar[l]<3pt>^{\Phi_{n-1} }M_{n-1}^\oplus(\sP)  \ar@(ur,ul)_{\Gamma_n}  \ar[r]<3pt>^{\Psi_n} & \ar[l]<3pt>^{\Phi_n} M_n^\oplus(\sP) 
}
\]
\begin{enumerate}
\item there is a reduction 
\begin{align}\label{dia:cont:tower}
\xymatrixrowsep{0.03in}
\xymatrixcolsep{0.35in}
\xymatrix{
C^\oplus(\sP)    \ar@(ur,ul)_{\Gamma_0} \ar[r]<3pt>^{\Psi_{n,0}}  & \ar[l]<3pt>^{\Phi_{n,0}} M_n^\oplus(\sP) 
}
\end{align}
with maps given by the formulas
\[
 \psi_{m,0} = \prod_{i=0}^n \Psi_i   \quad\quad   \Phi_{m,0} = \prod_{i=0}^m \Phi_i 
\quad\quad \Gamma_{n+1}= \Gamma + \sum_{i=0}^n \Phi_{i,0}\circ  \Gamma_i\circ \Psi_{i,0} \quad\quad 
\]

\item $\Gamma$ is a $\sP$-graded splitting homotopy and $\Gamma$ is perfect if any $\Gamma_i$ is perfect.

\end{enumerate}
\end{prop}


\subsection{Filtered Reductions}
 
An {\em $\sL$-filtered reduction} is a diagram in the category $\bLFC$ 

\[
\xymatrixrowsep{0.03in}
\xymatrixcolsep{0.3in}
\xymatrix{
f \ar@(ur,ul)_{h}  \ar[r]<3pt>^{\psi} & \ar[l]<3pt>^{\phi} m
}
\]

where $\psi\phi = id_M$ and $\phi\psi = id_C-(hd+dh)$.  Here  $M$ and $C$ are $\sL$-filtered chain complexes, $f$ and $g$ are $\sL$-filtered chain maps and $h$ is an $\sL$-filtered degree+1 map.  An $\sL$-filtered reduction is {\em strict} if $(M,d)$ is a Conley filtering.  The existence proof of~\cite{salamon}  furnish a filtered reduction.  

A filtered splitting homotopy is a degree+1 map $h:C\to C$ that is both a splitting homotopy and filtered with respect to $\sL$.  Again, one may define $\pi=id_C-(hd+dh)$ and $M=im(\pi)$.  Then $M$ is a subcomplex of $C$, $\pi i = id_M$ and $i\pi = id_C-(hd+dh)$.  The next result shows that $M$ is $\sL$-filtered. 

\begin{prop}\label{prop:filt:contract}
Let $f:\sL\to \Sub(C)$ be an $\sL$-filtered complex.  Let $h$ be a filtered splitting homotopy and $\pi = id-(hd+dh)$.  Define $M=im(\pi)$ and $m:\sL\to \Sub(M,d)$ as the map $$L\ni q\mapsto  \pi(f(q))\in \Sub(M)$$

Then $m:L\to \Sub(M,d)$ is an $\sL$-filtered complex.
\end{prop}
\begin{proof}
We begin with showing that $m(p \wedge q) = m(p)\wedge m(q)$.  We have that $$m(p\wedge q) = \pi(f(p\wedge q)) = \pi(f(p)\wedge f(q))$$

We must show that $\pi(f(p)\cap f(q)) = \pi(f(p))\cap \pi(f(q))$.  It is elementary set theory that $\pi(f(p)\cap f(q))\subseteq \pi(f(p))\cap \pi(f(q))$.  Now let $x\in \pi(f(p))\cap \pi(f(q))$.  Since $\pi(i(x))=x$ it suffices to show that $i(x)\in f(p)\cap f(q)$.  By definition there are $y\in f(p)$ and $y'\in f(q)$ such that $\pi(y) = x = \pi(y')$.  We have that $$i(x) = i(\pi(y)) = (id_C-hd - dh)(y)$$

The map on the right hand side is filtered since $h$ and $d$ are filtered.  Thus $i(x)\in f(p)$.  Similarly, $i(x)\in f(q)$.  Therefore $i(x)\in h(p)\cap h(q)$.  Finally, it is a straightforward consequence of the linearity of $\pi$ that $\pi(f(p\vee q)) = \pi(f(p))+\pi(f(q))$.


\end{proof}

For $f\in \bLFC(L,k)$ a filtered splitting homotopy $h:f\to f$ is {\em perfect} if for each $q\in J(L)$ the induced map $$h:C_q/C_{Pred(q)}\to C_q/C_{Pred(q)}$$ is perfect.  Such filtered homotopies give rise to Conley filterings.

\begin{cor}\label{cor:filt}
Filtered splitting homotopies and $\sL$-filtered reductions are in bijective correspondence.  Perfect filtered splitting homotopies are in bijective correspondence with strict $\sL$-filtered reductions.
\end{cor}
\begin{proof}
 The first statement follows from Proposition~\ref{prop:filt:contract}.  Consider $m:\sL\to \Sub(M,d^M)$ of the $\sL$-filtered reduction guaranteed by Corollary~\ref{cor:filt}.  If $h$ is perfect the differential $d^M$ must must obey $d^M(M_q) = (d-dhd)M_q\subseteq M_{Pred(q)}$.  Therefore $m:\sL\to \Sub(M,d^M)$ is a Conley filtered.
\end{proof}


The purpose of our computational section is to show how to furnish perfect filtered splitting homotopies.  


\begin{prop}
Let $f\colon\sL\to \Sub(C,d)$ be an $\sL$-Conley filtered complex.  Any filtered reduction
\[
\xymatrixrowsep{0.03in}
\xymatrixcolsep{0.3in}
\xymatrix{
f \ar@(ur,ul)_{h}  \ar[r]<3pt>^{\psi} & \ar[l]<3pt>^{\phi} m
}
\]
is strict.  Moreover $m$ and $f$ are filtered chain isomorphic.
\end{prop}
\begin{proof}
 The boundary operator is calculated as $d^M= d-dhd$.  Since $f$ is Conley-filtered  we have $d^M(M_q) = (d-dhd)M_q\subseteq M_{Pred(q)}$.  Thus $m$ is Conley-filtered.  Finally, $m$ and $f$ are filtered chain isomorphic by Proposition~\ref{prop:filt:cmiso}.
\end{proof}

For a tower of $\sL$-filtered reductions, we have:




\begin{prop}\label{prop:cont:tower}
For a tower of $\sL$-filtered reductions:
\[
\xymatrixrowsep{0.03in}
\xymatrixcolsep{0.35in}
\xymatrix{
f  \ar@(ur,ul)_{\gamma_0} \ar[r]<3pt>^{\psi_0} & \ar[l]<3pt>^{\phi_0 } m_0 \ar@(ur,ul)_{\gamma_1} \ar[r]<3pt>^{\psi_1} & \ar[l]<3pt>^{\phi_1 }  \ldots \ar[r]<3pt>^{\psi_{n-1}}  & \ar[l]<3pt>^{\phi_{n-1} }m_{n-1} \ar@(ur,ul)_{\gamma_n}  \ar[r]<3pt>^{\psi_n} & \ar[l]<3pt>^{\phi_n} m_n
}
\]

\begin{enumerate}
\item there is a reduction 
\begin{align}\label{dia:cont:tower}
\xymatrixrowsep{0.03in}
\xymatrixcolsep{0.45in}
\xymatrix{
f  \ar@(ur,ul)_{\gamma_{n+1} } \ar[r]<3pt>^{\psi_{n,0}} & \ar[l]<3pt>^{\phi_{n,0}} m_n
}
\end{align}
with the maps given by the formulas
\[
 \psi_{m,0} = \prod_{i=0}^n \psi_i  \quad\quad  \phi_{m,0} = \prod_{i=0}^m \phi_i 
\quad\quad
\gamma_{n+1}= \gamma + \sum_{i=0}^n \phi_{i,0}\circ  \gamma_i\circ \psi_{i,0} \quad\quad 
\]

\item $\gamma$ is a splitting homotopy and $\gamma$ is perfect if any $\gamma_i$ is perfect.

\end{enumerate}
\end{prop}


\subsection{Graded Morse Theory}\label{sec:gmt}

A graded version of discrete Morse theory may be done straightforwardly.   Let $f:\cX\to \sP$ be a $\sP$-graded cell complex.  
\begin{defn}
{\em
We say that $(A,w)$ is a {\em graded acyclic partial matching} if it satisfies $w(Q)=K$ only if $K,Q\in X_p$ for some $p\in P$.  That is, matchings may only occur in the same fiber of the valuation.  
}
\end{defn}

The idea of graded macthings can be found many places in the literature, for instance, see~\cite{mn} and~\cite[Patchwork Theorem]{koz}.

\begin{prop}
Let $f\colon \cX \to \sJ(\sL)$ be a graded cell complex and $(\cA,w)$ a graded acyclic matching.  Then the associated splitting homotopy $\Gamma$ for $(C^\oplus(\sJ(\sL)),\Delta)$ is $\sP$-graded and fits into a graded reduction.
\end{prop}
\begin{proof}
By Proposition~\ref{prop:matchinghomotopy} there is a splitting homotopy $\Gamma:C(X)\to C(X)$ associated to the matching $(\cA,w)$.  Let $p\in J(L)$.  Consider $(\cA_p,w_p)$ the matching restricted to the fiber $X_p = f^{-1}(p)$.  We have $$\cA_p = \cA\cap \cX_p\quad\quad \quad \quad  w_p:\cQ\cap X_p\to \cK\cap \cX_p$$

By assumption this is an acyclic partial matching on the fiber $X_p$.  By Proposition~\ref{prop:matchinghomotopy} there is a unique splitting homotopy $\gamma_p:C(X_p)\to C(X_p)$.    Therefore we have $$C(X) = \bigoplus_{p\in \sP} C(X_p) \quad\quad \Gamma_{pp} = \gamma_p$$ 

%Thus $\Gamma$ is diagonal, and in particular a graded-splitting homotopy.
{\bf still needs proof that $\Gamma$ is graded, }

Since $\Gamma$ is graded by Proposition~\ref{prop:grad:contract} there is an associated reduction
\[
\xymatrixrowsep{0.03in}
\xymatrixcolsep{0.3in}
\xymatrix{
 C^\oplus(\sP) \ar@(ur,ul)_{\Gamma}  \ar[r]<3pt>^{\Psi} & \ar[l]<3pt>^{\Phi}A^\oplus(\sP) 
}
\]
\end{proof}

In general, one needs a basis/graded basis on which to operate with discrete Morse theory.  For instance in~\cite{koz2} free chain complexes are used and based complexes are used in~\cite{sko}.  Typically this comes from the input data.  Otherwise, these exist via Theorem~\ref{prop:bases}.




\subsection{Connection Matrix Algorithm}\label{sec:reductions:alg}

In this section we introduce the algorithm for computing a connection matrix based on the Morse theory described above.

{\bf Algorithm}
\begin{enumerate}
\item Given a graded cell complex $f_0\colon \cX\to \sJ(\sL)$ as input
\item {\bf do}
\item Apply~\cite[Algorithm 3.6]{focm} to the fibers $\{X_q\colon X_q = f^{-1}(q)\}$ to produce a graded acyclic partial matching $(\cA,w:\cQ\to \cK)$
\item Apply~\cite[Algorithm 3.12]{focm} to produce a graded splitting homotopy $\Gamma:C^\oplus(\sP)\to C^\oplus(\sP)$ and graded reduction
\[
\xymatrixrowsep{0.03in}
\xymatrixcolsep{0.3in}
\xymatrix{
 C^\oplus(\sP) \ar@(ur,ul)_{\Gamma}  \ar[r]<3pt>^{\Psi} & \ar[l]<3pt>^{\Phi}A^\oplus(\sP) 
}
\]
\item {\bf while $|\cA|<|\cX|$}
\end{enumerate}


\begin{thm}
The above algorithm terminates, i.e. after finitely many iterations the sequence $(f_n)$ stabilizes to a final $L$-filtered complex $f_\infty$.  Moreover, $f_\infty$ is a Conley complex.
\end{thm}
\begin{proof}

\end{proof}

\begin{rem}
Morse theory operates on graded complexes.  However it also produces results for lattice-filtered complexes.  The connection matrix algorithm produces a graded basis, where each fiber gives a basis for the homology, and the basis is invariant under the differential.
\end{rem}






%%
%%
%!TEX root = ../main.tex

\section{Persistent Homology}\label{sec:PH}

Let $f:X\to \sP$ be a graded cell complex.  Let $\sT=\{0,\ldots, |\sP|\}\subseteq \N$ with order $\leq$ inherited from $(\N,\leq)$.  A {\em linear extension of $\sP$} is a bijective poset morphism $\sP\to \sT$.  By Birkhoff's theorem the composition $X\xrightarrow{f}\sP\to \sT$ induces a lattice morphism $\sO(f):\sO(\sT)\to Sub_{Cl}(X,\leq)$.  The map $\sO(f)$ is called a {\em filtration} as $\sO(\sT)$ is of the form
\[
\ldots \subseteq [0,n] \subseteq [0,n+1]\subseteq \ldots
\]

and $im\sO(f)$ 
\[
\ldots \subseteq X^n \subseteq X^{n+1}\subseteq \ldots
\]

The inclusion $\iota:X$
%%
%!TEX root = ../main.tex

\subsection{Persistence through Morse Theory and Reductions}



The classical notion of barcode is due to Carlsson et al.  Our notion of {\em barcode} is a chain-level version.  

%\begin{defn}[Barcode]
%{\em
%
%}
%\end{defn}


There are algorithms for computing persistent homology and barcodes based on discrete Morse theory, these can be found in~\cite[Algorithm 7]{dw} and \cite[Algorithm 2]{gbmr}.  

The basic idea is to construct a matching by pairing a cell with a face of maximal valuation.  For instance, if $\partial(\xi)\neq 0$ consider $Q = \{f(\xi'):\kappa(\xi,\xi'\neq 0\}$.  We'll pair $\xi$ with some $\xi'$ such that $\kappa(\xi,\xi')\neq 0$ such that $f(\xi')=\max Q$.  Construct the matching as follows
\begin{align*}
\cK = \{\xi\} \quad\quad \cQ &= \{\xi' \} \quad\quad w(\xi')=\xi \\
\quad\quad \cA &= \cX~\backslash~(\cK\sqcup \cQ)
\end{align*}

Notice that this is not a graded matching, since we are pairing outside of the filtration levels.  For this matching there is an associated splitting homotopy $\gamma$.  The map $\gamma$ is not a graded homotopy, however the associated projection $\pi$ is graded.

\begin{rem}
The key point of persistence is that you must match a cell with face of maximal valuation in order to get a graded chain map.  This guarantees that you are computing an isomorphism and a true splitting.
\end{rem}

\begin{prop}
Let $\gamma$ be the associated matching, then $\gamma$ is a splitting homotopy and $$\pi=id-\partial \gamma-\gamma \partial $$ is a $\sT$-graded map.  
\end{prop}
\begin{proof}
We describe $\pi$ on the basis of distinguished cells.     Then
\begin{align*}
\pi|_\cA = id \quad\quad
\pi(\xi) =  0 \quad\quad \pi(\xi') = \xi' - \partial (\xi)
\end{align*}

Since $\xi'$ is maximal with respect to $Q$, the chain $\xi-\partial(\xi)$ consists of cells with valuations less than $\xi'$.  Therefore $\pi$ is $\sT$-graded.
\end{proof}

Although $\pi$ is $\sT$-graded the reduction itself is not $\sT$-graded since $\gamma$ is not a graded homotopy.  Since $\pi$ is graded the graded complex $M^\oplus(\sT)$ splits as $im(\pi)$ and $\ker \pi$.  $\ker \pi$ is an acyclic complex, which stores the persistence pair $(\xi',\xi)$.  
   

{\bf Algorithm}
\begin{enumerate}
\item Given a filtration $f:\cX\to \sT$ where $\sT = \{0,\ldots, n\}$
\item {\bf for i=0\ldots n do}
\item Find a matching $\cA_i,w_i:\cQ_i\to \cK_i$ using Algorithm that matches king to highest queen in boundary
\item Apply~\cite[Algorithm 3.12]{focm} to produce a splitting homotopy $\gamma_i:C^\oplus(\sT)\to C^\oplus(\sT)$ and chain reduction
\[
\xymatrixrowsep{0.03in}
\xymatrixcolsep{0.3in}
\xymatrix{
M_i^\oplus(\sT)  \ar[r]<3pt>^{\phi} & \ar[l]<3pt>^{\psi} M_{i-1}^\oplus(\sT) \ar@(ul,ur)^{\gamma}
}
\]
%\item Apply~\cite[Algorithm 3.6]{focm} to the fibers $\{X_q\colon X_q = f^{-1}(q)\}$ to produce a graded acyclic partial matching $(\cA,w:\cQ\to \cK)$
\end{enumerate}

The set $\{\gamma_i\}$ of splitting homotopies produced by the algorithm contribute to a tower of reductions:
\[
\xymatrixrowsep{0.03in}
\xymatrixcolsep{0.35in}
\xymatrix{
H(C^\oplus(\sT)) \ar[r]^{\cong} & \ar[l]  H(M_n) \ar[r]<3pt>^{\phi_n} & \ar[l]<3pt>^{\psi_n } M_{n-1} \ar@(ul,ur)^{\gamma_n} \ar[r]<3pt>^{\phi_{n-1}} & \ar[l]<3pt>^{\psi_{n-1}}  \ldots \ar[r]<3pt>^{\phi_1}  & \ar[l]<3pt>^{\psi_1 }M_0 \ar@(ul,ur)^{\gamma_1}  \ar[r]<3pt>^{\phi_0} & \ar[l]<3pt>^{\psi_0} C \ar@(u,r)^{\gamma_0}
}
\]

With our previous results we have a single reduction, where $H^\oplus(\sT) = im(\Psi^{n+1})$.
 \[
\xymatrixrowsep{0.03in}
\xymatrixcolsep{0.3in}
\xymatrix{
H(C^\oplus(\sT)) \ar[r]<3pt>^{\Phi^{n+1}}   & \ar[l]<3pt>^{\Psi^{n+1}} C^\oplus(\sT) \ar@(ul,ur)^{\Gamma} 
}
\]
We have the splitting $C = im(\Psi^{n+1})\oplus \ker(\Psi^{n+1})$.  Moreover, an inductive argument shows that 
\[
\ker(\Psi^{n+1}) = \bigoplus_{i=0}^n \ker (\Psi^i)
\]

Each $\ker\psi^i$ is an acyclic $\sT$-graded subcomplex of the form 
\[
\ldots 0 \to k\langle \xi^i\rangle \xrightarrow{ \kappa (\xi^i,\eta^i)} k\langle \eta^i\rangle \to 0\to \ldots
\]

Since $\nu$ takes values in $\R$, we may form the difference between valuations $\nu(\xi^i) - \nu(\xi'^i)$.  This is often called the {\em persistence}.  

\begin{rem}
Here's is a remark note for the paper but for author's benefit:
The differential $\Delta|_{\ker(\psi^{n+1})}$ is the diagonal matrix of the form
\[
\begin{blockarray}{cccccc}
& \xi^0 & \xi^1 & \ldots & \xi^{n-1} & \xi^n \\
\begin{block}{c(ccccc)}
 \eta^0 & \kappa(\xi^0,\eta^0) & 0 & \ldots & 0 & 0 \\
 \eta^1 & 0 & \kappa(\xi^1,\eta^1) & \ldots & 0 & 0  \\
 \vdots &  \vdots & \vdots & & \vdots & \vdots  \\
  \eta^{n-1} & 0 & 0 & \ldots & \kappa(\xi^{n-1},\eta^{n-1}) & 0  \\
  \eta^n & 0 & 0 & \ldots & 0 & \kappa(\xi^n,\eta^n)  \\
\end{block}
\end{blockarray}
 \]

\end{rem}

If one runs the Algorithm on a general $\sT$-filtered cell complex, then it is possible to obtain subcomplexes of the form
 \[
\ldots 0 \to k\langle \xi^i\rangle \xrightarrow{ \kappa (\xi^i,\eta^i)} k\langle \eta^i\rangle \to 0\to \ldots
\]
where $\nu(\xi^i)=\nu(\eta^i)$.  This corresponds to a point of zero persistence.  We can now show that if one runs the Algorithm on a Conley complex, the only pairs found will be of nonzero persistence.  Consider

 \[
\xymatrixrowsep{0.03in}
\xymatrixcolsep{0.3in}
\xymatrix{
H(C^\oplus(\sT)) \ar[r]^{\cong} &\ar[l]  H(M^\oplus(\sT))  \ar[r]<3pt>^{\phi} & \ar[l]<3pt>^{\psi} M^\oplus(\sT) \ar@(ul,ur)^{\gamma}  \ar[r]<3pt>^{\Phi} & \ar[l]<3pt>^{\Psi} C^\oplus(\sT) \ar@(ul,ur)^{\Gamma} 
}
\]

 The right hand diagram is a $\sT$-graded reduction, with $\Gamma$ a splitting homotopy and $M^\oplus(\sT)$ a Conley complex.  The left hand diagram is a chain reduction in $Ch(k)$, $\gamma$ holds persistence pairs and $H(C^\oplus(T))$ contains infinite persistence.

 \[
\xymatrixrowsep{.4in}
\xymatrixcolsep{0.4in}
\xymatrix{
&  \ar[dl]<3pt>^{\psi}  M^\oplus(\sT)  \ar[dr]<3pt>^{\phi}  &\\
H(C^\oplus(\sT)) \ar[ur]<3pt>^{\phi} \ar[rr]<3pt>^{\phi}  & & \ar[ll]<3pt>^{\Psi} C^\oplus(\sT) \ar@(ul,ur)^{\Gamma} \ar[ul]<3pt>^{\psi}
}
\]

The left hand diagram is a chain reduction in $Ch(k)$, $\gamma$ holds persistence pairs and $H(C^\oplus(T))$ contains infinite persistence.   Therefore, we think of the sequence of two reductions:
 


Persistence as invariant of graded and filtered homotopy equivalence.  Given the data $H(C^\oplus(\sT)$ and $\gamma$ we can reconstruct the filtered homotopy type by a direct sum of the cyclic (homology) and acyclic complexes (associated to $\gamma$).

The difference between the two approaches, i.e. computing connection matrix first, is that one does not get pairs $(a,\gamma(a))$ that are on the same level of the filtration in the second approach.

With this decomposition, we can see the following theorem, akin to~\cite{usher}.

\begin{thm}
Algorithm produces a graded basis for the persistent homology lattice $\PH$.
\end{thm}

\begin{thm}\label{thm:pers:inv}
Consider the category $\bLFC(\sT,k)$.  
\begin{enumerate}
\item Concise barcodes classify up to chain homotopy equivalence, i.e. isomorphism in $\bKLFC(\sT,k)$.

\item Verbose barcodes classify up to chain isomorphism, i.e. isomorphism in $\bLFC(\sT,k)$.
\end{enumerate}
\end{thm}
\begin{proof}


\end{proof}

\begin{rem}
Our Algorithm  for computing the connection matrix (i.e. the right hand side of the reduction) is akin to phrase $r=1$ for computing persistence via the sweeping method for spectral sequence~\cite{} - (i.e. doing persistence only on diagonal blocks and computing all pairs with zero persistence.  This is also similar to computing persistence in chunks by Bauer et al~\cite{}. 
\end{rem}

\begin{rem}
In fact, one can re-arrange the Morse theory to produce a sequence of $r$-connection matrices, meaning boundary maps that are zero along the $0$ through $r$ diagonals, e.g. a strictly upper triangular matrix is a $0$-connection matrix.  This computes persistence pairs in order, and computing the connection matrix is thus computing all pairs with zero persistence.  The tower is then ordered in terms of length of persistence intervals, and on the very left hand side is infinite persistence.
\end{rem}



\begin{rem}
Algorithm is akin to obtaining sequences of connection matrices via the sweeping algorithm of~\cite{}.    In that work they attempt continuation by algebraic cancellation.
\end{rem}


\subsection{Homotopy Theory for Filtrations}

We outline a homotopy theory for filtrations.  In this case, we enlarge the class of homotopies we consider.   Let $(\sT,\leq) = \{0,1,\ldots,n\}$.

\begin{defn}
{\em
Let $f\colon \sO(\sT) \to \Sub(C,d)$. Let $C^i=f(\downarrow i)$. We say that a map $\gamma$ has order $s$ if
\[
\gamma(C^i)\subseteq C^{i+s}
\]

%$$\gamma( f\downarrow (i)) \subseteq f( \downarrow (i+s))$$
}
\end{defn}
\begin{rem}
Filtered maps have order zero.
\end{rem}


We can introduce a new equivalence relation $\sim_s$ where $\gamma$ has order $s$ and
\[
\phi-\psi = \gamma d+d\gamma\implies \phi\sim_s \psi \
\]  
The associated homotopy category $\bKGCC_{\leq s}(\sT,k)$ has hom-sets are defined as
\[
Hom_{\bKGCC_{\leq s}}(\phi,\psi) = Hom_{\bGCC}(\phi ,\psi )/\sim_s
\]

Notice that $\bKGCC(\sT,k) = \bKGCC_{\leq 0}(\sT ,k)$.   For each category there is a notion of connection matrix, which is all zero up to the appropriate off-diagonal.  An $s$-connection matrix is a $\sP$-graded map that satisfies 
\[
\Delta_{pq} = 0 \text{ for all } q-p\leq s
\]

An $s$-connection matrix has zeroes along the diagonals from $0$ to $s$.  A connection matrix is a $0$-connection matrix.


\begin{prop}
Let $\Gamma$ be order $s$ and $\Delta$ be an $s$-connection matrix.  Then $\Gamma\Delta$ and $\Delta\Gamma$ are strictly upper triangular.
\end{prop}



\begin{prop}
  Any reduction in $\bGCC (\sT,k)$ where the $\sT$-graded chain complex $C^\oplus(\sT)$ with an $s$-connection matrix, then
 \[
\xymatrixrowsep{0.03in}
\xymatrixcolsep{0.3in}
\xymatrix{
C^\oplus(\sT) \ar@(ur,ul)_{\Gamma}  \ar[r]<3pt>^{\Psi} & \ar[l]<3pt>^{\Phi} M^\oplus(\sT) 
}
\]
 $\Delta^M$ has an $s$-connection matrix and $M^\oplus(\sT)\cong C^\oplus(\sT)$.
\end{prop}

We can compute 




We can cook up the following sequence of reductions.  Each individual reduction is such that $\gamma_s$ is order $s$ and $\Delta_s$ is an $s$-connection matrix.   There the $s$-th reduction is an an isomorphism in $\bKGCC_{\leq i}$.  
Each $\gamma_i$ holds pairs with persistence $i$.


\[
\xymatrixrowsep{0.03in}
\xymatrixcolsep{0.35in}
\xymatrix{
C^\oplus(\sT)   \ar@(ur,ul)_{\Gamma_0} \ar[r]<3pt>^{\Psi_0} & \ar[l]<3pt>^{\Phi_0 } M_0^\oplus(\sT) \ar@(ur,ul)_{\Gamma_1} \ar[r]<3pt>^{\Psi_1} & \ar[l]<3pt>^{\phi_1 }  \ldots \ar[r]<3pt>^{\Psi_{n-1}}  & \ar[l]<3pt>^{\Phi_{n-1} }M_{n-1}^\oplus(\sT) \ar@(ur,ul)_{\Gamma_n}  \ar[r]<3pt>^{\Psi_n} & \ar[l]<3pt>^{\Phi_n} M_n^\oplus(\sT)
}
\]



\begin{rem}
The is equivalent to the sweeping algorithm for spectral sequences.
\end{rem}




%\subsection{Lattices}
%
%
%For lattices that are not totally ordered we may do the same for any $a< b$ via 
%\[
%\frac{Z_a}{Z_a\wedge B_b} \cong \frac{Z_a\vee B_b}{B_b}
%\]
%
%It is straightforward that for any filtered homotopy equivalence we have isomorphisms on the persistent homology groups.
%
%\begin{prop}
%If $f\cong g$ in ${\bf KLFC}$ then $f$ and $g$ have the same persistent homology.
%\end{prop}
%
%
%If $\sL$ is totally ordered we may apply the persistence equivalence theorem of~\cite{} to get that their diagrams are the same.
%
%\begin{cor}
%Same persistence diagram.
%\end{cor}
%
%In fact, there should be a category that holds all persistent homology groups and morphisms between them, see Zeeman's work.  Filtered homotopy equivalence should imply isomorphism of categories.
%
%\subsection{Persistence as invariant of graded homotopy equivalence}
%
%In Usher and Zhang paper they show that barcodes classify chain homotopy equivalence and chain isomorphism.
%
%Is there some analogous result for CM and lattices? 

%
%
%Consider $h\colon \sL\to Sub(C,d)$.  
%
%
%Let $\sT=\{0,\ldots, |\sP|\}\subseteq \N$ with order $\leq$ inherited from $(\N,\leq)$.  A {\em linear extension of $\sP$} is a bijective poset morphism $\sP\to \sT$.  By Birkhoff's theorem the composition $X\xrightarrow{f}\sP\to \sT$ induces a lattice morphism $\sO(f):\sO(\sT)\to Sub_{Cl}(X,\leq)$.  The map $\sO(f)$ is called a {\em filtration} as $\sO(\sT)$ is of the form
%\[
%\ldots \subseteq [0,n] \subseteq [0,n+1]\subseteq \ldots
%\]
%
%and $im\sO(f)$ 
%\[
%\ldots \subseteq X^n \subseteq X^{n+1}\subseteq \ldots
%\]
%
%The inclusion $\iota:X$

%%%
%%!TEX root = ../main.tex

\section{Conley Theory for Maps}\label{sec:maps}

The Conley theory for discete dynamical systems as been developed by Robbin and Salamon and Franks and Richeson.
In the case of a discrete dynamical system the appropriate notion is to have a to endow the chain data with an automorphism.    Let $F:f\to f$ be a filtered automorphism.
\begin{prop}
Let $F\colon f\to f$ be a $\sL$-filtered endomorphism.  For the chain contraction $\bCF(\sL)$

\[
\xymatrixrowsep{0.03in}
\xymatrixcolsep{0.3in}
\xymatrix{
m  \ar[r]<3pt>^{\phi} & \ar[l]<3pt>^{\psi} f \ar@(u,r)^{h}
}
\]
The map $G\colon m\to m$ defined by 
\[
G:= \psi\circ F \circ  \phi
\]
is a $\sL$-filtered automorphism of $m$.  Moreover, it induces isomorphisms of the associated braids.
\end{prop}
\begin{proof}
We first show that $G$ is a $\sL$-filtered endomorphism of $m$.
\[
d^m G - G d^m = d^m (\psi\circ  F\circ \phi) - (\psi \circ F\circ  \phi) d^m = \psi d^f F \phi - \psi F d^f \phi = \psi(d^f F- F d^f)\phi = 0
\]

Moreover, we have 
\begin{align*}
\phi \circ G &= (\phi \circ \psi) \circ F\circ \phi = (id_f-d^fh-hd^f) \circ F \circ \phi = F\circ \phi - d^f(hF\phi) - (hF\phi)d^m\\
G \circ \psi &= \psi \circ F\circ (\phi \circ \psi) =  \psi \circ F \circ (id_f-d^f h - hd^f) = \psi\circ F -  d^m( \psi F h) - (\psi F h )d^f
\end{align*}

Therefore $\phi\circ G\sim F\circ \phi$ and $G\circ \psi \sim \psi \circ F$ where $\sim$ means $\sL$-filtered chain homotopic.  


\end{proof}

{\bf Is $G$ an automorphism?}


%\subsection{Preliminaries}
%If $C$ is a category we may define a new category ${\bf End}(C)$ as follows: the objects of ${\bf End}(C)$ are pairs $(A,a)$ where $A$ is an object of $C$ and $a\colon A\to A$ is a morphism of $C$.  A morphism of $\psi\colon (A,a)\to (B,b)$ of $C$ is a morphism $\psi\colon A\to B$ such that $\psi\circ a = b\circ \psi$.
%
%\begin{ex}
%Consider ${\bf Ch}(k)$.  $End(Ch(k))$.
%\end{ex}
%
%
%
%\subsection{Endomorphisms of Chains}
%
%
%An object in ${\bf End}(\bCF)$ is a pair $(f,\phi)$ with $f\colon \sL\to Sub(C,d)$ and $F\colon C\to C$ with 
%\[
%F(f(q))\subseteq f(q)
%\]


%%%





\begin{thebibliography}{50}

\bibitem{cmdb}
Z.~Arai, W.~Kalies, H.~Kokubu, K.~Mischaikow, H.~Oka, and P.~Pilarczyk.
\newblock{A Database Schema for the Analysis of Global Dynamics of Multiparameter Systems},
\newblock{\em SIAM Journal of Applied Dynamical Systems}, 8: 757--789, (2009).

\bibitem{bk}
H.~Ban and W.~Kalies, 
\newblock{ A computational approach to Conley's decomposition theorem}
\newblock{\em  Journal of Computational Nonlinear Dynamics}, 1:312--319, (2006).


\bibitem{bar}
M.~Barakat and S.~Maier-Paape.
\newblock{Computation of connection matrices using the software package conley}
\newblock{\em Internatational Journal of Bifurcation and Chaos}, 19(9):3033--3056 (2009).

\bibitem{bar2} 
M.~Barakat and D.~Robertz.
\newblock{ conley: computing connection matrices in Maple},
\newblock{\em Journal of Symbolic Computation} 44(5): 540--557 (2009).

\bibitem{barnes:lambe}
Barnes, D.W. and Lambe, L.A., 1991. 
\newblock{A fixed point approach to homological perturbation theory.}
\newblock{ Proceedings of the American Mathematical Society, 112(3), pp.881-892.}

\bibitem{bauer}
Bauer, U. and Edelsbrunner, H., 2016. 
\newblock{The Morse theory of Cech and Delaunay complexes.}
\newblock{ Transactions of the American Mathematical Society.}

\bibitem{bl}
J.B.~van den Berg and J.P.~Lessard. 
\newblock{Rigorous numerics in dynamics.}
\newblock{ Notices of the AMS} 62.9 (2015).

\bibitem{bush}
Bush, J., Cowan, W., Harker, S. and Mischaikow, K., 2016. 
\newblock{Conley--Morse Databases for the Angular Dynamics of Newton's Method on the Plane. SIAM Journal on Applied Dynamical Systems, 15(2), pp.736-766.}

\bibitem{cmdbchaos}
J.~Bush, M.~Gameiro, S.~Harker, H.~Kokubu, K.~Mischaikow, I.~Obayashi, P.~Pilarczyk.
\newblock{Combinatorial-topological framework for the analysis of global dynamics},
\newblock{\em Chaos}, 22, (2008).

\bibitem{bm}
Bush, J. and Mischaikow, K., 2014. 
\newblock{Coarse dynamics for coarse modeling: An example from population biology. Entropy, 16(6), pp.3379-3400.}

\bibitem{csz}
Carlsson, G., Singh, G., Zomorodian, A. J. (2010). 
\newblock{Computing multidimensional persistence. Journal of Computational Geometry, 1(1), 72-100.}


\bibitem{conley}
C.~Conley.
\newblock{\em  Isolated invariant sets and the Morse index},
\newblock{American Mathematical Society}, 1978.

\bibitem{cmdbProject}
Conley-Morse Database.
\newblock{\url{http://chomp.rutgers.edu/Projects/Databases_for_the_Global_Dynamics.html}}

\bibitem{dsgrnProject}
Dynamic Signatures of Genetic Regulatory Networks.
\newblock{\url{http://chomp.rutgers.edu/Projects/DSGRN/}.}

\bibitem{dsgrn}
B.~Cummins, T.~Gedeon, S.~Harker, K.~Mischaikow and K.~Mok,
\newblock{ Combinatorial Representation of Parameter Space for Switching Systems}
\newblock{ arXiv preprint arXiv:1512.04131. 2015}

\bibitem{davey:priestley}
Davey, B.A. and Priestley, H.A., 2002.
\newblock{ Introduction to lattices and order.}
\newblock{ Cambridge university press.}

\bibitem{dw}
Dlotko, P. and Wagner, H., 2012. 
\newblock{Computing homology and persistent homology using iterated Morse decomposition.}
\newblock{arXiv preprint arXiv:1210.1429.}

\bibitem{dhmo}
S.~Day, Y.~Hiraoka, K.~Mischaikow, and T.~Ogawa.  
\newblock{Rigorous numerics for global dynamics: A study of the Switch-Hohenberg equation}
\newblock{SIAM Journal of Applied Dynamical Systems}, 4: 1--31 (2005).

\bibitem{floer}
Floer, A., 1989. 
\newblock{Witten�s complex and infinite-dimensional Morse theory.}
\newblock{ J. Differential Geom, 30(1), pp.207-221.}

\bibitem{fran}
R.~Franzosa.
\newblock{The connection matrix theory for {M}orse decompositions},
\newblock{\em  Transactions of the AMS},  311(2): 561 -- 592 (1989).

\bibitem{fran2}
R.~Franzosa.
\newblock{Index filtrations and the homology index braid for partially ordered Morse decompositions},
\newblock{\em Transactions of the AMS}, 298(1):193 (1986).

\bibitem{fran3}
R.~Franzosa.
\newblock{The continuation theory for Morse decompositions and connection matrices},
\newblock{\em Transactions of the AMS}, 310(2):781 (1988).

\bibitem{atm}
Franzosa, R. and Mischaikow, K., 1998. 
\newblock{Algebraic transition matrices in the Conley index theory.}
\newblock{ Transactions of the American Mathematical Society}, 350(3), pp.889-912.

\bibitem{gh}
T.~Gedeon, S.~Harker, H.~Kokubu, K.~Mischaikow, and H.~Oka,
\newblock{Global dynamics for steep nonlinearities in two dimensions.}
\newblock{ Physica D: Nonlinear Phenomena.} (2016).

\bibitem{gelfand}
Gelfand, S.I. and Manin, Y.I., 2013. 
\newblock{Methods of homological algebra.} Springer Science \& Business Media.

%\bibitem{lya}
%Goullet, A., Harker, S., Mischaikow, K., Kalies, W. and Kasti, D., 2015. 
%\newblock{Efficient computation of Lyapunov functions for Morse decompositions. DCDS-S.}

\bibitem{gbmr}
Gonzalez-Lorenzo, A., Bac, A., Mari, J. L., \& Real, P. (2017). 
\newblock{Allowing cycles in discrete Morse theory.}
\newblock{Topology and its Applications, 228, 1-35.}


\bibitem{focm}
S.~Harker, K.~Mischaikow, M.~Mrozek and V.~Nanda.
\newblock{Discrete Morse Theoretic Algorithms for Computing Homology of Complexes and Maps}
\newblock{\em Foundations of Computational Mathematics}, (2013).

\bibitem{braids}
S.~Harker, K.~Mischaikow, K.~Spendlove, and R.~Vandervorst.
\newblock{Computing Connection Matrices for a Morse Theory on Braids}
\newblock{Preprint} 2017.

\bibitem{kmm}
Kaczynski, T., Mischaikow, K. and Mrozek, M., 2006. 
\newblock{Computational homology (Vol. 157). Springer Science \& Business Media.}


\bibitem{kmv}
W.D.~Kalies, K.~Mischaikow, and R.C.A.M.~Van der Vorst, 
\newblock{An algorithmic approach to chain recurrence},
\newblock{\em Foundations of Computational Mathematics} 5:409--449 (2005).

\bibitem{lsa}
W.D.~Kalies, K.~Mischaikow, and R.C.A.M.~van der Vorst,
\newblock{Lattice structures for attractors I}
\newblock{Journal of Computational Dynamics 1, pp. 307-338, 2014.}

\bibitem{lsa2}
Kalies, W.D., Mischaikow, K. and Vandervorst, R.C.A.M., 2016. 
\newblock{Lattice structures for attractors II.}
\newblock{ Foundations of Computational Mathematics,} 16(5), pp.1151-1191.

\bibitem{kmv3}
Kalies, W.D., Mischaikow, K. and Vandervorst, R.C.A.M., 2016. 
\newblock{Lattice structures for attractors III.}
\newblock{ Preprint} 2018.


\bibitem{koz}
D.~Kozlov.
\newblock{\em Combinatorial Algebraic Topology.}
\newblock{ Algorithms and Computation in Mathematics, vol. 21.}, Springer, Berlin (2008).

\bibitem{koz2}
Kozlov, D.N., 2005. 
\newblock{Discrete Morse theory for free chain complexes. }
\newblock{Comptes Rendus Mathematique, 340(12), pp.867-872.}

\bibitem{lefschetz}
Lefschetz,S. 
\newblock{AlgebraicTopology.American Mathematical Society Colloquium Publications, vol.27.}
\newblock{American Mathematical Society,} New York (1942)

\bibitem{mpmw}
Maier-Paape, S., Mischaikow, K. and Wanner, T., 
\newblock{2007. Structure of the attractor of the Cahn-Hilliard equation on a square.}
\newblock{\em International Journal of Bifurcation and Chaos}, 17(4), pp.1221-1263.

\bibitem{mcmodels}
McCord, C., 2000. 
\newblock{Simplicial models for the global dynamics of attractors.}
\newblock{ Journal of Differential Equations,} 167(2), pp.316-356.

\bibitem{scalar}
McCord, C. and Mischaikow, K., 1996. 
\newblock{On the global dynamics of attractors for scalar delay equations. }
\newblock{Journal of the American Mathematical Society}, 9(4), pp.1095-1133.

\bibitem{mm}
Mischaikow, K. and Mrozek, M., 1995. 
\newblock{Chaos in the Lorenz equations: a computer-assisted proof. Bulletin of the American Mathematical Society, 32(1), pp.66-72.}

\bibitem{mn}
K.~Mischaikow and V.~Nanda. 
\newblock{ Morse Theory for Filtrations and Efficient Computation of Persistent Homology.}
\newblock{\em Discrete and Computational Geometry}, 50(2):330--353, 2013.

\bibitem{mc} 
C.~McCord, 1988. 
\newblock{Mappings and homological properties in the Conley index theory. Ergodic Theory and Dynamical Systems, 8(8*), pp.175-198.}

\bibitem{mcr}
C.~McCord and J.~Reineck.
\newblock{Connection matrices and transition matrices},
\newblock{\em Banach Center Publications}, 47(1): 41--55 (1999).

\bibitem{richeson}
D. Richeson. 
\newblock{Connection matrix pairs for discrete Conley index.}
\newblock{ Thesis, Northwestern University, 1998.}

\bibitem{robbin:salamon2} Robbin, J.W. and Salamon, D.A., 
\newblock{ 1992. Lyapunov maps, simplicial complexes and the Stone functor. }
\newblock{Ergodic Theory and Dynamical Systems}, 12(01), pp.153-183.

\bibitem{roman}
Roman, S., 2008. 
\newblock{Lattices and ordered sets.} Springer Science \& Business Media.

\bibitem{smale}
Smale, S., 
\newblock{1967. Differentiable dynamical systems. }
\newblock{\em Bulletin of the American mathematical Society}, 73(6), pp.747-817.

%\bibitem{rv}
%Rot, T.O. and Vandervorst, R.C., 2014. 
%\newblock{MorseÐConleyÐFloer homology. Journal of Topology and Analysis, 6(03), pp.305-338.}
%
%\bibitem{rvII}
%Rot, T.O. and Vandervorst, R.C.A.M., 2014. 
%\newblock{Functoriality and duality in MorseÐConleyÐFloer homology. Journal of Fixed Point Theory and Applications, 16(1-2), pp.437-476.}

%\bibitem{sal}
%Salamon, D., 1990. 
%\newblock{Morse Theory, the Conley Index and Floer Homology.  Bullettin of London Mathematical Society, 22, 113-140.}

\bibitem{sko}
Sk\"{o}ldberg,E.
\newblock{Morse theory from an algebraic viewpoint.}
\newblock{\em Transactions of the AMS}, 358(1):115--129,2006.

\bibitem{sko2}
Sk\"{o}ldberg, E., 2013. 
\newblock{Algebraic Morse theory and homological perturbation theory.}
\newblock{arXiv preprint arXiv:1311.5803.}

\bibitem{homalg}
The homalg project.
\newblock{\url{http://homalg-project.github.io/index.html}.}

\bibitem{w}
Tucker, W. (2002). A rigorous ODE solver and SmaleÕs 14th problem. Foundations of Computational Mathematics, 2(1), 53-117.

\bibitem{uz}
Usher, M.,  Zhang, J. (2016). 
\newblock{Persistent homology and Floer�Novikov theory.}
\newblock{ Geometry and Topology, 20(6), 3333-3430.}

\bibitem{weibel}
Weibel, C.A., 1995. 
\newblock{An introduction to homological algebra (No. 38).}
\newblock{ Cambridge university press.}



\end{thebibliography}
\end{document}
