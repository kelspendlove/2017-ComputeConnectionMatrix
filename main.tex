\documentclass[12pt, reqno]{amsart}
 \usepackage[margin=1in]{geometry}
\usepackage{amsmath, amssymb, amsthm}
\usepackage[osf]{mathpazo}
\usepackage[euler-digits]{eulervm}

% Xy-Pic
\usepackage[all,2cell]{xy}
\UseAllTwocells
% Commutative diagram
\usepackage{pb-diagram}
\usepackage{pb-xy}

\setcounter{tocdepth}{2}
\renewcommand{\mathbf}{\mathbold}
\renewcommand{\P}{\mathbb{P}}
\newcommand{\Q}{\mathbb{Q}}

\newcommand{\incar}{\ar@{^{(}->}}
\newcommand{\proar}{\ar@{->>}}
\newcommand{\two}{\Rightarrow}

 \newcommand{\under}{\mathbin{\mkern-3mu/\mkern-6mu/}\mkern-3mu}
	\newcommand{\fiber}{\under}

\usepackage{tikz}
\usetikzlibrary{matrix,arrows}

\usepackage[pdftex]{hyperref}
 \hypersetup{
    colorlinks,%
    citecolor=red,%
    filecolor=black,%
    linkcolor=blue,%
    urlcolor=blue     % can put red here to visualize the links
 }
 \urlstyle{same}

\newcommand{\myline}{\setlength{\parindent}{0in}\rule{\textwidth}{1pt}\setlength{\parindent}{0.1in}}


\setlength{\parindent}{0.1in}

\usepackage[osf]{mathpazo}
\usepackage[euler-digits]{eulervm}
%\usepackage{fourier}
%\usepackage{librebaskerville}
%\usepackage[T1]{fontenc}

\newcommand{\Qint}{\mathbf{Q}_{\text{Int}}}
\newcommand{\Con}{\mathrm{CH}}
\DeclareMathOperator{\Int}{\mathrm{Int}}

\newcommand{\Icat}{\mathrm{\mbox{{\bf Int}}}}

%\newcommand{\bI}{\mathbf{I}}
\newcommand{\bJ}{\mathbf{J}}
\newcommand{\Ch}{\mathrm{\textbf{Ch}}}


\newcommand{\inc}{\hookrightarrow}
\newcommand{\proj}{\twoheadrightarrow}

\newcommand{\SES}{\mathcal{S}}
\newcommand{\LES}{\mathcal{L}}
\newcommand{\CS}{\Con^{\oplus}} %conley-(direct)sum??
\newcommand{\HS}{H}

%\setcounter{tocdepth}{3}

% THEOREMS
\newtheorem{thm}{Theorem}[section]
\newtheorem{lem}[thm]{Lemma}
\newtheorem{defn}[thm]{Definition}
\newtheorem{cor}[thm]{Corollary}
\newtheorem{prop}[thm]{Proposition}
\newtheorem{ex}[thm]{Example}
\newtheorem{rem}[thm]{Remark}
\newtheorem{exer}[thm]{Exercise}
\newtheorem{alg}[thm]{Algorithm}
\newtheorem{com}[thm]{Comment}
\newtheorem{conj}[thm]{Conjecture}
%\newtheorem*{defn*}{Definition}

% COUNTERS
%\renewcommand{\thesection}{\thechapter.\arabic{section}}
%\renewcommand{\theequation}{\thesection.\arabic{theorem}}
%\renewcommand{\theequation}{\thesection.\arabic{equation}}
%\renewcommand{\thefigure}{\thechapter.\arabic{figure}}
%\newcommand\step{\stepcounter{theorem}}



% SET-RELATED MACROS
%\newcommand{\set}[1]{\left| {#1}\right|}
\newcommand{\setof}[1]{\left\{ {#1}\right\}}
\newcommand{\mvmap}{\raisebox{-0.2ex}{$\,\overrightarrow{\to}\,$}}
\newcommand{\setdef}[2]{\left\{{#1}\,\left|\,{#2}\right.\right\}}
\newcommand{\bdf}{\hbox{\rm bd$_f$}}
\newcommand{\supp}[1]{\lfloor {#1}\rfloor}

\newcommand{\cb}{\mbox{\sf{cb}}}
\newcommand{\bd}{\mbox{\sf{bd}}}
\newcommand{\birth}{\text{\sf{b}}}


%BOLD MACROS
\newcommand{\bCCB}{{\bf CCB}}
\newcommand{\bCh}{{\bf Ch}}
\newcommand{\bFDLat}{{\bf FDLat}}

% BOLD LETTERS
\newcommand{\bg}{{\bf g}}
\newcommand{\bh}{{\bf h}}
\newcommand{\bk}{{\bf k}}
\newcommand{\bq}{{\bf q}}
\newcommand{\bu}{{\bf u}}
\newcommand{\bv}{{\bf v}}
\newcommand{\bw}{{\bf w}}
\newcommand{\bx}{{\bf x}}
\newcommand{\by}{{\bf y}}
\newcommand{\bz}{{\bf z}}


\newcommand{\bA}{{\bf A}}
\newcommand{\bB}{{\bf B}}
\newcommand{\bC}{{\bf C}}
\newcommand{\bD}{{\bf D}}
\newcommand{\bG}{{\bf G}}
\newcommand{\bK}{{\bf K}}
\newcommand{\bM}{{\bf M}}
\newcommand{\bR}{{\bf R}}
\newcommand{\bI}{{\bf I}}


% BLACKBOARD BOLD LETTERS
\newcommand{\C}{{\mathbb{C}}}
\newcommand{\D}{{\mathbb{D}}}
\newcommand{\F}{{\mathbb{F}}}
\newcommand{\N}{{\mathbb{N}}}
\newcommand{\R}{{\mathbb{R}}}
\newcommand{\T}{{\mathbb{T}}}
\newcommand{\Z}{{\mathbb{Z}}}


% HOMOTOPY SYMBOL -- CHANGE THIS
\newcommand{\sh}{{\mathsf h}}

% FRAK LETTERS
%\newcommand{\cF}{{\mathfrak{f}}} % FOR MULTIVALUED MAPS

% CALIGRAPHIC LETTERS
\newcommand{\cA}{{\mathcal A}}
\newcommand{\cB}{{\mathcal B}}
\newcommand{\cC}{{\mathcal C}}
\newcommand{\cD}{\mathcal{D}}
\newcommand{\cE}{{\mathcal E}}
\newcommand{\cF}{{\mathcal F}}
\newcommand{\cG}{{\mathcal G}}
\newcommand{\cH}{{\mathcal H}}
\newcommand{\cI}{{\mathcal I}}
\newcommand{\cJ}{{\mathcal J}}
\newcommand{\cK}{{\mathcal K}}
\newcommand{\cL}{{\mathcal L}}
\newcommand{\cM}{{\mathcal M}}
\newcommand{\cN}{{\mathcal N}}
\newcommand{\cO}{{\mathcal O}}
\newcommand{\cP}{{\mathcal P}}
\newcommand{\cQ}{{\mathcal Q}}
\newcommand{\cR}{{\mathcal R}}
\newcommand{\cS}{{\mathcal S}}
\newcommand{\cT}{{\mathcal T}}
\newcommand{\cU}{{\mathcal U}}
\newcommand{\cV}{{\mathcal V}}
\newcommand{\cW}{{\mathcal W}}
\newcommand{\cX}{{\mathcal X}}
\newcommand{\cY}{{\mathcal Y}}
\newcommand{\cZ}{{\mathcal Z}}

\newcommand{\cAKQ}{{\mathcal{AKQ}}}
\newcommand{\wed}{\mathop{w}\nolimits}
\newcommand{\ci}{{\mathsf{i}}}
\newcommand{\cj}{{\mathsf{j}}}

% San Serif LETTERS
\newcommand{\sA}{{\mathsf A}}
\newcommand{\sB}{{\mathsf B}}
\newcommand{\sC}{{\mathsf C}}
\newcommand{\sD}{{\mathsf D}}
\newcommand{\sE}{{\mathsf E}}
\newcommand{\sH}{{\mathsf H}}
\newcommand{\sI}{{\mathsf I}}
\newcommand{\sJ}{{\mathsf J}}
\newcommand{\sK}{{\mathsf K}}
\newcommand{\sL}{{\mathsf L}}
\newcommand{\sM}{{\mathsf M}}
\newcommand{\sN}{{\mathsf N}}
\newcommand{\sP}{{\mathsf P}}
\newcommand{\sQ}{{\mathsf Q}}
\newcommand{\sR}{{\mathsf R}}
\newcommand{\sT}{{\mathsf T}}
\newcommand{\sU}{{\mathsf U}}
\newcommand{\sV}{{\mathsf{V}}}
\newcommand{\sW}{{\mathsf W}}
\newcommand{\sZ}{{\mathsf Z}}


\newcommand{\im}{{\mathsf i}}
\newcommand{\jm}{{\mathsf j}}
\newcommand{\km}{{\mathsf k}}
\newcommand{\lm}{{\mathsf l}}


\newcommand{\fF}{{\mathsf F}}
\newcommand{\fG}{{\mathsf G}}
\newcommand{\fH}{{\mathsf H}}
\newcommand{\fM}{{\mathsf M}}
\newcommand{\fS}{{\mathsf S}}

\newcommand{\bs}{{\mathbf s}}
\newcommand{\be}{{\mathbf e}}

\newcommand{\ind}{{\text{ind}}}
\newcommand{\mul}{{\text{mul}}}


\newcommand{\Que}{{\sf{Que}}}


% MISC CHARACTERS
%\newcommand{\Chi}{\raise .75ex\hbox{$\chi$}}

% MISC COMMANDS
%\newcommand{\proof}{\noindent{\em Proof:$\quad$}}
%\newcommand{\eproof}{\hfill{\vrule height5pt width5pt depth0pt}\medskip}


%  Maps and Arrows

\def\mapright#1{\stackrel{#1}{\longrightarrow}}
\def\smapright#1{\stackrel{#1}{\rightarrow}}

\def\mapleft#1{\stackrel{#1}{\longleftarrow}}
\def\mapdown#1{\Big\downarrow\rlap{$\vcenter{\hbox{$\scriptstyle#1$}}$}}
\def\mapup#1{\Big\uparrow\rlap{$\vcenter{\hbox{$\scriptstyle#1$}}$}}
\def\mapne#1{\nearrow\rlap{$\vcenter{\hbox{$\scriptstyle#1$}}$}}
\def\mapse#1{\searrow\rlap{$\vcenter{\hbox{$\scriptstyle#1$}}$}}
\def\mapsw#1{\swarrow\rlap{$\vcenter{\hbox{$\scriptstyle#1$}}$}}
\def\mapnw#1{\nwarrow\rlap{$\vcenter{\hbox{$\scriptstyle#1$}}$}}

\def\relup#1{\stackrel{#1}{\to}}


\def\setof#1{\left\{{#1}\right\}}
% For <blah>:
\def\ang#1{\langle {#1} \rangle}
% For <blah1, blah2>
\def\ip#1{\left\langle {#1} \right\rangle}
% For math words without mbox
\def\w#1{\mbox{#1}}
% For homology classes?
\def\eqcls#1{\left[#1\right]}

\newcommand{\inv}{^{-1}}
\newcommand{\id}{\text{id}}

% INDEXING MACRO

\newcommand{\symindex}[1]{{\index{#1}}}

% MATH 
%\newcommand{\isdef}{\stackrel{\rm def}{=}}
%\newcommand{\opensubset}{\stackrel{\rm open}{\subset}}
%\newcommand{\conv}{{\hbox{conv}\,}}
%\newcommand{\dist}{{\rm dist}\,}
%\newcommand{\rank}{\hbox{\rm rank}\,}
%\newcommand{\cl}{{\rm cl}\,}%{\text{cl}}
%\newcommand{\interior}{\hbox{\rm int}\,}
%\newcommand{\diam}{{\rm diam}\,}


% geometric
%\newcommand{\cl}{\mathop{\mathrm{cl}}\nolimits}
%\newcommand{\bd}{\mathop{\mathrm{bd}}\nolimits}
%\newcommand{\st}{\mathop{\mathrm{st}}\nolimits}

%algebraic
%\newcommand{\bdy}{\mathop{\mathrm{bdy}}\nolimits}
%\newcommand{\cbdy}{\mathop{\mathrm{cbdy}}\nolimits}
%\newcommand{\cbd}{\mathop{\mathrm{cbd}}\nolimits}  % to be removed


%\newcommand{\Int}{\mathop{\mathrm{int}}\nolimits} 
%\newcommand{\Inv}{\mathop{\mathrm{Inv}}\nolimits}
%\newcommand{\id}{\mathop{\mathrm{id}}\nolimits}
%
%
%\newcommand{\dist}{{\rm dist}}
%\newcommand{\rank}{\hbox{\rm rank}}
%\newcommand{\image}{\mathop{\rm image}}

\newcommand{\interior}{\hbox{\rm int}}
\newcommand{\diam}{{\rm diam}}

% MATH OPERATORS
\DeclareMathOperator{\vol}{vol}
\DeclareMathOperator{\img}{img}
\newcommand{\HG}{{H}}
\DeclareMathOperator{\dist}{d}
\DeclareMathOperator{\Oh}{O}

%\DeclareMathOperator{\diam}{diam}


% HATS

\newcommand{\hxi}{\widehat{\xi}}
\newcommand{\heta}{\widehat{\eta}}
\newcommand{\hbeta}{\widehat{\beta}}
\newcommand{\hzeta}{\widehat{\zeta}}

\newcommand{\higma}{\widehat{\sigma}}
\newcommand{\hphi}{\widehat{\phi}}
\newcommand{\hpsi}{\widehat{\psi}}
\newcommand{\hamma}{\widehat{\gamma}}
\newcommand{\happa}{\widetilde{\kappa}}
\newcommand{\bambda}{{\bf \lambda}}

% MISC
\newcommand{\divs}{{\bf ~|~}}
\newcommand{\pprec}{{\curlyeqprec}}
\newcommand{\mord}{{\lhd~}}

% MISC
%\newcommand{\bh}{{\bf h}}
%\newcommand{\divs}{{\bf ~|~}}
\newcommand{\tab}{\hspace{15pt}}
\newcommand{\Tub}{\text{Tub}}


% Make commands for the fancy quotes


\newenvironment{shadequote}%
{\begin{snugshade}\begin{quote}}
{\hfill\end{quote}\end{snugshade}}

\definecolor{shadecolor}{rgb}{0.8,0.8,0.8}

%\newcommand{\tab}{\hspace*{0.5em}}

\newcommand{\sweetline}{ 
\begin{center}
\nointerlineskip\vspace{-0.1in}
        $\diamond$\hfill\rule{0.90\linewidth}{1.0pt}\hfill$\diamond$
\par\nointerlineskip\vspace{0.1in}
\end{center}}


\title{A Computational Framework for Connection Matrices}
\author{Shaun Harker, Konstantin Mischaikow, Kelly Spendlove}

\begin{document}

\maketitle

\begin{abstract}
The connection matrix is a powerful algebraic topological tool from Conley index theory that captures relationships between isolated invariant sets.  Conley index theory is a topological generalization of Morse theory, in which the connection matrix subsumes the role of the Morse boundary operator.  Over the last few decades, the ideas of Conley has been cast into a purely computational framework~\cite{kmv,lsa,cmdb}.  In this paper we introduce a framework for computing the connection matrix.  This contribution transforms the computational Conley theory into a computational homological theory for dynamical systems.  Within these margins we have two goals:
\begin{enumerate}
\item We introduce a homotopy category which models the connection matrix theory.  We identify objects of this category which correspond to connection matrices and may be computed within the computational Conley theory paradigm.
\item We describe an algorithm for this computation based on discrete/algebraic Morse theory; we advertise publicly available packages in python and C++ which use the algorithm these computations.
\end{enumerate}

\end{abstract}

\myline




%%!TEX root = ../main.tex




\section{Introduction}\label{sec:intro}

The last few decades have seen the development of algebraic topological techniques for the analysis of data derived from experiment or computation.
An essential step is to make use of the data to construct a finite complex from which the algebraic topological information is computed.
For most applications this results in a high dimensional complex that, because of its lack of structure, provides limited insight into the problems of interest.
The purpose of this paper is to present an efficient algorithm for transforming the complex so that it posses a particularly simple boundary operator called the \emph{connection matrix}.
This process is not universally applicable; it requires the existence of distributive lattice that is coherent with the information that is to be extracted from the complex.
However, there are at least two settings in which we believe that it offers significant potential.

We begin by considering persistent homology which is a primary tool for the rapidly developing field of topological data analysis \cite{edelsbruner:harer, oudot, ***}.
The input is a cell complex $\cX$ along with a filtration $\cX_0 \subset \cX_1 \subset \cdots \subset \cX_N = \cX$.
Heuristically, persistent homology keeps track of the how homology generators from one level of the filtration are mapped to generators in another level of the filtration, i.e.\ $\iota_\bullet \colon H_\bullet(\cX_n) \to H_\bullet(\cX_m)$ where $\iota_*$ is induced by the inclusion $\cX_n\subset \cX_m$.
This information is tabulated as a bar code or persistence diagram.
In the context of this paper, the filtration is a  distributive lattice\footnote{See Section 2 for formal definitions associated with order theory and algebraic topology.}  with a total ordering given by the indexing $0 < 1 < \cdots < N$.
This is a serious limitation and has spurred the development of multidimensional persistent homology \cite{****} which remains a topic of current research.

A simple generalization is to assume that a decomposition of $\cX$ is given in the form of a distributive lattice.
To be more precise, assume that $\sL$ be a finite distributive lattice with  partial order denoted by $\leq$.
Let $\setof{\cX_a \subset \cX \mid a\in \sL}$ be an isomorphic lattice (the indexing provides the isomorphism) with operations $\cap$ and $\cup$ and minimal and maximal elements $\emptyset$ and $\cX$, respectively.
The content of this paper is to provide an efficient algorithm for computing a boundary operator, called the \emph{connection matrix},
\begin{equation}
\label{eq:connectionMatrix}
\Delta \colon \bigoplus_{a\in \sJ(\sL)} H_\bullet(\cX_a,\cX_{Pred(a)}) \to \bigoplus_{a\in \sJ(\sL)} H_\bullet(\cX_a,\cX_{Pred(a)})
\end{equation}
that is strictly upper triangular with respect to  $\leq$ where $\sJ(\sL)$ denotes the set of join-irreducible elements of $\sL$ and $Pred(a)$ denotes the unique predecessor of $a$, again with respect to $\leq$.

To put this into context, consider the classical handle body decomposition of a manifold.
In this case we have a filtration, i.e.\ $\sL$ is totally ordered and every element of the filtration $\cX_a$ is join-irreducible.
In this setting $\Delta$ is the classical Morse boundary operator.
As a consequence, it should be clear that the connection matrix encodes considerable information concerning the relationships between the homology generators of the elements of the lattice. 
Furthermore, as is shown in Section~\ref{sec:persistentHomology} with regard to persistent homology no information is lost using the chain complex with the connection matrix as the boundary operator.
More precisely, we prove the following theorem.

\begin{thm}
\label{thm:PH}
Let $\cX$ be a finite cell complex with associated chain complex $(C(\cX),\partial)$.
Let $\sL$ be a finite distributive lattice with  partial order denoted by $\leq$.
Let $\setof{\cX_a \subset \cX \mid a\in \sL}$ be an isomorphic lattice with operations $\cap$ and $\cup$ and minimal and maximal elements $\emptyset$ and $\cX$, respectively.
Let 
\[
\Delta \colon \bigoplus_{a\in \sJ(\sL)} H_\bullet(\cX_a,\cX_{Pred(a)}) \to \bigoplus_{a\in \sJ(\sL)} H_\bullet(\cX_a,\cX_{Pred(a)})
\]
be an associated connection matrix.
Let $\leq'$ be a linear extension of $\leq$.
The persistence diagrams for the filtration using the ordering $\leq'$ computed using the complex $(C(\cX),\partial)$ and the connection matrix complex will be the same.  
\end{thm}

The primary motivation for this paper is goal of developing efficient techniques for the analysis of time series data and computer-assisted proofs associated with deterministic nonlinear dynamics.
As background we recall that C. Conley developed a framework for the global analysis of nonlinear dynamics that makes use of two fundamental ideas \cite{conley:cbms}.
The first is the Conley index of an isolated invariant set \cite{salamon, robbin:salamon:1, mrozek} that is an algebraic topological generalization of the Morse index. 
The second is the use of attractors to organize the gradient-like structure of the dynamics.

To explain the relevance of these concepts in greater detail requires a digression.
Let $\varphi$ denote a dynamical system, e.g.\ a continuous semiflow or a continuous map, defined on a locally compact metric space.
Let $X$ denote a compact invariant set under $\varphi$.
The set of attractors  in $X$ is a bounded distributive lattice \cite{lsa}.
A \emph{Morse decomposition} of $X$ consists of a finite collection of mutually disjoint compact invariant sets $M(p)\subset X$, called \emph{Morse sets} and indexed by a partial order $(\sP,\leq)$, such that if $x\in X\setminus \bigcup_{p\in \sP}M(p)$, then in forward time $x$ limits to a Morse set $M(q)$, in backward time an orbit through $x$ limits to a Morse set $M(p)$, and $p < q$.
To each Morse decomposition there is associated a finite lattice of attractors $\sA$ such that each Morse set $M(p)$ is the maximal invariant set of $A\setminus Pred(A)$ where $A\in\sJ(\sA)$ and hence has a unique predecessor under the ordering of $\sA$ \cite{kalies:mischaikow:vandervorst:18}. 
In fact, Birkhoff's theorem (see Theorem~\ref{thm:birkhoff}) provides an isomorphism between the poset $\sP$ and $\sJ(\sA)$.

In general invariant sets such as attractors and Morse sets are not computable.
Instead one needs to focus on attracting blocks, these are compact subsets of $X$ such that they are mapped immediately in forward time into their interior.  
The set of all attracting blocks forms a bounded distributive lattice  under $\cap$ and $\cup$.
An essential fact is that if $\cX$ is a cell complex for $X$, then starting with a directed graph defined on $\cX$ that acts as an appropriate approximation of $\varphi$ it is possible to rigorously compute attracting blocks. 
Typically, the lattice of all attracting blocks has uncountably many elements.
However, as is shown in \cite{lsa} given a finite lattice of attractors $\sA$ and a fine enough cellular decomposition $\cX$ of $X$, then there exists a lattice of attracting blocks $\sABlock$ constructed using elements of  $\cX$ that is isomorphic (via taking omega limit sets) to $\sA$.
It is worth noting that $\sABlock$ defines a decomposition of $\cX$.

Consider a Morse decomposition of $X$ with its associated lattice of attractors $\sA$.
Let  $\sABlock$ denoted an isomorphic lattice of attracting blocks as described above.
In the context of semiflows (see Section~\ref{sec:conley} for case of maps)  the homology Conley index of any Morse set $M(p)$ is given by 
\[
CH_\bullet M(p) = H_\bullet (N, Pred(N))
\]
for the appropriate choice of $N\in \sJ(\sABlock)$.
Thus, the connection matrix of \eqref{eq:connectionMatrix} can be rewritten as
\begin{equation}
\label{eq:connectionMatrix2}
\Delta \colon \bigoplus_{p\in P} CH_\bullet M(p) \to \bigoplus_{p\in P} CH_\bullet M(p).
\end{equation}

The existence of a $\Delta$ expressed in the form of \eqref{eq:connectionMatrix2} is originally due to R. Franzosa~\cite{fran}.
The name connection matrix arose, since $\Delta$ can be used to identify and give lower bounds on the structure of connecting orbits between Morse sets \cite{scalar,mischaikow,mcmodels}.
While Franzosa's proof is constructive it is not effective with regard to computations. 
This is not surprising since the input to Franzosa's proof is the Conley index information associated with the Morse decomposition, and thus much more general than the description presented here (see Section~\ref{sec:CMT}).
J. Robbin and D. Salamon \cite{robbin:salamon2} provided an alternative proof for the existence of connection matrices and explicitly introduced the language of posets and lattices that we have employed, but again the input is the Conley index information.

In the context of experimental and computational data, the natural starting point is note the Conley index information, but rather a cell complex $\cX$ and its decomposition in the form of a distributive lattice.
Our contribution begins in Section~\ref{sec:cf} where we introduce the category of lattice filterings of a chain complexes.
This is closely related to ideas of \cite{robbin:salamon2}, but provides explicit ties to finite complexes.
Within this category we identify a specific type of filtering, that we call a \emph{Conley filtering}, for which the associated boundary operator is a connection matrix.
In Section~\ref{sec:computation} we produce an algorithm, based on discrete Morse theoretic manipulations of complexes, that is guaranteed to produce a Conley filtering.

A formal proof that the output of our algorithm is a connection matrix in the sense of Franzosa is presented in Section~\ref{sec:CMT}.
We conclude the paper in Section~\ref{sec:PH} with the proof of Theorem~\ref{thm:PH}.

%{\color{blue}We should discuss implementation and applications of these ideas somewhere.
%Here or perhaps in a concluding section}
%
%Proposed structure:
%\begin{enumerate}
%\item Preliminaries - algebraic, order theoretic
%\item Lattice filtered complexes, chain contractions
%\item Cellular complexes, Graded Complexes, discrete-algebraic Morse theory, 
%\item Connection Matrix Theory, Theorem
%\end{enumerate}






%!TEX root = ./main.tex


\section{Preliminaries}\label{sec:prelims}


In this section we review the necessary mathematical prerequisites.  We begin with the computational Conley paradigm.


\subsection{Computational Dynamics}

In the last few decades an algorithmic approach to Conley's approach to dynamical systems~\cite{conley} has been established~\cite{kmv, cmdb, cmdbchaos}.  This combinatorial-topological framework is central to our motivation for computing the connection matrix.  It has been made clear that one of the most prominent objects of the theory is the lattice of attractors and lattice of attracting blocks~\cite{kmv,lsa,lsa2,salamon}.

Let $f:X\times Z\to X$ be a dynamical system with $X,Z$ compact metric spaces.  We recall the standard pipeline:

\begin{itemize}
\item Select {\em grids} $\cX,\cZ$ on $X$ and $Z$.  In practice $\cX,\cZ$ are cubical complexes.
\item Compute a lattice of subcomplexes which serve as attracting blocks of $f$
\item For each join irreducible, compute a Conley index -  an algebraic topological invariant which provides a coarse measurement of the unstable dynamics associated with $M_\zeta$. This may be done efficiently by interpreting $M_\zeta$ as a subcomplex of $\cX$ and using computational homology 
\item Invoke theorems~\cite{cmdb,cmdbchaos} to lift the computational results to rigorous results for the continuous system $f$ 

%\item Construct an {\em outer approximation} $\cF:\cX\times \cZ\to \cX$ of $f$, i.e. a relation between $\cX\times \cZ$ and $\cX$ such that for all $\xi\in \cX$ and $\zeta\in \cZ$ $$f_{|\zeta|}(|\xi|)\subseteq int |(\cF(\xi,\zeta)|$$
%\item For $\zeta\in \cZ$, $\cF_\zeta := \cF(\cdot, \zeta)$ be decomposed into recurrent and gradient-like parts, in analogy to Conley's decomposition theorem~\cite{conley,kmv}.  This is done by interpreting $\cF_\zeta$ as a directed graph with vertex set $\cX$ and applying Tarjan-like algorithms~\cite{cmdbchaos}
%\item For each recurrent set $M_\zeta$ of $\cF_\zeta$ compute a Conley index - an algebraic topological invariant which provides a coarse measurement of the unstable dynamics associated with $M_\zeta$. This may be done efficiently by interpreting $M_\zeta$ as a subcomplex of $\cX$ and using computational homology algorithms~\cite{cmdbchaos}.
%\item Invoke theorems~\cite{cmdb,cmdbchaos} to lift the computational results to rigorous results for the continuous system $f$
\end{itemize}

As we will show, the lattice of subcomplexes can (via Birkhoff's theorem) be codified in morphism $(X,\preceq)\to (P,\leq)$ between posets, where $(X,\preceq)$ is the face poset of $X$ and $(P,\leq)$ is the set of join-irreducibles.


The connection matrix is a boundary operator on the Conley indices and promotes the Conley theory into a homology theory.  In this setting, an algorithm for the connection matrix promotes the computational Conley theory to a computational homology theory.




\subsection{Algebraic Topology}\label{sec:prelims:AT}

We review some algebraic topology.  This exposition follows~\cite{weibel}.  Let $\F$ be a field.  A {\em chain complex} $C_\bullet$ of $\F$-vector spaces is a family $\{C_n\}_{n\in \N}$ of vector spaces over field $\F$ together with linear maps $\partial=\partial_n:C_n\to C_{n-1}$.  When the context is clear we will abbreviate $C_\bullet$ by $C$.  A morphism $f:A\to B$ is a {\em chain map}, that is a family of linear maps $f_n:A_n\to B_n$ such that $f_{n-1}\partial^A = \partial^B f_n$. Chain complexes and chain maps make up a category denoted $Ch(\F)$.  

A chain complex $B$ is called a {\em subcomplex} of $C$ if each $B_n$ is a subspace of $C_n$ and $\partial(B_n)\subset B_{n-1}$, i.e. that the inclusion map $i:B\to C$ is a chain map.  In this case we may assemble the quotients $C_n/B_n$ into a chain complex denoted $C/B$ called the {\em quotient complex}.   The $n$th homology of $C$ is the quotient $H_n(C):= \ker \partial_n/im \partial_{n+1}$.  The graded vector space $H_\bullet(C) := \{H_n(C)\}_{n\in \N}$ is the {\em homology} of $C_\bullet$.  Chain maps induce linear maps on homology.  A chain map $A\to B$ is a {\em quasi-isomorphism} if the maps $H_n(A)\to H_n(B)$ are all isomorphisms.

Two chain maps $f,g:A\to B$ are {\em chain homotopic} if there exists degree +1 maps $h_n:A_n\to B_{n+1}$ such that $$f-g = h\partial_A+\partial_Bh$$  We say that $f:A\to B$ is a {\em chain homotopy equivalence} if there is a chain map $g:B\to A$ such that $fg$ and $gf$ are chain homotopic to the respective identity maps of $A$ and $B$.  Chain homotopy equivalence is an equivalence relation on $Hom(A,B)$.  The set of such equivalence classes $Hom_K(A,B)$ is an abelian group.  The category $K$ consisting of chain complexes with hom sets given by $Hom_K(A,B)$ is called the homotopy category.  The isomorphisms in this category are precisely the equivalence classes of the chain homotopy equivalences.


The rest of this exposition follows~\cite{focm,mn}.  This particular definition of complex dates back to Lefschetz.

\begin{defn}
{\em
Consider a finite graded set $\cX = \bigsqcup_{q\in \Z} \cX_q$ along with a function $\kappa:\cX\times\cX\to \F$ and denote $\xi\in \cX_q$ by $\dim \xi = q$.  Then $(\cX,\kappa)$ is a {\em complex} if the following hold:
\begin{enumerate}
\item \label{cond:1} For each $\xi$ and $\xi'$ in $\cX$:
$$\kappa(\xi,\xi')\neq 0\quad\text{implies}\quad \dim \xi = \dim \xi'+1$$
\item\label{cond:2} For each $\xi$ and $\xi''$ in $\cX$,
$$\sum_{\xi'\in \cX} \kappa(\xi,\xi')\cdot \kappa(\xi',\xi'')=0$$
\end{enumerate}
}
\end{defn}

An element $\xi\in \cX$ is called a {\em cell} and $\dim \xi$ is the {\em dimension} of $\xi$.  The function $\kappa$ is the {\em incidence function} of the complex $(\cX,\kappa)$.   The {\em face partial order} $\preceq$ is induced on the elements of $\cX$ by the transitive closure of the generating relation $\prec$ given as follows: For $\xi, \xi'\in \cX$ $$\xi' \prec \xi \quad \text{if} \quad \kappa(\xi,\xi')\neq 0$$
Let $(\cX,\kappa)$ be a complex.  The {\em associated chain complex} consists of free vector spaces $C_q(\cX)$ where the basis elements are the cells $\xi \in \cX_q$ and the boundary operator is generated by the maps $$\partial_q \xi := \sum_{\xi' \in \cX} \kappa(\xi, \xi')\xi'$$ It is straightforward to verify that the associated chain complex of a complex is indeed a chain complex.











\subsection{Order Theory}\label{sec:prelims:order}

Order theory is the study of posets and lattices.  Order theory has a strong relationship to both algebraic topology and dynamical systems, e.g.~\cite{salamon,lsa,lsa2}.  An intuition for posets, lattices and their correspondence via Birkhoff's theorem will be very helpful for understanding the paper.


\subsubsection{Posets}

A morphism of posets is a map $h:(P,\leq_P)\to (Q,\leq_Q)$ such that if $p\leq_P q$ then $h(p)\leq_Q h(q)$. Posets and their morphisms form the category {\bf Poset}.

The face poset $(X,\preceq)$ provides a powerful method of thinking about a complex $(X,\kappa)$.  One reason is that subcomplexes of $X$ have a nice characterization in terms of {\em convex sets} of $(X,\preceq)$.   Let $P$ be a poset.  An {\em upper set} of $P$ is a subset $U\subset P$ such that if $p\in U$ and $p\leq q$ then $q\in U$.  For $p\in P$ the {\em upset} at $p$ is $\uparrow(p):=\{q\in P:p \leq q\}$.  A {\em lower set} of $P$ is a set $D\subset P$ such that if $q\in D$ and $p\leq q$ then $p\in D$.  The {\em downset} at $q$ is $\downarrow(q):=\{p\in P: p \leq q\}$.  A subset $I\subset P$ is an {\em convex set} if $p,q\in I, r\in P$ and $ p < r < q$ implies that $r\in I$.  Any convex set in $P$ can be obtained by an intersection of a lower and upper set.  
 
\begin{defn}
{\em
A collection $(I_1,\ldots, I_N)$ of convex sets of $(P,\leq)$ is called {\em adjacent} if
\begin{enumerate}
\item $I_1,\ldots,I_n$ are mutually disjoint
\item $\bigcup_{i=1}^n I_i$ is a convex set in $P$
\item $p\in I_i, q\in I_j, i < j$ imply $q \nless p$
\end{enumerate}
}
\end{defn}

We will be primarily interested in adjacent pair of convex sets $(I,J)$.  We write the union (a convex set) $I\cup J$ as $IJ$.  We will denote the set of convex sets as $I(P)$ and the set of adjacent tuples and triples of convex sets as $I_2(P)$ and $I_3(P)$.  This notation agrees with~\cite{fran}.
%The convex sets are particularly important for complexes:

\begin{prop}\label{prop:subcomplex}
Let $(\cX,\kappa)$ be a complex.  Let $(\cX,\preceq)$ be its face poset.  If $\cX'$ be an convex set in $(\cX,\preceq)$ then $(\cX',\partial_{\cX'})$ is a subcomplex.
\end{prop}
\begin{proof}
We must show that $\partial_{\cX'}\circ \partial_{\cX'}=0$.  If $\xi,\xi''\in \cX'$ and $\xi'' \prec \xi$ then all $\xi'\in \cX$ such that $\xi'' \prec \xi' \prec \xi$ must be in $\cX'$ since $\cX'$ is an interval.  Thus~(\ref{cond:1}) and~(\ref{cond:2}) imply that $\sum_{\xi'\in \cX'} \kappa(\xi, \xi')\cdot \kappa(\xi',\xi'')=0$ which implies $\partial_{\cX'}^2=0$.
\end{proof}

\begin{cor}\label{cor:clsubcomplex}

If $\cX'$ is a lower set then this is a closed subcomplex.

\end{cor}

Posets have a topology known as the Alexandrov topology.  It is easy to see that a map $h:(P,\leq_P)\to (Q,\leq_Q)$ is a morphism of posets if and only if $h$ is continuous with respect to the Alexandrov topologies.


\subsubsection{Lattices}

\begin{defn}
{\em
A {\em lattice} is a set $L$ with the binary operations $\vee,\wedge:L\times L\to L$ satisfying the following axioms:

\begin{enumerate}
\item (idempotent) $a\wedge a = a \vee a = a$ for all $a\in L$
\item (commutative) $a\wedge b = b\wedge a$ and $a\vee b = b \vee a$ for all $a,b\in L$
\item (associative) $a\wedge b(b\wedge c) = (a\wedge b)\wedge c$ and $a\vee(b\vee c) = (a\vee b)\vee c$ for all $a,b,c\in L$
\item (absorption $a\wedge (a\vee b) = a\vee (a\wedge b)=a$ for all $a,b\in L$

A lattice $L$ is {\em distributive} if it satisfies the additional axiom:

\item (distributive) $a\wedge (b\vee c) = (a\wedge b)\vee (a\wedge c)$ and $a\vee (b\wedge c) = (a\vee b) \wedge (a\vee c)$ for all $a,b,c\in L$

A lattice $L$ is {\em bounded} if there exist elements $0_L$ and $1_L$ such that

\item $0_L\wedge a = 0_L, 0_L\vee a = a, 1_L\wedge a = a, 1_L\vee a = 1_L$ for all $a\in L$
\end{enumerate}
}
\end{defn}

A lattice morphism $h:L\to M$ is a map such that if $a,b\in L$ then $f(a\wedge b) = f(a)\wedge f(b)$ and $f(a\vee b) = f(a)\vee f(b)$.  If $L$ and $M$ are bounded lattices then we also require that $f(0_L)=0_M$ and $f(1_L)=1_M$.    Bounded, distributive lattices and their morphisms form the category {\bf BDLat}.

We say that $q$ {\em covers} $p$ if $p\leq q$ and there does not exist an $r$ with $p\leq r \leq q$.  If $q$ covers $p$ then we say $p$ is a {\em predecessor} of $q$.  An element $a\in L$ is {\em join-irreducible} if it has a unique predecessor.   A subset $K\subset L$ is called a sublattice of $L$ if $a,b\in K$ implies that $a\vee b\in K$ and $a\wedge b\in K$.  A lattice $L$ has an associated poset structure given by $a\leq b$ if $a=a\wedge b$ or if $b=a\vee b$.

\begin{defn}
{\em
For a lattice $L$ define $Pred:L\backslash \{0_L\} \to L$ via $Pred(p) = \bigwedge \{q: \text{$p$ covers $q$}\}$, i.e. the meet of the predecessors.
}
\end{defn}

 Notice for join irreducible elements that $Pred$ yields the unique predecessor.

\begin{lem}[\cite{roman}, Theorem 4.29]\label{lem:join}
Let $L$ be a bounded distributive lattice.  Any $p\in L$ can written as the irredundant join of join-irreducibles.
\end{lem}


\subsubsection{Birkhoff's Correspondence}

For a lattice $L$ we denote its join-irreducibles as $J(L)$.  $J(L)$ has a poset structure.  $J$ is a functor $J:{\bf BDLat}\to {\bf Poset}$.  For a poset $(P,\leq)$ we denote set of downsets by $O(P)$. $O(P)$ has the structure of a distributive lattice.  $O$ is a functor $O:{\bf Poset}\to {\bf BDLat}$.  This is formalized via Birkhoff's theorem.  See~\cite{lsa,lsa2,salamon} for a discussion in the context of dynamics.

\begin{thm}[\cite{lsa}]\label{thm:birkhoff}
$J$ and $O$ are contravariant functors and provide an equivalence of categories {\bf Poset} and {\bf BDLat}, i.e. $$L\cong O(J(L))\quad\quad P\cong J(O(P))$$

\end{thm}

It is often the case that lattices are related to algebraic structures via a study of their substructures.  For instance, consider a vector space $V$.  It is straightforward to verify that the collection of subspaces of $V$ forms a bounded lattice under the operations $\cap$ and $+$ (span). However this lattice of subspace is not distributive.  Similarly, for a chain complex $C$, it is again straightforward that the collection of subcomplexes of $C$ form a bounded lattice under the operations $\cap$ and $+$.  We denote this lattice $S(C)$.  Again, note that $S(C)$ is not distributive.











%!TEX root = ../main.tex

\section{Connection Matrix Theory}\label{sec:CMT}

In this section we will review the connection matrix theory.  Unfortunately, the theory comes with a fairly large overhead of mathematical machinery.  This section is included for completeness, and the Conley theory cognoscenti.  For non-experts, it may be skipped upon first reading.

The connection matrix theory was first developed by R. Franzosa in a sequence of papers based on his dissertation, which was directed by C. Conley~\cite{fran2,fran,fran3}.  These ideas were reinterpreted by J. Robbin and D. Salamon in their paper~\cite{salamon}.  We will discuss both approaches.  The connection matrix is a generalization of the Morse boundary operator for the Conley theory.   It is a boundary operator defined on Conley indices.  Its basic utility is to prove existence of connecting orbits~\cite{mpmw}.  At a higher level, it serves as an algebraic representation of global dynamics and may used to construct (semi)-conjugacies of the global attractor~\cite{dhmo,mcmodels,scalar}. Its preeminent function is to complete the Conley theory to a homological theory~\cite{mc}.  

We first review the connection matrix theory as presented by Franzosa in~\cite{fran}.

\subsection{The Category Of Chain Complex Braids}
It was Conley's observation~\cite{conley} that focusing on the attractors of a dynamical system provides a generalization of the Spectral Decomposition of Smale~\cite[Theorem 6.2]{smale}.  There is a lattice structure to the attractors of a dynamical system~\cite{salamon,lsa,lsa2}.  Thus a natural object of study in Conley theory is some finite sublattice of attractors, and an associated sublattice of attracting blocks.  A sublattice of attracting blocks is what Franzosa calls an index filtration.

In his work, Franzosa introduces the notion of a {\em chain complex braid} as a data structure to hold the chains that arise from the topological data within the index lattice.  The chain complex braid is organized by the poset of join-irreducibles.  Implicit in Franzosa's work is a description of a category for chain complex braids over a fixed poset.  We now describe this category and denote it by ${\bf CCB}(P,\leq)$.

\begin{defn}
{\em
A sequence of chain complexes and chain maps $$C_1\xrightarrow{i} C_2 \xrightarrow{p} C_3$$
is called {\em weakly exact} if $i$ is injective, $p\circ i = 0$ and $p:C_2/im(i)\to C_3$ induces an isomorphism on homology.
}
\end{defn}
\begin{rem}
Every short exact sequence is weakly exact.  Franzosa needs weakly for complications which arise when working with singular homology and a lattice of attracting blocks.
\end{rem}

\begin{defn}
{\em
A {\em chain complex braid} over $(P,\leq)$ is a collection of chain complexes and chain maps such that:
\begin{enumerate}
\item for each $I\in I(P)$ there is a chain complex $(C(I),\partial(I))$
\item for each $(I,J)\in I_2(P)$ there are chain maps $$i(I,IJ):C(I)\to C(IJ)\quad\quad p(IJ,J):C(IJ)\to C(J)$$ which satisfy:
\begin{enumerate}
\item $C(I)\xrightarrow{i(I,IJ)} C(IJ)\xrightarrow{p(IJ,J)} C(J)$ is weakly exact,
\item if $I$ and $J$ are noncomparable then $p(JI,I)i(I,IJ)=id|_{C(I)}$
\item if $(I,J,K)\in I_3(P)$ then the following braid diagram commutes:
\[
\xymatrixrowsep{0.03in}
\xymatrixcolsep{0.3in}
\xymatrix{
& & C(J) \incar[dr] & &  \\
& C(IJ) \proar[ur] \incar[dr] & & C(JK) \proar[dr] &  \\
C(I) \incar[ur] \ar@{_{(}->}[rr]& & C(IJK) \proar[rr] \proar[ur]& & C(K) 
}
\]
\end{enumerate}

\end{enumerate}
}
\end{defn}

Chain complex braids are the objects of our category ${\bf CCB}(P,\leq)$.  A morphism $\Psi:\cC\to \cC'$ between chain complex braids $\cC$ and $\cC'$ is a collection of chain maps $\Psi(I):C(I)\to C'(I)$ for each $I\in I(<)$ such that for $(I,J)\in I_2(<)$ the following diagram commutes:
\[
\xymatrixcolsep{0.4in}
\xymatrixrowsep{0.4in}
\xymatrix{
C(I) \incar[r] \ar@{->}[d]_{\Psi(I)} & C(IJ) \ar@{->}[d]_{\Psi(IJ)} \proar[r] & C(J) \ar@{->}[d]^{\Psi(J)}  \\
C'(I) \incar[r] & C'(IJ) \proar[r] & C'(J)
}
\] 

Different index lattices for the same dynamical system may yield different chain complex braids.  However the homology of these chain groups are an invariant.  This is the motivation for the idea of a graded module braid, which formalizes the notion of `homology' for a chain complex braid.


\begin{defn}
{\em
A {\em graded module braid} over $(P,\leq)$ is a collection of graded modules and maps between graded modules satisfying:
\begin{enumerate}
\item for each $I\in I(P)$ there is a graded module $G(I)$
\item for each $(I,J)\in I_2(P)$ there are maps:
\begin{align*}
i(I,IJ):G(I)\to G(IJ) \text{ of degree 0,}\\
p(IJ,J):G(IJ)\to G(J) \text{ of degree 0,}\\
\partial(J,I):G(J)\to G(I) \text{ of degree -1}
\end{align*}
which satisfy:
\begin{enumerate}
\item $\ldots \xrightarrow{i} G(I)\to G(IJ)\xrightarrow{p} G(J) \xrightarrow{\partial} \ldots$ is exact,
\item if $I$ and $J$ are noncomaprable then $p(JI,I)i(I,IJ)=id|_{G(I)}$
\item if $(I,J,K)\in I_3(P)$ then the following braid diagram commutes:
\[
\xymatrixrowsep{0.15in}
\xymatrixcolsep{0.3in}
\xymatrix{
\vdots \ar@{->}[d] && \cdots \ar@{->}[drr] \ar@{->}[dll] && \vdots \ar@{->}[d]\\
G(I)\ar@/_2pc/[dd]_{i} \ar@{->}[drr]_{i} &&  && G(K)\ar@{->}[dll]_{\partial} \ar@/^2pc/[dd]_{\partial}  \\
& &G(IJ) \ar@{->}[dll]_{i} \ar@{->}[drr]_{p} &&\\
G(IJK) \ar@/_2pc/[dd]_{p}  \ar@{->}[drr]_{p} &&  && G(J)\ar@{->}[dll]_{i}  \ar@/^2pc/[dd]_{\partial}  \\
& &G(JK) \ar@{->}[dll]_{p} \ar@{->}[drr]_{\partial} &&\\
G(K)\ar@/_2pc/[dd]_{\partial}    \ar@{->}[drr]_{\partial} &&  && G(I)\ar@{->}[dll]_{i}   \ar@/^2pc/[dd]_{i} \\
& &G(IJ) \ar@{->}[dll]_{p} \ar@{->}[drr]_{i} &&\\
G(J)\ar@{->}[d] \ar@{->}[drr] &&  && G(IJK)\ar@{->}[dll]  \ar@{->}[d] \\
\vdots && \cdots && \vdots}
\] 
%C(I) \incar[r] \ar@{->}[d]_{\Psi(I)} & C(IJ) \ar@{->}[d]_{\Psi(IJ)} \proar[r] & C(J) \ar@{->}[d]^{\Psi(J)}  \\
%C'(I) \incar[r] & C'(IJ) \proar[r] & C'(J)
\end{enumerate}

\end{enumerate}
}
\end{defn}

A morphism $\theta:\cG\to \cG'$ of graded module braids is a collection of linear maps $\theta(I):G(I)\to G'(I)$, $I\in I(P)$ such that for each $(I,J)\in I_2(P)$ the following diagram commutes:
\[
\xymatrixrowsep{0.4in}
\xymatrixcolsep{0.45in}
\xymatrix{
\ldots \ar@{->}[r] & G(I) \ar@{->}[d]^{\theta(I)} \ar@{->}[r]^{i} & G(IJ) \ar@{->}[d]_{\theta(IJ)} \ar@{->}[r]^{p} & G(J) \ar@{->}[d]_{\theta(I)} \ar@{->}[r]^{\partial} & G(I) \ar@{->}[d]_{\theta(I)} \ar@{->}[r] & \ldots\\
\ldots \ar@{->}[r] & G'(I) \ar@{->}[r]^{i} & G'(IJ) \ar@{->}[r]^{p} & G'(J) \ar@{->}[r]^{\partial} & G'(I) \ar@{->}[r] &\ldots
}
\]
%As observed by Robbin and Salamon~\cite{sal}, the correct way to think about ${\bf CCB}(P,\leq)$ is as a homotopy category.

We label the category of graded module braids over $(P,\leq)$ by ${\bf GB}(P,\leq)$.

Franzosa describes a functor from $\cH:{\bf CCB}(P,\leq)\to {\bf GB}(P,\leq)$ which is the analogy of homology.  This is basically the content of~\cite[Proposition 2.7]{fran}.

\subsection{Connection Matrices}

The connection matrix is a boundary operator on Conley indices.  The Conley indices are graded vector spaces.  Let $(P,\leq)$ be a poset.  Let $\{V_p\}_{p\in P}$ be a collection of graded vector spaces indexed by $P$.  Define $$V^\oplus(P) := \bigoplus_{p\in P} V_p$$  For collections $\{V_p\}_{p\in P}$ and $\{W_p\}_{p\in P}$ a morphism $\phi(P):V^\oplus(P)\to W^\oplus(P)$ may be thought of as a matrix of linear maps: $$[\phi(p,q):V_p\to W_q\mid p,q\in P]$$


As $V^\oplus(P),W^\oplus(P)$ are indexed by a poset, we say that a morphism $\phi(P)$ is {\em diagonal with respect to $P$} if $\phi(p,q)=0$ for $p\neq q$.  It is {\em upper triangular with respect to $P$} if $\phi(p,q)=0$ for $p\nless q$.  We will also be concerned with endomorphisms.  An endomorphism $\Delta(P):V^\oplus(P)\to V^\oplus(P)$ is {\em a boundary map} if each $\Delta(p,q)$ is a degree -1 map and $\Delta(P)\circ \Delta(P) = 0$.  Identifying an endomorphism $\Delta(P)$ with its matrix structure is origin of the term {\em connection matrix}.  

One often examines subspaces of $V^\oplus(P)$.  For $I\subseteq P$ we define $V^\oplus(I)= \bigoplus_{p\in I} V_p$.  We let $\phi(I)$ be the restriction of $\phi(P)$ to $V^\oplus(I)\to V^\oplus(I)$.  It is easy to see that $\phi(I) = \pi_{P,I} \circ \phi(P) \circ \iota_{I,P}$ where $\iota_{I,P}:V^\oplus(I)\to V^\oplus(P)$ and $\pi_{P,I}:V^\oplus(P)\to V^\oplus(I)$ are the natural inclusion and projection.  The following sequence of propositions shows that upper triangular boundary maps can be used to build chain complex braids.




\begin{prop}[Proposition 3.2,~\cite{fran}]
If $\Delta(P):V^\oplus(P)\to V^\oplus(P)$ is an upper triangular boundary map then $\Delta(I)$ upper triangular boundary map for any convex set $I$.   
\end{prop}

\begin{prop}[Proposition 3.3,~\cite{fran}]\label{prop:fran:3.3}
If $\Delta(P):V^\oplus(P)\to V^\oplus(P)$ is an upper triangular boundary map, then for any adjacent pair $(I,J)\in I_2(P)$ the following sequence is a short exact sequence of chain complexes: $$0\to (V^\oplus(I),\Delta(I)\to (V^\oplus(IJ),\Delta(IJ))\to (V^\oplus(J),V(J))\to 0$$ where the maps are the natural inclusion and projections.
\end{prop}

\begin{prop}[\cite{fran}, Proposition 3.4]
Given an upper triangular boundary map $\Delta(P):C^\oplus(P)\to C^\oplus(P)$ the collection, denoted $C\Delta(P)$, consisting of the chain complexes $C(I)$ with boundary map $\partial(I)$ for each $I\in I(P)$ and the obvious chain maps $i(I,IJ)$ and $p(IJ,J)$ for each $(I,J)\in I_2(P)$ is a chain complex braid over $P$.
\end{prop}






\begin{defn}
{\em
Let $\cG$ be a graded module braid.  Let $\{C(p)\}_{p\in P}$ be a collection of graded vector spaces.  Let $\Delta:C^\oplus(P)\to C^\oplus(P)$ be an upper triangular boundary map.  Let $\cC$ be the associated chain complex braid.  If $\cH(\cC)\cong \cG$, then $\Delta$ is a {\em C-connection matrix} for $\cC$.  If in addition $\Delta(p,p)=0$ for all $p\in P$ then $\Delta$ is a {\em connection matrix} for $\cC$.
}
\end{defn}

It is clear that if $\Delta(p,p)=0$ for all $p\in P$ then for each $p$ $HC(p)\cong C(p)$ and $\Delta$ may be written as a boundary map $\Delta:\bigoplus_{p\in P} HC(p)\to \bigoplus_{p\in P} HC(p)$.

One way to motivate the connection matrix is to observe that chain complex braids generated by upper triangular boundary maps are particularly simple.  For instance given $I\in I(P)$ $C(I) = \bigoplus_{p\in I} C(p)$.  Below is one of Franzosa's theorems on existence of connection matrices:

\begin{thm}[\cite{fran}, Theorem 4.8]
Let $\cC$ be a chain complex braid over $(P,\leq)$.  Let $B=\{B(p)\}_{p\in P}$ be a collection of free chain complexes such that $HB(p) \cong HC(p)$, the homology of the chain complex $C(p)$ in $\cC$.  Then there exists an upper triangular boundary map, $$\Delta:\bigoplus_{p\in P} B(p)\to \bigoplus_{p\in P}B(p)$$ and a chain map $\Psi:\cB\to \cC$, from the chain complex braid induced by $\Delta$ such that $\cH\Psi:\cH\cB\to \cH\cC$ is a graded module braid isomorphism.
\end{thm}

Here's a simple application of Franzosa's theorem.  Let $\cC$ be a chain complex braid.  Choose $B = \{C(p)\}_{p\in P}$.  The theorem says that there exists an upper triangular boundary map $\Delta:C^\oplus(P)\to C^\oplus(P)$ and a chain map $\Psi:\cB\to \cC$ such that $\cH\Psi$ is an isomorphism.  Therefore for any chain complex braid there is a simple representative (one generated by an upper triangular boundary map) in its {\em derived equivalence class}.  If $HC(p)$ is free, then we may choose $B=\{HC(p)\}$ where $HC(p)$ is considered as chain complex with zero differentials.  In this case $\Delta$ is a connection matrix.


\subsection{Connection Matrix Redux}



J. Robbin and D. Salamon have a related development of the connection matrix~\cite{salamon}.   Instead of a chain complex braid, their idea is to assign a subcomplex to each lower set of the poset.  As the lower sets form a lattice, the assignment is required to be a lattice homomorphism.  A $P$-filtered chain complex is a chain complex $(C,\partial)$ with a collection of subcomplexes $\{C_\alpha\}_{\alpha\in O(P)}$ such that $$C_{\alpha\cap \beta} = C_\alpha\cap C_\beta, \quad C_{\alpha\cup \beta} = C_\alpha + C_\beta,\quad C_\emptyset = \{0\},\quad C_P = C$$ and that $$\partial(C_\alpha)\subset C_\alpha$$

A mapping $\phi:A\to B$ of $P$-filtered chain complexes is said to {\em preserve the filtration} if $\phi(A_\alpha)\subset B_\alpha$.  A morphism of $P$-chain map $\phi:A\to B$ of $P$-filtered chain complexes is a chain map $A\to B$ which preserves the filtration.

It is implicit in~\cite{salamon} that the correct model for connection matrix theory is homotopy theory.    Two $P$-chain maps $\phi,\psi:A\to B$ are called $P$-chain homotopic if there is a map $\gamma:A\to B$ which preserves the filtration and satisfies $$\phi-\psi = \partial_B \circ \gamma + \gamma\circ \partial_A$$

Two $P$-filtered chain complexes $A$ and $B$ are called $P$-chain equivalent if there are morphisms $\phi:A\to B$ and $\psi:B\to A$ such that both $\phi\circ \psi:B\to A$ and $\psi\circ \phi:A\to B$ are $P$-chain homotopic to the identity.  

A $P$-connection matrix is a $P$-filtered chain complex $(C,\Delta)$ with the property that $$\Delta(C_\beta)\subset C_{\beta \backslash p}$$ whenever $p$ is maximal in $\beta$.  

Using the idea of homotopy equivalence you can build a homotopy category as in Section~\ref{sec:prelims:AT}.  This makes homology implicit, and one then does not need to define the equivalent of a `graded module braid'.  






%!TEX root = ../main.tex
\section{Chain Fibrations}\label{sec:cf}

In this section we will introduce our fundamental tools: cell fibrations and chain fibrations.


 {\bf Should the names of these be changed?  There may be confusion with fibrations/lifting properties/homotopy equivalent fibers, etc.  Perhaps we can call these $P$ or $J(L)$-filtered cell/chain complexes}


\subsection{The Model of Cell Fibrations}

In applications data often comes in the form of a cell complex $(X,\kappa,\preceq)$ filtered by a partial order $(P,\leq)$.  This is codified in terms of a map $f:(X,\kappa, \preceq)\to (P,\leq)$ which restricts to a poset morphism $f:(X,\preceq)\to (P,\leq)$. As we are explicitly thinking of $(X,\preceq)$ as being a face poset of some cell complex, we call such order-preserving maps {\em cell fibrations}.

\begin{defn}
{\em
A $P$-filtered complex/cell fibration is a map $f:(X,\kappa,\preceq)\to (P,\leq)$ that is order-preserving with respect to $(X,\preceq)$ and $(X,\leq)$.
}
\end{defn}

{\bf There is perhaps some forget functor between cell complexes and posets?  What is a morphism of cell complexes?}

\begin{rem}
If $f:(X,\kappa,\preceq)\to (P,\leq)$ is a $P$-filtered complex then $f:(X,\preceq)\to (P,\leq)$ is a poset morphism.  In this case, Birkhoff's theorem provides a map $O(f):O(P,\leq)\to O(X,\preceq)$, between the lattice of downsets.  For a cell complex $(X,\kappa,\preceq)$ the set of downsets is precisely the set of subcomplexes.   Thus a $P$-filtered complex gives rise to a parameterization of subcomplexes in $(X,\kappa,\preceq)$.
\end{rem}

% In computational dynamics, $f$ is a combinatorial model for dynamics.  The fibers of $f$ parameterize the recurrent sets, and the order structure on $(P,\leq)$ organizes the gradient-like behavior.  

%We give a few illustrations of how a cell fibration may arise in practice:
%
%\begin{description}
%
%\item[Applied Topology]  Within applied topology, often data comes as a height function $X\to \R$, and one examines the change in the topology of the sublevel sets $f^{-1}(-\infty,t]$.  For instance, when $X$ is a collection of pixels, a new function is then defined on a cubical complex $\cX$ corresponding to the image such that the sublevel sets are subcomplexes.  This produces a cell fibration $(\cX,\preceq)\to (\R,\leq)$.  
%
%\item[Morse Theory] For a Morse function $f:M\to \R$ on smooth manifold $M$ one often examines the flow defined generated by $\dot x = -\nabla f(x)$.  The fixed points of the flow are indexed by a poset~\cite{smale} and their unstable manifolds carve out a CW decomposition of the manifold.  The map sending each cell in the CW-complex to its index within the poset is a cell fibration.
%
%\item[Dynamics] In computational dynamics, especially the database approaches, one often has a transitive relation defined on a cubical complex.  The transitive relation partitions the complex into recurrent and gradient-like behavior which takes the form of a chain fibration.  See the braids paper for a concrete example.
%
%\item[Combinatorics] Mrozek's multivector field
%
%\end{description}


\subsection{Chain Fibrations}

We lift the concept of a cell fibration to that of a {\em chain fibration} by promoting cells to chains.  Lifting to chains will facilitate the use of algebraic Morse theory.  This will also help us relate more directly to chain complex braids.  Let $(C,\partial)$ be a chain complex and $L\in {\bf FDLat}$.  Promoting cells to chains ({\bf why?}) implies we will consider functions $f:(C,\partial)\to L$.  

We write down axioms for a chain fibration:

\begin{defn}
{\em 
A {\em chain fibration} is a map $f:C\to L$ such that
\begin{enumerate}
\item $f\circ \partial(a) \leq f(a)$

\item $f(a+ b) \leq f(a)\vee f(b)$

\item $f(\lambda a) = f(a)$ for $\lambda\in k$ and $\lambda \neq 0$ 

\item $f^{-1}(0_L) = \{0 \in C_n: n\in \N\}$
\end{enumerate}
}
\end{defn}

\begin{rem}
These axioms are related to non-Archimedean norms on chain complexes.
\end{rem}

It is straightforward that a chain fibration organizes subcomplexes of $C$, which is the content of the next proposition.

\begin{prop}\label{prop:cfDown}
Let $f:C\to L$ be a chain fibration and let $p\in L$.  Then $B(p,f):=f^{-1}\{q\in L: q\leq p\}$ is a subcomplex.  
\end{prop}
%\begin{proof}
%$B(p)$ inherits a grading from $C$. Let $B_n\subset C_n$  be the $n$th component of $B(p)$.  Let $a\in B_n$.  Then property (1) gives $f\partial (a)\leq f(a)\leq p$, implying $\partial a\in B_n$.  Therefore $\partial(B_n)\subset B_n$.  To see $B_n$ is a subspace, observe that:
%\begin{enumerate}
%\item $0\in B_n$ as $0_L \leq p$ and $f(0_n)=0_L$ by property (4)
%\item For any $a,b\in B_n$ we have $f(a+b)\leq f(a)\vee f(b)\leq p \vee p = p$ from property (2).  Therefore $a+b\in B_n$.
%\item For any $a\in B_n$ and $\lambda\in k$ with $\lambda\neq 0$ we have $f(\lambda a ) = f(a) \leq p$ from (3). Therefore $\lambda a \in B_n$.
%\end{enumerate}
%\end{proof}


Therefore a chain fibration has an associated assignment $O_f:L\to Sub(C)$ given by $$L\ni p\mapsto f^{-1}\{q\in L:q\leq p\}\in Sub(C)$$ This assignment resembles the contravariant down-set functor $O:{\bf Poset}\to {\bf FDLat}$ of Birkhoff's theorem and motivates both the notation and the following definition.  


\begin{defn}\label{def:coherent}
{\em
Let $(C,\partial)$ be a chain complex and $L\in {\bf FDLat}$.  We say that a chain fibration $f:C\to L$ is {\em coherent} if the associated assignment $O_f:L\to Sub(C)$ is a lattice homomorphism.
}
\end{defn}


Recall that in the terminology of~\cite{salamon} $P$-filtered chain complex refers to a lattice homomorphism $O(P)\to Sub(C)$ for a chain complex $C$.  Therefore a chain fibration $f:C\to L$ is coherent if $O_f$ is a $P$-filtered complex.   Notice that if $O_f$ is a lattice homomorphism then the homomorphic image of $L$ in $S(C)$ is a distributive sublattice.

%
%In fact, if $f:C\to L$ is any function such that the assignment $L\mapsto \mapsto f^{-1}\{q\in L:q\leq p\}\in P(C)$ is a lattice homomorphism $L\to Sub(C)$, then one can show $f$ is a chain fibration.
%
%\begin{prop}\label{prop:cf}
%Let $f:C\to L$ such that    Then $f$ satisfies the following conditions:
%\begin{enumerate}
%\item $f\circ \partial(a) \leq f(a)$
%
%\item $f(a+ b) \leq f(a)\vee f(b)$
%
%\item $f(\lambda a) = f(a)$ for $\lambda\in k$ and $\lambda \neq 0$ 
%
%\item $f^{-1}(0_L) = \{0 \in C_n: n\in \N\}$
%
%\end{enumerate}
%\end{prop}
%\begin{proof}
%\begin{enumerate}
%\item Let $a\in C$.  Consider $q=f(a)$.  Then $a\in O_f(q)$ by construction.  Since $O_f(q)$ is a subcomplex, $\partial(a)\in O_f(q)$.  By definition this implies $f(\partial (a)) \leq q=f(a)$.
%
%\item Let $a,b\in C$.  Let $p=f(a)$ and $q=f(b)$.  Then $a\in O_f(a)$ and $b\in O_f(b)$.  Thus $a+b\in O_f(p)+O_f(q)$.  Since $O_f$ is a lattice homomorphism $O_f(p)+O_f(q)= O_f(p\wedge q)$.  Therefore $a+b\in O_f(p\wedge q)$ and $f(a+b)\leq p\wedge q = f(a)\wedge f(b)$.
%
%\item Let $a\in C$ and $\lambda\in k$ with $\lambda \neq 0$.  Let $p=f(x)$ and $q=f(\lambda x)$.  Then $O_f(p)$ and $O_f(q)$ are subcomplexes.  Therefore $\lambda x\in O_f(p)$ and $x\in O_f(q)$.  Thus $f(\lambda x) \leq p = f(x) \leq q = f(\lambda x)$, implying $f(\lambda x ) = f(x)$.
%
%\item From our definition of lattice morphism we have $0_L\mapsto 0_{S(C)}$.  Thus from definition $f^{-1}(0_L) = O_f(0_L) = 0_{S(C)} = \{0_n:n\in \N\}$.
%
%\end{enumerate}
%\end{proof}



%Conditions $(1)-(4)$ of Proposition~\ref{prop:cf} can be shown to guarantee that the preimage of a sublattice $\{p\in L: p \leq q\}$ is a subcomplex.  However, if one takes conditions $(1)-(4)$ of Proposition~\ref{prop:cf} as a definition for `chain fibration', then one can concoct examples of a chain fibration that will not induce a chain complex braid.  



Given a cell fibration $g:(X,\kappa, \preceq)\to (P,\leq)$ the cells $\{a_i\}$ of $(X,\kappa,\preceq)$ define a distinguished basis of $C(X)$.  In this case one may define a map $f:C(X)\to O(P,\leq)$ where $O(P)$ is the lattice of lower sets of $P$ as follows.  Every element $x\in C(X)$ can be expressed uniquely as a sum $x=\sum_i \lambda_i a_i$.  Let $\cI_x = \{i: \lambda_i\neq 0\}$.  Then for each $x\in C(X)$ define
\begin{align} \label{eqn:cell2chain}
f(x) = f(\sum_i \lambda_i a_i):= \bigcup_{i\in \cI_x} \downarrow (g(a_i))\quad\quad\quad\quad\quad\quad
f(0_d):= 0_L
\end{align}

It is easy to see from the definition that $f(\sum \lambda_i a_i)$ is indeed a lower set of $P$ since it is expressed as a union of down sets.  Thus $f$ is indeed a map $C(X)\to O(P)$.  We now show that it is a coherent chain fibration.

\begin{prop}
Let $g:(X,\preceq)\to (P,\leq)$ be a cell fibration.  Let $f:C(X)\to O(P,\leq)$ be defined as in Eqn.~(\ref{eqn:cell2chain}).  Then $f$ is a coherent chain fibration.
\end{prop}
\begin{proof}
Consider $O(g):O(P,\leq)\to O(X,\leq)$ provided by Theorem~\ref{thm:birkhoff}. For $q\in O(P,\leq)$ we have that $O(g)(q)$ is a downset, and therefore a subcomplex of $(X,\leq)$.  Moreover, as a set of cells in $(X,\leq)$ $O(g)(q)$ provies a basis for the set $O_f(q)$.  The fact that $O_f$ is a lattice homomorphism follows from the fact that $O(g)$ is a lattice homomorphism.

\end{proof}

%\begin{proof}
%
%
%It is easy to see from the definition that $f(\sum \lambda_i a_i)$ is indeed a lower set of $P$ since it is expressed as a union of down sets.  Thus $f$ is indeed a map $C(X)\to O(P)$.  There are four things to show.
%
%\begin{enumerate}
%\item Let $x\in C(X)$. If $x=0$, then $\partial x =0$ and $f\partial x = fx$.  If $x\neq 0$, then $x = \sum_i \lambda_i a_i$ for $\lambda_i\neq 0$ and $a_i\in (X,\preceq)$.  Thus $$f\partial x = f\partial (\sum_i \lambda_i a_i) =  f (\sum_i \lambda_i \partial a_i) $$  
%
%We can write $\partial a_i = \sum_j \beta_{i,j} y_{i,j}$ for $y_{i,j}\in (X,\preceq)$ and $y_{i,j}\preceq a_i$.  Then $$f\partial x = f \sum_i \lambda_i (\sum_j \beta_{i,j} y_{i,j}) \subseteq \bigcup_{i,j} \downarrow ( g(y_{i,j})) $$  
%
%The final relation is not a strict equality as there may be some cancellation of $y_{i,j}$.  As $y_{i,j}\preceq x_i$ we have $g(y_{i,j})\leq g(a_i)$, implying $\downarrow (g(y_{i,j}))\subseteq \downarrow ( g(a_i))$.  Therefore $$f(\partial x)\subseteq  \bigcup_{i,j} \downarrow ( g(y_{i,j})) \subseteq \bigcup_i \downarrow( g(a_i)) = f(x)$$
%
%
%\item Let $a,b\in C_n(X)$.  Since $X^n$ forms a basis we have $a = \sum_i \lambda_i x_i$ and $b=\sum_i \beta_i x_i$ for $x_i\in (X,\preceq)$.  As it may be that some $\beta_i=0$ or $\lambda_i=0$,  $$f(a),f(b)\subseteq \bigcup_i \downarrow g(x_i) $$  Now $$f(a+b) = f(\sum_i (\lambda_i+\beta_i)x_i) \subseteq \bigcup_i \downarrow(g(x_i))$$  Again, the final relation is not strict equality, as it may be the case that $\lambda_i+\beta_i=0$ for some $i$.  
%
%\item $f(\lambda a) = f(a)$ for $\lambda\neq 0$. Let $a\in C$.  Then we may write $a = \sum_i \lambda_i x_i$ with $\lambda_i\neq 0$.  Thus $\lambda\lambda_i\neq 0$.  Then $$f(\lambda a) = f(\lambda \sum_i \lambda_i x_i) = f(\sum_i \lambda \lambda_i x_i) = \bigcup_i \downarrow g(x_i) = f(a)$$
%
%\item  $f^{-1}(0_L) = \{0 \in C_n:n\in \N\}$.  We have $\{0\in C_n:n\in \N\}\subset f^{-1}(0_L)$ as by the construction we set $f(0)=0_L$ for $0\in C_n$.  Now let $a\in f^{-1}(0_L)$.  If $a\neq 0$, then we may write $a = \sum_i \lambda_i x_i$ for $\lambda_i\neq 0$ for $x_i\in (X,\preceq)$.  Thus $$f(a) = \bigcup_i \downarrow (g(x_i)) \neq \emptyset$$  This forces $a=0$.
%
%\end{enumerate}  
%
%%We will use Corollary~\ref{cor:cfEquiv} to show that $f$ is a chain fibration.  As $g:(X,\preceq) \to (P,\leq)$ is a cell fibration, one may apply the contravariant functor provided by Birkhoff's theorem (Theorem~\ref{thm:birkhoff}) to obtain a lattice homomorphism $O(g):O(P)\to O(X,\preceq)$. Notice that $O(X,\preceq)$ are lower sets of the complex $(X,\preceq)$.  Therefore they are closed subcomplexes by Corollary~\ref{cor:clsubcomplex}.  For each $p\in L$, $O(g)(p)$ gives a distinguished basis for $f^{-1}(q\in L:q\leq p$.  Therefore these are subcomplexes.  
%%
%%
%%
%%
%%
%%
%%
%%Then $D_L:=O(g)(L)$ is a closed subcomplex in $(X,\preceq)$ with a distinguished basis.  We claim that $O(g)(L)$ is a distinguished basis for $D_L$.  Let $x\in D_L$.  We can write $x = \sum \sigma_i a_i$ with $a_i\in X$.  We want to show that $a_i\in O(g)(L)$.  Since $x\in D_L$ we have $f(\sum \sigma_i a_i) = f(x) \leq L$.  By definition $f(\sum\sigma_i a_i) = \bigcup_i \downarrow(f_c(a_i))$.  Thus $\downarrow(f_c(a_i))\leq L$ in the partial order (inclusion) on $O(P)$.  Therefore $f_c(a_i)\in L$.  So $a_i\in O(f_c)(L)$.
% 
%% From the fact that $O(f_c)$ is a lattice homomorphism it follows that the mapping $L\mapsto D_L$ is also a lattice homomorphism.
%
%
%
%\end{proof}
 
 For a cell fibration $f:(\cX,\preceq)\to (P,\leq)$ we call $f:C(X)\to O(P,\leq)$ defined as in Eqn.~(\ref{eqn:cell2chain}) the {\em associated chain fibration}.


The idea of chain fibration is this: just as join-irreducibles act as a basis for the lattice, the chains associated to join-irreducibles should act as a basis for the chain complex.   This is captured in the next result, Theorem~\ref{thm:cfdecomp:basis}.


\begin{thm} \label{thm:cfdecomp:basis}
Let $f:C\to L$ be a chain fibration.  There is a filtered basis $B=\{b_\alpha\}$ of $C$ with $f(b_\alpha)\in J(L)$ such that for any $p\in L$ the set $\{b\in B: f(b)\leq p\}$ is a basis for $B(p)$.  

\end{thm}
\begin{proof}

%%
%% 
%%
%% $(2)\implies (1)$.  We wish to show that the map $p\mapsto D_p$ is a lattice homomorphism.  By hypothesis $D_p = \bigoplus_{q\in J(L),q\leq p} C_q$. The assignment is a homomorphism as
%%
%%\begin{align*}
%%O(f)(p) \cap O(f)(r) = D_p\cap D_q = \bigoplus_{q\in J(L),q\leq p} C_q \cap \bigoplus_{q\in J(L), q\leq r} C_q &= \bigoplus_{q\in J(L),q\leq p\wedge r} C_q \\&= D_{p\wedge r} = O(f)(p\wedge r)
%%\end{align*}
%%
%%\begin{align*}
%%O(f)(p) + O(f)(r) = D_p+ D_q = \bigoplus_{q\in J(L),q\leq p} C_q + \bigoplus_{q\in J(L), q\leq r} C_q &= \bigoplus_{q\in J(L),q\leq p\vee r} C_q \\&= D_{p\vee r} = O(f)(p\vee r)
%%\end{align*}

 We will first construct subspaces associated to each $q\in J(L)$ then choose a basis for these subspaces.  Let $q\in J(L)$.  Consider the set $\{p_1,\ldots,p_n\}$ of maximal $p_i\in J(L)$ with $p_i < q$.  Then $\bigvee_i p_i = Pred(q)$.   Since $f$ is coherent, $D_{Pred(q)} = D_{p_1}+\ldots + D_{p_n}$.  Choose a subspace $V_q$ such that $D_q = V_q \oplus D_{Pred(q)}$.  Notice that for any minimal $q\in J(L)$ we have $Pred(q)=0_L$.  Thus $D_{Pred(q)}=0$ implying $V_q = D_q$.

We claim that $C = \bigoplus_{q\in J(L)} V_q$.  We first show that $V_q\cap V_p=0$ for $q\neq p\in J(L)$.  Let $x\in V_q\cap V_p$.  Then $x\in D_q\cap D_p = D_{q\wedge p}$.  However, $D_{q\wedge p}\subseteq D_{Pred(q)}$ and $D_{q\wedge p}\subseteq D_{Pred(p)}$.  Thus $x=0$ by choice of $V_q$ and $V_p$.  

Now we wish to show that $\bigoplus_{q\in J(L)} V_q$ span $C$.  We will prove this by strong induction.  We will induct over a linear extension of $L$.  The base case is to consider the minimal element, $0_L\in L$.  By (4) of~\ref{def:cf} if $f(x)=0_L$ then $x=0$, which is in the span.   Now fix $p\in L$.  The strong inductive hypothesis is to assume that for any $q< p$ any $x$ with $f(x)=q$ is in span $\bigoplus_{q\in J(L)} V_q$.  Let $p\in L$.  By Lemma~\ref{lem:join} we may write $p$ as an irredundant join $p=\bigvee_i q_i$ with $q_i \in J(L)$. Notice if $p\in J(L)$ then the decomposition is trivially written as $p=p$.  Coherence implies that $D_p = D_{q_1}+D_{q_2}+\ldots+D_{q_n}$.  Thus $x= \sum_i \lambda_i x_{q_i}$.  There are two cases.  First, if $p\not\in J(L)$, then $q_i< p$ and each $x_{q_i}$ belongs to the span.  For the second case, $p\in J(L)$ and we may write $D_p = V_p \bigoplus D_{Pred(p)}$.  Thus $x = v_p + x_{Pred(p)}$.  Since $Pred(p)<p$ the inductive hypothesis implies that $x_{Pred(p)}$ is in the span. 


Choose a basis $B_q$ for each $V_q$.  We have shown that $\bigsqcup_{q\in J(L)} B_q$ is a basis for $C$ and $\bigsqcup_{q\leq p, q\in J(L)} B_q$ is a basis for $D_p$.   We now show $f(b)=q$ for $b\in B_q$. Suppose that $f(b)\neq q$.  As $b\in B_q\subset D_q$ then $f(b)< q$.  Therefore $f(b)\leq Pred(q)$, implying $b\in D_{Pred(q)}$ and forcing $b=0$ by our choice of $V_q$.  This contradicts our choice of $b$, therefore $f(b)=q$. 

\end{proof}





%Theorem~\ref{thm:cfdecomp:basis} shows that the idea of coherence is the appropriate model for a cell fibration - cells may be thought of as basis elements, which map to $J(L)$.  
 
\begin{defn}
{\em
Let $f:C\to L$ be a chain fibration. For $p\in L, p\neq 0_L$ we have a chain map $B(Pred(p))\hookrightarrow B(p)$.   We denote by $C(p)$ quotient chain complex $C(p):= B(p)/B(Pred(p))$.  We say that this is the {\em center complex at $p$}.
}
\end{defn}
 
 
 
 \begin{cor}\label{thm:cfdecomp:JLDecomp}
 Let $f:C\to L$ be a chain fibration.  There is a decomposition into center subcomplexes at join-irreducibles $C= \bigoplus_{q\in J(L)} C(q)$ so that for each $p\in L$ $$B(p)= \bigoplus_{q\in J(L), q\leq p} C(q)$$ 
 \end{cor}
 \begin{proof}
  We first give a characterization of the center subcomplexes.  Let $q\in J(L)$.  By hypothesis $B_q:= \{b_i: f(b_i)\leq q\}$ is a basis for $D_q$.  Set $A_q =  \{b\in B:f(b)=q\}$.  Then $B_q = A_q\bigsqcup \{b\in B:f(b)< q\}$.  Notice that $\{b:f(b)<q\} = \{b_i:f(b_i)\leq Pred(q)\}$ which is a basis for $D_{Pred(q)}$.    $C_q := D_q/ D_{Pred(q)} = span(A_q)$.  As $B = \bigsqcup_{q\in J(L)} A_q$ we have that $C=\bigoplus_{q\in J(L)} C_q$.  Let $p\in L$.  Then $B_p$ is a basis for $D_p$ and $B_p = \bigsqcup_{q\in J(L),q\leq p} A_q$.  Thus $D_q = \bigoplus_{q\in J(L),q\leq p} C_q$.



 \end{proof}

 
 
 We call the decomposition of Corollary~\ref{thm:cfdecomp:JLDecomp} the join-irreducible decomposition.   
 
 
 
 
 Rewriting $\partial$ with respect to the decomposition, the condition that $f\partial \leq f$ implies that for $p\not\leq q$ $\partial(p,q):=C_p\hookrightarrow C \xrightarrow{\pi} C_q =0$.   Therefore $\partial$ is an {\em upper triangular boundary map} in the sense of Franzosa~\cite[Definition 3.1]{fran}.   We formalize this as a corollary:

%Thus $\partial$ is an {\em upper triangular boundary map} with respect to the $J(L)$ decomposition.  

% The decomposition of (2) is basically what~\cite{salamon} labels as $P$-splitting for an $P$-filtered module.  We call this decomposition a $J(L)$-decomposition of $f:C\to L$.  A $P$-splitting determines a $P$-filtered module (in our parlance,  we observed this in the proof as the $J(L)$ decomposition determining a coherent chain fibration).  
 
 %Consider $p\not\leq q$.  If $x\in D_p$ then $f\partial x \leq fx \leq p$.  Thus $f\partial x \not\leq q$.  Therefore $\partial$ is upper triangular with respect to the decomposition $C=\bigoplus_{q\in J(L)} C_q$.  
 
 
 \begin{cor}\label{cor:ccf:ut}
 Let $f:C\to L$ be coherent.  Then $\partial$ is an upper triangular boundary map with respect to the $J(L)$ decomposition $C=\bigoplus_{q\in J(L)}C_q$.
 \end{cor}

%
%The following observation of Franzosa implies that for $f:C\to L$ coherent there is an induced chain complex braid $\cC(f)$.
%
%\begin{prop}[\cite{fran}, Proposition 3.4]
%Given an upper triangular boundary map $$\Delta:\bigoplus_{q\in J(L)} C_q\to \bigoplus_{q\in J(L)} C_q$$ the collection, denoted $\cC\Delta(J(L))$, consisting of the chain complexes $C(I)$ with boundary map $\partial(I)$ for each $I\in I(J(L))$ and the obvious chain maps $i(I,IJ)$ and $p(IJ,J)$ for each $(I,J)\in I_2(J(L))$ is a chain complex braid over $J(L)$.
%\end{prop}
%
%
%

Corollary~\ref{cor:ccf:ut}, in combination with Proposition~\ref{prop:fran:3.4}, provides a assignment of a chain complex braid to each chain fibration.  We remark more upon this in Section~\ref{sec:homotopy}.


\subsection{Connection Fibrations}

We now describe {\em connection fibrations}, which are particularly simple chain fibrations. This is our analogue of the connection matrix of Franzosa.

\begin{defn}\label{def:connection}
{\em
A {\em connection fibration} is a chain fibration $f:C\to L$ if $f\circ\partial < f$
}
\end{defn}



Due to the $J(L)$-decomposition connection fibrations may be characterized by the center subcomplexes at join-irreducibles. 


\begin{prop}
Let $f:C\to L$ be a coherent chain fibration.  Then $f$ is a connection fibration if and only if $C_p$ has zero boundary map for all $p\in J(L)$.
\end{prop}



\begin{rem}
Notice that $\partial_p = 0$ for  $p\in J(L)$.  Thus the join-irreducible decomposition of a connection fibration $f:C\to L$ may be written in terms of the homology, i.e. in form $$C=\bigoplus_{q\in J(L)} HC_q$$
\end{rem}

Here are some more results about connection fibrations: {\bf Why are these needed?}

\begin{prop}
Let $f:C\to L$ be a connection fibration.  Suppose $\phi\in Hom(g,f)$ is injective.  Then $g$ is a connection fibration.
\end{prop}
\begin{proof}
Consider $g:C'\to L$.  We show that $g\partial < g$.  Let $x\in C'$.  Then 


\end{proof}



\begin{prop}
Let $\phi\in Hom(f,g)$.  Then $\ker \phi$ and $im\phi$ are chain fibrations.
\end{prop}






%\begin{rem}[KS]
%For our own purposes we can show that the following definition is equivalent.
%\end{rem}
%\begin{prop}
%An equivalent definition of connection fibration is as follows: a chain fibration $f:C\to L$ such that every center subcomplex has a trivial boundary map.  For each $p$ we have that $(C_p,\partial_p)$ the center subcomplex at $p$ has $\partial_p \equiv 0$.
%\end{prop}
%\begin{proof}
%Let $p\in J(L)$.  Then $q:=Pred(p) < p$ and is unique.  Consider $D_p:= O_f(p)$ and $D_{q} := O_f(q)$.  $D_p/D_q$.  For $x\in D_p$ we have $f\partial x < f x$.  Thus $f\partial x \in D_q$.  Therefore $\partial_p = 0$.  For $p\not\in J(L)$ we can write $D_p$ as a direct sum over join-irreducibles (see the $J(L)$ decomposition) and repeat the argument.
%
%On the other hand, assume this new definition.  Let $x\in C$.  Consider $p=f(x)$.  Consider $D_p$.  Then $\partial_p (x) = 0$ so $\partial (x) \in D_{Pred(p)}$.  Thus $f\partial x < fx$.
%
%
%\end{proof}


%\begin{defn}\label{def:connection}
%{\em
%A {\em connection fibration} is a chain fibration $f:C\to L$ such that every center subcomplex has a trivial boundary map.  For each $p$ we have that $(C_p,\partial_p)$ the center subcomplex at $p$ has $\partial_p \equiv 0$.
%}
%\end{defn}

%
%
%
%\begin{rem}
%Here's an alternate definition of connection fibration that may be easier: $f$ is a {\em connection fibration} if  $f:C\to L$ such that for all $x\in C$ we have $f\partial x < f x$.
%\end{rem}





 

%!TEX root = ./main.tex
\section{Coherent Chain Fibrations}

In this section we strengthen chain fibrations using the idea of coherence.  This idea is critical to relate chain fibrations to chain complex braids.  Recall that Section~\ref{sec:prelims:order} we introduced $S(C)$, the lattice of subcomplexes for a chain complex $(C,\partial)$ with operations $+$ and $\cap$. 

\begin{defn}\label{def:coherent}
{\em
Let $f:C\to L$ be a chain fibration.  Let $O_f:L\to S(C)$ denote the assignment $$L\ni p\mapsto B(p,f)= f^{-1}\{q\in L: q\leq p\}\in S(C)$$

We say that $f$ is {\em coherent} if $O_f$ is a lattice homomorphism.
}
\end{defn}

The idea of coherent chain fibration is this: just as join-irreducibles act as a basis for the lattice, the chains associated to join-irreducibles should act as a basis for the chain complex. This idea is fundamental and can be seen in the work of~\cite{salamon}.  Recall that their term $P$-filtered chain complex refers to a lattice homomorphism $O(P)\to S(C)$ for a chain complex $C$.  Without the assumption of coherent, one can concoct examples of a chain fibration that will not induce a chain complex braid.  Notice that if $O(f)$ is a lattice homomorphism then the homomorphic image of $L$ in $S(C)$ is a distributive sublattice.  It is insightful to have equivalent characterizations of coherent.  


\begin{defn}
{\em
Let $f:C\to L$ be a chain fibration. For $p\in L, p\neq 0_L$ we have a chain map $B(Pred(p))\hookrightarrow B(p)$.   We denote by $C(p)$ quotient chain complex $C(p):= B(p)/B(Pred(p))$.  We say that this is the {\em center complex at $p$}.
}
\end{defn}





\begin{thm}\label{thm:cfdecomp}
Let $f:C\to L$ be a chain fibration.  Then the following are equivalent:

\begin{enumerate}
\item $f$ is coherent


\item \label{thm:cfdecomp:JLDecomp} there is a decomposition into center subcomplexes at join-irreducibles $C= \bigoplus_{q\in J(L)} C(q)$ so that for each $p\in L$ $$B(p)= \bigoplus_{q\in J(L), q\leq p} C(q)$$ 

\item \label{thm:cfdecomp:basis} There is a filtered basis $B=\{b_\alpha\}$ of $C$ with $f(b_\alpha)\in J(L)$ such that for any $p\in L$ the set $\{b\in B: f(b)\leq p\}$ is a basis for $B(p)$.  


\end{enumerate}

\end{thm}
\begin{proof}



%\item $(3)\implies (2)$ is straightforward.  


 $(3)\implies (2)$.  We first give a characterization of the center subcomplexes.  Let $q\in J(L)$.  By hypothesis $B_q:= \{b_i: f(b_i)\leq q\}$ is a basis for $D_q$.  Set $A_q =  \{b\in B:f(b)=q\}$.  Then $B_q = A_q\bigsqcup \{b\in B:f(b)< q\}$.  Notice that $\{b:f(b)<q\} = \{b_i:f(b_i)\leq Pred(q)\}$ which is a basis for $D_{Pred(q)}$.    $C_q := D_q/ D_{Pred(q)} = span(A_q)$.  As $B = \bigsqcup_{q\in J(L)} A_q$ we have that $C=\bigoplus_{q\in J(L)} C_q$.  Let $p\in L$.  Then $B_p$ is a basis for $D_p$ and $B_p = \bigsqcup_{q\in J(L),q\leq p} A_q$.  Thus $D_q = \bigoplus_{q\in J(L),q\leq p} C_q$.



%Let $A_q:= \{b\in B: f(b)=q\}$.  By hypothesis $B=\bigsqcup_{q\in J(L)} A_q$.  We set $C_q:= span(A_q)$.  As $B$ is a basis for $C$ we have $C=\bigoplus_{q\in J(L)} C_q$.  For $q\in J(L)$ we have $B_q:=\{b\in B:f(b)\leq q\}$ is a basis for $D_q$.  Notice that $B_q = A_q \bigsqcup \{b\in B:f(b)< q\}$. Notice that $\{b_i:f(b_i)<q\} = \{b_i:f(b_i)\leq Pred(q)\}$, a basis for $D_{Pred(q)}$.  Thus $C_q = D_p / D_{Pred(q)}$ and is
 


%We can partition this as $$B_q = \{b_i:f(b_i)=q\}~\bigsqcup ~\{b_i:f(b_i)< q\}$$  Notice that $\{b_i:f(b_i)<q\} = \{b_i:f(b_i)\leq Pred(q)\}$.  This is a basis for $D_{Pred(q)}$.  Set $A_q := \{b_i:f(b_i)=q\}$.  Thus we can identify $span A_q = C_q := D_q/D_{Pred(q)}$.  As $B=\bigsqcup_{q\in J(L)} A_q$ we have $C=\bigoplus_{q\in J(L)} C_q$ with $D_p = \bigoplus_{
%
% Now let $p\in L$.  We have that $A=\{b_i:f(b_i)\leq p\}$ is basis for $D_p$.  There is a partition of $A$ by $$A=\bigsqcup_{q\in J(L)} \{b_\alpha\in A: f(b_\alpha) = q\} = \bigsqcup_{q\in J(L), q\leq p} \{b_\alpha\in B: f(b_\alpha)=q\}$$
% 
% Thus $D_p = span (A) = \bigoplus_{q\in J(L),q\leq p} C_q$.

 

 $(2)\implies (1)$.  We wish to show that the map $p\mapsto D_p$ is a lattice homomorphism.  By hypothesis $D_p = \bigoplus_{q\in J(L),q\leq p} C_q$. The assignment is a homomorphism as

\begin{align*}
O(f)(p) \cap O(f)(r) = D_p\cap D_q = \bigoplus_{q\in J(L),q\leq p} C_q \cap \bigoplus_{q\in J(L), q\leq r} C_q &= \bigoplus_{q\in J(L),q\leq p\wedge r} C_q \\&= D_{p\wedge r} = O(f)(p\wedge r)
\end{align*}

\begin{align*}
O(f)(p) + O(f)(r) = D_p+ D_q = \bigoplus_{q\in J(L),q\leq p} C_q + \bigoplus_{q\in J(L), q\leq r} C_q &= \bigoplus_{q\in J(L),q\leq p\vee r} C_q \\&= D_{p\vee r} = O(f)(p\vee r)
\end{align*}

 $(1)\implies (3)$.  Let $f:C\to L$ be coherent.  We will first construct subspaces associated to each $q\in J(L)$ then choose a basis for these subspaces.  Let $q\in J(L)$.  Consider the set $\{p_1,\ldots,p_n\}$ of maximal $p_i\in J(L)$ with $p_i < q$.  Then $\bigvee_i p_i = Pred(q)$.   Since $f$ is coherent, $D_{Pred(q)} = D_{p_1}+\ldots + D_{p_n}$.  Choose a subspace $V_q$ such that $D_q = V_q \oplus D_{Pred(q)}$.  Notice that for any minimal $q\in J(L)$ we have $Pred(q)=0_L$.  Thus $D_{Pred(q)}=0$ implying $V_q = D_q$.

We claim that $C = \bigoplus_{q\in J(L)} V_q$.  We first show that $V_q\cap V_p=0$ for $q\neq p\in J(L)$.  Let $x\in V_q\cap V_p$.  Then $x\in D_q\cap D_p = D_{q\wedge p}$.  However, $D_{q\wedge p}\subseteq D_{Pred(q)}$ and $D_{q\wedge p}\subseteq D_{Pred(p)}$.  Thus $x=0$ by choice of $V_q$ and $V_p$.  

Now we wish to show that $\bigoplus_{q\in J(L)} V_q$ span $C$.  We will prove this by strong induction.  We will induct over a linear extension of $L$.  The base case is to consider the minimal element, $0_L\in L$.  By (4) of~\ref{def:cf} if $f(x)=0_L$ then $x=0$, which is in the span.   Now fix $p\in L$.  The strong inductive hypothesis is to assume that for any $q< p$ any $x$ with $f(x)=q$ is in span $\bigoplus_{q\in J(L)} V_q$.  Let $p\in L$.  By Lemma~\ref{lem:join} we may write $p$ as an irredundant join $p=\bigvee_i q_i$ with $q_i \in J(L)$. Notice if $p\in J(L)$ then the decomposition is trivially written as $p=p$.  Coherence implies that $D_p = D_{q_1}+D_{q_2}+\ldots+D_{q_n}$.  Thus $x= \sum_i \lambda_i x_{q_i}$.  There are two cases.  First, if $p\not\in J(L)$, then $q_i< p$ and each $x_{q_i}$ belongs to the span.  For the second case, $p\in J(L)$ and we may write $D_p = V_p \bigoplus D_{Pred(p)}$.  Thus $x = v_p + x_{Pred(p)}$.  Since $Pred(p)<p$ the inductive hypothesis implies that $x_{Pred(p)}$ is in the span. 

%Now we do two cases.  First, consider $x\in C$ with $f(x)=q\in J(L)$.  Then $x\in D_q = V_q\bigoplus D_{Pred(q)}$.  Thus we can write $x= x_q + x_{Pred(q)}$.  

%We do two cases.  First, let $x\in C$ with $f(x)=q\in J(L)$.  Consider $D_q=V_q\bigoplus D_{Pred(q)}$.  We can write $x = x_q + x_{Pred(q)}$.  We can do downward induction with $x_{Pred(q)}$.  The base case is that for $Pred(q)=0_L$ then $x = x_q + x_{Pred(q)}$ with $x=x_q\in D_q=V_q$ and $x_{Pred(q)}=0$. Now let $x\in C$ with $f(x)=p$.  We can write $p$ as an irredundant join $p = \bigvee_i q_i$ with $q_i\in J(L)$ by Lemma~\ref{lem:join}.  Since $D_p = D_{q_1} + D_{q_2}+\ldots + D_{q_n}$ we can write $x = \sum_i \lambda_i x_{q_i}$ and use the above argument.

Choose a basis $B_q$ for each $V_q$.  We have shown that $\bigsqcup_{q\in J(L)} B_q$ is a basis for $C$ and $\bigsqcup_{q\leq p, q\in J(L)} B_q$ is a basis for $D_p$.   We now show $f(b)=q$ for $b\in B_q$. Suppose that $f(b)\neq q$.  As $b\in B_q\subset D_q$ then $f(b)< q$.  Therefore $f(b)\leq Pred(q)$, implying $b\in D_{Pred(q)}$ and forcing $b=0$ by our choice of $V_q$.  This contradicts our choice of $b$, therefore $f(b)=q$. %Therefore $\bigsqcup_{q\leq p,q\in J(L)} B_q= \{b_i:f(b_i)\leq p\}$ is a basis for $D_p$.

%
%
%thus $D_{q\wedge p} \subseteq D_{Pred(q)\wedge Pred(p)}$.  
%
%Let $B_q$ be a basis for $V_q$.  We claim that for $b\in B_q$ we have $f(b)=q$.  If not, then $f(b)< q$.  However, this implies that $f(b)\leq Pred(q)$, thus $f(b)\in span D_{p_1\vee \ldots \vee p_n}$.  This contradicts the fact that $b\in V_q$.  We claim that $\bigsqcup_{q\in J(L)} B_q$ is a basis.  
%
%Suppose that $V_{q_1}\cap V_{q_2}\neq 0$.  
%
%Now we must show that $\{b_i:f(b_i)\leq q\}$ is a basis for $D_q$.  For each $p\in J(L)$ with $\leq q$ we have $B_q$  
%
%
%
%Consider the set of $A$ of noncomparable, minimal $q\in J(L)$.   For each $q\in A$ we may choose a basis $B_q$ for $D_q$.  Since $f^{-1}0_L = \{0_d\}$, for any $b\in B_q$ we have that $f(b)=q$.
%
%Now fix $p\in J(L)$.  Assume by induction that we have such bases $B_q$ for all $q\in J(L)$ with $q<p$.  Consider the set $Y$ of noncomparable, maximal $q\in J(L)$ such that $q<p$. Consider 
%
%
%
%Now we induct.  Let $q\in J(L)$.  For all  maximal $p\in J(L)$ with $p<q$ we have chosen a basis for $D_p$ that maps into $p$.  These $p$ are noncomparable so the intersection of their subspaces is 0 (lattice property). Consider the disjoint union of these basis.  Extend this to a basis for $D_q$.  We claim that the vectors in this extension must map to $q$.  To see, notice that the span of the disjoint union is $D_{Pred(q)}$.  Thus if $x$ is in the extension but not in the union, and doesn't map to $q$ then $f(x)<q$.  Thus $f(x)\leq Pred(q)$.  However this implies its in the span, a contradiction that it is linearly independent with the disjoint union.  
%
%Then $\{D_p\}_{p\in L}$ generate a distributive sublattice in the lattice of subspaces of $C$.  Thus by the Proposition there is a basis $B=\{x_1,\ldots,x_n\}$ of $C$ such that each $D_p$ is generated by a subset of $B$.  We now wish to show that each $f(x_i)\in J(L)$ for each $i$.  Suppose that $f(x_i)=p\not\in J(L)$.  We can write in terms of an irredundant join of join-irreducibles: $p = q_1\vee q_2 \vee \ldots \vee q_m$ with $q_i\in J(L)$.  Consider $D_{q_1}+\ldots +D_{q_m}$.  Notice that $x_i\not\in D_{q_i}$ and thus $x_i$ is not in this span  as it is linearly independent.  Therefore $D_p\neq D_{q_1}+\ldots +D_{q_m}$ which is a contradiction of coherence.  

\end{proof}





 Part (\ref{thm:cfdecomp:basis}) of Theorem~\ref{thm:cfdecomp} shows that the idea of coherence is the appropriate model for a cell fibration - cells may be thought of as basis elements, which map to $J(L)$.  We call the decomposition of~(\ref{thm:cfdecomp:JLDecomp}) in Theorem~\ref{thm:cfdecomp}  the join-irreducible decomposition.   Rewriting $\partial$ with respect to the decomposition, the condition that $f\partial x \leq f$ implies that for $p\not\leq q$ $\partial(p,q):=C_p\hookrightarrow C \xrightarrow{\pi} C_q =0$.   Therefore $\partial$ is an {\em upper triangular boundary map} in the sense of Franzosa~\cite[Definition 3.1]{fran}.   We formalize this as a corollary:

%Thus $\partial$ is an {\em upper triangular boundary map} with respect to the $J(L)$ decomposition.  

% The decomposition of (2) is basically what~\cite{salamon} labels as $P$-splitting for an $P$-filtered module.  We call this decomposition a $J(L)$-decomposition of $f:C\to L$.  A $P$-splitting determines a $P$-filtered module (in our parlance,  we observed this in the proof as the $J(L)$ decomposition determining a coherent chain fibration).  
 
 %Consider $p\not\leq q$.  If $x\in D_p$ then $f\partial x \leq fx \leq p$.  Thus $f\partial x \not\leq q$.  Therefore $\partial$ is upper triangular with respect to the decomposition $C=\bigoplus_{q\in J(L)} C_q$.  
 
 
 \begin{cor}\label{cor:ccf:ut}
 Let $f:C\to L$ be coherent.  Then $\partial$ is an upper triangular boundary map with respect to the $J(L)$ decomposition $C=\bigoplus_{q\in J(L)}C_q$.
 \end{cor}

%
%The following observation of Franzosa implies that for $f:C\to L$ coherent there is an induced chain complex braid $\cC(f)$.
%
%\begin{prop}[\cite{fran}, Proposition 3.4]
%Given an upper triangular boundary map $$\Delta:\bigoplus_{q\in J(L)} C_q\to \bigoplus_{q\in J(L)} C_q$$ the collection, denoted $\cC\Delta(J(L))$, consisting of the chain complexes $C(I)$ with boundary map $\partial(I)$ for each $I\in I(J(L))$ and the obvious chain maps $i(I,IJ)$ and $p(IJ,J)$ for each $(I,J)\in I_2(J(L))$ is a chain complex braid over $J(L)$.
%\end{prop}
%
%
%

 The proposition provides an assignment from coherent chain fibrations to chain complex braids.  We remark more upon this in Section~\ref{sec:homotopy}.


A cell fibration $f:(X,\preceq)$ induces a chain fibration $f:C(X)\to O(P)$.  Moreover, Birkhoff's theorem guarantees that for the cell fibration $f:(X,\preceq)\to (P,\leq)$ there is a lattice homomorphism $O(f):O(P)\to O(X)$ where $O(f)(p)$ provides a basis for $D_p=\{x\in C:f(x)\leq p \}$.  This observation yields the following:

\begin{prop}
Let $f:(X,\preceq) \to (P,\leq)$ be a cell fibration. Then the associated chain fibration $f:C(X) \to O(P)$ is  coherent.
\end{prop}



Due to the $J(L)$-decompositon, coherent chain fibrations have a simpler characterization of the connection fibration expressed only in terms of join-irreducibles. 


\begin{prop}
Let $f:C\to L$ be a coherent chain fibration.  Then $f$ is a connection fibration if and only if $C_p$ has zero boundary map for all $p\in J(L)$.
\end{prop}



\begin{rem}
Notice that $\partial_p = 0$ for  $p\in J(L)$.  Thus the join-irreducible decomposition of a coherent connection fibration $f:C\to L$ may be written in terms of the homology, i.e. in form $$C=\bigoplus_{q\in J(L)} HC_q$$
\end{rem}



\section{A Category of Chain Fibrations}


For the purposes of this paper, we will fix the target lattice $L$.  We will construct a category whose objects are coherent chain fibrations over $L$.  We call this category $\bChF(\F, L)$.  As $\F$ is fixed for the purposes of this paper, we will use $\bChF(L)$ and suppress the dependence on $\F$.


\begin{defn}
{\em
Let $f:C\to L$ and $f':C'\to L$ be chain fibrations.   A map $\phi:C\to C'$ is {\em order-preserving} between $f$ and $f'$ if we have that $$f'\circ \phi \leq f$$
}
\end{defn}

%\begin{defn}[Fibration Morphism]
%{\em
%A {\em fibration morphism} from $f:\cC\to L$ to $f':\cC'\to L$ consists of a chain map $\phi:\cC\to \cC'$ and a lattice morphism $\phi':L\to L'$ such that $$f'\circ \phi \leq \phi'\circ f$$
%
% If $L=L'$ and $\phi' = id$, then we say $(\phi,\phi')$, or $\phi$, is a {\em strict fibration morphism}.
%}
%\end{defn}
%This is best visualized as a diagram:
%\[
%\xymatrixcolsep{0.25in}
%\xymatrixrowsep{.25in}
%\xymatrix{
% & C \ar[r]^{f} \ar[d]_{\phi} & L  \\
%& C' \ar[ur]_{f'} & 
%}
%\]

Notice that for $f:(C,\partial)\to L$ the boundary operator $\partial:C\to C$ is order-preserving.  For $f,f'\in \bChF(L)$ we set $Hom(f,f')$ to be the set of order-preserving chain maps $C\to C'$.  We will call such maps {\em fibration morphisms}.

\begin{prop}\label{prop:compCF}
Let $\phi\in Hom(C,C')$ with $\phi$ injective.  Let $f\in \bChF(C',L)$.  Then the composition $f\circ\phi:C\to L$ is a chain fibration.
\end{prop}
\begin{proof}
We will show that $f\circ \phi$ satisfies the hypotheses of Corollary~\ref{cor:cfEquiv}.  Let $p\in L$.  We must show that $D_p:= \{x\in C: f\circ\phi(x)\leq p\}$ is a subcomplex.  Let $x\in D_p$.  We first show $\partial(x)\in D_p$.  By definition of $x$, $\phi(x)\in \{y\in C':f(y)\leq p\}$.  This is a subcomplex since $f$ is a chain fibration thus $f\partial \phi x \leq p$.  As $\phi$ is a chain map $f\circ \phi(\partial x) = f \partial (\phi x) \leq p$.  We now show $D_p$ is a subspace.   $f\phi ( 0 ) = f (0) = 0_L$ so $0\in D_p$.  Let $x,y\in D_p$.  Then $f(\phi(x+y))=f(\phi x + \phi y)\leq f\phi x \vee f\phi y \leq p$ and for $\lambda\neq 0$ we have $f\phi (\lambda x) = f(\lambda \phi x) = f\phi x \leq p$.    

It remains to prove $f\circ \phi^{-1}(0_L) = \{0_d\}$.  This follows from the fact that $\phi$ is injective.
\end{proof}

The following corollary uses the fact that $\phi^{-1}$ preserves images and unions.
\begin{cor}\label{prop:compCoherent}
Let $f:C'\to L$ be a coherent chain fibration.  Let $\phi\in Hom(C,C')$ be injective.  Then $f\circ \phi$ is coherent.  
\end{cor}
\begin{proof}
To show that $f\circ \phi$ is coherent, we must show that the map $L\ni p\mapsto (f\circ \phi)^{-1}(p)\in Sub(C')$ is a lattice homomorphism.  Fix $p,q\in L$.    Let $D_p = f^{-1}p$ and $D_q = f^{-1}q$.  Then $D_p \cap D_q = \cap D_{p\wedge q}$ and $D_p+D_q = \cap D_{p\vee q}$.  

We have that $D_p = (f\circ \phi)^{-1}(p) = \phi^{-1}(f^{-1}D_p)$.

$$(f\circ \phi)^{-1}p = $$












\end{proof}

%\begin{rem}
%For $\phi\in Hom(C,C')$ and $f\in \bChF(C',L)$ we have a chain fibration $f\circ\phi:C\to L$.  Thus $\phi$ induces a map between $$\phi^*:\bChF(C';L)\to \bChF(C;L)$$ by $$\phi^*(f) \mapsto f'\circ \phi$$ 
%\end{rem}

%
%\begin{rem}
%The coherent chain fibrations form a subcategory of $\bChF(L)$.
%
%\end{rem}



%Let $f:C\to L,g:C'\to L$ be coherent chain fibrations.  A fibration morphism $\phi:f\to g$ induces an upper-triangular map on the associated $J(L)$-decompositions $\phi:\bigoplus_{q\in J(L)}C_q\to \bigoplus_{q\in J(L)} C_q'$ such that $\phi\partial_f = \partial_g\phi$.  The following observation shows that this induces a morphism between the associated chain complex braids $\cC(f)\to \cC(g)$.
%
%\begin{prop}[\cite{atm}, Proposition 3.2]
%Let $\Delta,\Delta'$ be an upper triangular boundary map for $\bigoplus_{q\in J(L)} C_q$ and $\bigoplus_{q\in J(L) C_q'}$, respecitvely.  If $T:\bigoplus_{q\in J(L)} C_q\to \bigoplus_{q\in J(L)} C_q'$ is upper triangular with $T\Delta = \Delta'T$ then $\cT:=\{T(I)\}_{I\in I(P)}$ is a chain complex braid morphism from $\cC(\Delta)\to \cC(\Delta')$.
%\end{prop}
%
%Therefore there is a functor $F:\bChF(L)\to CCB(J(L))$ where $CCB(J(L))$ is the category of chain complex braids over $J(L)$.











 


%!TEX root = ../main.tex

\section{The Homotopy Category of Chain Fibrations}\label{sec:homotopy}


  In this section we will build a homotopy category for $\bChF$.  This construction will be analogous to the relationship between $K$ and $Ch(k)$, which was discussed in Section~\ref{sec:prelims:AT}.

\begin{defn}
{\em
Let $f,g\in \bChF$.  Let $\phi,\psi:f\to g$ be fibration morphisms.  A {\em fibration homotopy} from $\gamma:\phi\two \psi$ is a map that is (1) a chain homotopy between $\phi$ and $\psi$ and (2) order-preserving $f\to g$, i.e.
\begin{enumerate}
\item $\phi - \psi = \partial_{C'}\circ \gamma + \gamma\circ \partial_C$
\item $g\circ \gamma \leq f$
\end{enumerate}
}
\end{defn}




If there exists a fibration homotopy from $\phi$ to $\psi$ we say they are {\em homotopic} and denote this by $\phi\sim \psi$.  

\begin{prop}
 Let $f,g\in \bChF$.  $\phi \sim \psi$ is an equivalence relation on $Hom(f,g)$.
\end{prop}
%\begin{proof}
%We have three things to prove.
%
%\begin{enumerate}
%\item Reflexive: Let $\phi \in Hom(f,g)$. We must show that $\phi\sim \phi$.  Define $\gamma=0$.
%
%\item Symmetric:  Let $\phi, \psi \in Hom(f,g)$.  Suppose that $\phi\sim \psi$ by $\gamma$. We must show $\psi\sim \phi$.  Choose $-\gamma$.
%
%\item Transitive: Suppose $\phi\sim \psi$ and $\psi \sim \theta$.  We must show $\phi\sim \theta$.  We have that there exists $\gamma:\phi \two \psi$ and $\gamma':\psi \two \theta$.  Choose $h:= \gamma+\gamma'$.  Then 
%
%$$\phi - \theta = (\phi-\psi)+(\psi-\theta) = (\partial\gamma + \gamma\partial) + (\partial\gamma'+\gamma'\partial) = \partial(\gamma+\gamma')+(\gamma+\gamma')\partial = \partial h + h \partial$$
%
%For the second condition, we have that $$f'hx = f'(\gamma x+\gamma' x) \leq f'\gamma x \vee f'\gamma' x \leq f'x \vee f' x = f'x$$
%
%\end{enumerate}
%\end{proof}


The define the category $\bK(L)$ whose objects are coherent chain fibrations over $L$ and whose hom-sets are obtained by the quotient $Hom_{\bK(L)}(f,g) := Hom_{\bChF}(f,g)/\sim$.  %This is analogous to the derived category.

\begin{defn}
{\em
Let $f,g\in \bChF$.  We say $f,g$ are {\em homotopy equivalent} if there exists morphisms $\phi:f\to g$ and $\psi:g\to f$ such that $\phi\circ \psi\sim id_g$ and $\psi\circ \phi \sim id_f$. Such maps $\phi,\psi$ are representatives of isomorphisms in $\bK(L)$.
}
\end{defn}

The next theorem is proved algorithmically with algebraic Morse theory:


\begin{thm}\label{thm:exist}
Let $f\in \bChF$.  Then there is a connection fibration $g$ which is homotopy equivalent to $f$.
\end{thm}
\begin{proof}
$f_i$ is coherent.  Thus there exists a basis by Theorem~\ref{thm:cfdecomp}, (\ref{thm:cfdecomp:basis}).  Use~\cite[Algorithm 3.6]{focm} to obtain an acyclic partial matching.  This induces a new fibration $f_{i+1}:M\to L$.  The number of basis elements must decrease monotonically, thus this process stabilizes at some point.

\end{proof}


\begin{rem}
Thus any chain fibration has a connection fibration within its homotopy equivalence class.  This is the analogue of Franzosa's existence of a connection matrix~\cite{fran}.
\end{rem}


An equivalence of chain fibrations induces an equivalence of their associated chain complex braids.  Thus the upshot is that we have the following:

\begin{prop}
Let $f,g\in \bChF$ and $f\sim g$. Then $H\cC(f)$ and $H\cC(g)$, the graded module braids induced by the chain complex braids $\cC(f)$ and $\cC(g)$, are isomorphic.
\end{prop}

Finally, we may relate our construction to Franzosa's with the following theorem:

\begin{thm}\label{thm:cfcm}
Let $f\in \bChF$.  Let $g:(C,\Delta)\to L$ be a connection fibration for $f$.  Then $\Delta$ is a connection matrix, in the sense of Franzosa~\cite[Definition 3.6]{fran}, for $H\cC(f)$, the graded module braid induced by the chain complex braid $\cC(f)$.
\end{thm}


%Another way of thinking of a connection fibration is as some sort of initial object in the homotopy equivalence class.




%!TEX root = ../main.tex



\section{Computational Technique}\label{sec:computation}

In this section we review discrete and algebraic Morse Theory - the computational tools we utilize to compute Conley complexes and connection matrices.    We use the formulation and algorithms for discrete Morse theory found in~\cite{focm}.  These will be applied to $J(L)$-filtered complexes.    We then use the ideas of algebraic Morse theory~\cite{sko} to produce splitting homotopies for the associated $L$-filtered chain complex.  Repeated applications of the algorithm lead to a Conley complex.


 \subsection{Discrete Morse Theory}
 
 Discrete Morse theory is applied to a $J(L)$-filtered complex.  We start by reviewing the theory for a complex $(X,\kappa,\preceq)$.  
 
 \begin{defn}
 {\em
 An {\em acyclic partial matching} of $(X,\kappa,\preceq)$ consists of a partition of $X$ into three sets $\cA$, $\cK$, and $\cQ$ along with a bijection $w:\cQ\to \cK$ such that the following hold:
 \begin{enumerate}
 \item {\em Incidence:} $\kappa(w(Q),Q)\neq 0$
 
 \item {\em Acyclicity:} the transitive closure of the binary relation $$Q \ll Q' \text{ if and only if } Q \preceq w(Q')$$
 
 generates a partial order $\lhd$ on $\cQ$.
 \end{enumerate}
 }
 \end{defn} 

The acyclic partial matching is used to construct a new incidence function $\kappa$.  Given an acyclic matching $(\cA,\mu:\cQ\to \cK)$ a {\em gradient path} is a nonempty sequence of cells $\rho = (Q_1,\mu(Q_1),\ldots, Q_M,\mu(Q_M))$ with $Q_i\in \cQ$ such that $Q_i\neq Q_{i+1}\preceq_\kappa Q_i$ for each $i$.  Thus successive elements from $\cQ$ in a gradient path are strictly monotonically decreasing with respect to the partial order $\lhd$.  As a consequence, no gradient path can be a cycle.  The initial cell $Q_1$ of $\rho$ is denoted ${\bf q}_\rho\in \cQ$ and the final cell $\mu(Q_M)$ by ${\bf k}_\rho \in \cK$.  The index $\nu(\rho)$ of $\rho$ is defined as $$\nu(\rho):= \frac{\prod_{i=1}^{M-1} \kappa(\mu(Q_i),Q_{i+1})}{\prod_{i=1}^{M} -\kappa(\mu(Q_i),Q_i) }$$

Given cells $A$ and $A'$ in $\cA$ a gradient path $\rho$ is a {\em connection} from $A$ to $A'$ if ${\bf q}_\rho\prec A$ and $A'\prec {\bf k}_\rho$.  The relationship is denoted by $A\stackrel{\rho}{\rightsquigarrow} A'$.  The {\em multiplicity} of a connection $\rho$ is defined to be $$m(\rho):= \kappa(A,{\bf q}_\rho)\cdot \nu(\rho)\cdot \kappa({\bf k}_\rho,A')$$

Define a new function $\tilde \kappa:\cA\to \cA\to \F$ by $$\tilde\kappa (A,A')=\kappa(A,A')+\sum_{A\stackrel{\rho}{\rightsquigarrow} A'} m(\rho)$$

Here the sum is taken over all connections $\rho$ from $A$ to $A'$.  It is defined to be 0 if no such connections exist.

\begin{prop}[\cite{mn}, Theorem 2.4]
Let $(X,\kappa,\preceq)$ be a complex over $k$.  Consider an acyclic partial matching $(\cA,\mu:\cQ\to \cK)$.  Then $(\cA,\tilde \kappa)$ is a complex and $H_\bullet(\cX)\cong H_\bullet(\cA)$.
\end{prop}

Acyclic partial matchings are relatively easy to produce, see~[Algorithm 3.6 (Coreduction-based Matching)]\cite{focm}.  Moreover, a filtered version can be done straightforwardly.  
Let $f:(X,\kappa,\preceq)\to (P,\leq)$ be a $P$-filtered cell complex.  We say that $(A,\mu)$ is a {\em filtered acyclic partial matching} if it satisfies $\mu(Q)=K$ only if $K,Q\in f^{-1}(p)$ for some $p\in P$.  That is, matchings may only occur in the same fiber.  This allows for a filtered version of the theory~\cite{mn}.
 
 \subsection{Algebraic Morse Theory}
 
 We introduce some notation.  Let $V$ be a vector space and $W\subseteq V$ be a subspace.  Let $f:V\to V$ be a linear map with $f(W)=0$.  Then there are maps $(f]:V/W\to V$, $[f):V\to V/W$ and $[f]:V/W\to V/W$ induced by $f$.   We have chosen this notation to agree with the conventional order of composition of maps.

\begin{defn}
{\em
Let $(C,d)$ be a chain complex.  A {\em splitting homotopy} is a degree +1 map $\gamma:C\to C$ such that $\gamma^2=0$ and $\gamma d\gamma = \gamma$.
}
\end{defn}


It is observed in~\cite{sko,focm} that is that acyclic partial matchings produce splitting homotopies.

\begin{prop}[\cite{sko,focm}]\label{prop:matchinghomotopy}
An acyclic partial matching $(A,\mu:K\to Q)$ induces a unique splitting homotopy $\gamma:C(X)\to C(X)$ with $im\gamma = C(A)\oplus C(K)$ and $ker\gamma = C(A)$.
\end{prop}

Moreover, given an acyclic partial matching there is an efficient algorithm to produce the associated splitting homotopy~\cite[Algorithm 3.12 (Gamma Algorithm)]{focm}.  



Given $\gamma$ we may define a graded vector space as $M^\gamma = \ker\gamma/ im \gamma$.  We can equip $M^\gamma$ with a differential $\Delta^\gamma:=[d-d\gamma d]$.

\begin{prop}
Let $(C,d)$ be a chain complex and $\gamma:C\to C$ be a splitting homotopy.  Then $\Delta^\gamma = [d-d\gamma d]$ is a differential on graded vector space $M^\gamma$.
\end{prop}
\begin{proof}
One can check that $d-d\gamma d$ is a degree -1 map with $im(d-d\gamma d)\subseteq \ker \gamma$.  Thus $d-d\gamma d$ restricts to a map $\ker\gamma^n \to \ker\gamma^{n-1}$. Moreover, $(d-d\gamma d)im\gamma= 0$.  Define $\Delta_\gamma=[d - d\gamma d]$.  A quick computation shows $\Delta_\gamma^2 = 0$ so $(M_\gamma,\Delta_\gamma)$ is a chain complex.
\end{proof}


The chain complex $(M^\gamma,\Delta^\gamma)$ is called the {\em associated Morse complex} of data $(C,d),\gamma$.  There are chain equivalences $\phi^\gamma:C\to M^\gamma$ and $\psi^\gamma:M^\gamma\to C$ given by $$\phi_\gamma =[id - \partial \circ \gamma)\quad\quad \psi_\gamma = (id - \gamma\circ \partial]$$

\begin{prop}
The maps $\phi^\gamma$ and $\psi^\gamma$ are chain maps.
\end{prop}
\begin{proof}
It is straightforward that the map $im(id_C-d\gamma)\subseteq \ker \gamma^n$ and $(id-d\gamma)\gamma = \gamma$.  Therefore the map $id_C-d\gamma$ fixes $im\gamma$ and induces a map $[id-d\gamma):C_n\to \ker\gamma^n/im\gamma^{n-1}$.  A straightforward computation shows that $\phi$ is a chain map.

Consider the map $\ker\gamma^n \to C_n$ given by $id-\gamma d$.  It is straightforward that that $(id-\gamma d)im\gamma^{n-1}=0$.  Thus there is a map $M^\gamma_n\to C_n$.  Another quick computation shows that this is a chain map.
\end{proof}

 The pair $(\phi_\gamma,\psi_\gamma)$ are called {\em Morse equivalences}.

\begin{prop}\label{prop:MorseEquiv}
For Morse equivalences $(\phi,\psi)$ we have that 
\begin{enumerate}
\item $\phi\circ \psi = id_{M_\gamma}$
\item $id_C - \psi\circ \phi = \gamma\partial_C + \partial_C \gamma$ ($\gamma$ is a chain homotopy)
\end{enumerate}
\end{prop}
\begin{proof}

\end{proof}

As a corollary, $H_\bullet M_\gamma\cong H_\bullet C$.  We say that a splitting homotopy is {\em perfect} if $\partial = \partial\gamma\partial$.  Notice if $\gamma$ is perfect then $\Delta_\gamma\equiv 0$.%\footnote{A degree +1 map $s:C_\bullet \to C_{\bullet+1}$ with $\partial=\partial s\partial$ is called a {\em splitting map}. If $C_\bullet$ admits a splitting map, then it splits as a complex~\cite[Ex. 1.4.2]{weibel}.}  

Akin to discrete Morse theory, a filtered version of the theory is straightforward.  Let $L\to Sub(C)$ be an $L$-filtered chain complex.  A {\em filtered splitting homotopy} is a splitting homotopy that is filtered.


\begin{prop}
Let $\gamma$ be a filtered splitting homotopy.  Then $\psi_\gamma,\phi_\gamma$ are filtered chain maps.
\end{prop}


\begin{prop}
Let $f:(X,\kappa,\preceq)\to J(L)$ be a filtered cell complex and $(\mu,\cA)$ a filtered acyclic matching.  Then the associated $\gamma$ is a filtered splitting homotopy the associated $L$-filtered chain complex $f:L\to Sub(C(X))$.
\end{prop}
\begin{proof}
Let $p\in P$.  Consider the $M_p$, the matching restricted to the fiber $f^{-1}p$.  By assumption this is an acyclic partial matching on the fiber $f^{-1}p$.  Thus by Proposition~\ref{prop:matchinghomotopy} there is a splitting homotopy $\gamma_p:C(f^{-1}p)\to C(f^{-1}p)$.     We have that $C=\bigoplus_{p\in P} C(f^{-1}p)$.  Consider $\Gamma:= \bigoplus_{p\in P} \gamma_p$.  $\Gamma$ is a diagonal endomorphism $C\to C$ and $\Gamma=\gamma_M$, the matching associated to $M$.  Since $\Gamma$ is diagonal with respect to the join-irreducible decomposition of $C$, it is order-preserving.

\end{proof}

Moreover, a filtered splitting homotopy for $L\to Sub(C)$ allows us to filter the associated Morse complex.  For any chain map $\psi:D\to C$ it is straightforward that the preimage of a subcomplex is a subcomplex.  Thus there is an induced map $\psi^{-1}:Sub(C)\to Sub(D)$ given by $B\in Sub(C)\mapsto \psi^{-1}(B)\in Sub(D)$.  In general this is not a lattice morphism.


\begin{prop}
Let $h:L\to Sub(C)$ be an $L$-filtered complex.  Let $\gamma$ be a filtered splitting homotopy and $(M_\gamma,\phi_\gamma,\psi_\gamma$ the associated Morse data. Define $m:L\to Sub(M_\gamma)$ as the composition $$L\xrightarrow{h} Sub(C) \xrightarrow{\psi^{-1}} Sub(M_\gamma)$$

Then  $m:L\to Sub(M_\gamma)$ is an $L$-filtered complex.
\end{prop}
\begin{proof}
We check the lattice properties.  

\end{proof}




%Finally, we record a proposition on how to create new $L$-filtered chain complexes using chain maps.
%
%\begin{prop}
%Let $h:L\to Sub(C)$.  Let $\phi:C\to D$ be a surjective chain map.  Define $h':L\to Sub(D)$ via $$h'(q):= \psi(h(q))$$  $h'$ is an $L$-filtered chain complex.
%\end{prop}
%\begin{proof}
%As $\phi$ is a chain map, it takes subcomplexes to subcomplexes.  Thus $\phi(h(q))$ is a subcomplex of $D$.  It remains to show that $h'$ is a lattice homomorphism.  
%
%As $\phi $ is a homomorphism, it preserves spans.  Since $h$ is a lattice morphism we have $$h'(q\vee p) = \phi(h(q\vee p)) = \phi(h(q)+ h(p)) = \phi(h(q))+ \phi(h(p))=h'(q)+h'(p)$$
%
%It is also straightforward that $$h'(q\wedge p) = \phi(h(q\wedge p)) = \phi(h(q)\cap h(p)) \subseteq \phi(h(q))\cap \phi(h(p))$$ It remains to show $\phi(h(q))\cap \phi(h(p))\subseteq \phi(h(q)\cap \phi(h(p))$.  
%
%
%
% For $x\in \phi(h(q))\cap \phi(h(p))$ we have $x = \psi(y)$ with $y\in h(q)$ and $x=\psi(y')$ for $y'\in h(p)$.  Since $\psi$ is a surjective it admits a section $s:D\to C$ with $\psi \circ s = id_D$. Thus $s(x) = s$
%
%
%\end{proof}


\subsection{Connection Matrix Algorithm}

In this section we introduce the algorithm for computing a connection matrix based on the Morse theory described above.

{\bf Algorithm}
\begin{enumerate}
\item Given a $L$-filtered chain complex
\item Choose a compatible basis (existence is guaranteed from Theorem~\ref{thm:cfdecomp} (\ref{thm:cfdecomp:basis})
\item Use~\cite[Algorithm 3.6]{focm} to produce an acyclic partial matching $(A,\mu:Q\to K)$
\item Use~\cite[Algorithm 3.12]{focm} to produce a splitting homotopy, with associated equivalences $(\phi,\psi)$
\item Construct filtered chain complex $f_{n+1}=\psi^{-1}\circ f_n$
\item Proceed until $f_{n+1}=f_n$ for some $n$
\end{enumerate}

\begin{prop}[Algorithm Correctness]
 The above algorithm terminates, i.e. after finitely many iterations the sequence $(f_n)$ stabilizes to a final fibration $f_\infty$.  Moreover, $f_\infty$ is a connection fibration strictly equivalent to $f_0$.
\end{prop}






 


%!TEX root = ../main.tex

\subsection{Algorithm}

A splitting fibration homotopy is a splitting homotopy and a fibration homotopy.

\begin{prop}
Let $f:C\to L$ be a chain fibration.  Let $\gamma$ be a splitting fibration homotopy.  Let $(M_\gamma, \Delta_\gamma)$ be the associated Morse complex.  Let $(\phi,\psi)$ be the Morse equivalences.  Define $g:(M_\gamma,\Delta_\gamma) \to L$ via $g:= f\circ \psi$.  Then $g$ is a coherent chain fibration and $g$ is homotopy equivalent to $f$.  

\end{prop}
\begin{proof}
$\psi$ is a section of $\phi$ by Proposition~\ref{prop:MorseEquiv}, thus it is injective.  Propositions~\ref{prop:compCF},~\ref{prop:compCoherent} imply $g$ is a coherent chain fibration.  We claim that the pair $(\phi,\psi)$ are equivalences between $f$ and $g$.  We first show that $\phi:f\to g$ and $\psi:g\to f$ are order-preserving maps.  $f\circ \psi = g$ thus $\psi$ is order-preserving.  $g\circ \phi = (f\circ \psi)\circ \phi = f(id_C - \gamma\partial_C - \partial_C \gamma) \leq f(x)$ as $\gamma$ and $\partial_C$ are order-preserving. Thus $\phi:f\to g$ is order-preserving.   Now $\phi \circ \psi = id_M$ and $\psi \circ \phi = id_C - \gamma \partial_C - \partial_C\gamma$, so $f$ and $g$ are homotopy equivalent.


\end{proof}

\begin{prop}
Let $f_c:X\to P$ be a cell fibration.  Let $(A,\mu)$ be an acyclic partial matching satisfying $\mu(Q)=K$ only if $K,Q\in f^{-1}p$ for some $p\in P$.  Then $\gamma_M$, the splitting homotopy from $M$ is order-preserving $f\to f$.
\end{prop}


\begin{proof}
Let $p\in P$.  Consider the $M_p$, the matching restricted to the fiber $f^{-1}p$.  By assumption this is an acyclic partial matching on the fiber $f^{-1}p$.  Thus by Proposition~\ref{prop:matchinghomotopy} there is a splitting homotopy $\gamma_p:C(f^{-1}p)\to C(f^{-1}p)$.     We have that $C=\bigoplus_{p\in P} C(f^{-1}p)$.  Consider $\Gamma:= \bigoplus_{p\in P} \gamma_p$.  $\Gamma$ is a diagonal endomorphism $C\to C$ and $\Gamma=\gamma_M$, the matching associated to $M$.  Since $\Gamma$ is diagonal with respect to the join-irreducible decomposition of $C$, it is order-preserving.

\end{proof}


{\bf Algorithm}
\begin{enumerate}
\item Given a coherent fibration $f_n$
\item Choose a compatible basis (existence is guaranteed from Theorem~\ref{thm:cfdecomp} (\ref{thm:cfdecomp:basis})
\item Use~\cite[Algorithm 3.6]{focm} to produce an acyclic partial matching $(A,\mu:Q\to K)$
\item Use~\cite[Algorithm 3.12]{focm} to produce a splitting homotopy, with associated equivalences $(\phi,\psi)$
\item Construct a new fibration $f_{n+1}:=f_n\circ \psi$
\item Proceed until $f_{n+1}=f_n$ for some $n$
\end{enumerate}

\begin{prop}[Algorithm Correctness]
 The above algorithm terminates, i.e. after finitely many iterations the sequence $(f_n)$ stabilizes to a final fibration $f_\infty$.  Moreover, $f_\infty$ is a connection fibration strictly equivalent to $f_0$.
\end{prop}



















\begin{thebibliography}{50}

\bibitem{cmdb}
Z.~Arai, W.~Kalies, H.~Kokubu, K.~Mischaikow, H.~Oka, and P.~Pilarczyk.
\newblock{A Database Schema for the Analysis of Global Dynamics of Multiparameter Systems},
\newblock{\em SIAM Journal of Applied Dynamical Systems}, 8: 757--789, (2009).

\bibitem{bk}
H.~Ban and W.~Kalies, 
\newblock{ A computational approach to ConleyÕs decomposition theorem}
\newblock{\em  Journal of Computational Nonlinear Dynamics}, 1:312--319, (2006).


\bibitem{bar}
M.~Barakat and S.~Maier-Paape.
\newblock{Computation of connection matrices using the software package conley}
\newblock{\em Internatational Journal of Bifurcation and Chaos}, 19(9):3033--3056 (2009).

\bibitem{bar2} 
M.~Barakat and D.~Robertz.
\newblock{ conley: computing connection matrices in Maple},
\newblock{\em Journal of Symbolic Computation} 44(5): 540--557 (2009).

\bibitem{bauer}
Bauer, U. and Edelsbrunner, H., 2016. 
\newblock{The Morse theory of Cech and Delaunay complexes.}
\newblock{ Transactions of the American Mathematical Society.}

\bibitem{bl}
J.B.~van den Berg and J.P.~Lessard. 
\newblock{Rigorous numerics in dynamics.}
\newblock{ Notices of the AMS} 62.9 (2015).

\bibitem{bush}
Bush, J., Cowan, W., Harker, S. and Mischaikow, K., 2016. 
\newblock{Conley--Morse Databases for the Angular Dynamics of Newton's Method on the Plane. SIAM Journal on Applied Dynamical Systems, 15(2), pp.736-766.}

\bibitem{cmdbchaos}
J.~Bush, M.~Gameiro, S.~Harker, H.~Kokubu, K.~Mischaikow, I.~Obayashi, P.~Pilarczyk.
\newblock{Combinatorial-topological framework for the analysis of global dynamics},
\newblock{\em Chaos}, 22, (2008).

\bibitem{bm}
Bush, J. and Mischaikow, K., 2014. 
\newblock{Coarse dynamics for coarse modeling: An example from population biology. Entropy, 16(6), pp.3379-3400.}


\bibitem{conley}
C.~Conley.
\newblock{\em  Isolated invariant sets and the Morse index},
\newblock{American Mathematical Society}, 1978.

\bibitem{cmdbProject}
Conley-Morse Database.
\newblock{\url{http://chomp.rutgers.edu/Projects/Databases_for_the_Global_Dynamics.html}}

\bibitem{dsgrnProject}
Dynamic Signatures of Genetic Regulatory Networks.
\newblock{\url{http://chomp.rutgers.edu/Projects/DSGRN/}.}

\bibitem{dsgrn}
B.~Cummins, T.~Gedeon, S.~Harker, K.~Mischaikow and K.~Mok,
\newblock{ Combinatorial Representation of Parameter Space for Switching Systems}
\newblock{ arXiv preprint arXiv:1512.04131. 2015}

\bibitem{dhmo}
S.~Day, Y.~Hiraoka, K.~Mischaikow, and T.~Ogawa.  
\newblock{Rigorous numerics for global dynamics: A study of the Switch-Hohenberg equation}
\newblock{SIAM Journal of Applied Dynamical Systems}, 4: 1--31 (2005).

\bibitem{fran}
R.~Franzosa.
\newblock{The connection matrix theory for {M}orse decompositions},
\newblock{\em  Transactions of the AMS},  311(2): 561 -- 592 (1989).

\bibitem{fran2}
R.~Franzosa.
\newblock{Index filtrations and the homology index braid for partially ordered Morse decompositions},
\newblock{\em Transactions of the AMS}, 298(1):193Ð213 (1986).

\bibitem{fran3}
R.~Franzosa.
\newblock{The continuation theory for Morse decompositions and connection matrices},
\newblock{\em Transactions of the AMS}, 310(2):781Ð803 (1988).

\bibitem{atm}
Franzosa, R. and Mischaikow, K., 1998. 
\newblock{Algebraic transition matrices in the Conley index theory.}
\newblock{ Transactions of the American Mathematical Society}, 350(3), pp.889-912.

\bibitem{gh}
T.~Gedeon, S.~Harker, H.~Kokubu, K.~Mischaikow, and H.~Oka,
\newblock{Global dynamics for steep nonlinearities in two dimensions.}
\newblock{ Physica D: Nonlinear Phenomena.} (2016).

\bibitem{gelfand}
Gelfand, S.I. and Manin, Y.I., 2013. 
\newblock{Methods of homological algebra.} Springer Science \& Business Media.

%\bibitem{lya}
%Goullet, A., Harker, S., Mischaikow, K., Kalies, W. and Kasti, D., 2015. 
%\newblock{Efficient computation of Lyapunov functions for Morse decompositions. DCDS-S.}


\bibitem{focm}
S.~Harker, K.~Mischaikow, M.~Mrozek and V.~Nanda.
\newblock{Discrete Morse Theoretic Algorithms for Computing Homology of Complexes and Maps}
\newblock{\em Foundations of Computational Mathematics}, (2013).

\bibitem{braids}
S.~Harker, K.~Mischaikow, K.~Spendlove, and R.~Vandervorst.
\newblock{Computing Connection Matrices for a Morse Theory on Braids}
\newblock{Preprint} 2017.

\bibitem{kmm}
Kaczynski, T., Mischaikow, K. and Mrozek, M., 2006. 
\newblock{Computational homology (Vol. 157). Springer Science \& Business Media.}


\bibitem{kmv}
W.D.~Kalies, K.~Mischaikow, and R.C.A.M.~Van der Vorst, 
\newblock{An algorithmic approach to chain recurrence},
\newblock{\em Foundations of Computational Mathematics} 5:409--449 (2005).

\bibitem{lsa}
W.D.~Kalies, K.~Mischaikow, and R.C.A.M.~van der Vorst,
\newblock{Lattice structures for attractors I}
\newblock{Journal of Computational Dynamics 1, pp. 307-338, 2014.}

\bibitem{lsa2}
Kalies, W.D., Mischaikow, K. and Vandervorst, R.C.A.M., 2016. 
\newblock{Lattice structures for attractors II.}
\newblock{ Foundations of Computational Mathematics,} 16(5), pp.1151-1191.


\bibitem{koz}
D.~Kozlov.
\newblock{\em Combinatorial Algebraic Topology.}
\newblock{ Algorithms and Computation in Mathematics, vol. 21.}, Springer, Berlin (2008).

\bibitem{lefschetz}
Lefschetz,S. 
\newblock{AlgebraicTopology.American Mathematical Society Colloquium Publications, vol.27.}
\newblock{American Mathematical Society,} New York (1942)

\bibitem{mpmw}
Maier-Paape, S., Mischaikow, K. and Wanner, T., 
\newblock{2007. Structure of the attractor of the Cahn-Hilliard equation on a square.}
\newblock{\em International Journal of Bifurcation and Chaos}, 17(4), pp.1221-1263.

\bibitem{mcmodels}
McCord, C., 2000. 
\newblock{Simplicial models for the global dynamics of attractors.}
\newblock{ Journal of Differential Equations,} 167(2), pp.316-356.

\bibitem{scalar}
McCord, C. and Mischaikow, K., 1996. 
\newblock{On the global dynamics of attractors for scalar delay equations. }
\newblock{Journal of the American Mathematical Society}, 9(4), pp.1095-1133.

\bibitem{mm}
Mischaikow, K. and Mrozek, M., 1995. 
\newblock{Chaos in the Lorenz equations: a computer-assisted proof. Bulletin of the American Mathematical Society, 32(1), pp.66-72.}

\bibitem{mn}
K.~Mischaikow and V.~Nanda. 
\newblock{ Morse Theory for Filtrations and Efficient Computation of Persistent Homology.}
\newblock{\em Discrete and Computational Geometry}, 50(2):330--353, 2013.

\bibitem{mc} 
C.~McCord, 1988. 
\newblock{Mappings and homological properties in the Conley index theory. Ergodic Theory and Dynamical Systems, 8(8*), pp.175-198.}

\bibitem{mcr}
C.~McCord and J.~Reineck.
\newblock{Connection matrices and transition matrices},
\newblock{\em Banach Center Publications}, 47(1): 41--55 (1999).

\bibitem{salamon} Robbin, J.W. and Salamon, D.A., 
\newblock{ 1992. Lyapunov maps, simplicial complexes and the Stone functor. }
\newblock{Ergodic Theory and Dynamical Systems}, 12(01), pp.153-183.

\bibitem{roman}
Roman, S., 2008. 
\newblock{Lattices and ordered sets.} Springer Science \& Business Media.

\bibitem{smale}
Smale, S., 
\newblock{1967. Differentiable dynamical systems. }
\newblock{\em Bulletin of the American mathematical Society}, 73(6), pp.747-817.

%\bibitem{rv}
%Rot, T.O. and Vandervorst, R.C., 2014. 
%\newblock{MorseÐConleyÐFloer homology. Journal of Topology and Analysis, 6(03), pp.305-338.}
%
%\bibitem{rvII}
%Rot, T.O. and Vandervorst, R.C.A.M., 2014. 
%\newblock{Functoriality and duality in MorseÐConleyÐFloer homology. Journal of Fixed Point Theory and Applications, 16(1-2), pp.437-476.}

%\bibitem{sal}
%Salamon, D., 1990. 
%\newblock{Morse Theory, the Conley Index and Floer Homology.  Bullettin of London Mathematical Society, 22, 113-140.}

\bibitem{sko}
E.~Sk\"{o}ldberg.
\newblock{Morse theory from an algebraic viewpoint.}
\newblock{\em Transactions of the AMS}, 358(1):115--129,2006.

\bibitem{homalg}
The homalg project.
\newblock{\url{http://homalg-project.github.io/index.html}.}

\bibitem{w}
Tucker, W. (2002). A rigorous ODE solver and SmaleÕs 14th problem. Foundations of Computational Mathematics, 2(1), 53-117.

\bibitem{weibel}
Weibel, C.A., 1995. 
\newblock{An introduction to homological algebra (No. 38).}
\newblock{ Cambridge university press.}



\end{thebibliography}
\end{document}
